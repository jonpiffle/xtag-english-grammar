\begin{abstract}
This document describes a sizable grammar of English written in the TAG
formalism and implemented for use with the XTAG system. This report and the
grammar described herein supersedes the TAG grammar described in
\cite{abeille90:TECH}. The English grammar described in this report is based on
the TAG formalism developed in \cite{joshi75}, which has been extended to
include lexicalization (\cite{schabes88}), and unification-based feature
structures (\cite{vijay91}).  The grammar discussed in this report extends the
grammar presented in \cite{abeille90:TECH} in at least two ways. First, this
grammar has more detailed linguistic analyses, and second, the grammar
presented in this paper is fully implemented.  The range of syntactic phenomena
that can be handled is large and includes auxiliaries (including inversion),
copula, raising and small clause constructions, topicalization, relative
clauses, infinitives, gerunds, passives, adjuncts, it-clefts, wh-clefts, PRO
constructions, noun-noun modifications, extraposition, determiner sequences,
genitives, negation, noun-verb contractions, sentential adjuncts and
imperatives.  The XTAG grammar has been relatively stable since November 1993,
although new analyses are always being added and old ones being modified.
\end{abstract}

