\documentstyle[11pt,leqno,lingmac,titlepage,aaai-named,twoside]{xtag-report}
\title{A  Lexicalized Tree Adjoining Grammar for English}
\author{The XTAG Research Group \\ \\
Institute for Research in Cognitive Science\\
University of Pennsylvania \\
3401 Walnut St., Suite 400C \\
Philadelphia, PA 19104-6228 \\ \\
xtag-request@linc.cis.upenn.edu
}
\date{March 9, 1995\\
\bigskip
IRCS 95-03}
%\makeindex
\oddsidemargin 0.25in
\evensidemargin 0.25in
\topmargin 0in
\textheight 8.5in
\textwidth 6.25in
%\typein {do the ``includeonly''}

\newcommand{\vertical}[1]{
\setlength{\unitlength}{0.012500in}%
%%%\hspace*{6pt}
\hspace*{2pt}
\begin{picture}(12,12)(0,0)
\put(0,0){\makebox(0,0)[lb]{\raisebox{0pt}[0pt][0pt]{\special{ps:gsave currentpoint currentpoint translate
%%%-90.0 rotate neg exch neg exch translate}\twlrm #1\special{ps:currentpoint
%%grestore moveto}}}}
-90.0 rotate neg exch neg exch translate}\elvrm #1\special{ps:currentpoint grestore moveto}}}}
\end{picture}
%%\hspace*{-18pt}
\hspace*{-18pt}
}

\newcommand{\xtagdef}[1]{{\sc #1}}
\newcommand{\xtagcheck}{$\surd$}

\setcounter{secnumdepth}{3}
\setcounter{tocdepth}{3}

\begin{document}
\setcounter{bottomnumber}{20}
\setcounter{topnumber}{20}
\renewcommand{\bottomfraction}{1}
\renewcommand{\topfraction}{1}
\setcounter{totalnumber}{30}
\renewcommand{\textfraction}{0}
\renewcommand{\floatpagefraction}{0}
\input{psfig}
\pagestyle{plain}

\maketitle
\pagenumbering{roman}
\tableofcontents
\listoffigures
\begin{abstract} 
 
This document describes a sizable grammar of English written in the TAG 
formalism and implemented for use with the XTAG system. This report and the 
grammar described herein supersedes the TAG grammar described in 
\cite{tech-rept98}. The English grammar described in this report is based 
on the TAG formalism developed in \cite{joshi75}, which has been extended 
to include lexicalization (\cite{schabes88}), and unification-based feature 
structures (\cite{vijay91}).  The range of syntactic phenomena that can be 
handled is large and includes auxiliaries (including inversion), copula, 
raising and small clause constructions, topicalization, relative clauses, 
infinitives, gerunds, passives, adjuncts, ditransitives (and datives), 
ergatives, it-clefts, wh-clefts, PRO constructions, noun-noun 
modifications, extraposition, determiner sequences, genitives, negation, 
noun-verb contractions, sentential adjuncts, imperatives and resultatives. 
The XTAG grammar is continuously updated with the addition of new analyses 
and modification of old ones, and an online version of this report can be 
found at the XTAG web page: {\tt http://www.cis.upenn.edu/\~{}xtag/}. 
 
\end{abstract} 
 

\pagestyle{plain}
\null\vfil
\begin{center}
{\bf Acknowledgements}
\end{center}
\setcounter{page}{0}

We are immensely grateful to Aravind Joshi for supporting this
project. 

The following people have contributed to the development of grammars
in the project: Anne Abeille, Jason Baldridge, Rajesh Bhatt, Kathleen
Bishop, Raman Chandrasekar, Sharon Cote, Beatrice Daille, Christine
Doran, Dania Egedi, Tim Farrington, Jason Frank, Caroline Heycock,
Beth Ann Hockey, Roumyana Izvorski, Karin Kipper, Daniel Karp, Seth
Kulick, Young-Suk Lee, Heather Matayek, Patrick Martin, Megan Moser,
Sabine Petillon, Rashmi Prasad, Laura Siegel, Yves Schabes, Victoria
Tredinnick and Raffaella Zanuttini.

The XTAG system has been developed by: Tilman Becker, Richard
Billington, Andrew Chalnick, Dania Egedi, Devtosh Khare, Albert Lee,
David Magerman, Alex Mallet, Patrick Paroubek, Rich Pito, Gilles
Prigent, Carlos Prolo, Anoop Sarkar, Yves Schabes, William Schuler,
B. Srinivas, Fei Xia, Yuji Yoshiie and Martin Zaidel.

We would also like to thank Michael Hegarty, Lauri Karttunen, Anthony
Kroch, Mitchell Marcus, Martha Palmer, Owen Rambow, Phillip Resnik,
Beatrice Santorini and Mark Steedman.

In addition, Jeff Aaronson, Douglas DeCarlo, Mark-Jason Dominus, Mark
Foster, Gaylord Holder, David Mageman, Ken Noble, Steven Shapiro and
Ira Winston have provided technical support.  Adminstrative support
was provided by Susan Deysher, Carolyn Elken, Jodi Kerper, Christine
Sandy and Trisha Yannuzzi.

This work was partially supported by ARO grant DAAL03-89-0031, ARPA
grant N00014-90-J-1863, NSF STC grant DIR-8920230, and Ben Franklin
Partnership Program (PA) grant 93S.3078C-6. 

\newpage


\pagenumbering{arabic}
\pagestyle{headings}
\part{General Information}
\chapter{Getting Around}

This technical report presents the English XTAG grammar as implemented by the
XTAG Research Group at the University of Pennsylvania.  The technical report is
organized into four parts, plus a set of appendices.  Part 1 contains general
information about the XTAG system and some of the underlying mechanisms that
help shape the grammar.  Chapter~\ref{intro-FBLTAG} contains an introduction to
the formalism behind the grammar and parser, while Chapter~\ref{overview}
contains information about the entire XTAG system.  Linguists interested solely
in the grammar of the XTAG system may safely skip Chapters~\ref{intro-FBLTAG}
and \ref{overview}.  Chapter~\ref{underview} contains information on some of
the linguistic principles that underlie the XTAG grammar, including the
distinction between complements and adjuncts, and how case is handled.

The actual description of the grammar begins with Part 2, and is contained in
the following three parts.  Parts 2 and 3 contains information on the verb
classes and the types of trees allowed within the verb classes, respectively,
while Part 4 contains information on trees not included in the verb classes
(e.g.  NP's, PP's, various modifiers, etc).  Chapter~\ref{table-intro} of Part
2 contains a table that attempts to provide an overview of the verb classes and
tree types by providing a graphical indication of which tree types are allowed
in which verb classes.  This has been cross-indexed to tree figures shown in
the tech report.  Chapter~\ref{verb-classes} contains an overview of all of the
verb classes in the XTAG grammar.  The rest of Part 2 contains more details on
several of the more interesting verb classes, including ergatives, sentential
subjects, sentential complements, small classes, ditransitives, and it-clefts.

Part 3 contains information on some of the tree types that are available within
the verb classes.  These tree types correspond to what would be transformations
in a movement based approach.  Not all of these types of trees are contained in
all of the verb classes.  The table (previously mentioned) in Part 2 contains a
list of the tree types and indicates which verb classes each occurs in.  

Part 4 focuses on the non-verb class trees in the grammar.  NP's and
determiners are presented in Chapter~\ref{det-comparitives}, while the various
modifier trees are presented in Chapter~\ref{modifiers}.  Auxiliary verbs,
which are classed separate from the verb classes, are presented in
Chapter~\ref{auxiliaries}, while certain types of conjunction are shown in
Chapter~\ref{conjunction}.  The XTAG treatment of comparatives is
presented in Chapter~\ref{compars-chapter}, and our treatment of
punctuation is discussed in Chapter~\ref{punct-chapt}.

Throughout the technical report, mention is occasionally made of
changes or analyses that we hope to incorporate in the future.
Appendix~\ref{future-work} details a list of these and other future
work.  The appendices also contain information on some of the nitty
gritty details of the XTAG grammar, including  a system of metarules which
can be used for grammar development and maintenance
(Appendix~\ref{metarules}), the tree naming
conventions (Appendix~\ref{tree-naming}), and a comprehensive list of the features
used in the grammar (Appendix~{\ref{features}).
Appendix~\ref{evaluation} contains an evaluation of the XTAG grammar,
including comparisons with other wide coverage grammars.



\section{Introduction}
The English grammar described in this report is based on the TAG formalism
developed in Joshi, Levy, and Takahashi \cite{joshi75}, which has been extended
to include lexicalization \cite{schabes88}, and unification-based feature
structures \cite{vijay91}. Tree Adjoining Languages (TALs) fall into the class
of mildly context-sensitive languages, and as such are more powerful than
context free languages.  The TAG formalism in general, and lexicalized TAGs in
particular, are well-suited for linguistic applications.  As first shown by
\cite{joshi85} and \cite{kj87}, the properties of TAGs permit us to encapsulate
diverse syntactic phenomena in a very natural way.  For example, TAG's extended
domain of locality and its factoring of recursion from local dependencies lead,
among other things, to a localization of so-called unbounded dependencies.

\subsection{TAG formalism}

The primitive elements of the standard TAG formalism are known as {\sc
elementary trees}.  Elementary trees are of two types: {\sc initial trees} and
{\sc auxiliary trees} (see Figure \ref{elem-fig}).  In describing natural
language, {\sc initial trees} are minimal linguistic structures that contain no
recursion, i.e. trees containing the phrasal structure of simple sentences,
NPs, PPs, and so forth.  Initial trees are characterized by the following: 1)
all internal nodes are labeled by non-terminals, 2) all leaf nodes are labeled
by terminals, or by non-terminal nodes marked for substitution. An initial tree
is called an X-type initial tree if its root is labeled with type X.  

\begin{figure}[ht]
\centering
\rule[.1in]{\textwidth}{0.01in} 
\psfig{figure=ps/intro-files/schematic-elem-trees.ps,height=2.0in}
\caption{Elementary trees in TAG}
\rule[.1in]{\textwidth}{0.01in} 
\label{elem-fig}
\end{figure}

Recursive structures are represented by {\sc auxiliary trees}, which represent
constituents that are adjuncts to basic structures (e.g. adverbials).  {\sc
Auxiliary trees} are characterized as follows: 1) all internal nodes are
labeled by non-terminals, 2) all leaf nodes are labeled by terminals, or by
non-terminal nodes marked for substitution, except for exactly one non-terminal
node, called the foot node, which can only be used to adjoin the tree to
another node\footnote{A null adjunction constraint (NA) is systematically put
on the footnode of an auxiliary tree. This disallows adjunction of a tree onto
a footnode itself.}, 3) the foot node has the same label as the root node of
the tree.

There are two operations defined in the TAG formalism,
substitution\footnote{Technically, substitution is a specialized version of
adjunction, but it is useful to make a distinction between the two.} and
adjunction.  In the substitution operation, the root node on an initial tree is
merged into a non-terminal leaf node marked for substitution in another initial
tree, producing a new tree.  The root node and the substitution node must have
the same name.  Figure \ref{proto-subst} shows two initial trees and the tree
resulting from the substitution of one tree into the other.

\begin{figure}[ht]
\centering
\rule[.1in]{\textwidth}{0.01in} 
\psfig{figure=ps/intro-files/schematic-subst2.ps,height=2.0in}
\caption{Substitution in TAG}
\rule[.1in]{\textwidth}{0.01in} 
\label{proto-subst}
\end{figure}

In an adjunction operation, an auxiliary tree is grafted onto a non-terminal
node anywhere in an initial tree.  The root and foot nodes of the auxiliary
tree must match the node at which the auxiliary tree adjoins.  Figure
\ref{proto-adjunction} shows an auxiliary tree and an initial tree, and the
tree resulting from an adjunction operation.

\begin{figure}[ht]
\centering
\rule[.1in]{\textwidth}{0.01in} 
\psfig{figure=ps/intro-files/schematic-adjunction2.ps,height=2.0in}
\caption{Adjunction in TAG}
\rule[.1in]{\textwidth}{0.01in} 
\label{proto-adjunction}
\end{figure}

The tree set of a TAG $G$, ${\cal T}(G)$ is defined to be the set
of all derived trees starting from S-type initial trees in $I$ whose frontier
consists of terminal nodes (all substitution nodes having been filled). The
string language generated by a TAG, ${\cal L}(G)$, is defined to be the
set of all terminal strings on the frontier of the trees in ${\cal T}(G)$.

\subsection{Lexicalization}

`Lexicalized' grammars systematically associate each elementary structure with
a lexical anchor. This means that in each structure there is a lexical item
that is realized.  It does not mean simply adding feature structures (such as
head) and unification equations to the rules of the formalism.  These resultant
elementary structures specify extended domains of locality (as compared to
CFGs) over which constraints can be stated.

Following \cite{schabes88} we say that a grammar is `lexicalized' if it
consists of 1) a finite set of structures each associated with a lexical item,
and 2) an operation or operations for composing the structures.  Each lexical
item will be called the {\it anchor} of the corresponding structure, which
defines the domain of locality over which constraints are specified.  Note
then, that constraints are local with respect to their anchor.

Not every grammar is in a `lexicalized' form.\footnote{Notice the similarity of
the definition of `lexicalized' grammar with the off line parsibility
constraint \cite{kaplan83}. As consequences of our definition, each structure
has at least one lexical item (its anchor) attached to it and all sentences are
finitely ambiguous.} In the process of lexicalizing a grammar, the
`lexicalized' grammar is required to be strongly equivalent to the original
grammar, i.e., it must produce not only the same language, but the same
structures or tree set as well.

\begin{figure*}[ht]
\centering
\rule[.1in]{\textwidth}{0.01in} 
\begin{tabular}{cccc}
{{\psfig{figure=ps/intro-files/john.ps,height=1.0in}}\label{fig1a}}  &
{{\psfig{figure=ps/intro-files/walked.ps,height=1.4in}}\label{fig1b}}  & 
{{\psfig{figure=ps/intro-files/to.ps,height=1.7in}} \label{fig1c} }  & 
{{\psfig{figure=ps/intro-files/philly.ps,height=1.0in}} \label{fig1d}} \\
(a)&(b)&(c)&(d)\\
\end{tabular}\\
\caption {Lexicalized Elementary Trees}
\rule[.1in]{\textwidth}{0.01in} 
\label {lex-elem-trees}
\end{figure*}

In Figure \ref{lex-elem-trees}, which shows sample initial and auxiliary trees,
substitution sites are marked by a ($\downarrow$), and foot nodes are marked by
an ($\ast$).  This notation is standard and is followed in the rest of this
report.


\subsection{Unification-based features}

In a unification framework, a feature structure is associated with each node in
an elementary tree.  This feature structure contains information about how the
node interacts with other nodes in the tree.  It consists of a top part, which
generally contains information relating to the supernode, and a bottom part,
which generally contains information relating to the subnode.  Substitution
nodes, however, have only the top features, since the tree substiuting in must
logically carry the bottom features.

The notions of substitution and adjunction must be augmented to fit within this
new framework.  The feature structure of a new node created by substitution
inherits the union of the features of the original nodes.  The top feature of
the new node is the union of the top features of the two original nodes, while
the bottom feature of the new node is simply the bottom feature of the top node
of the subsituting tree (since the substitution node has no bottom feature).
Feature \ref{subst-fig} shows this more clearly.

\begin{figure}[ht]
\centering
\rule[.1in]{\textwidth}{0.01in} 
\psfig{figure=ps/intro-files/schematic-feat-subst.ps,height=2.0in}
\caption{Substitution in FB-LTAG}
\rule[.1in]{\textwidth}{0.01in} 
\label{subst-fig}
\end{figure}

Adjunction is only slightly more complicated.  The node being adjoined into
splits, and its top feature unifies with the top feature of the root
adjoining node, while its bottom feature unifies with the bottom feature of the
foot adjoining node.  Again, this is easier shown graphically, as in Figure
\ref{adjunct-fig}.

\begin{figure}[ht]
\centering
\rule[.1in]{\textwidth}{0.01in} 
\psfig{figure=ps/intro-files/schematic-feat-adjunction.ps,height=2.0in}
\caption{Adjunction in FB-LTAG}
\label{adjunct-fig}
\rule[.1in]{\textwidth}{0.01in} 
\end{figure}


The embedding of the TAG formalism in a unification framework allows us to
dynamically specify local constraints that would have otherwise had to have
been made statically within the trees.  Constraints that verbs make on their
complements, for instance, can be implemented through the feature structures.
The notions of Obligatory and Selective Adjunction, crucial to the formation of
lexicalized grammars, can also be handled through the use of
features\footnote{The remaining constraint, Null Adjunction (NA), must still be
specified directly on a node.} Perhaps more important to developing a grammar,
though, is that the trees can serve as a schemata to be instantiated with
lexical-specific features when an anchor is associated with the tree.  To
illustrate this, Figure \ref{lex-with-features} shows the same tree lexicalized
with two different verbs, each of which instantiates the features of the tree
according to its lexical semantics.

\begin{figure*}[ht]
\centering
\begin{tabular}{cc}
{\psfig{figure=ps/intro-files/think-feat.ps,height=5.0in}}  &
{\psfig{figure=ps/intro-files/want-feat.ps,height=5.0in}} \\
{\it think} tree&{\it want} tree\\
\end{tabular}\\
\caption {Lexicalized Elementary Trees with Features}
\label {lex-with-features}
\end{figure*}

In Figure \ref{lex-with-features}, the lexical item {\it thinks} takes an
indicative sentential complement, as you might find in a sentence such as {\it
John thinks that Mary loves Sally}.  {\it Want} takes a sentential complement
as well, but an infinitive one, as in {\it John wants to love Mary}.  This
distinction is easily captured in the features and passed to other nodes to
constrain which trees this tree can adjoin into, both cutting down the number
of separate trees needed and enforcing conceptual Selective Adjunctions (SA).



\chapter{Components of the XTAG System}
\label{overview}

This section focuses on the various components that comprise the
parser and English grammar in the XTAG system.  Persons interested
only in the linguistic analyses in the grammar may skip this section
without loss of continuity, although a quick glance at the tagset used
in XTAG and the set of non-terminal labels used will be useful. We may
occasionally refer back to the various components mentioned in this
section.

\section{System Description}

Figure~{\ref{flowchart}} shows the overall flow of the system when
parsing a sentence; a summary of each component is presented in
Table~\ref{sys-table}. At the heart of the system is a parser for
lexicalized TAGs (\cite{schabesjoshi88,schabes90}) which produces all
legitimate parses for the sentence. The parser has two phases: {\bf
Tree Selection} and {\bf Tree Grafting}.

\begin{figure}[t]
\hspace{0.35in}
\centering
{\psfig{figure=ps/flowchart.eps,height=3.0in,angle=270}}
\caption[XTAG system diagram]{Overview of XTAG system }
\label{flowchart}
\end{figure}

\begin{table}[ht]
\small
\centering
\begin{tabular}{|l|l|} \hline
Component & Details \\ \hline
Morphological & Consists of approximately 317,000 inflected items \\ 
Analyzer and & derived from over 90000 stems. \\ 
Morph Database & Entries are indexed on the inflected form and return \\
& the root form, POS, and inflectional information.\\ \hline
POS Tagger & Wall Street Journal-trained
trigram tagger (\cite{kwc88})  \\ 
and  Lex Prob & extended to output N-best POS sequences  \\
Database & (\cite{soong90}). Decreases the time to parse \\
&a sentence by an average of 93\%. \\\hline
Syntactic &  More than 30,000 entries. \\
Database & Each entry consists of: the uninflected form of the word, \\
& its POS, the list of trees or tree-families associated with \\
& the word, and a list of feature equations that capture \\
&lexical idiosyncrasies. \\ \hline
Tree Database &  1004 trees, divided into 53 tree families and 221 individual \\
& trees. Tree families represent subcategorization frames; \\
& the trees in a tree family would be related to each other \\

& transformationally in a movement-based approach. \\ \hline
X-Interface & Menu-based facility for creating and modifying tree files. \\
&  User controlled parser parameters: parser's start category, \\ 
& enable/disable/retry on failure for POS tagger. \\
& Storage/retrieval facilities for elementary and parsed trees.\\
& Graphical displays of tree and feature data structures. \\
& Hand combination of trees by adjunction or substitution \\
& for grammar development. \\ 
& Ability to manually assign POS tag \\
& and/or Supertag before parsing \\ \hline
\end{tabular}
\caption{System Summary}
\label{sys-table}
\end{table}

\subsection{Tree Selection}

Since we are working with lexicalized TAGs, each word in the sentence
selects at least one tree. The advantage of a lexicalized formalism
like LTAGs is that rather than parsing with all the trees in the
grammar, we can parse with only the trees selected by the words in the
input sentence.

In the XTAG system, the selection of trees by the words is done in
several steps. Each step attempts to reduce ambiguity, i.e. reduce the
number of trees selected by the words in the sentence.

\begin{description}
\item[Morphological Analysis and POS Tagging] The input sentence is
  first submitted to the {\bf Morphological Analyzer} and the {\bf
    Tagger}. The morphological analyzer~(\cite{karp92}) consists of a
  disk-based database (a compiled version of the derivational rules)
  which is used to map an inflected word into its stem, part of speech
  and feature equations corresponding to inflectional information.
  These features are inserted at the anchor node of the tree
  eventually selected by the stem. The POS Tagger can be disabled in
  which case only information from the morphological analyzer is used.
  The morphology data was originally extracted from the Collins
  English Dictionary (\cite{ced79}) and Oxford Advanced Learner's
  Dictionary (\cite{oald74}) available through ACL-DCI
  (\cite{liberman89}), and then cleaned up and augmented by hand
  (\cite{karp92}).
    
\item[POS Blender] The output from the morphological analyzer and the
  POS tagger go into the {\bf POS Blender} which uses the output of
  the POS tagger as a filter on the output of the morphological
  analyzer. Any words that are not found in the morphological database
  are assigned the POS given by the tagger.
  
\item[Syntactic Database] The syntactic database contains the mapping
  between particular stem(s) and the tree templates or tree-families
  stored in the {\bf Tree Database} (see Table~\ref{sys-table}). The
  syntactic database also contains a list of feature equations that
  capture lexical idiosyncrasies. The output of the POS Blender is
  used to search the {\bf Syntactic Database} to produce a set of
  lexicalized trees with the feature equations associated with the
  word(s) in the syntactic database unified with the feature equations
  associated with the trees. Note that the features in the syntactic
  database can be assigned to any node in the tree and not just to the
  anchor node. The syntactic database entries were originally
  extracted from the Oxford Advanced Learner's Dictionary
  (\cite{oald74}) and Oxford Dictionary for Contemporary Idiomatic
  English (\cite{cie75}) available through ACL-DCI
  (\cite{liberman89}), and then modified and augmented by hand
  (\cite{EgediMartin94}).  There are more than 31,000 syntactic
  database entries.\footnote{This number does not include trees
    assigned by default based on the part-of-speech of the word.}
  Selected entries from this database are shown in
  Table~\ref{syn-entries}.
    
\item[Default Assignment] For words that are not found in the
  syntactic database, default trees and tree-families are assigned
  based on their POS tag.
  
\item[Filters] Some of the lexicalized trees chosen in previous stages
  can be eliminated in order to reduce ambiguity. Two methods are
  currently used: structural filters which eliminate trees which have
  impossible spans over the input sentence and a statistical filter
  based on unigram probabilities of non-lexicalized trees (from a hand
  corrected set of approximately 6000 parsed sentences). These methods
  speed the runtime by approximately 87\%.
  
\item[Supertagging] Before parsing, one can avail of an optional step
  of {\em supertagging} the sentence. This step uses statistical
  disambiguation to assign a unique elementary tree (or {\em
    supertag}) to each word in the sentence. These assignments can
  then be hand-corrected. These supertags are used as a filter on the
  tree assignments made so far. More information on supertagging can
  be found in (\cite{srini97diss,srini97iwpt}).

\end{description}

\begin{table}[htb]
\begin{verbatim}
<<INDEX>>porousness<<ENTRY>>porousness<<POS>>N
<<TREES>>^BNXN ^BN ^CNn
<<FEATURES>>#N_card- #N_const- #N_decreas- #N_definite- #N_gen- 
#N_quan- #N_refl-

<<INDEX>>coo<<ENTRY>>coo<<POS>>V<<FAMILY>>Tnx0V

<<INDEX>>engross<<ENTRY>>engross<<POS>>V<<FAMILY>>Tnx0Vnx1
<<FEATURES>>#TRANS+

<<INDEX>>forbear<<ENTRY>>forbear<<POS>>V<<FAMILY>>Tnx0Vs1
<<FEATURES>>#S1_WH- #S1_inf_for_nil

<<INDEX>>have<<ENTRY>>have<<POS>>V<<ENTRY>>out<<POS>>PL
<<FAMILY>>Tnx0Vplnx1
\end{verbatim}   
\caption{Example Syntactic Database Entries.}

\label{syn-entries}
\end{table}

\subsection{Tree Database}
\label{tree-db}

The {\bf Tree Database} contains the tree templates that are
lexicalized by following the various steps given above. The lexical
items are inserted into distinguished nodes in the tree template
called the {\em anchor nodes}.  The part of speech of each word in the
sentence corresponds to the label of the anchor node of the trees.
Hence the tagset used by the POS Tagger corresponds exactly to the
labels of the anchor nodes in the trees.  The tagset used in the XTAG
system is given in Table~\ref{tagset}. The tree templates are
subdivided into tree families (for verbs and other predicates), and
tree files which are simply collections of trees for lexical items
like prepositions, determiners, etc%
\footnote{ The nonterminals in the tree database are {\tt A, AP, Ad,
    AdvP, Comp, Conj, D, N, NP, P, PP, Punct, S, V, VP}.}%
.

\subsection{Tree Grafting}

Once a particular set of lexicalized trees for the sentence have been
selected, XTAG a parsing algorithm for LTAGs
(\cite{schabesjoshi88,schabes90}) to find all derivations for the
sentence. The derivation trees and the associated derived trees can be
viewed using the X-interface (see Table~\ref{sys-table}). The
X-interface can also be used to save particular derivations to disk.

The output of the parser for the sentence {\it I had a map yesterday} is
illustrated in Figure~\ref{sentence}. The parse tree\footnote{The feature
structures associated with each note of the parse tree are not shown here.}
represents the surface constituent structure, while the derivation tree
represents the derivation history of the parse. The nodes of the derivation
tree are the tree names anchored by the lexical items\footnote{Appendix
\ref{tree-naming} explains the conventions used in naming the trees.}.  The
composition operation is indicated by the nature of the arcs: a dashed line is
used for substitution and a bold line for adjunction.  The number beside each
tree name is the address of the node at which the operation took place.  The
derivation tree can also be interpreted as a dependency graph with unlabeled
arcs between words of the sentence.

\begin{figure}[htb]
\centering
\begin{tabular}{cc}
{{\psfig{figure=ps/overview-files/derived.ps,height=3.0in}}}  &
{{\psfig{figure=ps/overview-files/derivation.ps,height=2.0in,width=2.7in}}} \\
Parse Tree  & Derivation Tree \\
\end{tabular}
\caption{Output Structures from the Parser}
\label{sentence}
\end{figure}

\begin{table*}[ht]
\small
\centering
\begin{tabular}{|l|l|} \hline
Part of Speech & Description \\ \hline
A & Adjective \\ \hline
Ad & Adverb \\ \hline
Comp & Complementizer \\ \hline
D & Determiner \\ \hline
G & Genitive Noun \\ \hline
I & Interjection \\ \hline
N & Noun \\ \hline
P & Preposition \\ \hline
PL & Particle \\ \hline
Punct & Punctuation \\ \hline
V & Verb \\ \hline
\end{tabular}
\caption{XTAG tagset}
\label{tagset}
\end{table*}

The morphological and syntactic databases are documented in
(\cite{karp92,EgediMartin94}). XTAG also has a parsing and grammar
development interface documented in (\cite{PSJ92}). This interface
includes a tree editor, the ability to vary parameters in the parser,
work with multiple grammars and/or parsers, and use metarules for more
efficient tree editing and construction (\cite{becker94}). These
papers refer to an earlier version of the grammar development
tools. They have been replaced by newer implementations which are
described on the XTAG web page. Also, more information and download
instructions about the XTAG grammar, the parser and the grammar
development interface can be obtained on the XTAG web page at \\ {\tt
http://www.cis.upenn.edu/\~{}xtag/}.


\chapter{Underview}
\label{underview}

The morphological, syntactic, and tree databases together comprise the
English grammar\footnote{ See Chapter~\ref{overview} for details on
these levels of representation. }.  A lexical item that is not in the
databases receives a default tree selection and features for its part
of speech and morphology.  

In designing the grammar, a decision was made early on to err on the
side of acceptance whenever there are conflicting opinions as to
whether or not a construction is grammatical.  In this sense, the XTAG
English grammar is intended to function primarily as an acceptor
rather than a generator of English sentences.  The range of syntactic
phenomena that can be handled is large and includes auxiliaries
(including inversion), copula, raising and small clause constructions,
topicalization, relative clauses, infinitives, gerunds, passives,
adjuncts, it-clefts, wh-clefts, PRO constructions, noun-noun
modifications, determiner sequences, genitives, negation, noun-verb
contractions, clausal adjuncts and imperatives.

This chapter is meant to be an introductory look at grammar
organization in XTAG. It serves as basic information the reader would
need to read the more detailed information about the grammar in
subsequent parts of the technical report.

\section{Tree Families and Subcategorization Frames}
\label{subcat-frames}

Elementary trees for predicative words\footnote{ Such as non-auxiliary
verbs or predicative nouns. } are used to represent the linguistic
notion of subcategorization.  The anchor of the elementary tree
subcategorizes for the other elements that appear in the tree, forming
a clausal or sentential structure.  Tree families group together trees
that belong to the same subcategorization frame. Consider the
following uses of the verb {\it buy}:

\enumsentence{Srini bought a book.}
\enumsentence{Srini bought Beth a book.}

In sentence (\ex{-1}), the verb {\it buy} subcategorizes for a direct
object NP.  The elementary tree anchored by {\it buy} is shown in
Figure~\ref{subcat-trees}(a) and includes nodes for the NP complement
of {\it buy} and for the NP subject.  In addition to this declarative
tree structure, the tree family also contains the trees that would be
related to each other transformationally in a movement based approach,
i.e passivization, imperatives, wh-questions, relative clauses, and so
forth.  Each tree family selected by its anchor represents all the
various syntactic environments that it can appear in.

Sentence (\ex{0}) shows that {\it buy} also subcategorizes for
a double NP object.  This means that {\it buy} also selects the double
NP object subcategorization frame, or tree family, with its own set of
transformationally related sentence structures.
Figure~\ref{subcat-trees}(b) shows the declarative structure for this
set of sentence structures.

\begin{figure}[ht]
\centering
\begin{tabular}{ccc}
{\psfig{figure=ps/compl-adj-files/alphanx0Vnx1_bought.ps,height=1.8in}} & 
\hspace*{0.5in} & 
{\psfig{figure=ps/compl-adj-files/alphanx0Vnx2nx1_bought.ps,height=1.8in}}\\
(a) & \hspace*{0.5in} & (b) \\ 
\end{tabular}
\caption{Different subcategorization frames for the verb {\it buy}}
\label{subcat-trees}
\end{figure}

Entire classes of anchors select a tree family. All the transitive
verbs form a class which selects the transitive tree family. Recall
that a tree family is a group of trees related by some syntactic
transformations. Since an entire class of anchors are selecting this
tree family, the assumption is that these syntactic transformations
are valid for each member of this class. For instance, wh- extraction
is a syntactic transformation that will apply regardless of any
idiosyncratic properties of any particular anchor of the tree family.
Chapter~\ref{verb-classes} contains more information about
the different tree families in the grammar.

There are some syntactic transformations, however, that are sensitive
to the properties of a particular anchor within the same
subcategorization frame. The ergative (or transitive-inchoative)
alternation for transitive verbs is one such transformation. Only a
subset of the transitive verbs can undergo this transformation. A more
rigorous definition of tree family that accounts for such lexical
idiosyncrasies within an otherwise homogeneous family is discussed in
detail in Appendix~\ref{families}.

\section{Complements and Adjuncts}
\label{compl-adj}

Complements and adjuncts have very different structures in the XTAG grammar.
Complements are included in the elementary tree anchored by the verb that
selects them, while adjuncts do not originate in the same elementary tree as
the verb anchoring the sentence, but are instead added to a structure by
adjunction.  The contrasts between complements and adjuncts have been
extensively discussed in the linguistics literature and the classification of a
given element as one or the other remains a matter of debate (see
\cite{rizzi90},
\cite{larson88}, \cite{jackendoff90}, \cite{larson90}, \cite{cinque90}, 
\cite{obernauer84}, \cite{lasnik-saito84}, and \cite{chomsky86}).  The guiding
rule used in developing the XTAG grammar is whether or not the sentence is
ungrammatical without the questioned structure.\footnote{Iteration of a
structure can also be used as a diagnostic: {\it Srini bought a book at the
bookstore on Walnut Street for a friend}.} Consider the following
sentences:

\enumsentence{Srini bought a book.}
\enumsentence{Srini bought a book at the bookstore.}
\enumsentence{Fei ventured into the cave.}
\enumsentence{$\ast$Fei ventured.}

Prepositional phrases are common adjuncts, and when they are used as
adjuncts they have a tree structure such as that shown in
Figure~\ref{compl-adjunct}(a).  This adjunction tree would adjoin into
the tree shown in Figure~\ref{subcat-trees}(a) to generate sentence
(\ex{-2}).  There are verbs, however, such as {\it venture}, {\it
hunger} and {\it differentiate}, that take prepositional phrases as
complements.  Sentences (\ex{-1}) and (\ex{0}) clearly show that the
prepositional phrase are not optional for {\it venture}.
% For these
%sentences, the prepositional phrase will be an initial tree (as shown
%in Figure~\ref{compl-adjunct}(b)) that substitutes into an elementary
%tree, such as the one anchored by the verb {\it arrange} in
%Figure~\ref{compl-adjunct}(c).
For verbs such as this, the prepositional phrase is articulated within the
verb's elementary tree, as shown in Figure~\ref{compl-adjunct}(b). The
preposition and its complement noun phrase are substituted into this elementary
tree. 


\begin{figure}[ht]
\centering
\begin{tabular}{ccc}
{\psfig{figure=ps/compl-adj-files/betavxPnx_at.ps,height=1.8in}} &
\hspace{0.5in} &
%{\psfig{figure=ps/compl-adj-files/alphaPXPnx_for.ps,height=1.3in}} &
%\hspace{0.5in} & 
{\psfig{figure=ps/compl-adj-files/alphanx0Vpnx1_ventured_.ps,height=1.8in}}\\
(a) & \hspace{0.5in} & (b) \\ 
\end{tabular}
\caption{Trees illustrating the difference between Complements and Adjuncts}
\label{compl-adjunct}
\label{2;1,9}
\end{figure}


Virtually all parts of speech, except for main verbs, function as both
complements and adjuncts in the grammar.  More information is available in this
report on various parts of speech as complements: adjectives (e.g. section
\ref{nx0Vax1-family}), nouns (e.g.  section~\ref{nx0Vnx1-family}), and
prepositions (e.g. section~\ref{nx0Vpnx1-family}); and as adjuncts: adjectives
(section~\ref{adj-modifier}), adverbs (section~\ref{adv-modifier}), nouns
(section~\ref{noun-modifier}), and prepositions (section~\ref{prep-modifier}).

\section{Non-S constituents}

Although sentential trees are generally considered to be special cases
in any grammar, insofar as they make up a `starting category', it is
the case that any initial tree constitutes a phrasal constituent.
These initial trees may have substitution nodes that need to be filled
(by other initial trees), and may be modified by adjunct trees,
exactly as the trees rooted in S.  Although grouping is possible
according to the heads or anchors of these trees, we have not found
any classification similar to the subcategorization frames for verbs
that can be used by a lexical entry to create a tree family of the set
of trees selected by such entries.  These trees are selected one by
one by each lexical item, according to each lexical item's
idiosyncrasies.  The grammar described by this technical report places
them into several files for ease of use, but these files do not
constitute tree families in the way that the subcategorization frames
do.









% \section{Case Assignment}
\label{case-assignment}
\subsection{Approaches to Case}
\subsubsection{Case in GB theory}

GB (Government and Binding) theory proposes the following ``case filter'' as a
requirement on S-structure.\footnote{There are certain problems with applying
the case filter as a requirement at the level of S-structure.  These issues are
not crucial to the discussion of the English LTAG implementation of case and so
will not be discussed here.  Interested readers are referred to
\cite{lasnik-uriagereka88}.}

\begin{verse}
\underline{Case Filter}
Every overt NP must be assigned abstract case.
\end{verse}

Abstract case is taken to be universal.  Languages with rich morphological case
marking, such as Latin, and languages with very limited morphological case
marking, like English, are all presumed to have full systems of abstract case
that differ only in the extent of morphological realization.

In GB, abstract case is assigned to NPs by various case assigners, namely
verbs, prepositions, and INFL.  Verbs and prepositions are said to assign
accusative case to NPs that they govern, and INFL assigns nominative case to
NPs that it governs.  These governing categories are constrained in where they
can assign case by means of `barriers' based on `minimality conditions',
although these are relaxed in `exceptional case marking' situations.  The
details of the GB analysis are beyond the scope of this technical report, but
see \cite{chomsky86} for the original analysis or \cite{haegeman91} for an
overview.  Let it suffice for us to say that the notion of abstract case and
the case filter are useful in accounting for a number of phenomenon including
the distribution of nominative and accusative case, and the distribution of
overt NPs and empty categories (such as PRO).

\subsubsection{Minimalism and Case} 

A major conceptual difference between GB theories and minimalism is that in
minimalism, lexical items carry their features with them rather than being
assigned their features based on the nodes that they end up at.  For nouns,
this means that they carry case with them, and that case is 'checked' by
AGR$_s$ or AGR$_o$, which then disappears \cite{chomsky92}.

\subsection{Case in XTAG}

The English XTAG grammar adopts the notion of case and the case filter for
many of the same reasons argued in the GB literature.  However, the English
XTAG grammar implementation of case more closely resembles the treatment in
Chomsky's minimalism framework \cite{chomsky92} than the system outlined in the
GB literature \cite{chomsky86}.  As in minimalism, nouns in the XTAG approach
carry case with them, which is eventually 'checked' against the case values
assigned by the verb during the unification of the feature structures.  Unlike
Chomsky's minimalism, there is no separate AGR nodes; the case checking comes
from the verbs directly.

Most nouns in English do not have separate forms for nominative and accusative
case, and so they are ambiguous between the two.  Pronouns, of course, are
morphologically marked for case, and each carries the appropriate case in its
feature.  Figures \ref{nouns-with-case}a and \ref{nouns-with-case}b show the NP
tree anchored by a noun and a pronoun, respectively, along with the feature
values associated with each word.

\begin{figure*}[ht]
\centering
\rule[.1in]{3.5in}{0.01in} \\
\begin{tabular}{cc}
{\psfig{figure=ps/case-files/alphaNXN_books.ps,height=3.0in}}  &
{\psfig{figure=ps/case-files/alphaNXN_she.ps,height=3.2in}} \\
(a)&(b)\\
\end{tabular}\\
\caption {Lexicalized NP trees with case markings}
\rule[.1in]{3.5in}{0.01in}
\label {nouns-with-case}
\end{figure*}

\subsection{Case Assigners}

\subsubsection{Prepositions}
\label{prep-case}

Case is assigned in the XTAG English grammar by two components - verbs and
prepositions\footnote{{\it For} also assigns case as a complementizer.  See
section \ref{for-complementizer} for more details.}.  Prepositions assign
accusative case ({\bf acc})through their {\bf assign-case} feature, which is
linked directly to the {\bf case} feature of their objects.  Figure
\ref{PXPnx-with-case}a shows a lexicalized preposition tree, while
\ref{PXPnx-with-case}b shows the same tree with the NP tree from
\ref{nouns-with-case}a substituted into the NP position.  Figure
\ref{PXPnx-with-case}c is the tree \ref{PXPnx-with-case}b after unification has
taken place.  Note that the case ambiguity of {\it books} has been resolved to
accusative case.

\begin{figure*}[ht]
\centering
\rule[.1in]{6.0in}{0.01in}
\begin{tabular}{ccc}
{\psfig{figure=ps/case-files/alphaPXPnx_of.ps,height=1.7in}}  &
{\psfig{figure=ps/case-files/NXN-substituted-into-PXPnx.ps,height=3.5in}}  &
{\psfig{figure=ps/case-files/NXN-substituted-into-PXPnx-unified.ps,height=2.8in}} \\
(a)&(b)&(c)\\
\end{tabular}\\
\caption {Assigning case in prepositional phrases}
\rule[.1in]{6.0in}{0.01in}
\label {PXPnx-with-case}
\end{figure*}

\subsubsection{Verbs}
\label{case-for-verbs}
Verbs are the other part of speech in XTAG that can assign case.  Because
XTAG does not distinguish INFL and VP nodes\footnote{See section
\ref{VP-INFL-collapse} for an explanation of how this was done.}, verbs must
provide case assignment on the subject position in addition to the
case assigned to their NP complements.

Assigning case to NP complements is handled by building the case values of the
complements directly into the tree that the case assigner (the verb) anchors.
Figures \ref{S-tree-with-case}a and \ref{S-tree-with-case}b show an S
tree\footnote{Features not pertaining to this discussion have been taken out to
improve readability and to make the trees easier to fit onto the page.} that
would be anchored\footnote{The diamond marker ($\diamond$) indicates the
anchor(s) of a structure if the tree has not yet been lexicalized.} by a
transitive and ditransitive verb, respectively.  Note that the case assignments
for the NP complements are already in the tree, even though there is not yet a
lexical item anchoring the tree.  Since every verb that selects these trees
(and other trees in each respective subcategorization frame) assigns the same
case to the complements, building case features into the tree has exactly the
same result as putting the case feature value in each verb's lexical entry.

\begin{figure*}[ht]
\centering
\rule[.1in]{5.0in}{0.01in}
\begin{tabular}{cc}
{\psfig{figure=ps/case-files/alphanx0Vnx1-case-features.ps,height=2.0in}}  &
{\psfig{figure=ps/case-files/alphanx0Vnx1nx2-case-features.ps,height=2.0in}} \\
(a)&(b)\\
\end{tabular}\\
\caption {Case assignment to NP complements}
\rule[.1in]{5.0in}{0.01in}
\label {S-tree-with-case}
\end{figure*}

The case assigned to the subject position varies with verb form.  Since the
XTAG grammar treats the inflected verb as a single unit rather than dividing
it into INFL and V nodes, case, along with tense and agreement, is expressed in
the features of verbs, and must be passed in the appropriate manner.  The trees
in Figure \ref {S-tree-with-case} show the path of linkages that joins the {\bf
assign-case} feature of the V to the {\bf case} feature of the subject NP.  The
morphological form of the verb determines the value of the {\bf assign-case}
feature.  Figures \ref{lexicalized-S-tree-with-case}a and
\ref{lexicalized-S-tree-with-case}b show the same tree anchored by different
morphological forms of the verb {\it sing}, which give different values for the
assign-case feature\footnote{Again, the feature structures shown have been
restricted to those that pertain to the V/NP interaction.}.

\begin{figure*}[ht]
\centering
\rule[.1in]{5.0in}{0.01in}
\begin{tabular}{cc}
{\psfig{figure=ps/case-files/alphanx0Vnx1_sings-case-features.ps,height=3.2in}}  &
{\psfig{figure=ps/case-files/alphanx0Vnx1_singing-case-features.ps,height=2.9in}} \\
(a)&(b)\\
\end{tabular}\\
\caption {Assigning case according to verb form}
\rule[.1in]{5.0in}{0.01in}
\label {lexicalized-S-tree-with-case}
\end{figure*}

The adjunction of an auxiliary verb onto the VP node breaks the {\bf
assign-case} link from the main V and substitutes a link from the auxiliary
verb instead\footnote{see section \ref{aux-non-inverted} for a more complete
explanation of how this relinking occurs.}. The progressive form of the verb in
Figure \ref{lexicalized-S-tree-with-case}b assigns case {\bf none}, but this is
overridden by the adjunction of the appropriate form of the auxiliary word {\it
be}.  Figure \ref{Vvx-with-case}a shows the lexicalized auxiliary tree, while
\ref{Vvx-with-case}b shows it adjoined into the transitive tree shown in Figure
\ref{lexicalized-S-tree-with-case}b.  The case value passed to the NP is now
{\bf nom} (nominative).

\begin{figure*}[ht]
\centering
\rule[.1in]{5.0in}{0.01in}
\begin{tabular}{cc}
{\psfig{figure=ps/case-files/betaVvx_is-with-case.ps,height=2.1in}}  &
{\psfig{figure=ps/case-files/betaVvx_is-adjoined-into-nx0Vnx1_singing.ps,height=3.5in}} \\
(a)&(b)\\
\end{tabular}\\
\caption {Proper case assignment with auxiliary verbs}
\rule[.1in]{5.0in}{0.01in}
\label {Vvx-with-case}
\end{figure*}


\subsection{PRO in a unification based framework}

Most forms of a verb assign nominative case, although some forms, such as past
participle, assign no case whatsoever.  This is different than assigning case
{\bf none}, as the progressive form of the verb {\it sing} does in Figure
\ref{lexicalized-S-tree-with-case}b.  The distinction of a case {\bf none} from
no case is indicative of a divergence from the standard GB theory.  In GB
theory, the absence of case on an NP means that only PRO can fill that NP.  In
XTAG, the absence of case on an NP means that *any* NP can fill it,
regardless of its case.  This is due to the mechanism of unification, in which
if something is unspecified, it can unify with anything.  Thus we have a
specific case {\bf none} to handle verb forms that in GB theory do not assign
case.  PRO is the only NP with case {\bf none}.  Verbs forms that assign no
case, as the past participle mentioned above, can do so because they cannot
occur without an auxiliary verb which takes care of the case assignment.  Note
that although we are drawn to this treatment by our use of unification for
feature manipulation, \cite{watanabe93} proposes a very similar approach within
Chomsky's minimalist framework for entirely different reasons.
  This is now included from the compl-adj file.
\part{Verb Classes}
\chapter{Where to Find What}
\label{table-intro}

The two page table that follows gives an overview of what types of
trees occur in various tree families with pointers to discussion in
this report.  An entry in a cell of the table indicates that the
tree(s) for the construction named in the row header are included in
the tree family named in the column header. Entries are of two types.
If the particular tree(s) are displayed and/or discussed in this
report the entry gives a page number reference to the relevant
discussion or figure.\footnote{Since Chapter~\ref{verb-classes} has a
  brief discussion and a declarative tree for every tree family, page
  references are given only for other sections in which discussion or
  tree diagrams appear.}  Otherwise, a \xtagcheck \space indicates
inclusion in the tree family but no figure or discussion related
specifically to that tree in this report.  Blank cells indicate that
there are no trees for the construction named in the row header in the
tree family named in the column header.  Two tables are given below.
The first one gives the expansion of abbreviations in the table
headers. The second table gives the name given to each tree family in
the actual XTAG grammar. This makes it easier to find the description
of each tree family in Chapter~\ref{verb-classes} and to compare the
description with the online XTAG grammar.

\vspace{0.3in}

\small
\begin{tabular}{ll}
Abbreviation&Full Name\\
\hline
Sent. Subj. w. {\it to} & Sentential Subject with {\it to} PP complement \\
Pred. Mult-wd. ARB, P & Predicative Multi-word PP with Adv, Prep anchors\\
Pred. Mult-wd. A, P & Predicative Multi-word PP with Adj, Prep anchors\\
Pred. Mult-wd. N, P & Predicative Multi-word PP with Noun, Prep
anchors\\
Pred. Mult-wd. P, P & Predicative Multi-word PP with two Prep
anchors\\
Pred. Mult-wd. no int. mod. & Predicative Multi-word PP with no internal
modification\\
Pred. Sent. Subj., ARB, P & Predicative PP with Sentential Subject, and
Adv, Prep anchors\\
Pred. Sent. Subj., A, P & Predicative PP with Sentential Subject, and
Adj, Prep anchors\\
Pred. Sent. Subj., Conj, P & Predicative PP with Sentential Subject, and
Conj, Prep anchors\\
Pred. Sent. Subj., N, P & Predicative PP with Sentential Subject, and
Noun, Prep anchors\\
Pred. Sent. Subj., P, P & Predicative PP with Sentential Subject, and two
Prep anchors\\
Pred. Sent. Subj., no int-mod & Predicative PP with Sentential Subject,
no internal modification\\
Pred. Locative & Predicative anchored by a Locative Adverb\\
Pred. A Sent. Subj., Comp. & Predicative Adjective with Sentential
Subject and Complement\\
Sentential Comp. with NP&Sentential Complement with NP\\
Pred. Mult wd. V, P & Predicative Multi-word with Verb, Prep anchors \\
Adj. Sm. Cl. w. Sentential Subj.&Adjective Small Clause with Sentential Subject\\
NP Sm. Clause w. Sentential Subj.&NP Small Clause with Sentential Subject\\
PP Sm. Clause w. Sentential Subj.&PP Small Clause with Sentential Subject\\
NP Sm. Cl. w. Sent. Comp.&NP Small Clause with Sentential Complement\\
Adj. Sm. Cl. w. Sent. Comp.&Adjective Small Clause with Sentential
Complement\\
Exhaustive PP Sm. Cl.&Exhaustive PP Small Clause\\
Ditrans. Light Verbs w. PP Shift&Ditransitive Light Verbs with PP Shift\\
Ditrans. Light Verbs w/o PP Shift&Ditransitive Light Verbs without PP Shift\\
Y/N question&Yes/No question \\
Wh-mov. NP complement&Wh-moved NP complement \\
Wh-mov. S comp.&Wh-moved S complement \\
Wh-mov. Adj comp.&Wh-moved Adjective complement \\
Wh-mov. object of a mod.&Wh-moved object of a modifier \\
Wh-mov. PP&Wh-moved PP \\
Topic. NP complement&Topicalized NP complement \\
Det. gerund&Determiner gerund \\
Rel. cl. on NP comp.&Relative clause on NP complement \\
Rel. cl. on PP comp.& Relative clause on PP complement\\
Rel. cl. on NP object of P& Relative clause on NP object of P\\
Pass. with wh-moved subj.&Passive with wh-moved subject (with and without {\it by} phrase) \\
Pass. w. wh-mov. ind. obj.&Passive with wh-moved indirect object (with and without {\it by} phrase) \\
Pass. w. wh-mov. obj. of the {\it {\it by} phrase}&Passive with wh-moved object of the {\it by} phrase \\
Pass. w. wh-mov. {\it by} phrase&Passive with wh-moved {\it by} phrase \\
Trans. Idiom with V, D and N & Transitive Idiom with Verb, Det and
Noun anchors\\
Idiom with V, D, N & Idiom with V, D, and N anchors \\
Idiom with V, D, A, N & Idiom with V, D, A, and N anchors \\
Idiom with V, N & Idiom with V, and N anchor \\
Idiom with V, A, N & Idiom with V, A, and N anchors \\
Idiom with V, D, N, P & Idiom with V, D, N, and Prep anchors \\
Idiom with V, D, A, N, P & Idiom with V, D, A, N, and Prep anchors \\
Idiom with V, N, P & Idiom with V, N, and Prep anchors \\
Idiom with V, A, N, P & Idiom with V, A, N, and Prep anchors 
\end{tabular}
\normalsize

\small
\begin{tabular}{ll}
Full Name&XTAG Name\\
\hline
Intransitive Sentential Subject &  Ts0V\\
Sentential Subject with `to' complement &  Ts0Vtonx1\\
PP Small Clause, with Adv and Prep anchors & Tnx0ARBPnx1\\
PP Small Clause, with Adj and Prep anchors & Tnx0APnx1\\
PP Small Clause, with Noun and Prep anchors & Tnx0NPnx1\\
PP Small Clause, with Prep anchors & Tnx0PPnx1\\
PP Small Clause, with Prep and Noun anchors & Tnx0PNaPnx1\\
PP Small Clause with Sentential Subject, and Adv and Prep anchors & Ts0ARBPnx1\\
PP Small Clause with Sentential Subject, and Adj and Prep anchors & Ts0APnx1\\
PP Small Clause with Sentential Subject, and Noun and Prep anchors & Ts0NPnx1\\
PP Small Clause with Sentential Subject, and Prep anchors & Ts0PPnx1\\
PP Small Clause with Sentential Subject, and Prep and Noun anchors & Ts0PNaPnx1\\
Exceptional Case Marking & TXnx0Vs1\\
Locative Small Clause with Ad anchor & Tnx0nx1ARB\\
Predicative Adjective with Sentential Subject and Complement & Ts0A1s1\\
Transitive & Tnx0Vnx1\\
Ditransitive with PP shift & Tnx0Vnx1Pnx2\\
Ditransitive & Tnx0Vnx2nx1\\
Ditransitive with PP & Tnx0Vnx1pnx2\\
Sentential Complement with NP & Tnx0Vnx1s2\\
Intransitive Verb Particle & Tnx0Vpl\\
Transitive Verb Particle & Tnx0Vplnx1\\
Ditransitive Verb Particle & Tnx0Vplnx1nx2\\
Intransitive with PP & Tnx0Vpnx1\\
Sentential Complement & Tnx0Vs1\\
Light Verbs & Tnx0lVN1\\
Ditransitive Light Verbs with PP Shift & Tnx0lVN1Pnx2\\
Adjective Small Clause with Sentential Subject & Ts0Ax1\\
NP Small Clause with Sentential Subject &  Ts0N1\\
PP Small Clause with Sentential Subject & Ts0Pnx1\\
Predicative Multi-word with Verb, Prep anchors & Tnx0VPnx1\\
Adverb It-Cleft & TItVad1s2\\
NP It-Cleft & TItVnx1s2\\
PP It-Cleft & TItVpnx1s2\\
Adjective Small Clause Tree & Tnx0Ax1\\
Adjective Small Clause with Sentential Complement & Tnx0A1s1\\
Equative {\it BE} & Tnx0BEnx1\\
NP Small Clause & Tnx0N1\\
NP with Sentential Complement Small Clause & Tnx0N1s1\\
PP Small Clause & Tnx0Pnx1\\
Exhaustive PP Small Clause & Tnx0Px1\\
Intransitive & Tnx0V\\
Intransitive with Adjective & Tnx0Vax1\\
Transitive Sentential Subject &  Ts0Vnx1\\
Idiom with V, D and N & Tnx0VDN1\\
Idiom with V, D, A, and N anchors & Tnx0VDAN1\\
Idiom with V and N anchors & Tnx0VN1\\
Idiom with V, A, and N anchors & Tnx0VAN1\\
Idiom with V, D, N, and Prep anchors & Tnx0VDN1Pnx2\\
Idiom with V, D, A, N, and Prep anchors & Tnx0VDAN1Pnx2\\
Idiom with V, N, and Prep anchors & Tnx0VN1Pnx2\\
Idiom with V, A, N, and Prep anchors & Tnx0VAN1Pnx2
\end{tabular}
\normalsize

\clearpage
















%%Labels/pagerefs are formatted {table-no;row,column}
%%Rajesh: Add ref to \label{2;Npxnx0Vnx1}--need new rows in tables

%%% pagerefs take the following table to be `table 3'
%%% 

\begin{center}
\hspace*{-0.75in}  %%% the table is too wide and center doesn't work
\begin{tabular}{|p{2.4in}||*{15}{c|}}
\cline{2-16}
\multicolumn{1}{c||}{} & \multicolumn{15}{c|}{Tree families}\\
\hline
\vspace*{10em}
& & & & & & & & & & & & & & & \\
 &
\vertical{Intransitive Sentential Subj } &
\vertical{Sent. Subj. w. to } & %s0Vtonx1 
\vertical{Pred. Mult-wd. ARB, P } &
\vertical{Pred. Mult-wd. A, P } &
\vertical{Pred. Mult-wd. N, P } &
\vertical{Pred. Mult-wd. P, P } &
\vertical{Pred. Mult-wd. no int. mod. } &
\vertical{Pred. Sent. Subj., ARB, P } &
\vertical{Pred. Sent. Subj., A, P } &
\vertical{Pred. Sent. Subj., N, P } &
\vertical{Pred. Sent. Subj., P, P } &
\vertical{Pred. Sent. Subj., no int-mod } &
\vertical{ECM}  & %TXnx0Vs1 
\vertical{Pred. Locative} &  %Tnx0nx1ARB
\vertical{Pred. A Sent. Subj., Comp.} \\
%
%%% columns:
%
\hline\hline
\vspace*{-2.3em} \centerline{Constructions} \vspace*{0.5em}
Declarative & \xtagcheck & \xtagcheck &\xtagcheck &\xtagcheck
&\xtagcheck & \xtagcheck& \xtagcheck& \xtagcheck& \xtagcheck&
\xtagcheck &\xtagcheck &\xtagcheck & {\tiny \pageref{3;1,15}}  & {\tiny \pageref{3;nx0nx1ARB}} &\\
\hline
Passive w/ \& w/o {\it by} phrase & & & & & & & & &  & & & & {\tiny \pageref{3;2,15}} & &\\
\hline
Y/N quest. & & &  &  &  & &  & & & & & & & & \\
\hline
Wh-moved subject & \xtagcheck &  \xtagcheck & \xtagcheck & \xtagcheck
& \xtagcheck &  \xtagcheck & \xtagcheck& \xtagcheck& \xtagcheck &
\xtagcheck & \xtagcheck & \xtagcheck  & \xtagcheck & \xtagcheck &
\xtagcheck \\
\hline
Wh-mov.\ NP complement, DO or IO & & & & & & & & & & & & & & & \\
\hline
Wh-mov.\ S comp. & & & & & & & & & & & & & & & \\
\hline
Wh-mov.\ Adj. or Adv.\ comp. & & & & & & & & & & & & & & {\tiny \pageref{3;W1nx0nx1ARB}} & \\
        \hline
Wh-mov.\ object of a P & & & \xtagcheck & \xtagcheck & \xtagcheck & \xtagcheck & \xtagcheck & & & & & & & & \\
\hline
Wh-mov.\ PP & & & \xtagcheck & &  & \xtagcheck & \xtagcheck & & & & & & & & \\
\hline
Topic.\ NP comp.& & & & & & & & & & & & & & & \\
\hline
Imperative & & & \xtagcheck & & & \xtagcheck & \xtagcheck  & & & & & &
\xtagcheck  & \xtagcheck  & \\
\hline
Det.\ gerund & & & & & & & & & & & & & & & \\
\hline
NP gerund & & & \xtagcheck & & & & \xtagcheck  & & & & & & \xtagcheck & \xtagcheck &\\
\hline
Ergative & & & & & & & & & & & & & & & \\
\hline
Rel.\ cl.\ on subj. w/ NP  & & & \xtagcheck & \xtagcheck & \xtagcheck & \xtagcheck & \xtagcheck & & & & & & \xtagcheck & \xtagcheck &\\
\hline
Rel.\ cl.\ on subj. w/ Comp  & & & \xtagcheck & \xtagcheck & \xtagcheck & \xtagcheck & \xtagcheck & & & & & & \xtagcheck & \xtagcheck &\\
\hline
Rel.\ cl.\ on NP comp., DO, IO w/ NP & & & & & & & & & & & & & & & \\
\hline
Rel.\ cl.\ on NP comp., DO, IO w/ Comp & & & & & & & & & & & & & & & \\
\hline
Rel.\ cl.\ on PP comp. w/ pied-piping  & & & & & & \xtagcheck & \xtagcheck &  &  & & & & & & \\
\hline
Rel.\ cl.\ on NP object of P w/ NP & & & \xtagcheck & \xtagcheck &  \xtagcheck & \xtagcheck  & \xtagcheck & \xtagcheck & \xtagcheck & \xtagcheck & \xtagcheck & \xtagcheck & & &\\
\hline
Rel.\ cl.\ on NP object of P w/ Comp & & & \xtagcheck & \xtagcheck & & \xtagcheck  & \xtagcheck & \xtagcheck & \xtagcheck & \xtagcheck & \xtagcheck & \xtagcheck & & &\\
\hline
Rel.\ cl.\ on adjunct w/ PP & \xtagcheck & \xtagcheck & \xtagcheck &
\xtagcheck &  \xtagcheck & \xtagcheck  & \xtagcheck & \xtagcheck &
\xtagcheck & \xtagcheck & \xtagcheck & \xtagcheck & \xtagcheck & &
\xtagcheck \\
\hline
Rel.\ cl.\ on adjunct w/ Comp & \xtagcheck & \xtagcheck & \xtagcheck &
\xtagcheck &  \xtagcheck & \xtagcheck  & \xtagcheck & \xtagcheck &
\xtagcheck & \xtagcheck & \xtagcheck & \xtagcheck & \xtagcheck & &
\xtagcheck \\
\hline
Pass.\ w.\ wh-mov.\ subj.\ & & & & & & & & & & & & & & & \\
\hline
Pass.\ w.\ wh-mov.\ ind.\ obj.\ & & & & & & & & & & & & & & & \\
\hline
Pass.\ w.\ wh-mov.\ obj. of  {\it by} phrase  & & & & & & & & & & & & & & & \\
\hline
Pass.\ w.\ wh-mov.\ {\it by} phrase  & & & & & & & & & & & & & & & \\
\hline
\end{tabular}
\end{center}

\clearpage

%%% from here on into the document
%%% pagerefs take the following table to be `table 2'
%%% 

\vspace*{-0.5in}

\begin{center}
\hspace*{-0.75in}  %%% the table is too wide and center doesn't work
\begin{tabular}{|p{2.4in}||*{16}{c|}}
\cline{2-17}
\multicolumn{1}{c||}{} & \multicolumn{16}{c|}{Tree families}\\
\hline
\vspace*{12em} & & & & & & & & & & & & & & & & \\
 &
\vertical{Transitive} &
\vertical{Ditransitive with PP shift} &
\vertical{Ditransitive} &
\vertical{Ditransitive with PP} &
\vertical{Sentential Comp.\ with NP} &
\vertical{Intransitive Verb Particle} &
\vertical{Transitive Verb Particle} &
\vertical{Ditransitive Verb Particle} &
\vertical{Intransitive with PP} &
\vertical{Sentential Complement} &
\vertical{Trans. Light Vs (w \& w/o Dets)} &
\vertical{Ditrans.\ Light Vs (w \& w/o Dets)} &
\vertical{Adj.\ Sm.\ Cl.\ w.\ Sentential Subj.} &
\vertical{NP Sm.\ Cl.\ w.\ Sentential Subj.} &
\vertical{PP Sm.\ Cl.\ w.\ Sentential Subj.} &
\vertical{Pred.\ Mult.\ wd. V, P} \\

%
%%% columns:
%
\hline\hline
\vspace*{-2.3em} \centerline{Constructions} \vspace*{0.5em}
Declarative &{\tiny \pageref{2;1,1}} & {\tiny \pageref{2;1,2}} & {\tiny \pageref{2;1,3}}& \xtagcheck & \xtagcheck & \xtagcheck & \xtagcheck & \xtagcheck &{\tiny \pageref{2;1,9}}&{\tiny \pageref{2;Tnx0Vs1},\pageref{2;1,10}} & \xtagcheck & \xtagcheck & \xtagcheck & \xtagcheck & \xtagcheck &\xtagcheck \\
\hline
Passive w/ \& w/o {\it by} phrase &\xtagcheck & \xtagcheck & \xtagcheck & \xtagcheck & {\tiny \pageref{2;2,5}} & & \xtagcheck & \xtagcheck & & & & \xtagcheck & & & &\xtagcheck\\
\hline
Y/N quest.\ & & & & & & & & & & & & & & & &\\
\hline
Wh-moved subject & \xtagcheck& \xtagcheck& \xtagcheck& \xtagcheck& \xtagcheck&\xtagcheck &\xtagcheck &\xtagcheck &\xtagcheck &\xtagcheck  &\xtagcheck & &\xtagcheck & \xtagcheck& \xtagcheck &\xtagcheck \\
\hline
Wh-mov.\ NP complement, DO or IO  &{\tiny \pageref{2;5,1}}&\xtagcheck &{\tiny \pageref{2;5,3}}&\xtagcheck &\xtagcheck & &\xtagcheck &\xtagcheck & & & & & & & & \\
\hline
Wh-mov.\ S comp.\ & & & & & \xtagcheck & & & & & \xtagcheck & & & & & & \\
\hline
Wh-mov.\ Adj. or Adv.\ comp.  & & & & & & & & & & & & & \xtagcheck & & & \\
\hline
Wh-mov.\ object of a P  & &\xtagcheck & &{\tiny \pageref{2;8,4}}& & & & &\xtagcheck & & &\xtagcheck & & & &\xtagcheck  \\
\hline
Wh-mov.\ PP  & &\xtagcheck & &{\tiny \pageref{2;9,4}}& & & & &\xtagcheck & & &\xtagcheck & & & & \\
\hline
Topic.\ NP comp.  &\xtagcheck &\xtagcheck &\xtagcheck &\xtagcheck &\xtagcheck & &\xtagcheck &\xtagcheck & & & & & & & & \\
\hline
Imperative &{\tiny \pageref{2;11,1}}&\xtagcheck &\xtagcheck & \xtagcheck&\xtagcheck &\xtagcheck &\xtagcheck & \xtagcheck&\xtagcheck &\xtagcheck &\xtagcheck &\xtagcheck & & &  &\xtagcheck \\
\hline
Det.\ gerund &{\tiny \pageref{2;12,1}}&\xtagcheck &\xtagcheck &\xtagcheck &\xtagcheck &\xtagcheck &\xtagcheck &\xtagcheck &\xtagcheck &\xtagcheck &\xtagcheck &\xtagcheck & & & &\xtagcheck  \\
\hline
NP gerund &{\tiny \pageref{2;13,1}}&\xtagcheck &\xtagcheck &\xtagcheck &\xtagcheck & \xtagcheck& \xtagcheck& \xtagcheck& \xtagcheck& \xtagcheck &\xtagcheck &\xtagcheck & & &  &\xtagcheck \\
\hline
Ergative &{\tiny \pageref{2;14,1}}& & & & & & & & & & & & & & & \\
\hline
Rel.\ cl.\ on subj. w/ NP  & \xtagcheck & \xtagcheck & \xtagcheck & \xtagcheck & \xtagcheck & \xtagcheck & \xtagcheck & \xtagcheck & \xtagcheck & \xtagcheck & \xtagcheck & \xtagcheck & & & &\xtagcheck \\
\hline
Rel.\ cl.\ on subj. w/ Comp  & \xtagcheck & \xtagcheck & \xtagcheck & \xtagcheck & \xtagcheck & \xtagcheck & \xtagcheck & \xtagcheck & \xtagcheck & \xtagcheck & \xtagcheck & \xtagcheck &  & & &\xtagcheck \\
\hline
Rel.\ cl.\ on NP comp., DO, IO w/ NP & \xtagcheck & \xtagcheck & \xtagcheck & \xtagcheck & \xtagcheck & & \xtagcheck & \xtagcheck & \xtagcheck & & & \xtagcheck & & &  & \\
\hline
Rel.\ cl.\ on NP comp., DO, IO w/ Comp & \xtagcheck & \xtagcheck & \xtagcheck & \xtagcheck & \xtagcheck & & \xtagcheck & \xtagcheck & \xtagcheck & & & \xtagcheck & & &  & \\
\hline
Rel.\ cl.\ on PP comp. w/ pied-piping  & \xtagcheck & \xtagcheck & \xtagcheck & \xtagcheck & \xtagcheck & & \xtagcheck & & \xtagcheck & & & \xtagcheck & & & \xtagcheck &\\
\hline
Rel.\ cl.\ on NP object of P w/ NP & \xtagcheck & \xtagcheck & \xtagcheck & \xtagcheck &  \xtagcheck & & \xtagcheck & & \xtagcheck & & & \xtagcheck & & & \xtagcheck  &\xtagcheck \\
\hline
Rel.\ cl.\ on NP object of P w/ Comp & \xtagcheck & \xtagcheck & \xtagcheck & \xtagcheck &  \xtagcheck & & \xtagcheck & & \xtagcheck & & & \xtagcheck & & &  \xtagcheck  &\xtagcheck \\
\hline
Rel.\ cl.\ on adjunct w/ PP &  \xtagcheck & \xtagcheck & \xtagcheck & \xtagcheck &  \xtagcheck &\xtagcheck  & \xtagcheck & \xtagcheck & \xtagcheck & \xtagcheck & \xtagcheck & \xtagcheck & \xtagcheck & \xtagcheck & \xtagcheck  &\xtagcheck  \\
\hline
Rel.\ cl.\ on adjunct w/ Comp &  \xtagcheck & \xtagcheck & \xtagcheck & \xtagcheck &  \xtagcheck &\xtagcheck  & \xtagcheck & \xtagcheck & \xtagcheck & \xtagcheck & \xtagcheck & \xtagcheck & \xtagcheck & \xtagcheck &  \xtagcheck  &\xtagcheck \\
\hline
Parenthetical quoting clause &   &   &   &   & \xtagcheck & &   &   &
& \xtagcheck & &   & & & & \\
\hline %alpha_AV
Past-participal as arg Adj & \xtagcheck   &   &   &   &  & &   &   & & & &   & & &  &\xtagcheck  \\
\hline %betaVtransn
Past-participial NP pre-mod  & \xtagcheck   &   &   &   & & &   &   & & & &   & & &  &\xtagcheck  \\
\hline
Pass.\ w.\ wh-mov.\ subj. &\xtagcheck &\xtagcheck &\xtagcheck &\xtagcheck &\xtagcheck & & \xtagcheck&\xtagcheck & & & &\xtagcheck & & & &\xtagcheck  \\
\hline
Pass.\ w.\ wh-mov.\ ind.\ obj. & & \xtagcheck& \xtagcheck& \xtagcheck&
\xtagcheck& & &\xtagcheck & & & & \xtagcheck& & & & \\
\hline
Pass.\ w.\ wh-mov.\ obj. of {\it by} phrase & \xtagcheck & \xtagcheck & \xtagcheck & \xtagcheck & \xtagcheck & & \xtagcheck & \xtagcheck & & & & \xtagcheck & & & &\xtagcheck  \\
\hline
Pass.\ w.\ wh-mov.\ {\it by} phrase &\xtagcheck &\xtagcheck & \xtagcheck&\xtagcheck & \xtagcheck& &\xtagcheck &\xtagcheck & & & & & & &  &\xtagcheck \\
\hline
\end{tabular}
\end{center}

\clearpage

%%% pagerefs take the following table to be `table 1'
%%% 

\begin{center}
\hspace*{-0.75in}  %%% the table is too wide and center doesn't work
\begin{tabular}{|p{2.4in}||*{15}{c|}}
\cline{2-16}
\multicolumn{1}{c||}{} & \multicolumn{15}{c|}{Tree families}\\
\hline
\vspace*{10em}
& & & & & & & & & & & & & & & \\
 &
\vertical{Adverb It-Cleft } &
\vertical{NP It-Cleft } &
\vertical{PP It-Cleft } &
\vertical{Adj. Small Clause } &
\vertical{Adj.\ Sm.\ Cl.\ w.\ Sent.\ Comp.} &
\vertical{Equative {\it BE} } &
\vertical{NP Small Clause } &
\vertical{NP Sm.\ Cl.\ w.\ Sent.\ Comp.} &
\vertical{PP Small Clause} &
\vertical{Pred.\ N w.\ Sent.\ Subj.} &
\vertical{Exhaustive PP Sm. Cl. } &
\vertical{Intransitive} &
\vertical{Intransitive with Adjective} &
\vertical{Transitive Sentential Subj} \\
%
%%% columns:
%
\hline\hline
\vspace*{-2.3em} \centerline{Constructions} \vspace*{0.5em}
Declarative &\xtagcheck & \xtagcheck &{\tiny \pageref{1;1,3}}&{\tiny \pageref{1;1,4}}& \xtagcheck & {\tiny \pageref{1;1,6}} &{\tiny \pageref{1;1,7}}& \xtagcheck &{\tiny \pageref{1;1,9}}& \xtagcheck & \xtagcheck & \xtagcheck & \xtagcheck & \xtagcheck & {\tiny \pageref{1;1,16}} \\
\hline
Passive w/ \& w/o {\it by} phrase & & & & & & & & & & & & & & & \\
\hline
Y/N quest. & \xtagcheck & \xtagcheck & {\tiny \pageref{1;3,3}} & & & \xtagcheck & & & & & & & & & \\
\hline
Wh-moved subject & & & &{\tiny \pageref{1;4,4}} & \xtagcheck& &\xtagcheck &\xtagcheck &\xtagcheck &\xtagcheck &\xtagcheck &{\tiny \pageref{1;4,13}}& {\tiny \pageref{1;4,14}} &\xtagcheck & \xtagcheck \\
\hline
Wh-mov.\ NP complement, DO or IO & &\xtagcheck & & & & &\xtagcheck & &
& & & & & & \xtagcheck \\
\hline
Wh-mov.\ S comp. & & & & & & & & & & & & & &  & \\
\hline
Wh-mov.\ Adj. or Adv.\ comp. &\xtagcheck & & &\xtagcheck & & & & & &
& &  & {\tiny \pageref{1;7,14}} & & \\
\hline
Wh-mov.\ object of a P & & & & & & & & & \xtagcheck & & & & & & \\
\hline
Wh-mov.\ PP & & &\xtagcheck & & & & & &\xtagcheck & & & & & & \\
\hline
Topic.\ NP comp. & &\xtagcheck & & & & & \xtagcheck& & & & & & & & \\
\hline
Imperative & & & &\xtagcheck &\xtagcheck & &\xtagcheck &\xtagcheck &\xtagcheck & &\xtagcheck &\xtagcheck &\xtagcheck & \xtagcheck & \\
\hline
Det.\ gerund & & & & & & & & & & & & & & & \\
\hline
NP gerund & & & &\xtagcheck &\xtagcheck & &\xtagcheck &\xtagcheck &\xtagcheck & &\xtagcheck &\xtagcheck &\xtagcheck &  &  \\
\hline
Ergative & & & & & & & & & & & & & & & \\
\hline
Rel.\ cl.\ on subj. w/ NP  & & & & \xtagcheck & \xtagcheck &  & \xtagcheck & \xtagcheck & \xtagcheck &  & \xtagcheck & \xtagcheck & \xtagcheck &  \xtagcheck & \\
\hline
Rel.\ cl.\ on subj. w/ Comp  & & & & \xtagcheck & \xtagcheck & & \xtagcheck & \xtagcheck & \xtagcheck & & \xtagcheck & \xtagcheck & \xtagcheck & \xtagcheck & \\
\hline
Rel.\ cl.\ on NP comp., DO, IO w/ NP & & & & & & & & & &\xtagcheck & & & & & \\
\hline
Rel.\ cl.\ on NP comp., DO, IO w/ Comp & & & & & & & & & & & & & & & \\
\hline
Rel.\ cl.\ on PP comp. w/ pied-piping  & & & & & & & & & \xtagcheck & & & & & & \\
\hline
Rel.\ cl.\ on NP object of P w/ NP & & & & & &  & & & \xtagcheck & & & & & &\\
\hline
Rel.\ cl.\ on NP object of P w/ Comp & & & & & &  & & & \xtagcheck & & & & & &\\
\hline
Rel.\ cl.\ on adjunct w/ PP & \xtagcheck & \xtagcheck  & \xtagcheck & \xtagcheck &  \xtagcheck &  & \xtagcheck & \xtagcheck & \xtagcheck & \xtagcheck &  \xtagcheck & \xtagcheck  & \xtagcheck & \xtagcheck &  \xtagcheck \\
\hline
Rel.\ cl.\ on adjunct w/ Comp & \xtagcheck & \xtagcheck  & \xtagcheck & \xtagcheck &  \xtagcheck &  & \xtagcheck & \xtagcheck & \xtagcheck & \xtagcheck &  \xtagcheck & \xtagcheck  & \xtagcheck & \xtagcheck &  \xtagcheck\\
\hline %beta_Vintransn
Participial  NP pre-mod  & & & & & & & & & & & & \xtagcheck & & & \\
\hline
Pass.\ w.\ wh-mov.\ subj.\ & & & & & & & & & & & & & & & \\
\hline
Pass.\ w.\ wh-mov.\ ind.\ obj.\ & & & & & & & & & & & & & & & \\
\hline
Pass.\ w.\ wh-mov.\ obj. of  {\it by} phrase  & & & & & & & & & & & & & & & \\
\hline
Pass.\ w.\ wh-mov.\ {\it by} phrase  & & & & & & & & & & & & & & & \\
\hline
\end{tabular}
\end{center}

\clearpage


\begin{center}
%\hspace*{-0.75in} % the table is too wide and center doesn't work
\begin{tabular}{|p{2.4in}||*{8}{c|}}
\cline{2-9}
\multicolumn{1}{c||}{} & \multicolumn{8}{c|}{Tree families}\\
\hline
\vspace*{10em}
& & & & & & & & \\
 &
\vertical{Transitive Idiom with D, N } &
\vertical{Transitive Idiom with D, A, N } &
\vertical{Transitive Idiom with N } &
\vertical{Transitive Idiom with A, N} &
\vertical{Transitive Idiom with D, N, P } &
\vertical{Transitive Idiom with D, A, N, P } &
\vertical{Transitive Idiom with N, P } &
\vertical{Transitive Idiom with A, N, P} \\

%columns:

\hline\hline
\vspace*{-2.3em} \centerline{Constructions} \vspace*{0.5em}
Declarative & \xtagcheck & \xtagcheck &\xtagcheck &\xtagcheck
&\xtagcheck & \xtagcheck& \xtagcheck& \xtagcheck \\
\hline
Passive w/ \& w/o {\it by} phrase &\xtagcheck &\xtagcheck &\xtagcheck &\xtagcheck &\xtagcheck &\xtagcheck &\xtagcheck &\xtagcheck \\
\hline
Y/N quest. & & & & & & & & \\
\hline
Wh-moved subject & \xtagcheck & \xtagcheck & \xtagcheck & \xtagcheck & \xtagcheck & \xtagcheck & \xtagcheck& \xtagcheck \\
\hline
Wh-mov.\ NP complement, DO or IO & & & & & & & & \\
\hline
Wh-mov.\ S comp. & & & & & & & & \\
\hline
Wh-mov.\ Adj. or Adv.\ comp. & & & & & & & & \\
\hline
Wh-mov.\ object of a P & & & & & & & & \\
\hline
Wh-mov.\ PP & & & & & & & & \\
\hline
Topic.\ NP comp. & & & & & & & & \\
\hline
Imperative &\xtagcheck &\xtagcheck &\xtagcheck &\xtagcheck &\xtagcheck &\xtagcheck &\xtagcheck &\xtagcheck \\
\hline
Det.\ gerund & & & & & & & & \\
\hline
NP gerund &\xtagcheck &\xtagcheck &\xtagcheck &\xtagcheck &\xtagcheck &\xtagcheck &\xtagcheck &\xtagcheck \\
\hline
Ergative & & & & & & & & \\
\hline
Rel.\ cl.\ on subj. w/ NP & \xtagcheck & \xtagcheck &\xtagcheck &\xtagcheck &\xtagcheck &\xtagcheck &\xtagcheck &\xtagcheck \\
\hline
Rel.\ cl.\ on subj. w/ Comp  &\xtagcheck &\xtagcheck &\xtagcheck &\xtagcheck &\xtagcheck &\xtagcheck &\xtagcheck &\xtagcheck \\
\hline
Rel.\ cl.\ on NP comp., DO, IO w/ NP & & & & & & & & \\
\hline
Rel.\ cl.\ on NP comp., DO, IO w/ Comp & & & & & & & & \\
\hline
Rel.\ cl.\ on PP comp. w/ pied-piping  & & & & & & & & \\
\hline
Rel.\ cl.\ on NP object of P w/ NP & & & & & & & & \\
\hline
Rel.\ cl.\ on NP object of P w/ Comp & & & & & & & & \\
\hline
Rel.\ cl.\ on adjunct w/ PP & \xtagcheck & \xtagcheck & \xtagcheck & \xtagcheck &  \xtagcheck & \xtagcheck  & \xtagcheck & \xtagcheck \\
\hline
Rel.\ cl.\ on adjunct w/ Comp & \xtagcheck & \xtagcheck & \xtagcheck & \xtagcheck &  \xtagcheck & \xtagcheck  & \xtagcheck & \xtagcheck \\
\hline
Pass.\ w.\ wh-mov.\ subj.\ & & & & & & & & \\
\hline
Pass.\ w.\ wh-mov.\ ind.\ obj.\ & & & & & & & & \\
\hline
Pass.\ w.\ wh-mov.\ obj. of  {\it by} phrase & \xtagcheck & \xtagcheck &\xtagcheck &\xtagcheck &\xtagcheck &\xtagcheck &\xtagcheck &\xtagcheck \\
\hline
Pass.\ w.\ wh-mov.\ {\it by} phrase & \xtagcheck & \xtagcheck &\xtagcheck &\xtagcheck &\xtagcheck &\xtagcheck &\xtagcheck &\xtagcheck \\
\hline
Outer Pass.\ w.\ and wo.\ {\it by} phrase & & & & & \xtagcheck & \xtagcheck & \xtagcheck & \xtagcheck \\
\hline
Outer Pass.\ w.\ Rel.\ cl.\ on subj.\ w.\ Comp  & & & & & \xtagcheck & \xtagcheck & \xtagcheck & \xtagcheck \\
\hline
Outer Pass.\ w.\ Rel.\ cl.\ on subj.\ w.\ NP  & & & & & \xtagcheck & \xtagcheck & \xtagcheck & \xtagcheck \\
\hline
\end{tabular}
\end{center}

\clearpage





\section{Verb Classes}
\label{verb-classes}

Each main\footnote{Auxiliary verbs are handled under a different mechanism.
See section~\ref{auxiliaries} for details.} verb in the syntactic
lexicon selects 
at least one tree family\footnote{A tree family is a collection of trees that
would be considered related under a transformational grammar approach.  See
section~\ref{verb-transformations} for more details.}, or subcategorization
frame.  Since the tree database and syntactic lexicon are already separated for
space efficiency (see \ref{overview}), each verb can efficiently select a
large number of trees by specifying a tree family, as opposed to each of the
individual trees.  This approach allows for a considerably reduction in the
number of trees that must be specified for any given verb or form of a verb.

There are currently 38 tree families in the system\footnote{An explanation of
the naming convention used in naming the trees and tree families is available
in Appendix \ref{naming-conventions}.}.  This section gives a brief description
of each tree family and and shows the corresponding declarative tree (before
lexicalization - the $\diamond$ indicates the anchor of the tree), along with
any peculiar characteristics or trees.  It also tells which transformations are
in each tree family, and gives the approximate number of verbs that select that
family.  A few sample verbs are given, along with example sentences.


\subsection{Intransitive: Tnx0V}\index{verbs, intransitive}
\label{nx0V-family}

\begin{description}

\item[Description:]  This tree family is selected by verbs that do not 
require an object complement of any type.  Adverbs, prepositional phrases and
other adjuncts may adjoin on, but are required for the sentences to be
grammatical.  Approximately 2,091 verbs select this family.

\item[Examples:]  {\it eat}, {\it kill}, {\it dance} \\
{\it Al ate.} \\ 
{\it Seth killed.} \\ 
{\it Hyun danced.}

\item[Declarative tree:]  See Figure~\ref{nx0V-tree}.

\begin{figure}[ht]
\centering
\begin{tabular}{c}
\psfig{figure=ps/verb-class-files/alphanx0V.ps,height=4.0cm}
\end{tabular}
\caption{Declarative Intransitive Tree:  $\alpha$nx0V}
\label{nx0V-tree}
\end{figure}

\item[Other available trees\footnote{Please see 
Section~\ref{verb-transformations} for description of each of these types of
trees.}:] Wh-moved subject, subject relative clause, imperative, determiner
gerund, NP gerund.

\end{description}




\subsection{Transitive: Tnx0Vnx1}\index{verbs,transitive}
\label{nx0Vnx1-family}

\begin{description}

\item[Description:] This tree family is selected by verbs that require only a 
NP object complement.  The NP's may be complex structures, including NP's that
take sentential complements and gerund NPs.  This does not include light verb
constructions (see Sections~\ref{nx0lVN1-family} and \ref{nx0lVN1Pnx2-family}).
Approximately 4,335 verbs select the transitive tree family.

\item[Examples:] eat, dance, take, like\\
{\it Al ate an apple.} \\ 
{\it Seth danced the tango.} \\ 
{\it Hyun is taking an algorithms course.} \\
{\it Anoop likes the fact that the semester is over.}

\item[Declarative tree:] See Figure~\ref{nx0Vnx1-tree}.

\begin{figure}[ht]
\centering
\begin{tabular}{c}
\psfig{figure=ps/verb-class-files/alphanx0Vnx1.ps,height=4.0cm}
\end{tabular}
\caption{Declarative Transitive Tree:  $\alpha$nx0Vnx1}
\label{nx0Vnx1-tree}
\end{figure}

\item[Other available trees:] wh-moved subject, wh-moved object, subject
relative clause, object relative clause, imperative, pre-sentential adjunct,
post-sentential adjunct, determiner gerund, NP gerund, passive with {\it by}
phrase, passive without {\it by} phrase, passive with wh-moved subject and {\it
by} phrase, passive with wh-moved subject and no {\it by} phrase, passive with
wh-moved object out of the {\it by} phrase, passive with wh-moved {\it by}
phrase, passive with relative clause on subject and {\it by} phrase, passive
with relative clause on subject and no {\it by} phrase, passive with relative
clause on object on the {\it by} phrase, ergative, ergative with wh-moved
subject, ergative with subject relative clause.  In addition, two other trees
that allowed transitive verbs to function as adjectives {\it the stopped truck}
are also in the family.

\end{description}





\subsection{Transitive Idioms: Tnx0Vdn1}\index{verbs,idiomatic}
\label{nx0Vdn1-family}

\begin{description}

\item[Description:]  This tree family is selected by idiomatic phrases in which
the verb, determiner, and NP are all frozen (as in {\it He kicked the
bucket.}).  Only a limited number of transformations are allowed, as compared
to the normal transitive tree family (see Section~\ref{nx0Vnx1-family}).  The
analysis of idioms has not been done for idioms in general; this tree is
included to illustrate how they could be handled in XTAG.  Other idioms that
have the same structure as {\it kick the bucket}, and that are limited to the
same transformations would select this tree, and more trees would be built to
handle other idioms.  Note that {\it John kicked the bucket} is actually
ambiguous, and would result in two parses - an idiom (meaning that John died),
a simple transitive sentences (meaning that there is an physical bucket that
John hit with his foot).

\item[Examples:] {\it kick} \\
{\it He kicked the bucket.}

\item[Declarative tree:]  See Figure~\ref{nx0Vdn1-tree}.

\begin{figure}[ht]
\centering
\begin{tabular}{c}
\psfig{figure=ps/verb-class-files/alphanx0Vdn1.ps,height=4.0cm}
\end{tabular}
\caption{Declarative Transitive Idiom Tree:  $\alpha$nx0Vdn1}
\label{nx0Vdn1-tree}
\end{figure}

\item[Other available trees:]  wh-moved subject, subject relative clause, 
imperative.

\end{description}




\subsection{Di-Transitive: Tnx0Vnx1nx2}\index{verbs,ditransitive}
\label{nx0Vnx1nx2-family}

\begin{description}

\item[Description:]  This tree family is selected by verbs that take exactly 
two NP complements.  It does {\bf not} include verbs that undergo the
ditransitive verb shift (see Section~\ref{nx0Vnx1Pnx2-family}).  Various
alternations that change a sentence in this class into a NP followed by a PP
are handled by the transitive version of the verb (see
Section~\ref{nx0Vnx1-family}) followed by a PP that adjoins on.  In these
cases, the PP is optional.  Currently, only one verb in our database ({\it
ensure}) selects the DiTransitive with PP tree family (see
Section~\ref{nx0Vnx1pnx2-family}).  In this case, the verb does not have a
transitive alternation, and so the prepositional phrase is required.
Approximately 54 verbs select the Di-Transitive tree family.

\item[Examples:] {\it ask}, {\it cook}, {\it win} \\
{\it Christy asked Mike a question.} \\ 
{\it Doug cooked his father dinner.} \\
{\it Dania won her sister a stuffed animal.}

\item[Declarative tree:]  See Figure~\ref{nx0Vnx1nx2-tree}

\begin{figure}[ht]
\centering
\begin{tabular}{c}
\psfig{figure=ps/verb-class-files/alphanx0Vnx1nx2.ps,height=4.0cm}
\end{tabular}
\caption{Declarative DiTransitive Tree:  $\alpha$nx0Vnx1nx2}
\label{nx0Vnx1nx2-tree}
\end{figure}

\item[Other available trees:] wh-moved subject, wh-moved direct object, 
wh-moved indirect object, subject relative clause, direct object relative
clause, indirect object relative clause, imperative, determiner gerund, NP
gerund, passive with {\it by} phrase, passive without {\it by} phrase, passive
with wh-moved subject and {\it by} phrase, passive with wh-moved subject and no
{\it by} phrase, passive with wh-moved object out of the {\it by} phrase,
passive with wh-moved {\it by} phrase, passive with wh-moved indirect object
and {\it by} phrase, passive with wh-moved indirect object and no {\it by}
phrase,  passive with relative clause on subject and {\it by} phrase, passive
with relative clause on subject and no {\it by} phrase, passive with relative
clause on object of the {\it by} phrase, passive with relative clause on the
indirect object and {\it by} phrase, passive with relative clause on the
indirect object and no {\it by} phrase.


\end{description}





\subsection{DiTransitive with PP:Tnx0Vnx1pnx2}\index{verbs, NP with VP verbs}
\label{nx0Vnx1pnx2-family}

\begin{description}

\item[Description:]  This tree family is selected by ditransitive verbs that
take a noun phrase followed by a prepositional phrase.  The preposition is not
constrained.  The prepositional must be required and not optional - that is,
the sentence must be ungrammatical with just the noun phrase (i.e. {\it John
put the table}).  No verbs, therefore, should select both this tree family and
the transitive tree family.  (see Section~\ref{nx0Vnx1-family}).  This tree
family is also distinguished from the ditransitive verbs, such as {\it give}
that undergo verb shifting (see Section~\ref{nx0Vnx1Pnx2-family}).  There are
approximately 61 verbs that select this tree family.

\item[Examples:] {\it associate}, {\it put}, {\it refer} \\
{\it Rostenkowski associated money with power.}   \\
{\it He put his reputation on the line.}  \\
{\it He referred all questions to his attorney.}

\item[Declarative tree:]  See Figure~\ref{nx0Vnx1pnx2-tree}

\begin{figure}[ht]
\centering
\begin{tabular}{c}
\psfig{figure=ps/verb-class-files/alphanx0Vnx1pnx2.ps,height=4.0cm}
\end{tabular}
\caption{Declarative DiTransitive with PP Tree:  $\alpha$nx0Vnx1pnx2}
\label{nx0Vnx1pnx2-tree}
\end{figure}

\item[Other available trees:]  wh-moved subject, wh-moved direct object, 
wh-moved object of PP, wh-moved PP, subject relative clause, direct object
relative clause, object of PP relative clause, imperative, determiner gerund,
NP gerund, passive with {\it by} phrase, passive without {\it by} phrase,
passive with wh-moved subject and {\it by} phrase, passive with wh-moved
subject and no {\it by} phrase, passive with wh-moved object out of the {\it
by} phrase, passive with wh-moved {\it by} phrase, passive with wh-moved object
out of the PP and {\it by} phrase, passive with wh-moved object out of the PP
and no {\it by} phrase, passive with wh-moved PP and {\it by} phrase, passive
with wh-moved PP and no {\it by} phrase, passive with relative clause on
subject and {\it by} phrase, passive with relative clause on subject and no
{\it by} phrase, passive with relative clause on object of the {\it by} phrase,
passive with relative clause on the object of the PP and {\it by} phrase,
passive with relative clause on the object of the PP and no {\it by} phrase.

\end{description}




\subsection{Ditransitive with PP shift: Tnx0Vnx1Pnx2}\index{verbs,ditransitive
with PP shift}
\label{nx0Vnx1Pnx2-family}

\begin{description}

\item[Description:]  This tree family is selected by ditransitive verbs that
undergo a shift to a {\it to} prepositional phrase.  These ditransitive verbs
are clearly constrained so that when they shift, the prepositional phrase must
start with {\it to}, unlike the di-transitives in Section~{\it
nx0Vnx1nx2-family}, in which verbs may shift to any of a number of
prepositions.  Both the shifted and non-shifted trees are included.
Approximately 55 verbs select this family.

\item[Examples:] {\it give}, {\it promise}, {\it tell} \\
{\it Bill gave Hillary flowers.} \\ 
{\it Bill gave flowers to Hillary.} \\
{\it Whitman had promised the voters a tax cut.} \\
{\it Whitman had promised a tax cut to the voters.} \\
{\it Pinnochino told Gepetto a lie.} \\
{\it Pinnochino told a lie to Gepetto.}

\item[Declarative tree:]  See Figure~\ref{nx0Vnx1Pnx2-tree}

\begin{figure}[ht]
\centering
\begin{tabular}{cc}
\psfig{figure=ps/verb-class-files/alphanx0Vnx1Pnx2.ps,height=5.0cm} &
\psfig{figure=ps/verb-class-files/alphanx0Vnx2nx1.ps,height=4.0cm} \\
$\alpha$nx0Vnx1Pnx2 & $\alpha$nx0Vnx2nx1
\end{tabular}
\caption{Declarative Ditransitive with PP shift Trees}
\label{nx0Vnx1Pnx2-tree}
\end{figure}

\item[Other available trees:] 
{\bf Non-shifted:}  wh-moved subject, wh-moved direct object, 
wh-moved indirect object, subject relative clause, direct object relative
clause, indirect object relative clause, imperative, NP
gerund, passive with {\it by} phrase, passive without {\it by} phrase, passive
with wh-moved subject and {\it by} phrase, passive with wh-moved subject and no
{\it by} phrase, passive with wh-moved object out of the {\it by} phrase,
passive with wh-moved {\it by} phrase, passive with wh-moved indirect object
and {\it by} phrase, passive with wh-moved indirect object and no {\it by}
phrase,  passive with relative clause on subject and {\it by} phrase, passive
with relative clause on subject and no {\it by} phrase, passive with relative
clause on object of the {\it by} phrase, passive with relative clause on the
indirect object and {\it by} phrase, passive with relative clause on the
indirect object and no {\it by} phrase; \\
{\bf Shifted:} wh-moved subject, wh-moved direct object, 
wh-moved object of PP, wh-moved PP, subject relative clause, direct object
relative clause, object of PP relative clause, imperative, determiner gerund,
NP gerund, passive with {\it by} phrase, passive without {\it by} phrase,
passive with wh-moved subject and {\it by} phrase, passive with wh-moved
subject and no {\it by} phrase, passive with wh-moved object out of the {\it
by} phrase, passive with wh-moved {\it by} phrase, passive with wh-moved object
out of the PP and {\it by} phrase, passive with wh-moved object out of the PP
and no {\it by} phrase, passive with wh-moved PP and {\it by} phrase, passive
with wh-moved PP and no {\it by} phrase, passive with relative clause on
subject and {\it by} phrase, passive with relative clause on subject and no
{\it by} phrase, passive with relative clause on object of the {\it by} phrase,
passive with relative clause on the object of the PP and {\it by} phrase,
passive with relative clause on the object of the PP and no {\it by} phrase.


\end{description}




\subsection{Sentential Complement with NP: Tnx0Vnx1s2}\index{verbs,Sentential
Complement with NP} 
\label{nx0Vnx1s2-family}

\begin{description}

\item[Description:]  The tree family is selected by verbs that take both a NP
and sentential complement.  The sentential complement may be an infinitive,
indicative, or small clause (see Section~\ref{small-clauses}).  The type of
clause is specified by each individual verb in its syntactic lexicon entry.  A
given verb may select more than one type of sentential complement.  The
declarative tree, and many other trees in this family, are auxiliary trees, as
opposed to the more normal initial trees.  These auxiliary trees adjoin onto an
S node in an existing tree of the type specified by the sentential complement.
This is the mechanism by which TAGs are able to maintain long-distance
dependencies (see Section~\ref{long-distance-dependencies}), even over multiple
embeddings (ie. {\it What did Bill tell Mary that John said?}.  Approximately
106 verbs select this tree family.

\item[Examples:] {\it beg}, {\it promise}, {\it regard} \\
{\it Srini begged Mark to increase his disk quota.} \\
{\it Jim promised that he would feed the dogs.} \\
{\it Dania regarded Carl a jerk.}

\item[Declarative tree:]  See Figure~\ref{nx0Vnx1s2-tree}

\begin{figure}[ht]
\centering
\begin{tabular}{c}
\psfig{figure=ps/verb-class-files/betanx0Vnx1s2.ps,height=4.0cm}
\end{tabular}
\caption{Declarative Transitive Tree:  $\beta$nx0Vnx1s2}
\label{nx0Vnx1s2-tree}
\end{figure}

\item[Other available trees:]  wh-moved subject, wh-moved object, wh-moved
sentential complement, subject relative clause, object relative clause,
imperative, determiner gerund, NP gerund, passive with {\it by} phrase before
sentential complement, passive with {\it by} phrase after sentential
complement, passive without {\it by} phrase, passive with wh-moved subject and
{\it by} phrase before sentential complement, passive with wh-moved subject and
{\it by} phrase after sentential complement, passive with wh-moved subject and
no {\it by} phrase, passive with wh-moved object out of the {\it by} phrase,
passive with wh-moved {\it by} phrase, passive with relative clause on subject
and {\it by} phrase before sentential complement, passive with relative clause
on subject and {\it by} phrase after sentential complement, passive with
relative clause on subject and no {\it by} phrase.

\end{description}



\subsection{Intransitive Verb Particle: Tnx0Vpl}\index{verbs,verb particle,intransitive}
\label{nx0Vpl}

\begin{description}

\item[Description:]  The trees in this tree family are anchored by both the
verb and the verb particle.  Both appear in the syntactic lexicon and together
select this tree family.  Intransitive verb particles can be difficult to 
distinguish from intransitive verbs with adverbs adjoined on. The main
diagnostics for including verbs in this class was whether the meaning seemed
compositional or not, and whether there existed a transitive version of the
verb/verb particle combination with the same meaning.  The existence of an
alternate compositional meaning was a strong indication that a verb particle
construction consisted.  There are approximately 164 verb/verb particle
combinations.

\item[Examples:] {\it add up}, {\it come out}, {\it sign off} \\
{\it The number never quite added up.} \\
{\it John finally came out (of the closet).} \\
{\it I think that I will sign off now.}

\item[Declarative tree:]  See Figure~\ref{nx0Vpl-tree}

\begin{figure}[ht]
\centering
\begin{tabular}{c}
\psfig{figure=ps/verb-class-files/alphanx0Vpl.ps,height=4.0cm}
\end{tabular}
\caption{Declarative Intransitive Verb Particle Tree:  $\alpha$nx0Vpl}
\label{nx0Vpl-tree}
\end{figure}

\item[Other available trees:] Wh-moved subject, subject relative clause, 
imperative, determiner gerund, NP gerund.

\end{description}




\subsection{Transitive Verb Particle: Tnx0Vplnx1}\index{verbs,particle,transitive}
\label{nx0Vplnx1-family}

\begin{description}

\item[Description:]  Verb/Verb particle combinations that take an NP complement
select this tree family.  Both the verb and the verb particle are anchors of
the trees.  Distinguishing verb particles with an NP complement from
prepositional phrases can be somewhat controversial, but we fell back on a
purely syntactic test.  If the Prep/particle lexical item moved, then it was a
particle.  If it did not, then it was a preposition heading a prepositional
phrase.  In many, but not all, of the verb particle cases, there was an
alternate prepositional meaning in which the lexical item did not move.
(ie. {\it He looked up the number (in the phonebook).  He looked the number
up. He looked up the road (to see if any cars were coming).  *He looked the
road up.})  There are approximately 841 verb/verb particle combinations.

\item[Examples:] {\it blow off}, {\it make up}, {\it pick out} \\
{\it He blew off classes for the third time this year.} \\
{\it He blew classes off for the third time this year.} \\
{\it The dyslexic leprechaun made up the syntactic lexicon.} \\
{\it The dyslexic leprechaun made the syntactic lexicon up.} \\
{\it I would like to pick out a new computer.} \\
{\it I would like to pick a new computer out.} 

\item[Declarative tree:]  See Figure~\ref{nx0Vplnx1-tree}.

\begin{figure}[ht]
\centering
\begin{tabular}{cc}
\psfig{figure=ps/verb-class-files/alphanx0Vplnx1.ps,height=4.0cm} &
\psfig{figure=ps/verb-class-files/alphanx0Vnx1pl.ps,height=4.0cm} \\
$\alpha$nx0Vplnx1 & $\alpha$nx0Vnx1pl
\end{tabular}
\caption{Declarative Transitive Verb Particle Tree}
\label{nx0Vplnx1-tree}
\end{figure}

\item[Other available trees:] wh-moved subject with particle before the NP,
wh-moved subject with particle after the NP, wh-moved object, subject relative
clause with particle before the NP, subject relative clause with particle after
the NP, object relative clause, imperative with particle before the NP,
imperative with particle after the NP, determiner gerund with particle before
the NP, NP gerund with particle before the NP, NP gerund with particle after
the NP, passive with {\it by} phrase, passive without {\it by} phrase, passive
with wh-moved subject and {\it by} phrase, passive with wh-moved subject and no
{\it by} phrase, passive with wh-moved object out of the {\it by} phrase,
passive with wh-moved {\it by} phrase, passive with relative clause on subject
and {\it by} phrase, passive with relative clause on subject and no {\it by}
phrase, passive with relative clause on object of the {\it by} phrase.

\end{description}




\subsection{Ditransitive Verb Particle: Tnx0Vplnx1nx2}\index{verbs,particle,ditransitive}
\label{nx0Vplnx1nx2}

\begin{description}

\item[Description:]  Verb/verb particle combinations that select this tree
family take 2 NP complements.  Both the verb and the verb particle anchor the
trees, and the verb particle can occur before, between, or after the noun
phrases.  Perhaps because of the complexity of the sentence, these verbs do not
seem to have passive alternations (*{\it A new bank account was opened Michelle
by me.})  There are 4 verb/verb particle combinations that select this tree
family.  The exhaustive list is given in the examples.

\item[Examples:] {\it dish out}, {\it open up}, {\it pay off}, {\it rustle up}
\\
{\it I opened up Michelle a new bank account.} \\
{\it I opened Michelle up a new bank account.} \\
{\it I opened Michelle a new bank account up.}


\item[Declarative tree:]  See Figure~\ref{nx0Vplnx1nx2-tree}

\begin{figure}[ht]
\centering
\begin{tabular}{ccc}
\psfig{figure=ps/verb-class-files/alphanx0Vplnx1nx2.ps,height=4.0cm} &
\psfig{figure=ps/verb-class-files/alphanx0Vnx1plnx2.ps,height=4.0cm} &
\psfig{figure=ps/verb-class-files/alphanx0Vnx1nx2pl.ps,height=4.0cm} \\
$\alpha$nx0Vplnx1nx2 & $\alpha$nx0Vnx1plnx2 & $\alpha$nx0Vnx1nx2pl
\end{tabular}
\caption{Declarative Ditransitive Verb Particle Tree}
\label{nx0Vplnx1nx2-tree}
\end{figure}

\item[Other available trees:] wh-moved subject with particle before the NPs,
wh-moved subject with particle between the NPs, wh-moved subject with particle
after the NPs, wh-moved indirect object with particle before the NPs, wh-moved
indirect object with particle after the NPs, wh-moved direct object with
particle before the NPs, wh-moved direct object with particle between the NPs,
subject relative clause with particle before the NPs, subject relative clause
with particle between the NPs, subject relative clause with particle after the
NPs, indirect object relative clause with particle before the NPs, indirect
object relative clause with particle after the NPs, direct object relative
clause with particle before the NPs, direct object relative clause with
particle between the NPs, imperative with particle before the NPs, imperative
with particle between the NPs, imperative with particle after the NPs,
determiner gerund with particle before the NPs, NP gerund with particle before the NPs, NP gerund with particle between the NPs, NP gerund with
particle after the NPs.

\end{description}





\subsection{Intransitive with PP:Tnx0Vpnx1}\index{verbs,intransitive with PP}
\label{nx0Vpnx1-family}
\begin{description}

\item[Description:]  The verbs that select this tree family are not strictly 
intransitive, in that they {\bf must} be followed by a prepositional phrase.
Verbs that are intransitive and simply {\bf can} be followed by a prepositional
phrase do not select this family, but instead have the PP adjoin onto the
intransitive sentence.  Accordingly, there should be no verbs in both this
class and the intransitive tree family (see Section~\ref{nx0V-family}).  The
prepositional phrase is not restricted to be headed by any particular lexical
item.  Note that these are not transitive verb particles (see
Section~\ref{nx0Vplnx1-family}, since the head of the PP does not move.
Approximately 171 verbs select this tree family.

\item[Examples:] {\it grab}, {\it impinge}, {\it provide} \\
{\it Seth grabbed for the brass ring.} \\
{\it The noise gradually impinged on Dania's thoughts.} \\
{\it A good host provides for everyone's needs.}

\item[Declarative tree:]  See Figure~\ref{nx0Vpnx1-tree}

\begin{figure}[ht]
\centering
\begin{tabular}{c}
\psfig{figure=ps/verb-class-files/alphanx0Vpnx1.ps,height=4.0cm}
\end{tabular}
\caption{Declarative Intransitive with PP Tree:  $\alpha$nx0Vpnx1}
\label{nx0Vpnx1-tree}
\end{figure}

\item[Other available trees:]  wh-moved subject, wh-moved object of the PP,
wh-moved PP, subject relative clause, object of the PP relative clause,
imperative, determiner gerund, NP gerund, passive with {\it by} phrase, passive
without {\it by} phrase, passive with wh-moved subject and {\it by} phrase,
passive with wh-moved subject and no {\it by} phrase, passive with wh-moved
{\it by} phrase, passive with relative clause on subject and {\it by} phrase,
passive with relative clause on subject and no {\it by} phrase, passive with
relative clause on object of the {\it by} phrase.

\end{description}



\subsection{Sentential Complement: Tnx0Vs1}\label{verbs,sentential complement}
\label{nx0Vs1-family}

\begin{description}

\item[Description:]  The tree family is selected by verbs that take just a
sentential complement.  The sentential complement may be of type infinitive,
indicative, or small clause (see Section~\ref{small-clauses}).  The type of
clause is specified by each individual verb in its syntactic lexicon entry, and
a given verb may select more than one type of sentential complement.  The
declarative tree, and many other trees in this family, are auxiliary trees, as
opposed to the more normal initial trees.  These auxiliary trees adjoin onto an
S node in an existing tree of the type specified by the sentential complement.
This is the mechanism by which TAGs are able to maintain long-distance
dependencies (see Section~\ref{long-distance-dependencies}), even over multiple
embeddings (ie. {\it What did Bill think that John said?}  Approximately 318
verbs select this tree family.

\item[Examples:]  {\it consider}, {\it think}, {\it want} \\
{\it Dania considered the algorithm unworkable.}
{\it Srini thought that the program was working.} \\
{\it They wanted to make the sentence parse correctly.}

\item[Declarative tree:]  See Figure~\ref{nx0Vs1-tree}

\begin{figure}[ht]
\centering
\begin{tabular}{c}
\psfig{figure=ps/verb-class-files/betanx0Vs1.ps,height=4.0cm}
\end{tabular}
\caption{Declarative Sentential Complement Tree:  $\beta$nx0Vs1}
\label{nx0Vs1-tree}
\end{figure}

\item[Other available trees:]  Wh-moved subject, Wh-moved sentential
complement, subject relative clause, imperative, determiner gerund, NP gerund.

\end{description}




\subsection{Intransitive with Adjective: Tnx0Va1}\index{verbs,intransitive with adjective}
\label{nx0Va1-family}

\begin{description}

\item[Description:]  The verbs that select this tree family take an adjective
as a complement.  The adjective may be regular, comparative, or superlative.
It may also be formed from the special class of adjectives derived from the
transitive verbs (i.e. {\it agitated, broken}.  See
Section~\ref{nx0Vnx1-family} and \ref{adj-modifier}).
Unlike the Intransitive with PP verbs (see Section~\ref{nx0Vpnx1-family}), some
of these verbs may also occur as bare intransitives as well.  This distinction
is drawn because adjectives do not normally adjoin onto sentences, as
prepositional phrases do.  Other intransitive verbs can only occur with the
adjective, and these select only this family.  The verb class is also
distinguished from the adjective small clauses (see
Section''\ref{nx0Ax1-family}) because these verbs are not raising verbs.
Approximately 34 verbs select this tree family.

\item[Examples:] {\it become}, {\it grow}, {\it smell} \\
{\it The greenhouse became hotter.} \\
{\it The plants grew tall and strong.} \\
{\it The flowers smelled wonderful.}

\item[Declarative tree:]  See Figure~\ref{nx0Va1-tree}

\begin{figure}[ht]
\centering
\begin{tabular}{c}
\psfig{figure=ps/verb-class-files/alphanx0Va1.ps,height=4.0cm}
\end{tabular}
\caption{Declarative Intransitive with Adjective Tree:  $\alpha$nx0Va1}
\label{nx0Va1-tree}
\end{figure}

\item[Other available trees:]  Wh-moved subject, Wh-moved adjective 
({\it how}), subject relative clause, imperative, NP gerund.

\end{description}




\subsection{Sentential Subject:Ts0Vnx1}\index{verbs,sentential subject}
\label{s0Vnx1-family}

\begin{description}

\item[Description:] The verbs that select this tree family all take sentential
subjects, and are often referred to as 'psych' verbs, since they all refer to
some psychological state of mind.  The sentential subject can be indicative
(complementizer required) or infinitive (complementizer optional).
Approximately 100 verbs that select this tree family.

\item[Examples:] {\it delight}, {\it impress}, {\it surprise} \\
{\it That the tea had rosehips in it delighted Christy.} \\
{\it To even attempt a marathon impressed Dania.} \\
{\it For Jim to have walked the dogs surprised Beth.}

\item[Declarative tree:]  See Figure~\ref{s0Vnx1-tree}

\begin{figure}[ht]
\centering
\begin{tabular}{c}
\psfig{figure=ps/verb-class-files/alphas0Vnx1.ps,height=4.0cm}
\end{tabular}
\caption{Declarative Sentential Subject Tree:  $\alpha$s0Vnx1}
\label{s0Vnx1-tree}
\end{figure}

\item[Other available trees:]  Wh-moved subject, Wh-moved object.

\end{description}





\subsection{Light Verbs: Tnx0lVN1, Tnx0lVdxN1}\index{verbs, light}
\label{nx0lVN1-family}

\begin{description}

\item[Description:] The verb/noun pairs that select this tree family are pairs
in which the interpretation is non-compositional and the noun contributes
argument structure to the predicate (i.e. {\it The man took a walk.} vs {\it
The man took a radio}).  The verb and the noun occur together in the syntactic
database, and both anchor the trees.  The verbs in the light verb constructions
are {\it do}, {\it give}, {\it have}, {\it make}, and {\it take}.  The noun
following the light verb is (usually) a bare infinitive form (ie. {\it have a
good cry}) and usually occurs with {\it a(n)}.  However, we include deverbal
nominals ({\it take a bath}, {\it give a demonstration}) as well.  onstructions
with nouns that do not contribute an argument structure ({\it have a
cigarette}, {\it give NP a black eye}) are excluded.  In addition to semantic
considerations of light verbs, they differ syntactically from transitive verbs
(see Section~\ref{nx0Vnx1-family}) as well in that the noun in the light verb
construction does not extract.  Because the noun is an anchor in the tree,
there are two different tree families representing nouns that require
determiners and those that occur without them (see Section~\ref{nouns} for more
information on noun phrases in general).  There are approximately 96 verb/noun
pairs that select the light verb tree without determiners, and 242 that select
the light verb tree with determiners.

\item[Examples:] {\it give groan}, {\it have discussion}, {\it make comment} \\
{\it The audience gave a collective groan.} \\
{\it We had a big discussion about closing the libraries.} \\
{\it The professors made comments on the paper.}

\item[Declarative tree:]  See Figure~\ref{nx0lVN1-tree}.

\begin{figure}[ht]
\centering
\begin{tabular}{cc}
\psfig{figure=ps/verb-class-files/alphanx0lVN1.ps,height=4.0cm}
\psfig{figure=ps/verb-class-files/alphanx0lVdxN1.ps,height=4.0cm} \\
$\alpha$nx0lVN1 & $\alpha$nx0lVdxN1
\end{tabular}
\caption{Declarative Light Verb Trees}
\label{nx0lVN1-tree}
\end{figure}

\item[Other available trees:] Wh-moved subject, subject relative clause, 
imperative, determiner gerund, NP gerund.

\end{description}




\subsection{Ditransitive Light Verbs with PP Shift: Tnx0lVN1Pnx2,Tnx0lVdxN1Pnx2}\index{verbs,ditransitive light verbs with PP shift}
\label{nx0lVN1Pnx2-family}

\begin{description}

\item[Description:]  The verb/noun pairs that select this tree family are pairs
in which the interpretation is non-compositional and the noun contributes
argument structure to the predicate (i.e. {\it Dania made Srini a cake.} vs
{\it Dania made Srini a loan.}).  The verb and the noun occur together in the
syntactic database, and both anchor the trees.  The verbs in these light verb
constructions are {\it give} and {\it make}.  The noun following the light verb
is (usually) a bare infinitive form (ie. {\it make promise to Anoop}).
However, we include deverbal nominals ({\it make a payment to Anoop}) as well.
Constructions with nouns that do not contribute an argument structure are
excluded.  In addition to semantic considerations of light verbs, they differ
syntactically from the ditransitive with PP shift verbs (see
Section~\ref{nx0Vnx1Pnx2-family}) as well in that the noun in the light verb
construction does not extract.  Also, passivization is severely restricted.
Because the noun is an anchor in the tree, there are two different tree
families representing nouns that require determiners and those that occur
without them (see Section~\ref{nouns} for more information on noun phrases in
general).  There are approximately 10 verb/noun pairs that select the trees
without determiners, and 18 that select the trees with determiners.

\item[Examples:] {\it give look}, {\it give wave}, {\it make promise} \\
{\it Dania gave Carl a look that would kill.} \\
{\it Amanda gave a little wave as she left.} \\
{\it Dania made Doug a promise.} 

\item[Declarative tree:]  See Figure~\ref{nx0lVN1Pnx2-tree}

\begin{figure}[ht]
\centering
\begin{tabular}{cc}
\psfig{figure=ps/verb-class-files/alphanx0lVN1Pnx2.ps,height=4.0cm} &
\psfig{figure=ps/verb-class-files/alphanx0lVnx2N1.ps,height=4.0cm} \\
$\alpha$nx0lVN1Pnx2 & $\alpha$nx0lVnx2N1 \\
\vspace{1.5cm}
\psfig{figure=ps/verb-class-files/alphanx0lVdxN1Pnx2.ps,height=4.0cm} &
\psfig{figure=ps/verb-class-files/alphanx0lVnx2dxN1.ps,height=4.0cm} \\
$\alpha$nx0lVdxN1Pnx2 & $\alpha$nx0lVnx2dxN1 \\
\end{tabular}
\caption{Declarative Light Verbs with PP Tree}
\label{nx0lVN1Pnx2-tree}
\end{figure}

\item[Other available trees:]
{\bf Non-shifted:}  wh-moved subject,
wh-moved indirect object, subject relative clause, indirect object relative 
clause, imperative, NP gerund, passive with {\it by} phrase \\
{\bf Shifted:} wh-moved subject,  wh-moved object of PP, wh-moved PP, subject 
relative clause, object of PP relative clause, imperative, determiner gerund,
NP gerund, passive with {\it by} phrase.
\end{description}




\subsection{NP It-Cleft: TItVnx1s2}
\label{ItVnx1s2-family}

\begin{description}

\item[Description:] ** Beth to fill this in **

\item[Examples:] {\it it be} \\
{\it It was Beth who agreed to do the demo.}

\item[Declarative tree:]  See Figure~\ref{ItVnx1s2-tree}

\begin{figure}[ht]
\centering
\begin{tabular}{c}
\psfig{figure=ps/verb-class-files/alphaItVnx1s2.ps,height=4.0cm}
\end{tabular}
\caption{Declarative NP It-Cleft Tree:  $\alpha$ItVnx1s2}
\label{ItVnx1s2-tree}
\end{figure}

\item[Other available trees:]  inverted question, wh-moved object with
{\it be} inverted, wh-moved object with {\it be} not inverted.

\end{description}



\subsection{PP It-Cleft: TItVpnx1s2}
\label{ItVpnx1s2-family}

\begin{description}

\item[Description:]  ** Beth to fill this in **

\item[Examples:] {\it it be} \\
{\it It was at Kent State that the police shot all those students.}

\item[Declarative tree:]  See Figure~\ref{ItVpnx1s2-tree}

\begin{figure}[ht]
\centering
\begin{tabular}{c}
\psfig{figure=ps/verb-class-files/alphaItVpnx1s2.ps,height=4.0cm}
\end{tabular}
\caption{Declarative PP It-Cleft Tree:  $\alpha$ItVnx1s2}
\label{ItVpnx1s2-tree}
\end{figure}

\item[Other available trees:] inverted question, wh-moved prepositional phrase
with {\it be} inverted, wh-moved prepositional phrase with {\it be} not
inverted.

\end{description}

\subsection{Adverb It-Cleft: TItVad1s2}
\label{ItVad1s2-family}

\begin{description}

\item[Description:]  ** Beth to fill this in **

\item[Examples:] {\it it be} \\
{\it It was reluctantly that Dania agreed to do the tech report.}

\item[Declarative tree:]  See Figure~\ref{ItVad1s2-tree}

\begin{figure}[ht]
\centering
\begin{tabular}{c}
\psfig{figure=ps/verb-class-files/alphaItVad1s2.ps,height=4.0cm}
\end{tabular}
\caption{Declarative Transitive Tree:  $\alpha$ItVad1s2}
\label{ItVad1s2-tree}
\end{figure}

\item[Other available trees:]  inverted question, wh-moved adverb {\it how}
with {\it be} inverted, wh-moved adverb {\it how} with {\it be} not
inverted.

\end{description}



\subsection{Adjective Small Clause Tree: Tnx0Ax1}\index{verbs,small-clause}
\label{nx0Ax1-family}

\begin{description}

\item[Description:]  These trees are not anchored by verbs, but by adjectives.
They are explained in much greater detail in the section on small clauses (see
Section~\ref{sm-clause-xtag-analysis}).  The section is presented here for
completeness.  Approximately 3312 adjectives select this tree family.

\item[Examples:] {\it addictive}, {\it dangerous}, {\it wary}\\
{\it Cigarettes are addictive.} \\
{\it Smoking cigarettes is dangerous.} \\
{\it John seems wary of the Surgeon General's warnings.}

\item[Declarative tree:]  See Figure~\ref{nx0Ax1-tree}

\begin{figure}[ht]
\centering
\begin{tabular}{c}
\psfig{figure=ps/verb-class-files/alphanx0Ax1.ps,height=4.0cm}
\end{tabular}
\caption{Declarative Adjective Small Clause Tree:  $\alpha$nx0Ax1}
\label{nx0Ax1-tree}
\end{figure}

\item[Other available trees:]  wh-moved subject, wh-moved adjective {\it how},
relative clause on subject, imperative, NP gerund.

\end{description}

\subsection{Adjective Small Clause with Sentential Complement: Tnx0Ax1s2}
\label{nx0Ax1s2-family}

\begin{description}

\item[Description:]  This tree family is selected by adjectives that take 
sentential complements.  The sentential complements can be indicative or
infinitive.  Note that these trees are not anchored by adjectives, not verbs.
Most adjectives that take the Adjective Small Clause tree family (see
Section~\ref{nx0Ax1-family}) takes this family as well\footnote{No great
attempt has been made to go through and decide which adjective actually should
take this family and which should not.}.  Small clauses are explained in much
greater detail in Section~\ref{sm-clause-xtag-analysis}).  This section is
presented here for completeness.  Approximately 3229 adjectives select this
tree family.

\item[Examples:] {\it able}, {\it curious}, {\it disappointed} \\
{\it Christy was able to find the problem.} \\
{\it Christy was curious whether the new analysis was working.} \\
{\it Christy was disappointed that the old analysis failed.} 

\item[Declarative tree:]  See Figure~\ref{nx0Ax1s2-tree}

\begin{figure}[ht]
\centering
\begin{tabular}{c}
\psfig{figure=ps/verb-class-files/alphanx0Ax1s2.ps,height=4.0cm}
\end{tabular}
\caption{Declarative  Adjective Small Clause with Sentential Complement Tree:  $\alpha$nx0Ax1s2}
\label{nx0Ax1s2-tree}
\end{figure}

\item[Other available trees:] wh-moved subject, wh-moved adjective {\it how},
relative clause on subject, imperative, NP gerund.

\end{description}

\subsection{Adjective Small Clause with Sentential Subject: Ts0Ax1}
\label{s0Ax1-family}

\begin{description}

\item[Description:]  This tree family is selected by adjectives that take 
sentential subjects.  The sentential subjects can be indicative or infinitive.
Note that these trees are not anchored by adjectives, not verbs.  Most
adjectives that take the Adjective Small Clause tree family (see
Section~\ref{nx0Ax1-family}) takes this family as well\footnote{No great
attempt has been made to go through and decide which adjective actually should
take this family and which should not.}.  Small clauses are explained in much
greater detail in Section~\ref{sm-clause-xtag-analysis}).  This section is
presented here for completeness.  Approximately 3227 adjectives select this
tree family.

\item[Examples:] {\it decadent}, {\it incredible}, {\it uncertain} \\
{\it To eat raspberry chocolate truffle ice cream is decadent.} \\
{\it That Carl could eat a large bowl of it is incredible.} \\
{\it Whether he will actually survive the experience is uncertain.}

\item[Declarative tree:]  See Figure~\ref{s0Ax1-tree}

\begin{figure}[ht]
\centering
\begin{tabular}{c}
\psfig{figure=ps/verb-class-files/alphas0Ax1.ps,height=4.0cm}
\end{tabular}
\caption{Declarative Adjective Small Clause with Sentential Subject Tree:  $\alpha$s0Ax1}
\label{s0Ax1-tree}
\end{figure}

\item[Other available trees:]  Wh-moved subject.

\end{description}



\subsection{Equative {\it BE}: Tnx0BEnx1}
\label{nx0BEnx1-family}

\begin{description}

\item[Description:]  This tree family is selected only by the verb {\it be}.
It is distinguished from the predicative NPs (see Section~\ref{nx0N1-family},
or the section on English Copular (Section~\ref{copula-data} in that two
NP's are equated, and hence interchangeable.  The XTAG analysis for equative
{\it be} is explained in greater detail in
Section~\ref{equative-be-xtag-analysis}.

\item[Examples:] {\it be} \\
{\it That man is my uncle.}

\item[Declarative tree:]  See Figure~\ref{nx0BEnx1-tree}.

\begin{figure}[ht]
\centering
\begin{tabular}{c}
\psfig{figure=ps/verb-class-files/alphanx0BEnx1.ps,height=4.0cm}
\end{tabular}
\caption{Declarative Equitive {\it BE} Tree:  $\alpha$nx0BEnx1}
\label{nx0BEnx1-tree}
\end{figure}

\item[Other available trees:] Inverted-question.

\end{description}




\subsection{NP Small Clauses: Tnx0N1, Tnx0dxN1}
\label{nx0N1-family}

\begin{description}

\item[Description:]  These trees are not anchored by verbs, but by nouns.
Because they are anchored by nouns, there are two different tree families
representing nouns that require determiners and those that occur without them
(see Section~\ref{nouns} for more information on noun phrases in general).
Small clauses are explained in much greater detail in the
Section~\ref{sm-clause-xtag-analysis}).  The section is presented here for
completeness.  Approximately 9915 nouns select the tree family without
determiners, and 9888 nouns select the family with determiners.

\item[Examples:] {\it author}, {\it chair}, {dish} \\
{\it Dania is an author.} \\
{\it That blue, warped-looking thing is a chair.} \\
{\it Those broken pieces were dishes.}

\item[Declarative tree:]  See Figure~\ref{nx0N1-tree}

\begin{figure}[ht]
\centering
\begin{tabular}{cc}
\psfig{figure=ps/verb-class-files/alphanx0N1.ps,height=4.0cm} &
\psfig{figure=ps/verb-class-files/alphanx0dxN1.ps,height=4.0cm} \\
$\alpha$nx0N1 & $\alpha$nx0dxN1
\end{tabular}
\caption{Declarative NP Small Clause Trees}
\label{nx0N1-tree}
\end{figure}

\item[Other available trees:] Wh-moved subject, Wh-moved object, relative
clause on object, imperative, NP gerund.

\end{description}



%%  No nouns select this tree family.  What is up?
%%\subsection{NP Small Clauses with Sentential Complement: Tnx0dxN1s2, Tnx0N1s2}
%%\label{nx0N1s2-family}
%%
%%\begin{description}
%%
%%\item[Description:]
%%
%%\item[Examples:]
%%
%%\item[Declarative tree:]  See Figure~\ref{nx0N1s2-tree}
%%
%%\begin{figure}[ht]
%%\centering
%%\begin{tabular}{cc}
%%\psfig{figure=ps/verb-class-files/betanx0N1s2.ps,height=4.0cm} &
%%psfig{figure=ps/verb-class-files/betanx0dxN1s2.ps,height=4.0cm} \\
%%$\beta$nx0N1s2 &$\beta$nx0dxN1s2 
%%\end{tabular}
%%\caption{Declarative NP Small Clauses with Sentential Complement Tree}
%%\label{nx0N1s2-tree}
%%\end{figure}
%%
%%\item[Other available trees:]  Wh-moved subject, Wh-moved object, relative clause on object, imperative.
%%
%%\end{description}



\subsection{NP with Sentential Complement Small Clause: Tnx0dxN1s1, Tnx0N1s1}
\label{nx0N1s1-family}

\begin{description}

\item[Description:]  This tree family is selected by the small group of nouns
that take sentential complements by themselves (see
Section~\ref{NPA}).  The sentential complements can
be indicative or infinitive, depending on the noun.  Because the trees are
anchored by nouns, there are two different tree families representing nouns
that require determiners and those that occur without them (see
Section~\ref{nouns} for more information on noun phrases in general).  Small
clauses in general are explained in much greater detail in the
Section~\ref{sm-clause-xtag-analysis}).  The section is presented here for
completeness.  Approximately 216 nouns select both the family with and without
determiners.

\item[Examples:] {\it admission}, {\it claim}, {\it vow} \\
{\it The affidavits are admissions that they killed the sheep.} \\
{\it There is always is the claim that they were insane at the time.} \\
{\it This is his vow to fight the charges.}

\item[Declarative tree:]  See Figure~\ref{nx0N1s1-tree}

\begin{figure}[ht]
\centering
\begin{tabular}{cc}
\psfig{figure=ps/verb-class-files/alphanx0N1s1.ps,height=4.0cm} &
\psfig{figure=ps/verb-class-files/alphanx0dxN1s1.ps,height=4.0cm} \\
$\alpha$nx0N1s1 & $\alpha$nx0dxN1s1
\end{tabular}
\caption{Declarative NP with Sentential Complement Small Clause Tree}
\label{nx0N1s1-tree}
\end{figure}

\item[Other available trees:] Wh-moved subject, Wh-moved object, relative
clause on object, imperative, NP gerund.

\end{description}



\subsection{NP Small Clause with Sentential Subject: Ts0dxN1, Ts0N1}
\label{s0N1-family}

\begin{description}

\item[Description:]  This tree family is selected by nouns that take 
sentential subjects.  The sentential subjects can be indicative or infinitive.
Note that these trees are not anchored by nouns, not verbs.  Because they are
anchored by nouns, there are two different tree families representing nouns
that require determiners and those that occur without them (see
Section~\ref{nouns} for more information on noun phrases in general).  Most
nouns that take the NP Small Clause tree family (see
Section~\ref{nx0N1-family}) takes this family as well\footnote{No great attempt
has been made to go through and decide which nouns actually should take this
family and which should not.}.  Small clauses are explained in much greater
detail in Section~\ref{sm-clause-xtag-analysis}).  This section is presented
here for completeness.  Approximately 9888 nouns select both the tree family
with determiners and the tree family without determiners.

\item[Examples:] {\it dilemma}, {\it insanity}, {\it tragedy} \\
{\it Whether to keep the job he hates is a dilemma.} \\
{\it For Bill to invest all of his money in worms is insanity.} \\
{\it That the worms died is a tragedy.}

\item[Declarative tree:]  See Figure~\ref{s0N1-tree}

\begin{figure}[ht]
\centering
\begin{tabular}{cc}
\psfig{figure=ps/verb-class-files/alphas0N1.ps,height=4.0cm} &
\psfig{figure=ps/verb-class-files/alphas0dxN1.ps,height=4.0cm} \\
$\alpha$s0N1 & $\alpha$s0dxN1
\end{tabular}
\caption{Declarative NP Small Clause with Sentential Subject Tree}
\label{s0N1-tree}
\end{figure}

\item[Other available trees:]  Wh-moved subject.

\end{description}




\subsection{PP Small Clause: Tnx0Pnx1}
\label{nx0Pnx1-family}

\begin{description}

\item[Description:]  This family is selected by prepositions that can occur in
small clause constructions (see Section~\ref{sm-clause-xtag-analysis} for more
information on small clause constructions).  The section is presented here for
completeness.  Approximately 39 prepositions select this tree family.

\item[Examples:] {\it around}, {\it in}, {\it underneath} \\
{\it Chris is around the corner.} \\
{\it Trisha is in big trouble.} \\
{\it The dog is underneath the table.}

\item[Declarative tree:]  See Figure~\ref{nx0Pnx1-tree}

\begin{figure}[ht]
\centering
\begin{tabular}{c}
\psfig{figure=ps/verb-class-files/alphanx0Pnx1.ps,height=4.0cm}
\end{tabular}
\caption{Declarative PP Small Clause  Tree:  $\alpha$nx0Pnx1}
\label{nx0Pnx1-tree}
\end{figure}

\item[Other available trees:]  Wh-moved subject, Wh-moved object of PP, 
relative clause on subject, relative clause on object of PP, imperative, NP
gerund.

\end{description}





\subsection{Exhaustive PP Small Clause: Tnx0Px1}
\label{nx0Px1-family}

\begin{description}

\item[Description:] This family is selected by {\bf exhaustive} prepositions
that can occur in small clauses.  Exhaustive prepositions are prepositions that
function as prepositional phrases by themselves.  For more information on small
clause constructions, please see the Section~\ref{sm-clause-xtag-analysis} for
more information on small clause constructions).  The section is presented here
for completeness.  Approximately 8 exhaustive prepositions select this tree
family.

\item[Examples:] {\it abroad}, {\it below}, {\it outside} \\
{\it Dr. Joshi is abroad.} \\
{\it The workers are all below.} \\
{\it Clove is outside.}

\item[Declarative tree:]  See Figure~\ref{nx0Px1-tree}

\begin{figure}[ht]
\centering
\begin{tabular}{c}
\psfig{figure=ps/verb-class-files/alphanx0Px1.ps,height=4.0cm}
\end{tabular}
\caption{Declarative Exhaustive PP Small Clause Tree:  $\alpha$nx0Px1}
\label{nx0Px1-tree}
\end{figure}

\item[Other available trees:] Wh-moved subject, Wh-moved PP, relative clause 
on subject, imperative, NP gerund.

\end{description}


%%  I think that these trees are going away with the new sentential adjunct analysis.
%%\subsection{PP Small Clause with Sentential Complement: Tnx0Pnx1s2}
%%\label{nx0Pnx1s2-family}
%%
%%\begin{description}
%%
%%\item[Description:] Approximately 41 prepositions select this tree family.
%%
%%\item[Examples:] {\it } \\
%%
%%\item[Declarative tree:]  See Figure~\ref{nx0Pnx1s2-tree}
%%
%%\begin{figure}[ht]
%%\centering
%%\begin{tabular}{c}
%%\psfig{figure=ps/verb-class-files/alphanx0Pnx1s2.ps,height=4.0cm}
%%\end{tabular}
%%\caption{Declarative PP Small Clause with Sentential Complement Tree:  $\alpha$nx0Pnx1s2}
%%\label{nx0Pnx1s2-tree}
%%\end{figure}
%%
%%\item[Other available trees:]  Wh-moved subject, Wh-moved object of the PP, 
%%Wh-moved PP with wh+ noun, Wh-moved PP with wh+ preposition {\it where},
%%relative clause on subject, relative clause on object of PP, imperative, NP
%%gerund.
%%
%%\end{description}
%%
%%\subsection{Exhaustive PP Small Clause with Sentential Complement:Tnx0Px1s2}
%%\label{nx0Px1s2-family}
%%
%%\begin{description}
%%
%%\item[Description:]  Approximately 7 exhaustive prepositions select this
%%family. 
%%
%%\item[Examples:]
%%
%%\item[Declarative tree:]  See Figure~\ref{nx0Px1s2-tree}
%%
%%\begin{figure}[ht]
%%\centering
%%\begin{tabular}{c}
%%\psfig{figure=ps/verb-class-files/alphanx0Px1s2.ps,height=4.0cm}
%%\end{tabular}
%%\caption{Declarative Transitive Tree:  $\alpha$nx0Px1s2}
%%\label{nx0Px1s2-tree}
%%\end{figure}
%%
%%\item[Other available trees:]  Wh-moved subject,  Wh-moved PP with wh+ 
%%preposition {\it where}, relative clause on subject, imperative, NP gerund.
%%
%%\end{description}




\subsection{PP Small Clause with Sentential Subject: Ts0Pnx1}
\label{s0Pnx1-family}

\begin{description}

\item[Description:]  This tree family is selected by prepositions that take
sentential subjects.  The sentential subject can be indicative or infinitive.
Small clauses are explained in much greater detail in
Section~\ref{sm-clause-xtag-analysis}).  This section is presented here for
completeness.  Approximately 39 prepositions select this tree family.

\item[Examples:] {\it beyond}, {\it unlike} \\
{\it That Ken could forget to pay the taxes is beyond belief.} \\
{\it To explain how this happened is outside the scope of this discussion.} \\
{\it For Ken to do something right is unlike him.}


\item[Declarative tree:]  See Figure~\ref{s0Pnx1-tree}

\begin{figure}[ht]
\centering
\begin{tabular}{c}
\psfig{figure=ps/verb-class-files/alphas0Pnx1.ps,height=4.0cm}
\end{tabular}
\caption{Declarative PP Small Clause with Sentential Subject Tree:  $\alpha$s0Pnx1}
\label{s0Pnx1-tree}
\end{figure}

\item[Other available trees:] Wh-moved subject, relative clause on object of
the PP.

\end{description}




%%\section{Template}
%%
%%\begin{description}
%%
%%\item[Description:]
%%
%%\item[Declarative tree:]  See Figure~\ref{decl-???-tree}
%%
%%\begin{figure}[ht]
%%\centering
%%\begin{tabular}{c}
%%\psfig{figure=ps/verb-class-files/alpha???.ps,height=4.0cm}
%%\end{tabular}
%%\caption{Declarative Transitive Tree:  $\alpha$nx0Vnx1}
%%\label{decl-????-tree}
%%\end{figure}
%%
%%\item[Other available trees:]
%%
%%\item[Examples:]
%%
%%\end{description}



\chapter{Ergatives}
\label{ergatives}

Verbs in English that are termed ergative display the kind of
alternation shown in the sentences in (\ex{1}) below.

\enumsentence{The sun melted the ice .\\
The ice melted .}

The pattern of ergative pairs as seen in (\ex{0}) is for the object of the
transitive sentence to be the subject of the intransitive sentence.
The literature discussing such pairs is based largely on syntactic
models that involve movement, particularly GB.  Within that framework
two basic approaches are discussed:

\begin{itemize}
\item {\bf Derived Intransitive}\\ The intransitive member of the
ergative pair is derived through processes of movement and deletion from:
\begin{itemize}
\item a transitive D-structure (\cite{Burzio86}); or	
\item transitive lexical structure (\cite{HaleKeyser86,HaleKeyser87})
\end{itemize}

\item {\bf Pure Intransitive}\\ The intransitive member is intransitive at all levels of the
syntax and the lexicon and is not related to the transitive member
syntactically or lexically (\cite{Napoli88}).
\end{itemize}


Obviously, the Derived Intransitive approach's notions of movement in the
lexicon or in the grammar cannot be represented as such in lexicalized tag.
However, distinctions drawn in these arguments can be translated to the FB-LTAG
framework.  In the XTAG grammar the difference between these two approaches is
not a matter of movement but rather a question of tree family membership.  The
relation between sentences represented in terms of movement in other frameworks
is represented in XTAG by membership in the same tree family. Wh-questions and
their indicative counterparts are one example of this.  Adopting the Pure
Intransitive approach suggested by \cite{Napoli88} would mean placing the
intransitive ergatives in a tree family with other intransitive verbs and
separate from the transitive variants of the same verbs.  This would result in
a grammar that represented intransitive ergatives as more closely related to
other intransitives than to their transitive counterparts.  The only hint of
the relation between the intransitive ergatives and the transitive ergatives
would be that ergative verbs would select both tree families. While
this is a workable solution, it is an unattractive one for the English XTAG
grammar because semantic coherence is implicitly associated with tree families
in our analysis of other constructions.  In particular, constancy in thematic
role is represented by constancy in node names across sentence types within a
tree family. For example, if the object of a declarative tree is NP$_{1}$ the
subject of the passive tree(s) in that family will also be NP$_{1}$.

The analysis that has been implemented in the English XTAG grammar is an
adaptation of the Derived Intransitive approach. The ergative verbs select one
family, Tnx0Vnx1, that contains both transitive and intransitive trees.  The
{\bf$<$trans$>$} feature appears on the intransitive ergative trees with the
value {\bf --} and on the transitive trees with the value {\bf +}.  This
creates the two possibilities needed to account for the data.

\begin{itemize}
%%\item {\bf intransitive ergative only.} These verbs have the feature
%%value  {\bf$<$trans$>$=--}, so they can unify only with the
%%intransitive trees within Tnx0Vnx1. This correctly captures the
%%pattern shown in (\ex{1}) and (\ex{2}).
%%
%%\enumsentence{The leaves fell .}
%%\enumsentence{$\ast$The wind fell the leaves .}
%%
\item {\bf intransitive ergative/transitive alternation.}  These verbs
have transitive and intransitive variants as shown in sentences~(\ex{1}) and
(\ex{2}).

\enumsentence{The sun melted the ice cream .}
\enumsentence{The ice cream melted .}


In the English XTAG grammar, verbs with this behavior are left unspecified as
to value for the {\bf$<$trans$>$} feature.  This lack of specification allows
these verbs to anchor either type of tree in the Tnx0Vnx1 tree family because
the unspecified {\bf$<$trans$>$} value of the verb can unify with either {\bf
+} or {\bf --} values in the trees.

\item {\bf transitive only.}  Verbs of this type select only the
transitive trees and do not allow intransitive ergative variants as in
the pattern show in sentences~(\ex{1}) and (\ex{2}).

\enumsentence{Elmo borrowed a book .}
\enumsentence{$\ast$A book borrowed .}

The restriction to selecting only transitive trees is accomplished by
setting the {\bf$<$trans$>$} feature value to {\bf +} for these verbs.
\end{itemize}

\begin{figure}[htb]
\centering
\mbox{}
\psfig{figure=ps/erg-files/alphaEnx1V.ps,height=4.0cm}
\caption{Ergative Tree: $\alpha$Enx1V}
\label{decl-erg-tree}
\label{2;14,1}
\end{figure}

The declarative ergative tree is shown in Figure~\ref{decl-erg-tree} with the
{\bf $<$trans$>$} feature displayed.  Note that the index of the subject NP
indicates that it originated as the object of the verb.
























%talk about ECM and object control verbs
%being treated the same (at least, sharing a tree family).
%format for features? angle brackets or no

\section{Sentential Subjects and Sentential Complements}
\label{scomps-section}

In the LTAG formalism arguments of a lexical item, including
subjects, appear in the initial tree anchored by that lexical item.  A
sentential argument appears as an S node in the appropriate position
within an elementary tree anchored by the lexical item that selects
it. This is the case for sentential complements of verbs, prepositions
and nouns and for sentential subjects. The distribution of
complementizers in English is intertwined with the distribution of
embedded sentences.  A successful analysis of complementizers in
English must handle both the co-occurrence restrictions between
complementizers and various types of clauses, and the distribution of
the clauses themselves, in both subject and complement positions.

\subsection{S or VP complements?}
 
Two comparable grammatical formalisms, Generalized Phrase Structure
Grammar (GPSG) \cite{gazdar85} and Head-driven Phrase Structure
Grammar (HPSG) \cite{pollard87}, have rather different treatments of
sentential complements (S-comps).  They both treat embedded sentences
as VPs with subjects, which generates the correct structures but
misses the generalization that Ss behave similarly in both matrix and
embedded environments, and VPs behave quite differently.  Neither
account has PRO subjects of infinitival clauses-- they have
subjectless VPs instead.  GPSG has a complete complementizer system,
which appears to cover the same range of data as our analysis.  It is
not clear what sort of complementizer analysis could be implemented in
HPSG.

Following standard GB approach, the English LTAG grammar does not
allow VP complements but treats verb-anchored structures without overt
subjects as having PRO subjects. Thus, indicative clauses, infinitives
and gerunds all have a uniform treatment as embedded clauses using the
same trees under this approach. Furthermore, our analysis is able to
preserve the selectional and distributional distinction between Ss and
VPs, in the spirit of GB theories, without having to posit `extra'
empty categories.\footnote{I.e. empty complementizers. We do have PRO
and NP traces in the grammar.} Consider the alternation between {\it
that} and the null complementizer\footnote{Although we will continue
to refer to `null' complementizers, in our analysis this is actually
the absence of a complementizer.}, shown in (\ex{1}) and (\ex{2}).

\enumsentence{He hopes $\emptyset$ Muriel wins}
\enumsentence{He hopes that Muriel wins.}

 In GB both {\it Muriel wins} in (\ex{-1}) and {\it that Muriel wins}
in (\ex{0}) are CPs even though there is no overt complementizer to
head the phrase in (\ex{-1}).  Our grammar does not distinguish by
category label between the the phrases that would be labeled in GB as
{\it IP\/} and {\it CP\/}.  We label both of these phrases {\it S\/}.
The difference between these two levels is the presence or absence of
the complementizer (or extracted {\it WH\/} constituent), and is
represented in our system as a difference in feature values (here, of
the {\bf $<$comp$>$} feature), and the presence of the additional structure
contributed by the complementizer or extracted constituent.  This
illustrates an important distinction in LTAG, that between features
and node labels.  Because we have a sophisticated feature system, we
are able to make fine-grained distinctions between nodes with the same
label which in another system might have to be realized by
distinguishing node labels.
 
\subsection{Complementizers and Embedded Clauses in English:  The
Data}
\label{data}

Verbs selecting sentential complements (or subjects) place
restrictions on their complements, in particular, on the form of the
embedded verb phrase.\footnote{Other considerations, such as the
relationship between the tense/aspect of the matrix clause and the
tense/aspect of a complement clause are also important but are not
currently addressed in the English LTAG.}  Furthermore,
complementizers are constrained to appear with certain types of
clauses, again, based primarily on the form of the embedded VP.  For
example {\it think\/} selects both indicative and infinitival
complements. With an indicative complement, it may only have {\it
that\/} or null as possible complementizers; with an infinitival
complement, it may have either {\it if\/} or {\it whether\/}, but
never {\it that\/}.  The possible combinations of complementizers and
clause types is summarized in Table \ref{facts}.

\begin{table}[h]
\begin{tabular}{l|lllllll}
Complementizer:&&that&whether&if&for&null\\
\hline
Clause type&&&&&&&\\
\hline
indicative&subject&Yes&Yes&No&No&No\\
&complement&Yes&Yes&Yes&No&Yes\\
\hline
infinitive&subject&No&Yes&No&Yes&Yes\\
&complement&No&Yes&No&Yes&Yes\\
\hline
subjunctive&subject&Yes&No&No&No&No\\
&complement&Yes&No&No&No&Yes\\
\hline
gerundive\footnotemark\ &complement&No&No&No&No&Yes\\
\hline
base & complement & No & No & No & No & Yes \\
\hline
small clause & complement & No & No & No & No & Yes \\
\hline
\end{tabular}
\vspace{.2in}
\caption{Summary of Complementizer and Clause Combinations}
\label{facts}
\end{table}
\footnotetext{Most gerundive phrases are treated as NPs.  In
fact, all gerundive subjects are treated as NPs, and the only gerundive
complements which receive a sentential parse are those for which there
is no corresponding NP parse.  This was done to reduce duplication of
parses. See Chapter \ref{gerunds-chapter} for more discussion of gerunds.}

As can be seen in Table \ref{facts}, sentential subjects differ from
sentential complements in requiring the complementizer {\it that\/}
for all indicative and subjunctive clauses.  In sentential complements
{\it that\/} often varies freely with a null complementizer, as
illustrated in (\ex{1})-(\ex{6}).

\enumsentence{John hopes that Mary wins.}
\enumsentence{John hopes Mary wins.}
\enumsentence{Mary thinks that John is a liar.}
\enumsentence{Mary thinks John is a liar.}
\enumsentence{That John won so easily annoyed Max.}
\enumsentence{$\ast$ John won so easily annoyed Max.}

Another fact which must be account for in the  analysis is that in
infinitival clauses, the complementizer {\it for} must appear with an
overt subject NP, whereas a complementizer-less infinitival clause never
has an overt subject, as shown in (\ex{1})-(\ex{4}). (See Section
\ref{for-complementizer} for more discussion of the case assignment
issues relating to this construction.)

\enumsentence{To lose would be awful}
\enumsentence{For Mary to lose would be awful}
\enumsentence{$\ast$ For to lose would be awful}
\enumsentence{$\ast$ Mary to lose would be awful}

In addition, some verbs select {\it wh+} complements (either questions or
clauses with {\it whether} or {\it if}) \cite{grimshaw90}:

\enumsentence{Mary wondered who left}
\enumsentence{Mary wondered if John left}
\enumsentence{Mary wondered whether to leave.}
\enumsentence{Mary wondered whether John left}
\enumsentence{$\ast$Mary thought who left.}
\enumsentence{$\ast$Mary thought if John left.}
\enumsentence{$\ast$Mary thought whether to leave.}
\enumsentence{$\ast$Mary thought whether John left.}

\subsection{Features Required}
\label{s-features}

As we have seen above, clauses may be {\it wh+} or {\it wh--}, may
have one of several complementizers or no complementizer, and can be
of various clause types.  The TAG analysis uses three features to
capture these possibilities: {\bf $<$comp$>$} for the variation in
complementizers, {\bf$<$wh$>$} for the question vs.  non-question
alternation and {\bf $<$mode$>$} for clause types\footnote{{\bf
$<$Mode$>$} actually conflates several types of information, in
particular verb form and mood.}.  In addition to these three features,
the {\bf $<$assign-comp$>$} feature represents complementizer
requirements of the embedded verb.  More detailed discussion of the
{\bf $<$assign-comp$>$} feature appears below in the discussions of
sentential subjects and of infinitives.  The four features and their
possible values are shown in Table (\ref{feat}).


\begin{table}[t]
\centering
\begin{tabular}{l|c}
Feature&Values\\
\hline
{\bf $<$comp$>$}&that, if, whether, for, rel, nil\\
\hline
{\bf$<$mode$>$}&ind, inf, subjnt, ger, base, ppart, nom/prep\\
\hline
{\bf$<$assign-comp$>$}&that, if, whether, for, rel, ind\underline{~}nil, inf\underline{~}nil\\
\hline
{\bf$<$wh$>$}&$+,-$\\
\hline
\end{tabular}
\caption{Summary of Relevant Features}
\label{feat}
\end{table}


\subsection{Distribution of Complementizers}
Like other non-arguments, complementizers anchor an auxiliary tree
(shown in Figure 1) and adjoin to elementary clausal trees.  The
auxiliary tree for complementizers is the only alternative to having a
complementizer position `built into' every sentential tree.  The
latter choice would mean having an empty complementizer substitute for
every matrix sentence and complementizerless embedded sentence to fill
the substitution node.  Our choice follows the LTAG principle that
initial trees consist only of the arguments of the anchor -- the S
tree does not contain a slot for a complementizer, and the $\beta$COMP
tree has only one argument, an S with particular features determined by
the complementizer.  Complementizers select the type of clause to
which they adjoin through constraints on the {\bf $<$mode$>$} feature of the
S foot node in the $\beta$COMPs tree shown in Figure
(\ref{comp-tree}).These features also pass up to the root of
$\beta$COMP, so that they are `visible' to the tree where the embedded
sentence adjoins/substitutes.

\begin{figure}[h]
\centering
\hspace{0.0in}
\psfig{figure=/mnt/linc/extra/xtag/work/doc/tech-rept/ps/sent-comps-subjs-files/betaCOMPs.ps,height=3.0in}
\caption{Tree $\beta$COMPs, selected by all complementizers}
\label{comp-tree}
\end{figure}

The grammar handles the following complementizers: {\it that\/}, {\it
whether\/}, {\it if\/}, {\it for\/}, and no complementizer, and the
clause types: indicative, infinitival, gerundive, past participial,
subjunctive and small clause ({\bf $<$nom/prep$>$}).  The {\bf
$<$comp$>$} feature in a clausal tree reflects the value of the
complementizer if one has adjoined to the clause. 

The {\bf $<$comp$>$} and {\bf $<$wh$>$} features receive their root
node values from the particular complementizer which anchors the tree.
The $\beta$COMPs tree adjoins to an S node with the feature {\bf
$<$comp=nil$>$}; this feature indicates that the tree does not already
\underline{have} a complementizer adjoined to it.\footnote{ Because root Ss
cannot have complementizers, the parser checks that the roots S has
{\bf $<$comp=nil$>$} at the end of the derivation, when the S is also
checked for a tensed verb.} We ensure that there are no stacked
complementizers by requiring the foot node of $\beta$COMPs to have {\bf
$<$comp = nil$>$}, as well as using the {\bf $<$sub-conj = nil$>$} feature to
prevent complementizers from adjoining above subordinating conjunctions.

\subsection{Case assignment, {\it for\/} and the two {\it to\/}'s}
\label{for-complementizer}

The {\bf $<$assign-comp$>$} feature is used to represent the
requirements of particular types of clauses for particular
complementizers.  So while the {\bf $<$comp$>$} feature represents
constraints originating from the VP dominating the clause, the {\bf
$<$assign-comp$>$} feature represents constraints originating from the
highest VP in the clause. {\bf $<$Assign-comp$>$} is used to control the
appearance of subjects in infinitival clauses,  to
ensure the correct distribution of complementizers in sentential
subjects, and to block ``that-trace'' violations.

Examples (\ex{2}), (\ex{3}) and (\ex{4}) show that an accusative
case subject is obligatory in an infinitive clause if the
complementizer {\it for\/} is present. The infinitive clauses in both
(\ex{0}) and (\ex{1}) are analyzed in the English LTAG grammar as
having PRO subjects.  The apparent subject of {\it to win\/} in
(\ex{-4}) is taken to be an object of the verb rather than the subject
of the infinitive clause. 

\enumsentence{John wants her to win.}
\enumsentence{John wants to win.}
\enumsentence{John wants for her to win.}
\enumsentence{*John wants for she to win.}
\enumsentence{*John wants for to win.}
 
The {\it for-to\/} construction is particularly illustrative of the
difficulties and benefits faced in using a lexicalized grammar.  It is
commonly accepted that {\it for\/} is behaving as a case-assigning
complementizer in this construction, assigning accusative case to the
`subject' of the clause since the infinitival verb does not assign
case to its subject position.  However, in our featurized grammar, the
absence of a feature licenses anything, so we must have overt null
case assigned by infinitives to ensure the correct distribution of PRO
subjects. (See Section \ref{case-assignment} for more discussion of
case assignment.)  This null case assignment clashes with accusative
case assignment if we simply add {\it for\/} as a standard
complementizer, since NPs (including PRO) are drawn from the lexicon
already marked for case.  Thus, we must use the {\bf
$<$assign-comp$>$} feature, to pass information about the verb up to
the root of the embedded sentence.  To capture these facts, two
infinitive {\it to}'s are posited. One infinitive {\it to\/} has {\bf
$<$assign-case=none$>$} which forces a PRO subject, and {\bf
$<$assign-comp=inf\_nil$>$} which prevents {\it for\/} from
adjoining. The other infinitive {\it to\/} has no value at all for
{\bf $<$assign-case$>$} and has {\bf $<$assign-comp=for$>$}, so that
it can only occur with the complementizer {\it for\/}. In those
instances {\it for} supplies the {\bf $<$assign-case$>$} value and
assigns accusative case to the overt subject.
 
\subsection{Sentential Complement Trees}
\subsubsection{Sentential Complements of Verbs}

{\sc Tree families}: Tnx0Vs1, Tnx0Vnx1s2, TItVnx1s2, TItVpnx1s2, TItVad1s2,
Tnx0Ax1s2, Tnx0dxN1s1, Tnx0N1s1, Tnx0Pnx1s2, Tnx0Px1s2. 


Verbs that select sentential complements select the {\bf $<$mode$>$}
and {\bf $<$comp$>$} values for those complements. Since with very few
exceptions\footnote{For example, long distance extraction is not
possible from the S complement in it-clefts.} long distance extraction
is possible from sentential complements, the S complement nodes are
adjunction nodes. Figure \ref{think} shows the declarative tree
$\beta$nx0Vs1, anchored by {\it think}.  

\begin{figure}[h]
\centering
\hspace{0.0in}
\psfig{figure=/mnt/linc/extra/xtag/work/doc/tech-rept/ps/sent-comps-subjs-files/think.ps,height=2.0in}
\caption{$\beta$nx0Vs1}
\label{think}
\end{figure}

The need for an adjunction node rather than a substitution node  at
S$_{1}$ may not be obvious until one considers the derivation of
sentences with long distance extractions.  For example, the
declarative in (\ex{1}) is derived by adjoining the tree (b) in Figure
\ref{aard-emu} to the S$_{1}$ node of tree (a).  Since there are no
bottom features on S$_{1}$, the same final result could have been
achieved with a substitution node at S$_{1}$.

\enumsentence{The emu thinks that the aardvark smells terrible.}

\begin{figure}[t]
\begin{tabular}{cc}
\psfig{figure=/mnt/linc/extra/xtag/work/doc/tech-rept/ps/sent-comps-subjs-files/aard-smells.ps,height=2.5in}&\hspace{0.3in}
\psfig{figure=/mnt/linc/extra/xtag/work/doc/tech-rept/ps/sent-comps-subjs-files/emu-thinks.ps,height=2.5in}\\
(a)&(b)\\
\end{tabular}
\caption{Trees for {\it The emu thinks that the aardvark smells terrible}}  
\label{aard-emu}
\end{figure}

However, adjunction is crucial in deriving sentences with
long distance extraction, such as (\ex{1}).  

\enumsentence{Who does the emu think smells terrible?}
\enumsentence{Who did the elephant think the panda heard the emu say
smells terrible?} 

This example is derived from the trees for {\it who smells terrible?}
shown in figure \ref{who-smells} and for {\it the emu thinks} S shown
in figure \ref{aard-emu}(b), by adjoining the latter at the S$_r$ node of
the former. (See Section \ref{auxiliaries} for discussion of do-support.) This process is recursive allowing sentences like
\ex{0}. Such a representation has been shown by Kroch and Joshi
(1985)\nocite{kj85} to be well-suited for describing unbounded dependencies.

\begin{figure}[t]
\centering
\hspace{0.0in}
\psfig{figure=/mnt/linc/extra/xtag/work/doc/tech-rept/ps/sent-comps-subjs-files/who-smells.ps,height=3.0in}
\caption{Tree for {\it Who smells terrible?}}
\label{who-smells}
\end{figure}

In English, a complementizer may not appear on a complement with an extracted
subject (the ``that-trace'' configuration). This phenomenon
is illustrated in \ex{1}-\ex{3}:

\enumsentence{Which animal did the giraffe say that he likes?}
\enumsentence{*Which animal did the giraffe say that likes him?}
\enumsentence{Which animal did the giraffe say likes him?}

These sentences are derived in TAG by
adjoining the tree for {\it did the giraffe say S} at the S$_r$ node
of the tree for either {\it which animal likes him} (to yield \ex{0})
or {\it which animal he likes} (to yield \ex{-2}).  That-trace
violations are blocked by the presence of the feature {\bf $<$assign-comp
= inf\underline{~}nil/ind\underline{~}nil$>$} feature on the bottom of
the S$_r$ node of trees with extracted subjects, i.e. those used in
sentences such as \ex{0} and \ex{-1}.  This blocks (or ``filters'') any
other values of {\bf $<$assign-comp$>$} projected by the verb, and ensures
that no complementizer is able to adjoin at this node.
Complementizers may adjoin normally to object extraction trees such as
those used in \ex{-2}.

In the case of indirect questions, subjacency follows from the
principle that a given tree cannot contain more than one
wh- element. Extraction out of an indirect question is ruled out
because a sentence like:

\enumsentence{$\ast$ Who$_{i}$ do you wonder who loves e$_{i}$ ?}

\noindent would have to be derived from the adjunction of {\it do you
wonder} into {\it who$_{i}$ who loves e$_{i}$}, which is an ill-formed
elementary tree.\footnote{This does not mean that elementary trees
with more than one gap should be ruled out across the grammar. Such
trees might be required for dealing with parasitic gaps or gaps in
coordinated structures.}

\subsection{Sentential Subjects}
\label{sent-subjs}

{\sc Tree families}: Ts0Vnx1, Ts0Ax1, Ts0dxN1, Ts0N1, Ts0Pnx1.

Verbs that select sentential subjects anchor trees that have an S node
in the subject position rather that an NP node.  Since extraction is
not possible from sentential subjects, they are implemented as
substitution nodes in the English LTAG grammar.  Restrictions on
sentential subjects, such as the required {\it that} complementizer for
indicatives, are enforced by feature values specified on the S
substitution node in the elementary tree.  

Sentential subjects behave essentially like sentential complements,
with a few exceptions.  In general, all verbs which license sentential
subjects license the same set of clause types. Thus, unlike sentential
complement verbs which select particular complementizers and clause
types, the matrix verbs licensing sentential subjects merely license
the S argument. Information about the complementizer or embedded verb
is located in the tree features, rather than in the features of each
verb selecting that tree.  Thus, all sentential subject trees have the
same {\bf $<$mode$>$}, {\bf $<$comp$>$} and {\bf $<$assign-comp$>$} values shown in Figure
\ref{comparison}(a).

\begin{figure}[h]
\begin{tabular}{cc}
\psfig{figure=/mnt/linc/extra/xtag/work/doc/tech-rept/ps/sent-comps-subjs-files/perplexes-feats.ps,height=2.5in}&\hspace{0.3in}
\psfig{figure=/mnt/linc/extra/xtag/work/doc/tech-rept/ps/sent-comps-subjs-files/think-feats.ps,height=2.5in} 
\end{tabular}
\caption{Comparison of {\bf $<$assign-comp$>$} values for sentential
subjects (a) and sentential complements (b).}
\label{comparison}
\end{figure}

The major difference in clause types licensed by S-subjs and S-comps
is that indicative S-subjs obligatorily have a complementizer (see
examples in section \ref{data}). The {\bf $<$assign-comp$>$} feature is used
here to license a null complementizer for infinitival but not
indicative clauses. {\bf $<$Assign-comp$>$} has the same possible values as
{\bf $<$comp$>$}, with the exception that the {\bf $<$nil$>$} value is `split'
into {\bf $<$ind-nil$>$} and {\bf $<$inf-nil$>$}.  This difference in feature
values is illustrated in Figure \ref{comparison}.
%This allows us to specify precisely which environments license null
%complementizers. 
%Intuitively, {\bf $<$assign-comp$>$} passes information about what
%complementizers are licensed from the verb \underline{up} to its root,
%where it is `visible' to the extra-clausal environment.  

Another minor difference is that {\it whether\/} but not {\it if\/} is
grammatical with S-subjs (although some speakers also find {\it if\/}
as a complementizer only marginally grammatical in S-comps). Thus,
{\bf $<$if$>$} is not among the {\bf $<$comp$>$} values allowed in
S-subjs. The final difference from S-comps is that there are no
S-subjs with {\bf $<$mode = ger$>$}. As noted in footnote 4, gerundive
complements are only allowed when there is no corresponding NP
parse. In the case of gerundive S-subjs, there is always an NP parse
available.

\subsection{Nouns and Prepositions taking Sentential Complements}
\label{NPA}

{\sc Trees}: $\alpha$NXNs, $\alpha$NXdxNs, $\alpha$PXPs, $\beta$vxPs,
$\beta$Pss, $\beta$nxPx.

\begin{figure}[h]
\begin{tabular}{cc}
\psfig{figure=/mnt/linc/extra/xtag/work/doc/tech-rept/ps/sent-comps-subjs-files/with.ps,height=2.0in}&\hspace{0.3in}
\psfig{figure=/mnt/linc/extra/xtag/work/doc/tech-rept/ps/sent-comps-subjs-files/boast.ps,height=2.5in} 
\end{tabular}
\caption{Sample trees for preposition (a) and noun (b) taking
sentential complements}
\label{nounprep}
\end{figure}

Prepositions and nouns can also select sentential complements, using
the trees listed above.  These trees use the {\bf $<$mode$>$} and {\bf
$<$comp$>$} features as shown in Figure \ref{nounprep}.  For example, the noun {\it
boast} takes only indicative complements with {\it that}, while the
preposition {\it with} takes indicative or small clause complements.

%%Comparative adjs also take s-comps, e.g. the boys easiest to teach.
%%See Quirk, section 7.20 and others.



\chapter{The English Copula, Raising Verbs, and Small Clauses}
\label{small-clauses}

The English copula, raising verbs, and small clauses are all handled in XTAG by
a common analysis based on sentential clauses headed by non-verbal elements.
Since there are a number of different analyses in the literature of how these
phenomena are related (or not), we will present first the data for all three
phenomena, then various analyses from the literature, finishing with the
analysis used in the English XTAG grammar.\footnote{This chapter is strongly
based on \cite{heycock91}.  Sections \ref{sm-clause-data} and
\ref{sm-clause-other-analyses} are greatly condensed from her paper, while the 
description of the XTAG analysis in section \ref{sm-clause-xtag-analysis} is an
updated and expanded version.}


\section{Usages of the copula, raising verbs, and small clauses}
\label{sm-clause-data}

\subsection{Copula}
\label{copula-data}

The verb {\it be} as used in sentences ({\ex{1}})-({\ex{3}}) is often
referred to as the \xtagdef{copula}.  It can be followed by a noun, adjective, or
prepositional phrase.

\enumsentence{Carl is a jerk .}
\enumsentence{Carl is upset .}
\enumsentence{Carl is in a foul mood .}

Although the copula may look like a main verb at first glance, its syntactic
behavior follows the auxiliary verbs rather than main verbs.  In particular,

\begin{itemize}
\item Copula {\it be} inverts with the subject.
\enumsentence{is Beth writing her dissertation ?\\
		is Beth upset ?\\
		$\ast$wrote Beth her dissertation ?}

\item Copula {\it be} occurs to the left of the negative marker {\it
not}.
\enumsentence{Beth is not writing her dissertation .\\
		Beth is not upset .\\
		$\ast$Beth wrote not her dissertation .}

\item Copula {\it be} can contract with the negative marker {\it not}.
\enumsentence{Beth isn't writing her dissertation .\\
		Beth isn't upset .\\
		$\ast$Beth wroten't her dissertation .}

\item Copula {\it be} can contract with pronominal subjects.
\enumsentence{She's writing her dissertation .\\
		She's upset .\\
		$\ast$She'ote her dissertation .}

\item Copula {\it be} occurs to the left of adverbs in the unmarked order.
\enumsentence{Beth is often writing her dissertation .\\
		Beth is often upset .\\
		$\ast$Beth wrote often her dissertation .}
\end{itemize}

Unlike all the other auxiliaries, however, copula {\it be} is not followed by
a verbal category (by definition) and therefore must be the rightmost verb.  In
this respect, it is like a main verb.

The semantic behavior of the copula is also unlike main verbs.  In particular,
any semantic restrictions or roles placed on the subject come from the
complement phrase (NP, AP, PP) rather than from the verb, as illustrated in
sentences ({\ex{1}}) and ({\ex{2}}).  Because the complement phrases predicate
over the subject, these types of sentences are often called
\xtagdef{predicative} sentences.

\enumsentence{The bartender was garrulous .}
\enumsentence{?The cliff was garrulous .}


\subsection{Raising Verbs}
\label{raising-verbs}

Raising verbs are the class of verbs that share with the copula the property
that the complement, rather than the verb, places semantic constraints on
the subject.  

\enumsentence{Carl seems a jerk .\\
		Carl seems upset .\\
		Carl seems in a foul mood .}

\enumsentence{Carl appears a jerk .\\
		Carl appears upset .\\
		Carl appears in a foul mood .}

The raising verbs are similar to auxiliaries in that they order with other
verbs, but they are unique in that they can appear to the left of the
infinitive, as seen in the sentences in ({\ex{1}}).  They cannot, however,
invert or contract like other auxiliaries ({\ex{2}}), and they appear to the
right of adverbs ({\ex{3}}).

\enumsentence{Carl seems to be a jerk .\\
		Carl seems to be upset .\\
		Carl seems to be in a foul mood .}

\enumsentence{$\ast$seems Carl to be a jerk ?\\
		$\ast$Carl seemn't to be upset .\\
		$\ast$Carl`ems to be in a foul mood .}

\enumsentence{Carl often seems to be upset .\\
		$\ast$Carl seems often to be upset .}


\subsection{Small Clauses}

One way of describing small clauses is as predicative sentences without the
copula.  Since matrix clauses require tense, these clausal structures appear
only as embedded sentences.  They occur as complements of certain verbs, each
of which may allow certain types of small clauses but not others, depending on its
lexical idiosyncrasies.

\enumsentence{I consider [Carl a jerk] .\\
		I consider [Carl upset] .\\
		?I consider [Carl in a foul mood] .}

\enumsentence{I prefer [Carl in a foul mood] .\\
		??I prefer [Carl upset] .}


\subsection{Raising Adjectives}
\label{raising-adjs}

Raising adjectives are the class of adjectives that 
share with the copula and raising verbs the property
that the complement, rather than the verb, places semantic constraints on
the subject.  

They appear with the copula in a matrix clause, as in ({\ex{1}}).  However,
in other cases, such as that of small clauses ({\ex{2}}), they do not
have to appear with the copula.

\enumsentence{Carl is likely to be a jerk .\\
		Carl is likely to be upset .\\
		Carl is likely to be in a foul mood .\\
                Carl is likely to perjure himself .}

\enumsentence{I consider Carl likely to perjure himself .}



\section{Various Analyses}
\label{sm-clause-other-analyses}

\subsection{Main Verb Raising to INFL + Small Clause}

In \cite{pollack89} the copula is generated as the head of a VP, like any main
verb such as {\it sing} or {\it buy}. Unlike all other main verbs\footnote{with
the exception of {\it have} in British English. See
footnote~\ref{have-footnote} in Chapter~\ref{auxiliaries}.}, however, {\it be}
moves out of the VP and into Infl in a tensed sentence.  This analysis aims to
account for the behavior of {\it be} as an auxiliary in terms of inversion,
negative placement and adverb placement, while retaining a sentential structure
in which {\it be} heads the main VP at D-Structure and can thus be the only
verb in the clause.

Pollock claims that the predicative phrase is not an argument of {\it be},
which instead he assumes to take a small clause complement, consisting of a
node dominating an NP and a predicative AP, NP or PP. The subject NP of the
small clause then raises to become the subject of the sentence.  This accounts
for the failure of the copula to impose any selectional restrictions on the
subject.  Raising verbs such as {\it seem} and {\it appear}, presumably, take the
same type of small clause complement.

\subsection{Auxiliary + Null Copula}
\label{la}

In \cite{lapointe80} the copula is treated as an auxiliary verb that takes as its
complement a VP headed by a passive verb, a present participle, or a null verb
(the true copula). This verb may then take AP, NP or PP complements.  The
author points out that there are many languages that have been analyzed as
having a null copula, but that English has the peculiarity that its
null copula requires the co-presence of the auxiliary {\it be}.

\subsection{Auxiliary + Predicative Phrase}
\label{gpsg}

In GPSG (\cite{gazdar85}, \cite{sag85}) the copula is treated as an auxiliary
verb that takes an X$^{2}$ category with a + value for the head feature
[PRD] (predicative). AP, NP, PP and VP can all be [+PRD], but a
Feature Co-occurrence Restriction guarantees that a [+PRD] VP will be
headed by a verb that is either passive or a present participle.

GPSG follows \cite{chomsky70} in adopting the binary valued features [V] and
[N] for decomposing the verb, noun, adjective and preposition categories.  In
that analysis, verbs are [+V,$-$N], nouns are [$-$V,+N], adjectives [+V,+N] and
prepositions [$-$V,$-$N].  NP and AP predicative complements generally pattern
together; a fact that can be stated economically using this category
decomposition.  In neither \cite{sag85} nor \cite{chomsky70} is there any
discussion of how to handle the complete range of complements to a verb like
{\it seem}, which takes AP, NP and PP complements, as well as infinitives.  The
solution would appear to be to associate the verb with two sets of rules for
small clauses, leaving aside the use of the verb with an expletive subject and
sentential complement.

\subsection{Auxiliary + Small Clause}

\label{mo}
In \cite{moro90} the copula is treated as a special functional category - a
lexicalization of tense, which is considered to head its own projection. It
takes as a complement the projection of another functional category, Agr
(agreement). This projection corresponds roughly to a small clause, and is
considered to be the domain within which predication takes place.  An NP must
then raise out of this projection to become the subject of the sentence: it may
be the subject of the AgrP, or, if the predicate of the AgrP is an NP, this may
raise instead.  In addition to occurring as the complement of {\it be}, AgrP is
selected by certain verbs such as {\it consider}. It follows from this analysis
that when the complement to {\it consider} is a simple AgrP, it will always
consist of a subject followed by a predicate, whereas if the complement
contains the verb {\it be}, the predicate of the AgrP may raise to the left of
{\it be}, leaving the subject of the AgrP to the right.

\enumsentence{John$_{i}$ is [$_{AgrP}$ $t_{i}$ the culprit ] .}
\enumsentence{The culprit$_{i}$ is [$_{AgrP}$ John $t_{i}$ ] .}
\enumsentence{I consider [$_{AgrP}$ John the culprit] .}
\enumsentence{I consider [John$_{i}$ to be [$_{AgrP}$ $t_{i}$ the culprit ]] .}
\enumsentence{I consider [the culprit$_{i}$ to be [$_{AgrP}$ John $t_{i}$ ]] .}

Moro does not discuss a number of aspects of his analysis, including the
nature of Agr and the implied existence of sentences without VP's. 

\section{XTAG analysis}
\label{sm-clause-xtag-analysis}

\begin{figure}[htbp]
\centering
\begin{tabular}{ccccc}
{\psfig{figure=ps/sm-clause-files/alphanx0N1.ps,height=1.3in}} &
\hspace{0.5in} &
{\psfig{figure=ps/sm-clause-files/alphanx0Ax1.ps,height=1.4in}} &
\hspace{0.5in} &
{\psfig{figure=ps/sm-clause-files/alphanx0Pnx1.ps,height=1.4in}} \\
(a)&&(b)&&(c)\\
\end{tabular}
\caption{Predicative trees: $\alpha$nx0N1 (a), $\alpha$nx0Ax1 (b) and $\alpha$nx0Pnx1 (c)}
\label{predicative-trees}
\label{1;1,7}
\label{1;1,9}
\end{figure}

The XTAG grammar provides a uniform analysis for the copula, raising verbs and
small clauses by treating the maximal projections of lexical items that can be
predicated as predicative clauses, rather than simply noun, adjective and
prepositional phrases.  The copula adjoins in for matrix clauses, as do the
raising verbs.  Certain other verbs (such as {\it consider}) can take the
predicative clause as a complement, without the adjunction of the copula, to
form the embedded small clause.

The structure of a predicative clause, then, is roughly as seen in
({\ex{1}})-({\ex{3}}) for NP's, AP's and PP's.  The XTAG trees corresponding
to these structures\footnote{There are actually two other predicative trees in
the XTAG grammar.  Another predicative noun phrase tree is needed for noun
phrases without determiners, as in the sentence {\it They are firemen}, and
another prepositional phrase tree is needed for exhaustive prepositional
phrases, such as {\it The workers are below}.} are shown in
Figures~\ref{predicative-trees}(a),
\ref{predicative-trees}(b), and \ref{predicative-trees}(c), 
respectively.

\enumsentence{[$_{S}$ NP [$_{VP}$  N \ldots ]]}
\enumsentence{[$_{S}$ NP [$_{VP}$  A \ldots ]]}
\enumsentence{[$_{S}$ NP [$_{VP}$  P \ldots ]]}



The copula {\it be} and raising verbs all get the basic auxiliary tree as
explained in the section on auxiliary verbs (section \ref{aux-non-inverted}).
Unlike the raising verbs, the copula also selects the inverted auxiliary tree
set.  Figure~\ref{Vvx-with-nomprep} shows the basic auxiliary tree anchored by
the copula {\it be}.  The {\bf $<$mode$>$} feature is used to distinguish the
predicative constructions so that only the copula and raising verbs adjoin onto
the predicative trees.  

\begin{figure}[htb]
\centering
\begin{tabular}{c}
{\psfig{figure=ps/sm-clause-files/betaVvx_is-with-features.ps,height=5.7in}} \\
\end{tabular}
\caption{Copula auxiliary tree: $\beta$Vvx}
\label{Vvx-with-nomprep}
\end{figure}

There are two possible values of {\bf $<$mode$>$} that correspond to the
predicative trees, {\bf nom} and {\bf prep}.  They correspond to a modified
version of the four-valued [N,V] feature described in section \ref{gpsg}.  The
{\bf nom} value corresponds to [N+], selecting the NP and AP predicative
clauses.  As mentioned earlier, they often pattern together with respect to
constructions using predicative clauses.  The remaining prepositional phrase
predicative clauses, then, correspond to the {\bf prep} mode.

Figure~\ref{upset-with-features} shows the predicative adjective tree from
Figure~\ref{predicative-trees}(b) now anchored by {\it upset} and with the
features visible.  As mentioned, {\bf $<$mode$>$=nom} on the VP node prevents
auxiliaries other than the copula or raising verbs from adjoining into this
tree.  In addition, it prevents the predicative tree from occurring as a matrix
clause.  Since all matrix clauses in XTAG must be mode indicative ({\bf ind})
or imperative ({\bf imp}), a tree with {\bf $<$mode$>$=nom} or {\bf
$<$mode$>$=prep} must have an auxiliary verb (the copula or a raising verb)
adjoin in to make it {\bf $<$mode$>$=ind}.


\begin{figure}[htb]
\centering
\begin{tabular}{c}
{\psfig{figure=ps/sm-clause-files/alphanx0Ax1_upset-with-features.ps,height=6.3in}} \\
\end{tabular}
\caption{Predicative AP tree with features: $\alpha$nx0Ax1}
\label{upset-with-features}
\label{1;1,4}
\end{figure}

The distribution of small clauses as embedded complements to some verbs is also
managed through the mode feature.  Verbs such as {\it consider} and {\it
prefer} select trees that take a sentential complement, and then restrict that
complement to be {\bf $<$mode$>$=nom} and/or {\bf $<$mode$>$=prep},
depending on the lexical idiosyncrasies of that particular verb.  Many verbs
that don't take small clause complements do take sentential complements that
are {\bf $<$mode$>$=ind}, which includes small clauses with the copula
already adjoined.  Hence, as seen in sentence sets ({\ex{1}})-({\ex{3}}),
{\it consider} takes only small clause complements, {\it prefer} takes both
{\bf prep} (but not {\bf nom}) small clauses and indicative clauses, while {\it
feel} takes only indicative clauses.

\enumsentence{She considers Carl a jerk .\\
		?She considers Carl in a foul mood .\\
		$\ast$She considers that Carl is a jerk .}

\enumsentence{$\ast$She prefers Carl a jerk .\\
		She prefers Carl in a foul mood .\\
		She prefers that Carl is a jerk .}

\enumsentence{$\ast$She feels Carl a jerk .\\
		$\ast$She feels Carl in a foul mood .\\
		She feels that Carl is a jerk .}

\noindent
Figure \ref{consider-with-features} shows the tree anchored by {\it consider}
that takes the predicative small clauses.

\begin{figure}[htb]
\centering
\begin{tabular}{c}
{\psfig{figure=ps/sm-clause-files/betanx0Vs1_consider-with-features.ps,height=2.3in}} \\
\end{tabular}
\caption{{\it Consider} tree for embedded small clauses}
\label{consider-with-features}
\end{figure}

Raising verbs such as {\it seems} work essentially the same as the
auxiliaries, in that they also select the basic auxiliary tree, as in
Figure~\ref{Vvx-with-nomprep}.  The only difference is that 
the value of {\bf $<$mode$>$} 
on the VP foot node might be different, depending on what types of
complements the raising verb takes.  Also, two of the raising verbs take
an additional tree, $\beta$Vpxvx, shown in Figure~\ref{Vpxvx}, which
allows for an experiencer argument, as in {\it John seems to me
to be happy}.  

\begin{figure}[htb]
\centering
\begin{tabular}{c}
{\psfig{figure=ps/sm-clause-files/betaVpxvx.ps,height=2.0in}} \\
\end{tabular}
\caption{Raising verb with experiencer tree: $\beta$Vpxvx}
\label{Vpxvx}
\end{figure}


Raising adjectives, such as {\it likely}, take the tree shown in
Figure~\ref{Vvx-adj}.  This tree combines aspects of the auxiliary
tree $\beta$Vvx and the adjectival predicative tree shown in
Figure~\ref{predicative-trees}(b).  As with $\beta$Vvx, it adjoins
in as a VP auxiliary tree.  However, since it is anchored by an
adjective, not a verb, it is similar to the adjectival predicative
tree in that it has an $\epsilon$ at the V node, and a feature value
of {\bf $<$mode$>$=nom} which is passed up to the VP root indicates
that it is an adjectival predication.  This serves the same purpose
as in the 
case of the tree in Figure~\ref{upset-with-features}, and forces another
auxiliary verb, such as the copula, to adjoin in to make it
{\bf $<$mode$>$=ind}.

\begin{figure}[htb]
\centering
\begin{tabular}{c}
{\psfig{figure=ps/sm-clause-files/betaVvx-adj.ps,height=2.0in}} \\
\end{tabular}
\caption{Raising adjective tree: $\beta$Vvx-adj}
\label{Vvx-adj}
\end{figure}

\section{Non-predicative {\it BE}}
\label{equative-be-xtag-analysis}

The examples with the copula that we have given seem to indicate that {\it be}
is always followed by a predicative phrase of some sort.  This is not the case,
however, as seen in sentences such as ({\ex{1}})-({\ex{6}}).  The noun phrases in
these sentences are not predicative.  They do not take raising verbs, and they
do not occur in embedded small clause constructions.

\enumsentence{my teacher is Mrs. Wayman .}
\enumsentence{Doug is the man with the glasses .}

\enumsentence{$\ast$My teacher seems Mrs. Wayman .}
\enumsentence{$\ast$Doug appears the man with the glasses .}

\enumsentence{$\ast$I consider [my teacher Mrs. Wayman] .}
\enumsentence{$\ast$I prefer [Doug the man with the glasses] .}

In addition, the subject and complement can exchange positions in these type of
examples but not in sentences with predicative {\it be}.  Sentence ({\ex{1}})
has the same interpretation as sentence ({\ex{-4}}) and differs only in the
positions of the subject and complement NP's. Similar sentences, with a
predicative {\it be}, are shown in ({\ex{2}}) and ({\ex{3}}).  In this case,
the sentence with the exchanged NP's ({\ex{3}}) is ungrammatical.

\enumsentence{The man with the glasses is Doug .}
\enumsentence{Doug is a programmer .}
\enumsentence{$\ast$A programmer is Doug .}

The non-predicative {\it be} in ({\ex{-8}}) and ({\ex{-7}}), also called
\xtagdef{equative be}, patterns differently, both syntactically and
semantically, from the predicative usage of {\it be}.  Since these sentences
are clearly not predicative, it is not desirable to have a tree structure that
is anchored by the NP, AP, or PP, as we have in the predicative sentences.  In
addition to the conceptual problem, we would also need a mechanism to block
raising verbs from adjoining into these sentences (while allowing them for true
predicative phrases), and prevent these types of sentence from being embedded
(again, while allowing them for true predicative phrases).  

%%An additional
%%indication that distinct trees are necessary is the difference in the
%%grammaticality of extraction for the material to the right of the predicative
%%and non-predicative {\it be}.  The sentences in ({\ex{1}}) contain predicative
%%{\it be} as evidenced by the ungrammaticality of switching the subject {\it the
%%challenger} with the clause {\it to provide a non-stipulative solution}. In
%%contrast the grammaticality of the sentence produced by the equivalent switch
%%in ({\ex{2}}) shows that the {\it be} in those sentences is
%%non-predicative. Notice that extraction is possible from the clause with
%%predicative {\it be} but not with non-predicative {\it be}.
%%
%%\enumsentence{The challenger is to provide a non-stipulative solution .\\
%%$\ast$To provide a non-stipulative solution is the challenger .\\
%%What$_{i}$ is the challenger to provide $t_{i}$ ?}
%%
%%\enumsentence{The challenge is to provide a non-stipulative solution .\\
%%To provide a non-stipulative solution is the challenge .\\
%%		$\ast$What$_{i}$ is the challenge to provide $t_{i}$ ?}
%%
%%The ungrammaticality of the extracted NP in ({\ex{0}}) is explained if the
%%clause is a complement added by substitution, since extraction is not possible
%%from within substituted elements.  This is true in XTAG of the complements of
%%equative {\it be}.  In contrast, the predicative {\it be} in ({\ex{-1}})
%%adjoins to {\it What$_{i}$ the challenger to provide
%%$\epsilon_{i}$}\footnote{The elementary tree underlying this sentence would not
%%include the infinitive {\it to}.  Since the details of how {\it to} is adjoined
%%do not affect the thrust of the current argument, we assume a stage of
%%derivation where {\it to} has already been adjoined.}. Since the extraction is
%%already part of the elementary tree anchored by {\it provide} in the
%%predicative analysis, the extraction is unproblematic.

\begin{figure}[htb]
\centering
\begin{tabular}{ccc}
{\psfig{figure=ps/sm-clause-files/alphanx0BEnx1_is.ps,height=1.9in}} &
\hspace{1.0in}&
{\psfig{figure=ps/sm-clause-files/alphaInvnx0BEnx1_is.ps,height=2.5in}} \\
(a)&&(b)\\
\end{tabular}
\caption{Equative {\it BE} trees: $\alpha$nx0BEnx1 (a) and $\alpha$Invnx0BEnx1 (b)}
\label{equative-be}
\label{1;1,6}
\end{figure}

Although non-predicative {\it be} is not a raising verb, it does exhibit the
auxiliary verb behavior set out in section \ref{copula-data}.  It inverts,
contracts, and so forth, as seen in sentences ({\ex{1}}) and ({\ex{2}}), and
therefore can not be associated with any existing tree family for main verbs.
It requires a separate tree family that includes the tree for inversion.
Figures~\ref{equative-be}(a) and \ref{equative-be}(b) show the declarative and
inverted trees, respectively, for equative {\it be}.

\enumsentence{is my teacher Mrs. Wayman ?}
\enumsentence{Doug isn't the man with the glasses .} 



\chapter{Ditransitive constructions and dative shift}
\label{double-objs}

Verbs such as {\it give\/} and {\it put\/} that require two objects, as
shown in examples (\ex{1})-(\ex{4}), are termed ditransitive.

\enumsentence{Christy gave a cannoli to Beth Ann.}
\enumsentence{$\ast$Christy gave Beth Ann.}
\enumsentence{Christy put a cannoli in the refrigerator.} 
\enumsentence{$\ast$Christy put a cannoli.}


The indirect objects {\it Beth Ann\/} and {\it refrigerator\/} appear in
these examples in the form of PP's.  Within the set of ditransitive
verbs there is a subset that also allow two NP's as in (\ex{1}). As can
be seen from (\ex{1}) and (\ex{2}) this two NP, or double-object,
construction is grammatical for {\it give\/} but not for {\it put}.  

\enumsentence{Christy gave Beth Ann a cannoli}
\enumsentence{$\ast$Christy put the refrigerator the cannoli.}

The alternation between (\ex{-5}) and (\ex{-1}) is known as dative
shift.\footnote{In languages similar to English that have overt case marking
indirect objects would be marked with dative case. It has also been suggested
that for English the preposition {\it to} serves as a dative case marker.} In
order to account for verbs with dative shift the English XTAG grammar includes
structures for both variants in the tree family Tnx0Vnx1Pnx2.  The declarative
trees for the shifted and non-shifted alternations are shown in
Figure~\ref{dative-alt}.


\begin{figure}[htb]
\centering
\begin{tabular}{ccc}
{\psfig{figure=ps/double-obj-files/alphanx0Vnx1Pnx2.ps,height=2.0in}}&
\hspace*{0.5in} &
{\psfig{figure=ps/double-obj-files/alphanx0Vnx2nx1.ps,height=1.1in}}
\\
(a)&\hspace*{0.5in}&(b)\\
\end{tabular}
\caption{Dative shift trees: $\alpha$nx0Vnx1Pnx2 and $\alpha$nx0Vnx2nx1}
\label{dative-alt}
\label{2;1,2}
\end{figure}

The indexing of nodes in these two trees represents the fact that the semantic
role of the indirect object (NP$_2$) in Figure~\ref{dative-alt}(a) is the same
as that of the direct object (NP$_2$) in Figure~\ref{dative-alt}(b) (and vice
versa).  This use of indexing is consistent with our treatment of other
constructions such as passive and ergative.


Verbs that do not show this alternation and have only the NP PP structure
(e.g. {\it put\/}) select the tree family Tnx0Vnx1pnx2.  Unlike
the Tnx0Vnx1Pnx2 family, the Tnx0Vnx1pnx2 tree family does not contain trees for
the NP NP structure. Other verbs such as {\it ask} allow only the NP
NP structure as shown in (\ex{1}) and (\ex{2}).  

\enumsentence{Beth Ann asked Srini a question.}
\enumsentence{$\ast$Beth Ann asked a question to Srini.}

Verbs that only allow the NP NP structure select the tree family
Tnx0Vnx1nx2. This tree family does not have the trees for the NP PP
structure. 

Notice that in Figure~\ref{dative-alt}(a) the preposition {\it to\/} is
built into the tree.  There are other apparent cases of dative shift
with {\it for}, such as in (\ex{1}) and (\ex{2}), that we have taken to
be structurally distinct from the cases with {\it to}.  

\enumsentence{Beth Ann baked Dusty a biscuit.}
\enumsentence{Beth Ann baked a biscuit for Dusty.}

\cite{mccawley88} notes that the {\it to} and {\it for} cases
differ with respect to passivization; the indirect objects with {\it
to} may be the subjects of corresponding passives while the alleged
indirect objects with {\it for} can not, as in (\ex{1})-(\ex{4}).

\enumsentence{Beth Ann gave dinner to Clove.}
\enumsentence{Clove was given dinner (by Beth Ann).}
\enumsentence{Beth Ann made dinner for Clove.}
\enumsentence{$\ast$Clove was made dinner (by Beth Ann).} 

McCawley also notes:

\begin{quote}
A ``{\it for-dative}'' expression in underlying structure is external
to the V with which it is combined, in view of the fact that the
latter behaves as a unit with regard to all relevant syntactic
phenomena.
\end{quote}


In other words, the {\it for} PP's that appear to undergo dative shift
are actually adjuncts, not complements. Examples (\ex{1}) and (\ex{2})
demonstrate that PP's with {\it for} are
optional while ditransitive {\it to} PP's are not.

\enumsentence{Beth Ann made dinner.}
\enumsentence{$\ast$Beth Ann gave dinner.}

Consequently, in the XTAG grammar the apparent dative shift with {\it
for} PP's is treated as Tnx0Vnx1nx2 for the NP NP structure, and as a
transitive plus an adjoined adjunct PP for the NP PP structure.  Hence
the fact that the preposition {\it to} is built into the tree and is
the only preposition allowed in dative shift constructions correctly
accounts for the observed patterns.


\chapter{It-clefts}
\label{it-clefts}

There are several varieties of it-clefts in English.  All of the
it-clefts have four major components:

\begin{itemize}
\item {\bf the dummy subject:}  {\it it},
\item {\bf the main verb:}  {\it be},
\item {\bf the clefted element:}  A constituent (XP) compatible with
any gap in the clause,
\item {\bf the clause:}  A clause (e.g. S) with or without a gap.
\end{itemize}

\noindent
Examples of it-clefts are shown in (\ex{1})-(\ex{4}).

\enumsentence{It is [$_{XP}$ these demands$_{XP}$] [$_{S}$ to which
candidates from the left and the right are responding.$_{S}$] (WSJ)}
\enumsentence{It is [$_{XP}$ spirit$_{XP}$]  [$_{S}$ which gives life
to a community and causes it to cohere. (Brown corpus)}
\enumsentence{It is [$_{XP}$ here$_{XP}$]  [$_{S}$ that the ecumenical
must become local and the local become ecumenical.$_{S}$] (Brown corpus)}
\enumsentence{It was  [$_{XP}$ there$_{XP}$]  [$_{S}$ that she would
have to enact her renunciation.$_{S}$] (Brown corpus)}

The clefted element can be of a number of categories, for example NP, PP or
adverb. The clause can also be of several types. The English XTAG grammar
currently has a separate analysis for only a subset of the `specificational'
it-clefts\footnote{See e.g. \cite{Ball91},
\cite{Delin89} and \cite{Delahunty84} for more detailed discussion of
types of it-clefts.}, in particular the ones without gaps in the clause
(e.g. (\ex{-1}) and (\ex{-0})).  It-clefts that have gaps in the clause, such
as (\ex{-3}) and (\ex{-2}) are currently handled as relative clauses. Although
arguments have been made against treating the clefted element and the clause as
a constituent (\cite{Delahunty84}), the relative clause approach does capture
the restriction that the clefted element must fill the gap in the clause, and
does not require any additional trees.

In the `specificational' it-cleft without gaps in the clause, the
clefted element has the role of an adjunct with respect to the clause.
For these cases the English XTAG grammar requires additional trees.
These it-cleft trees are in separate tree families because, although
some researchers (e.g. \cite{Akmajian70}) derived it-clefts through
movement from other sentence types, most current researchers
(e.g. \cite{Delahunty84}, \cite{Knowles86}, \cite{gazdar85},
\cite{Delin89} and \cite{Sornicola88}) favor base-generation of the
various cleft sentences.  Placing the it-cleft trees in their own tree
families is consistent with the the current preference for base
generation, since in the XTAG English grammar, structures that would
be related by transformation in a movement-based account will appear
in the same tree family. Like the base-generated approaches, the
placement of it-clefts in separate tree families makes the claim that
there is no derivational relation between it-clefts and other sentence
types.

The three it-cleft tree families are virtually identical except for the
category label of the clefted element.  Figure~\ref{pp-it-clefts} shows the
declarative tree and an inverted tree for the PP It-cleft tree family.

\begin{figure}[htb]
\centering
\begin{tabular}{ccc}
{\psfig{figure=ps/it-cleft-files/alphaItVpnx1s2.ps,height=2.0in}} &
\hspace*{0.5in} &
{\psfig{figure=ps/it-cleft-files/alphaInvItVpnx1s2.ps,height=2.5in}} \\
(a)&\hspace*{0.5in}&(b)\\
\end{tabular}
\caption{It-cleft with PP clefted element: $\alpha$ItVpnx1s2 and $\alpha$InvItVpnx1s2}
\label{pp-it-clefts}
\label{1;1,3}
\label{1;3,3}
\end{figure}


The extra layer of tree structure in the VP represents that while {\it be} is a
main verb rather than an auxiliary in these cases, it retains some auxiliary
properties. The VP structure for the equative/it-cleft-{\it be} is identical to
that obtained after adjunction of predicative-{\it be} into
small-clauses.\footnote{For additional discussion of equative or
predicative-{\it be} see section~\ref{small-clauses}.}  The inverted tree in
Figure~\ref{pp-it-clefts}(b) is necessary because of {\it be}'s auxiliary-like
behavior.  Although {\it be} is the main verb in it-clefts, it inverts like an
auxiliary.  Main verb inversion cannot be accomplished by adjunction as is done
with auxiliaries and therefore must be built into the tree family. The tree in
Figure~\ref{pp-it-clefts}(b) is used for yes/no questions such as (\ex{1}).

\enumsentence{Was it in the forest that the wolf talked to Little Red
Riding Hood?}








\part{Sentence Types}
\chapter{Passives}
\label{passives}
In passive constructions such as (\ex{1}), the subject NP is
interpreted as having the same role as the direct object NP in the
corresponding declarative (\ex{2}).

\enumsentence{{\bf An airline buy-out bill} was approved by the House. (WSJ)}
\enumsentence{The House approved {\bf an airline buy-out bill}.}

\begin{figure}[hbt]
\centering
\begin{tabular}{ccccc}
\psfig{figure=ps/passives-files/betanx1Vs2-reduced-features.ps,height=6.5cm}&
\hspace{1.0in}&
\psfig{figure=ps/passives-files/betanx1Vbynx0s2.ps,height=6.5cm}&
\hspace{1.0in}&
\psfig{figure=ps/passives-files/betanx1Vs2bynx0.ps,height=6.5cm}\\
(a)&&(b)&&(c)
\end{tabular}
\caption{Passive trees in the Sentential Complement with NP tree family:
$\beta$nx1Vs2 (a), $\beta$nx1Vbynx0s2 (b) and $\beta$nx1Vs2bynx0 (c)}
\label{passive-trees}
\label{2;2,5}
\end{figure}

In a movement analysis, the direct object is said to have moved to the subject
position.  The original declarative subject is either absent in the passive or
is in a {\it by} headed PP ({\it by} phrase). In the English XTAG grammar,
passive constructions are handled by having separate trees within the
appropriate tree families.  Passive trees are found in most tree families that
have a direct object in the declarative tree (the light verb tree families, for
instance, do not contain passive trees).  Passive trees occur in pairs - one
tree with the {\it by} phrase, and another without it.  Variations in the
location of the {\it by} phrase are possible if a subcategorization includes
other arguments such as a PP or an indirect object. Additional trees are
required for these variations.  For example, the Sentential Complement with NP
tree family has three passive trees, shown in Figure~\ref{passive-trees}: one
without the {\it by}-phrase (Figure~\ref{passive-trees}(a)), one with the {\it
by} phrase before the sentential complement (Figure~\ref{passive-trees}(b)),
and one with the {\it by} phrase after the sentential complement
(Figure~\ref{passive-trees}(c)).

Figure~\ref{passive-trees}(a) also shows the feature restrictions imposed on
the anchor\footnote{A reduced set of features are shown for readability.}. Only
verbs with {\bf $<$mode$>$=ppart} (i.e. verbs with passive morphology) can
anchor this tree.  The {\bf $<$mode$>$} feature is also responsible for
requiring that passive {\it be} adjoin into the tree to create a matrix
sentence.  Since a requirement is imposed that all matrix sentences must have
{\bf $<$mode$>$=ind/imp}, an auxiliary verb that selects {\bf
$<$mode$>$=ppart} and {\bf $<$passive$>$=+} (such as {\it was}) must adjoin
(see Chapter~\ref{auxiliaries} for more information on the auxiliary verb
system).

 








\chapter{Extraction}
\label{extraction}

The discussion in this chapter covers constructions that are analyzed
as having wh-movement in GB, in particular, wh-questions and
topicalization. Relative clauses, which could also be considered
extractions, are discussed in Chapter~\ref{rel_clauses}.

Extraction involves a constituent appearing in a linear position to the left of
the clause with which it is interpreted. One clause argument position is
empty. For example, the position filled by {\it frisbee} in the declarative in
sentence~(\ex{1}) is empty in sentence~(\ex{2}). The wh-item {\it what} in
sentence~(\ex{2}) is of the same syntactic category as {\it frisbee} in
sentence~(\ex{1}) and fills the same role with respect to the
subcategorization.

\enumsentence{Clove caught a frisbee.}
\enumsentence{What$_{i}$ did Clove catch $\epsilon_{i}$?} 


The English XTAG grammar represents the connection between the extracted
element and the empty position with co-indexing (as does GB).  The {\bf
$<$trace$>$} feature is used to implement the co-indexing.  In extraction trees
in XTAG, the `empty' position is filled with an {\it $\epsilon$}.  The
extracted item always appears in these trees as a sister to the S$_{r}$
tree, with both dominated by a S$_{q}$ root node.  The S$_{r}$ subtrees in
extraction trees have the same structure as the declarative tree in the same
tree family.  The additional structure in extraction trees of the S$_{q}$ and
NP nodes roughly corresponds to the CP and Spec of CP positions in GB.

All sentential trees with extracted components (this does not include relative
clause trees) are marked {\bf $<$extracted$>$=+} at the top S node, while
sentential trees with no extracted components are marked {\bf
$<$extracted$>$=--}.  Items that take embedded sentences, such as nouns, verbs
and some prepositions can place restrictions on whether the embedded sentence
is allowed to be extracted or not.  For instance, sentential subjects and
sentential complements of nouns and prepositions are not allowed to be
extracted, while certain verbs may allow extracted sentential complements and
others may not (e.g. sentences (\ex{1})-(\ex{4})).

\enumsentence{The jury wondered [who killed Nicole].}
\enumsentence{The jury wondered [who Simpson killed].}
\enumsentence{The jury thought [Simpson killed Nicole].}
\enumsentence{$\ast$The jury thought [who did Simpson kill]?}
The {\bf $<$extracted$>$} feature is also used to block embedded topicalization
in infinitival complement clauses as exemplified in (\ex{1}). 
\enumsentence{* John wants [ Bill$_{i}$ [PRO to see t$_{i}$]]}
Verbs such as {\em want} that take non-{\em wh} infinitival complements
specify that the {\bf $<$extracted$>$} feature of their complement clause
(i.e. of the foot S node)
is {\bf --}. Clauses that involve topicalization have {\bf +} as the value
of their {\bf $<$extracted$>$} feature (i.e. of the root S node). 
Sentences like (\ex{0}) are thus ruled out.  

\begin{figure}[htb]
\centering
\mbox{}
\psfig{figure=ps/extraction-files/alphaW1nx0Vnx1.ps,height=10.0cm}
\caption{Transitive tree with object extraction: $\alpha$W1nx0Vnx1}
\label{alphaW1nx0Vnx1}
\label{2;5,1}
\end{figure} 


The tree that is used to derive the embedded sentence in (\ex{-2}) in
the English XTAG grammar is shown in
Figure~\ref{alphaW1nx0Vnx1}\footnote{Features not pertaining to this
  discussion have been taken out to improve readability.}.  The
important features of extracted trees are:

\begin{itemize}
\item The subtree that has S$_{r}$ as its root is identical to the
  declarative tree or a non-extracted passive tree, except for having
  one NP position in the VP filled by $\epsilon$.

\item The root S node is S$_{q}$, which dominates NP and S$_{r}$.
  
\item The {\bf $<$trace$>$} feature of the $\epsilon$ filled NP is
  co-indexed with the {\bf $<$trace$>$} feature of the NP daughter of
  S$_{q}$.

\item The {\bf $<$case$>$} and {\bf $<$agr$>$} features are passed
  from the empty NP to the extracted NP.  This is particularly
  important for extractions from subject NP's, since {\bf $<$case$>$}
  can continue to be assigned from the verb to the subject NP
  position, and from there be passed to the extracted NP.
  
\item The {\bf $<$inv$>$} feature of S$_{r}$ is co-indexed to the {\bf
    $<$wh$>$} feature of NP through the use of the {\bf $<$invlink$>$}
  feature in order to force subject-auxiliary inversion where needed
  (see section~\ref{topicalization} for more discussion of the {\bf
    $<$inv$>$}/{\bf$<$wh$>$} co-indexing and the use of these trees
  for topicalization).

\end{itemize}



\section{Topicalization and the value of the {\bf $<$inv$>$} feature}
\label{topicalization}

Our analysis of topicalization uses the same trees as wh-extraction.  For any
NP complement position a single tree is used for both wh-questions and for
topicalization from that position. Wh-questions have subject-auxiliary
inversion and topicalizations do not.  This difference between the
constructions is captured by equating the values of the S$_{r}$'s {\bf
$<$inv$>$} feature and the extracted NP's {\bf $<$wh$>$} feature.  This means
that if the extracted item is a wh-expression, as in wh-questions, the value of
{\bf $<$inv$>$} will be {\bf +} and an inverted auxiliary will be forced to
adjoin. If the extracted item is a non-wh, {\bf $<$inv$>$} will be {\bf --}
and no auxiliary adjunction will occur. An additional complication is that
inversion only occurs in matrix clauses, so the values of {\bf $<$inv$>$} and
{\bf $<$wh$>$} should only be equated in matrix clauses and not in embedded
clauses.  In the English XTAG grammar, appropriate equating of the {\bf
$<$inv$>$} and {\bf $<$wh$>$} features is accomplished using the {\bf
$<$invlink$>$} feature and a restriction imposed on the root S of a
derivation. In particular, in extraction trees that are used for both
wh-questions and topicalizations, the value of the {\bf $<$inv$>$} feature for
the top of the S$_{r}$ node is co-indexed to the value of the {\bf $<$inv$>$}
feature on the bottom of the S$_{q}$ node.  On the bottom of the S$_{q}$ node
the {\bf $<$inv$>$} feature is co-indexed to the {\bf $<$invlink$>$} feature.
The {\bf $<$wh$>$} feature of the extracted NP node is co-indexed to the value
of the {\bf $<$wh$>$} feature on the bottom of S$_{q}$. The linking between the
value of the S$_{q}$ {\bf $<$wh$>$} and the {\bf $<$invlink$>$} features is
imposed by a condition on the final root node of a derivation (i.e. the top S
node of a matrix clause) requires that {\bf $<$invlink$>$=$<$wh$>$}.  For
example, the tree in Figure~\ref{alphaW1nx0Vnx1} is used to
derive both (\ex{1}) and (\ex{2}).


\enumsentence{John, I like.}
\enumsentence{Who do you like?}

For the question in (\ex{0}), the extracted item {\it who} has the feature
value {\bf $<$wh$>$=+}, so the value of the {\bf $<$inv$>$} feature on VP is
also $+$ and an auxiliary, in this case {\it do}, is forced to adjoin.  For the
topicalization (\ex{-1}) the values for {\it John}'s {\bf $<$wh$>$} feature and
for S$_{q}$'s {\bf $<$inv$>$} feature are both {\bf --} and no auxiliary
adjoins.



\section{Extracted subjects}
\label{subject-extraction}

The extracted subject trees provide for sentences like (\ex{1})-(\ex{3}),
depending on the tree family with which it is associated.

\enumsentence{Who left?}
\enumsentence{Who wrote the paper?}
\enumsentence{Who was happy?}

Wh-questions on subjects differ from other argument extractions in
not having subject-auxiliary inversion.  This means that in subject
wh-questions the linear order of the constituents is the same as in
declaratives so it is difficult to tell whether the subject has moved
out of position or not (see \cite{heycock/kroch93gagl} for arguments
for and against moved subject). 

The English XTAG treatment of subject extractions assumes the
following:

\begin{itemize}
\item Syntactic subject topicalizations don't exist; and 
\item Subjects in wh-questions are extracted rather than in situ.
\end{itemize}

The assumption that there is no syntactic subject topicalization is reasonable
in English since there is no convincing syntactic evidence and since the
interpretability of subjects as topics seems to be mainly affected by discourse
and intonational factors rather than syntactic structure. As for the assumption
that wh-question subjects are extracted, these questions seem to have more
similarities to other extractions than to the two cases in English that have
been considered in situ wh: multiple wh questions and echo questions. In
multiple wh questions such as sentence~(\ex{1}), one of the wh-items is blocked
from moving sentence initially because the first wh-item already occupies the
location to which it would move.

\enumsentence{Who ate what?}

This type of `blocking' account is not applicable to
subject wh-questions because there is no obvious candidate to do the
blocking.  Similarity between subject wh-questions and echo questions
is also lacking.  At least one account of echo questions
(\cite{hockey94}) argues that echo questions are not ordinary
wh-questions at all, but rather focus constructions in which the
wh-item is the focus. Clearly, this is not applicable to subject
wh-questions. So it seems that treating subject wh-questions similarly
to other wh-extractions is more justified than an in situ treatment. 

Given these assumptions, there must be separate trees in each tree family for
subject extractions. The declarative tree cannot be used even though the linear
order is the same because the structure is different. Since topicalizations are
not allowed, the {\bf $<$wh$>$} feature for the extracted NP node is set in
these trees to {\bf +}.  The lack of subject-auxiliary inversion is handled
by the absence of the {\bf $<$invlink$>$} feature.  Without the presence of
this feature, the {\bf $<$wh$>$} and {\bf $<$inv$>$} are never linked, so
inversion can not occur.  Like other wh-extractions, the S$_{q}$ node is marked
{\bf $<$extracted$>$=+} to constrain the occurrence of these trees in
embedded sentences. The tree in Figure~\ref{alphaW0nx0V} is an example of a
subject wh-question tree.

\begin{figure}[htb]
\centering
\begin{tabular}{c}
\psfig{figure=ps/extraction-files/alphaW0nx0V.ps,height=10.3cm}
\end{tabular}
\caption{Intransitive tree with subject extraction: $\alpha$W0nx0V}
\label{alphaW0nx0V}
\label{1;4,13} 
\end{figure}



\section{Wh-moved NP complement}
\label{NP-extr}

Wh-questions can be formed on every NP object or indirect object that appears
in the declarative tree or in the passive trees, as seen in sentences
(\ex{1})-(\ex{6}).  A tree family will contain one tree for
each of these possible NP complement positions.
Figure~\ref{ditrans-extractions} shows the two extraction trees from the
ditransitive tree family for the extraction of the direct
(Figure~\ref{ditrans-extractions}(a)) and indirect object
(Figure~\ref{ditrans-extractions}(b)).

\enumsentence{Dania asked Beth a question.}
\enumsentence{Who$_{i}$ did Dania ask $\epsilon_{i}$ a question?}
\enumsentence{What$_{i}$ did Dania ask Beth $\epsilon_{i}$?}
\enumsentence{Beth was asked a question by Dania.}
\enumsentence{Who$_{i}$ was Beth asked a question by $\epsilon_{i}$??}
\enumsentence{What$_{i}$ was Beth asked $\epsilon_{i}$? by Dania?}

\begin{figure}[htb]
\centering
\begin{tabular}{ccc}
\psfig{figure=ps/extraction-files/alphaW2nx0Vnx2nx1.ps,height=6.0cm}&
\hspace{1.0in}&
\psfig{figure=ps/extraction-files/alphaW1nx0Vnx2nx1.ps,height=6.0cm}\\
(a)&&(b)
\end{tabular}
\caption{Ditransitive trees with direct object: $\alpha$W2nx0Vnx2nx1 (a) and
indirect object extraction: $\alpha$W1nx0Vnx2nx1 (b)}
\label{ditrans-extractions}
\label{2;5,3}
\end{figure}

%%\begin{figure}[htb]
%%\centering
%%\mbox{}
%%\psfig{figure=ps/extraction-files/alphaW0nx1Vpnx2bynx0.ps,height=8.0cm}
%%\caption{Tree:  $\alpha$W0nx1Vpnx2bynx0}
%%\label{alphaW0nx1Vpnx2bynx0} 
%%\label{2;21,4}
%%\end{figure}


\section{Wh-moved object of a P}
Wh-questions can be formed on the NP object of a complement PP as in
sentence~(\ex{1}).

\enumsentence{$[$Which dog$]_{i}$ did Beth Ann give a bone to $\epsilon_{i}$?}

The {\it by} phrases of passives behave like complements and can undergo the
same type of extraction, as in (\ex{1}).

\enumsentence{$[$Which dog$]_{i}$ was the frisbee caught by $\epsilon_{i}$?}

Tree structures for this type of sentence are very similar to those for the
wh-extraction of NP complements discussed in section~\ref{NP-extr} and have the
identical important features related to tree structure and trace and inversion
features.  The tree in Figure~\ref{alphaW2nx0Vnx1pnx2} is an example of this
type of tree.  Topicalization of NP objects of prepositions is handled the same
way as topicalization of complement NP's.

\begin{figure}[htb]
\centering
\mbox{}
\psfig{figure=ps/extraction-files/alphaW2nx0Vnx1pnx2.ps,height=6.0cm}
\caption{Ditransitive with PP tree with the object of the PP extracted: $\alpha$W2nx0Vnx1pnx2}
\label{alphaW2nx0Vnx1pnx2}
\label{2;8,4}
\end{figure}



\section{Wh-moved PP}
Like NP complements, PP complements can be extracted to form
wh-questions, as in sentence (\ex{1}).

\enumsentence{[To which dog]$_{i}$ did Beth Ann throw the frisbee $\epsilon_{i}$?}

As can be seen in the tree in Figure~\ref{alphapW2nx0Vnx1pnx2}, extraction of
PP complements is very similar to extraction of NP complements from the same
positions.

\begin{figure}[htb]
\centering
\mbox{}
\psfig{figure=ps/extraction-files/alphapW2nx0Vnx1pnx2.ps,height=6.0cm}
\caption{Ditransitive with PP with PP extraction tree: $\alpha$pW2nx0Vnx1pnx2}
\label{alphapW2nx0Vnx1pnx2} 
\label{2;9,4}
\end{figure}


The PP extraction trees differ from NP extraction trees in having a PP
rather than an NP left daughter node under S$_{q}$ and in having the
$\epsilon$ fill a PP rather than an NP position in the VP. In other
respects these PP extraction structures behave like the NP extractions,
including being used for topicalization.



\section{Wh-moved S complement}

Except for the node label on the extracted position, the trees for wh-questions
on S complements look exactly like the trees for wh-questions on NP's in the
same positions.  This is because there is no separate wh-lexical item for
clauses in English, so the item {\it what} is ambiguous between representing a
clause or an NP.  To illustrate this ambiguity notice that the question in
(\ex{1}) could be answered by either a clause as in (\ex{2}) or an NP as in
(\ex{3}).  The extracted NP in these trees is constrained to be {\bf
$<$wh$>$=+}, since sentential complements can not be topicalized.

\enumsentence{What does Clove want?}
\enumsentence{for Beth Ann to play frisbee with her}
\enumsentence{a biscuit}

%%\begin{figure}[htb]
%%\centering
%%\mbox{}
%%\psfig{figure=ps/extraction-files/betaW1nx0Vs1.ps,height=20cm}
%%\caption{Tree:  $\beta$W1nx0Vs1}
%%\label{wh-s-extr} 
%%\label{2;6,10}
%%\end{figure}



\section{Wh-moved Adjective complement}
In subcategorizations that select an adjective complement, that
complement can be questioned in a wh-question, as in sentence~(\ex{1}).

\enumsentence{How$_{i}$ did he feel $\epsilon_{i}$?}

\begin{figure}[htb]
\centering
\mbox{}
\psfig{figure=ps/extraction-files/alphaWA1nx0Vax1.ps,height=6.0cm}
\caption{Predicative Adjective tree with extracted adjective: $\alpha$WA1nx0Vax1}
\label{wh-adj-extr} 
\label{1;7,14}
\end{figure}


The tree families with adjective complements include trees for such adjective
extractions that are very similar to the wh-extraction trees for other
categories of complements.  The adjective position in the VP is filled by an
{\it $\epsilon$} and the trace feature of the adjective complement and of the
adjective daughter of S$_{q}$ are co-indexed.  The extracted adjective is
required to be {\bf $<$wh$>$=+}\footnote{{\it How} is the only {\bf
$<$wh$>$=+} adjective currently in the XTAG English grammar.}, so no
topicalizations are allowed.  An example of this type of tree is shown in
Figure~\ref{wh-adj-extr}.










\chapter{Relative Clauses}
\label{rel_clauses}

Relative clauses are NP modifiers, which involve extraction of an argument
or an adjunct. The NP head (the portion of the NP being modified by the
relative clause) is not directly related to the extracted element.  For
example in \ex{1}, {\it the person} is the head NP and is modified by the
relative clause {\it whose mother $\epsilon$ likes Chris}. {\em The person}
is not interpreted as the subject of the relative clause which is missing
an overt subject. In other cases, such as \ex{2}, the relationship between
the head NP {\em export exhibitions} may seem to be more direct but even
there we assume that there are two independent relationships: one between
the entire relative clause and the NP it modifies, and another between the
extracted element and its trace. In \ex{2}, the extracted element is
covert, whereas in \ex{1} it is overt. 

\enumsentence{[[ the person$_i$ ] [ whose$_i$ mother$_j$ [ $\epsilon_j$ likes Chris ]]]}
\enumsentence{[[ export exhibitions$_i$ ] [ $\epsilon_i$ that [ $\epsilon_i$ included high-tech items ]]]}

Our analysis is essentially identical to the GB analysis of relative
clauses. Relative clauses are represented in the English XTAG grammar by
auxiliary trees that adjoin to NP's. These trees are anchored by the verb
in the relative clause and appear in the appropriate tree families
representing various verb subcategorizations. Each family has groups of
relative clause trees based on the declarative tree and each passive tree
in that family. Within each of these groups, there is a separate relative
clause tree corresponding to each possible argument that can be extracted
from the clause. As stated above, there is no relationship between the
extracted position and the head NP. The relationship between the relative
clause and the head NP is treated as a semantic relationship which will be
provided by any reasonable compositional theory. The relationship between
the extracted element, NP$_w$ (which can be covert) and the position from
which it was extracted is captured by co-indexing the {\bf $<$trace$>$}
features of the two positions/nodes.\footnote{%
%
No adjunct traces are represented in the XTAG analysis of adjunct
extraction. Since relative clauses on adjuncts also do not have traces,
feature equations showing the trace coindexation are not present in such
trees. See Section~\ref{sec:adju-RC} for more discussion of adjunct
relative clauses.%
%
} If for example, it is {\bf NP$_{1}$} that is extracted, we have the
following feature equations (see Figure~\ref{trans-rel-clause-trees}(a)):

\begin{\itemize}

\item {\bf NP$_{w}$.t:$\langle$ trace $\rangle =$NP$_{1}$.t:$\langle$ trace $\rangle$}
\item {\bf NP$_{w}$.t:$\langle$ case $\rangle =$NP$_{1}$.t:$\langle$ case $\rangle$}
\item {\bf NP$_{w}$.t:$\langle$ agr $\rangle =$NP$_{1}$.t:$\langle$ agr $\rangle$}

\end{itemize}

One aspect of the implementation of relative clauses is to allow a covert
{\bf $+<$wh$>$} NP and/or a covert COMP. For example, \ex{1} has a covert
{\bf $+<$wh$>$} NP and overt COMP, \ex{2} has a covert COMP and overt {\bf
$+<$wh$>$}, and \ex{3} has both a covert {\bf $+<$wh$>$} NP and a covert
COMP.

\enumsentence{export exhibitions [[ {\footnotesize $_{NP_{w}}$} $\epsilon$ ]$_{i}$ that [ $\epsilon$$_{i}$ included high-tech items ]]}
\enumsentence{the export exhibition [ which$_{i}$ [ Muriel planned  $\epsilon$$_{i}$ ]]}
\enumsentence{the export exhibition [[ {\footnotesize $_{NP_{w}}$} $\epsilon$ ]$_{i}$ [ Muriel planned  $\epsilon$$_{i}$ ]]}

Consequently, our treatment of relative clauses has different trees to
handle relative clauses with an overt extracted {\em wh}-NP
(Section~\ref{sec:overt-extraction}) and relative clauses with a covert
extracted {\em wh}-NP (Section~\ref{sec:covert-extraction}). Covert and
overt COMP's are handled by adjunction with the already existing auxiliary
trees for complementizers, that are used for sentential complementation
(see Chapter~\ref{scomps-section}).

\section{Relative Clauses with overt extracted {\em wh}-phrases}
\label{sec:overt-extraction}

Relative clauses with an overt extracted {\em wh}-NP
(Figure~\ref{trans-rel-clause-trees}(a)) involve substitution of a $+${\bf
$<$wh$>$} NP into the (extracted) NP$_{w}$ node. The feature equation {\bf
NP$_{w}$.t:$<$wh$>=+$} allows only {\it wh}-phrases to substitute into
this node, such as {\em whose mother}, {\em who}, {\em whom}, {\em which}
(but not {\em when} and {\em where}, which are treated as exhaustive
$+${\em wh} PPs (Figure~\ref{trans-rel-clause-trees}(b))).

\begin{figure}[ htb ]
\begin{tabular}{cc}
\psfig{figure=ps/rel_clauses-files/betaN1nx0Vnx1.ps,height=10.0cm}&
\psfig{figure=ps/rel_clauses-files/betaNpxnx0Vnx1.ps,height=10.0cm}\\
(a)&(b)
\end{tabular}
\caption{Relative clause trees with overt {\em wh}-phrases in the transitive
tree family: (a) object extraction tree $\beta$N1nx0Vnx1, ({\it the man who
Muriel loved}), and (b) adjunct relative clause tree with PP pied-piping
$\beta$Npxnx0Vnx1, {\it the woman who killed the mice}.}
\label{trans-rel-clause-trees}
\label{2;16,1}
\label{2;15,1}
\end{figure}

Complementizers can never cooccur with the overt extracted {\bf $+<$wh$>$}
NP (cf. *{\it I saw the man who$_i$ that Muriel saw $\epsilon$$_{i}$}).
Consequently, the auxiliary $\beta$COMPs trees are prevented from adjoining
at the {\bf S$_r$} node in these trees by the equation \ex{1} in the
relative clause tree, which will always fail to unify with the (non-{\bf
nil}) values of the {\bf $<$comp$>$} feature in the $\beta$COMPs trees (see
for example Figure~\ref{that-comp-tree}).

\enumsentence{{\bf S$_{r}$.t:$\langle$comp$\rangle =$ nil}}

Examples \ex{1} and \ex{2} are examples for which the tree in
Figure~\ref{trans-rel-clause-trees}(a) is used. Cases of PP pied-piping, as
in \ex{3}, are handled in a similar fashion by building in a PP$_{w}$
substitution node (Figure~\ref{trans-rel-clause-trees}(b)).

\enumsentence{the man who Muriel likes}
\enumsentence{the man whose mother Muriel likes}
\enumsentence{the bowl in which Miriam ate her cereal}

\subsection{Contraints on the mode of the relative clause}
\label{sec:mode-restriction}

Relative clause trees that have {\bf NP$_{w}$} as a substitution node have
the feature equation in \ex{1}. The examples in \ex{2}--\ex{6} provide the
rationale for this feature setting.

\enumsentence{
{\bf S$_{r}$.t:$\langle$mode$\rangle =$ind}}
\enumsentence{
the boy [[whose mother ]$_{i}$ [ $\epsilon$$_{i}$ chased the cat ]] ({\bf S$_{r}$.t:$\langle$mode$\rangle =$ind})}
\enumsentence{
*the boy [[whose mother ]$_{i}$ [ $\epsilon$$_{i}$ to chase the cat ]] ({\bf S$_{r}$.t:$\langle$mode$\rangle =
$inf})}
\enumsentence{
*the boy [[whose mother ]$_{i}$ [ $\epsilon$$_{i}$ eaten the cake ]] ({\bf S$_{r}$.t:$\langle$mode$\rangle 
=$ppart})}
\enumsentence{
*the boy [[whose mother ]$_{i}$ [ $\epsilon$$_{i}$ chasing the cat ]] ({\bf S$_{r}$.t:$\langle$mode$\rangle =$
ger})}
\enumsentence{
the boy [[whose mother ]$_{i}$ [ Bill believes [ $\epsilon$$_{i}$ to chase the
cat ]]] ({\bf S$_{r}$.t:$\langle$mode$\rangle =$ind})}

The feature equation in trees with a {\bf PP$_{w}$} substitution node is
given in \ex{1} with the rationale provided by examples in \ex{2}--\ex{5}.%
\footnote{%
%
As is the case for {\bf NP$_{w}$} substitution, any $+${\bf $<$wh$>$}PP can
substitute under PP$_{w}$.  This is implemented by the following
equation: {\bf PP$_{w}$.t:$\langle$wh$\rangle=+$}.
%Not all cases of pied-piping involve substitution of {\bf PP$_{w}$}.  In
%some cases, the P may be built in. In cases where part of the pied-piped PP
%is part of the anchor, it continues to function as an anchor even after
%pied-piping i.e. the P node and the {\bf NP$_{w}$} nodes are represented
%separately.%
%
}

\enumsentence{
{\bf S$_{r}$.t:$\langle$mode$\rangle =$ ind/inf}}
\enumsentence{
the person [[by whom ]$_{i}$ [ this machine was invented $\epsilon$$_{i}$ ]] ({\bf S$_{r}$.t:$\langle$mode$\rangle =$ind})}
\enumsentence{
a baker [[in whom ]$_{i}$ [ to trust $\epsilon$$_{i}$ ]] ({\bf S$_{r}$.t:$\langle$mode$\rangle =$inf})}
\enumsentence{
*the fork [[with which ]$_{i}$ (Geoffrey) eaten the pudding $\epsilon$$_{i}$
]] ({\bf S$_{r}$.t:$\langle$
mode$\rangle =$ppart})}
\enumsentence{
*the person [[by whom ]$_{i}$ [ (this machine) inventing $\epsilon$$_{i}$
]] ({\bf S$_{r}$.t:$\langle$mode$\rangle =$ger})}

\section{Relative Clauses with Complementizers}
\label{sec:covert-extraction}

Relative clauses with a covert extracted {\em wh}-NP
(Figure~\ref{trans-rel-clause-trees2} ) have a NP$_{w}$ node headed by
$\epsilon$$_{w}$, which is built into the trees. Complementizers can adjoin
in at the {\bf S$_r$} node in a manner parallel to sentential
complementation (see Chapter~\ref{scomps-section}). The examples in
\ex{1}-\ex{4} are handled by this tree.

\enumsentence{the cake that Muriel said Steven ate}
\enumsentence{the book for Miranda to read}
\enumsentence{the man looking at Muriel}
\enumsentence{the librarian to issue the book}

\begin{figure}[ htb ]
\begin{tabular}{c}
\centerline{\psfig{figure=ps/rel_clauses-files/betaNc0nx0Vnx1.ps,height=12.0cm}}
\end{tabular}
\caption{Subject extraction tree in the transitive tree family with
non-overt {\em wh}-phrase, $\beta$Nc0nx0Vnx1}
\label{trans-rel-clause-trees2}
\end{figure}

There are three aspects to the implementation of these trees. Firstly, in a
manner parallel to the relative clause trees with overt NP$_w$, there are
constraints on the mode of the relative clause for these trees, which is
realized with the {\bf $<$mode$>$} feature
(Section~\ref{sec:clause-mode}). Secondly, there are cooccurrence constraints
between the mode of the relative clause and the complementizers that can
adjoin in -- these are realized with the {\bf $<$assign-comp$>$} and {\bf
$<$comp$>$} features (Section~\ref{sec:comp-selection}). The implementation of
the couccrrence constraints is entirely parallel to sentential
complementation, except in one respect: the occurrence of the null COMP
(which in the case of the relative clauses, is represented by disallowing
the adjunction of any COMP -- namely, the $\beta$COMPs auxiliary tree --
altogether) is subject to further constraints, which we realize (crucially)
with the {\bf $<$nocomp-mode$>$} feature (Section~\ref{sec:nocomp-mode}).

\subsection{Constraints on the mode of the relative clause}
\label{sec:clause-mode}

The mode of relative clause varies depending on which argument has been
extracted. For example, subject extraction can occur only in the
indicative, infinitive, and gerundive modes, as can be seen from examples
like \ex{1}--\ex{4}. Object extraction can only occur in the indicative and
infinitive modes, as shown in examples \ex{5}--\ex{8}. This restriction is
implemented by setting the {\bf S$_r$.t:$<$mode$>$} feature to the
appropriate values, such as {\bf ind}, {\bf inf}, {\bf ger},
etc.. Figure~\ref{trans-rel-clause-trees2} shows this restriction
implemented for the relative clause tree with subject extraction in the
transitive tree family.

\enumsentence{the dog that [ $\epsilon$ ate the cake ] ({\bf
S$_{r}$.t:$\langle$mode$\rangle =$ind})}
\enumsentence{the dog [ $\epsilon$ to chase the cat ] ({\bf
S$_{r}$.t:$\langle$mode$\rangle =$inf})}
\enumsentence{the girl [ $\epsilon$ reading the book ] ({\bf
S$_{r}$.t:$\langle$mode$\rangle =$ger})}
\enumsentence{*the woman [ $\epsilon$ seen the sight ] ({\bf
S$_{r}$.t:$\langle$mode$\rangle =$ppart})}
\enumsentence{the toy that [ Miranda likes $\epsilon$ ] ({\bf
S$_{r}$.t:$\langle$mode$\rangle =$ind})}
\enumsentence{the guava for [ Miranda to eat $\epsilon$ ] ({\bf
S$_{r}$.t:$\langle$mode$\rangle =$inf})}
\enumsentence{*the toy [ Miranda liking $\epsilon$ ] ({\bf
S$_{r}$.t:$\langle$mode$\rangle =$ger})}
\enumsentence{*the book [ Muriel torn $\epsilon$ ] ({\bf
S$_{r}$.t:$\langle$mode$\rangle =$ppart})}

The full set of mode restrictions on the different relative clause trees is
as follows:\\

For all non-passive cases of subject extraction, {\bf S$_{r}$.t:$\langle$mode$\rangle =$ ind/ger/inf} ( see \ex{1}--\ex{4}):

\enumsentence{the girl [ $\epsilon$ eating the cake ] ({\bf S$_{r}$.t:$\langle$mode$\rangle =$ger})}
\enumsentence{the girl [ $\epsilon$ to eat the cake ] ({\bf S$_{r}$.t:$\langle$mode$\rangle =$inf})}
\enumsentence{the girl that [ $\epsilon$ ate the cake ] ({\bf S$_{r}$.t:$\langle$mode$\rangle =$ind})}
\enumsentence{*the girl (that/for) [ $\epsilon$ eaten the cake ] ({\bf S$_{r}$.t:$\langle$mode$\rangle =$ppart})}

For all passive cases of subject extraction, {\bf S$_{r}$.t:$\langle$mode$\rangle =$ ind/ger/ppart/inf} (see \ex{1}--\ex{4}):

\enumsentence{the toy that [ $\epsilon$ was broken by the child ] ({\bf S$_{r}$.t:$\langle$mode$\rangle =$ind})}
\enumsentence{the toy [ $\epsilon$ being broken by the child ] ({\bf S$_{r}$.t:$\langle$mode$\rangle =$ger})}
\enumsentence{the toy [ $\epsilon$ to be broken by the child ] ({\bf S$_{r}$.t:$\langle$mode$\rangle =$inf})}
\enumsentence{the toy [ $\epsilon$ broken by the child ] ({\bf S$_{r}$.t:$\langle$mode$\rangle =$ppart})}

Finally, for all cases of non-subject extraction, {\bf S$_{r}$.t:$\langle$mode$\rangle =$ ind/inf} (see \ex{1}--\ex{4}): 

\enumsentence{the book [ John will tear $\epsilon$ ] ({\bf S$_{r}$.t:$\langle$mode$\rangle =$ind})}
\enumsentence{the book for [ John to tear $\epsilon$ ] ({\bf S$_{r}$.t:$\langle$mode$\rangle =$inf})}
\enumsentence{*the book (that/for) [ John tearing $\epsilon$ ] ({\bf S$_{r}$.t:$\langle$mode$\rangle =$ger})}
\enumsentence{*the book (that/for) [ John torn $\epsilon$ ] ({\bf S$_{r}$.t:$\langle$mode$\rangle =$ppart})}

Relative clause formation with {\bf $<$mode$>$=nom/prep} are also
allowed, but only with a covert $_{NP_{w}}$ and an covert COMP. Furthermore,
they can be formed only on the subject of the clause. Some families that
have these additional modes are Tnx0APnx1 \ex{1}, Tnx0ARBPnx1 \ex{2},
Tnx0nx1ARB \ex{3}.

\enumsentence{the accused [ $\epsilon$ void of all hope ] ({\bf S$_{r}$.t:$\langle$mode$\rangle =$prep})} 
\enumsentence{the dog [ $\epsilon$ next to the tree ] ({\bf S$_{r}$.t:$\langle$mode$\rangle =$prep})}
\enumsentence{the road [ $\epsilon$ seven miles away ] ({\bf S$_{r}$.t:$\langle$mode$\rangle =$nom})}

\subsection{Complementizer Selection}
\label{sec:comp-selection}
The {\bf VP.t:$<$assign-comp$>$} feature is assigned values which represent
constraints on COMP selection by the highest verb in the clause. The
feature values are passed up to the {\bf S$_r$} node of the relative clause
by the equation,

\enumsentence{{\bf S$_{r}$.b:$\langle$assign-comp$\rangle=$VP.t:$\langle$assign-comp$\rangle$}}

This ensures proper selection of the appropriate COMP since the auxiliary
tree anchored by each complementizer also has the {\bf $<$assign-comp$>$}
feature with a value appropriate to the particular complementizer in
question (see for example the $\beta$COMPs anchored by {\em that} in
Figure~\ref{that-comp-tree}). Adjunction of any complementizer can
therefore succeed only if the {\bf $<$assign-comp$>$} features in the COMP
tree and the relative clause tree have the same value.

\begin{figure}[ htb ]
\begin{tabular}{c}
\centerline{\psfig{figure=ps/rel_clauses-files/betaCOMPthat.ps,height=7.0cm}}
\end{tabular}
\caption{Tree $\beta$COMPs, anchored by {\it that}}
\label{that-comp-tree}
\end{figure}

So, while the subject extraction tree in
Figure~\ref{trans-rel-clause-trees2} allows {\it that} to adjoin, it
prevents {\it for} from adjoining because the {\bf
S$_r$.b:$<$assign-comp$>$=for} equation in the $\beta$COMPs tree anchored
by {\em for} will fail to unify with the {\bf VP.t:$<$assign-comp$>$=
that/ind\_nil/inf\_nil/ecm} equation, which is coindexed with the {\bf
S$_r$.b:$<$assign-comp$>$} feature in the relative clause tree.


\subsection{Further constraints on the null COMP}
\label{sec:nocomp-mode}

In our analysis, the {\it null} complementizer is not represented in the
structure of the relative clause at all -- realization of the null COMP
implies preventing any COMP from adjoining. However, this requires an
additional set of constraints, both for distributional and implementational
reasons. For example, the null COMP is not permitted in cases of subject
extraction with {\bf $<$mode$>$=ind} unless there is an intervening
clause. The evidence can be seen in \ex{1}-\ex{4}, especially in the
contrast between \ex{1} and \ex{2}.

\enumsentence{*the toy [ $\epsilon$$_{i}$ [ $\epsilon$$_{i}$ likes Dafna ]]] ({\bf $<$mode$>$=ind})}
\enumsentence{the toy [ Mary said [ $\epsilon$$_{i}$ likes Dafna ]]] ({\bf $<$mode$>$=ind})}
\enumsentence{the boy [ $\epsilon$$_{i}$ $\epsilon$$_{i}$ eating the guava ]]] ({\bf $<$mode$>$=ger})}
\enumsentence{the guava [ $\epsilon$$_{i}$ [ $\epsilon$$_{i}$ eaten by the
boy ]]] ({\bf $<$mode$>$=ppart})}
\enumsentence{the boy [ $\epsilon$$_{i}$ [ $\epsilon$$_{i}$ to eat the
guava ]]] ({\bf $<$mode$>$=inf})}
\enumsentence{the guava [ $\epsilon$$_{i}$ [ $\epsilon$$_{i}$ next to the tree ]]] ({\bf $<$mode$>$=prep})}
\enumsentence{the boy [ $\epsilon$$_{i}$ [ $\epsilon$$_{i}$ seven miles away ]]] ({\bf $<$mode$>$=nom})}

To model this paradigm, the feature {\bf $\langle$nocomp-mode$\rangle$} is
used in conjunction with the following equations.\footnote{%
%
The {\bf S$_{r}$.t:$\langle$nocomp-mode$\rangle$} value given here appears
in the relative clause trees with subject extraction. Trees with other
constituents extracted will have different values for this feature. For
example, in object extraction trees, this feature has the value {\bf ind}.%
%
}

\begin{itemize}

\item {\bf S$_{r}$.t:$\langle$nocomp-mode$\rangle =$ inf/ger/ppart} (in
relative clause trees with subject extraction)
\item {\bf S$_{r}$.b:$\langle$nocomp-mode$\rangle =$
S$_{r}$.b:$\langle$mode$\rangle$}

\end{itemize}

Given the two equations above, successful unification of the {\bf
S$_r$.t:$<$nocomp-mode$>$} and {\bf S$_r$.b:$<$nocomp-mode$>$} features
implies realization of the null COMP, which, in the subject extracted
relative clauses (see Figure~\ref{trans-rel-clause-trees2}), is possible
only if the relative clause is in the {\bf inf}, {\bf ger}, or {\bf ppart}
mode (see examples above). Since the {\it that} $\beta$COMPs tree selects a
clause in the indicative mode (See Figure~\ref{that-comp-tree}), {\it that}
will never be able to adjoin to a relative clause with the subject
extracted. However, if a clause adjoins first to the relative clause, as
would be the case in \ex{-5}, this adjunction puts the {\bf
S$_r$.t:$<$nocomp-mode$>$} and {\bf S$_r$.b:$<$nocomp-mode$>$} in different
nodes, thus preventing a feature clash between

The above feature equations also permit the mode of the relative clause to
be {\bf ind} just in case there is an intervening clause, as in
\ex{-5}. Adjunction of the clause puts the {\bf S$_r$.t:$<$nocomp-mode$>$}
and {\bf S$_r$.b:$<$nocomp-mode$>$} in different nodes, thus preventing a
unification failure. Note, however, that the feature mismatch induced by
the above equations is not remedied by adjunction of just any S-adjunct
since all other S-adjuncts are transparent to the {\bf
$\langle$nocomp-mode$\rangle$} feature because of the following equation,

\begin{itemize}
\item {\bf S$_{m}$.b:$\langle$nocomp-mode$\rangle =$
S$_{f}$.t:$\langle$nocomp-mode$\rangle$}
\end{itemize}

where {\bf S$_{f}$.t} is in the foot node of the adjoining adjunct.


The obligatory adjunction of complementizers implemented above for subject
extracted relative clauses contrasts with what we do with COMP adjunction
in subject extracted questions, where we disallow COMP from adjoining to
the embedded S ({\it *Who did Miranda say that likes Zed?}). We are thus
able to capture the facts related to {\it that-trace} violations in
English.\footnote{%
%
See Chapter~\ref{scomps-section} for a more detailed discussion related to
the {\it that-trace} violation.%
%
}

\section{External syntax}
A relative clause can combine with the NP it modifies in at least 
the following two ways:

\enumsentence{[ the [ toy [ $\epsilon$$_{i}$ [ Dafna likes $\epsilon$$_{i}$ ]]]]}
\enumsentence{[[ the toy ] [ $\epsilon$$_{i}$ [ Dafna likes $\epsilon$$_{i}$ ]]]}

Based on cases like \ex{1} and \ex{2}, which are problematic for the
structure in \ex{-1}, the structure in \ex{0} is
adopted.

\enumsentence{ [[ the man and the woman ] [ who met on the bus ]]}
\enumsentence{ [[ the man and the woman ] [ who like each other ]]} 

\begin{figure}[ htb ]
\begin{tabular}{cc}
\centerline{\psfig{figure=ps/rel_clauses-files/NbetaDnx.ps,height=10.0cm}}
\end{tabular}
\label{trans-rel-clause-trees3}
\caption{Determiner tree with {\bf $<$rel-clause$>$} feature: $\beta$Dnx}
\end{figure}

As it stands, the relative clause analysis sketched so far will combine in
two ways with the Determiner tree shown in
Figure~(\ref{trans-rel-clause-trees3}),%
%
\footnote{The determiner tree shown has the {\bf $<$rel-clause$>$} feature
built in. The relative clause analysis would give two parses in the absence
of this feature.%
%
} giving us both the possiblities shown in (\ref{n-attach-ex}) and
(\ref{np-attach-ex}). In order to block the structure exemplified in
(\ref{n-attach-ex}), the feature {\bf $\langle$rel-clause$\rangle$} is used
in combination with the following equations.

\enumsentence{{\bf NP$_{r}$.b:$\langle$rel-clause$\rangle=+$} on the
Relative Clause}, and 

\enumsentence{{\bf NP$_{f}$.t:$\langle$rel-clause$\rangle=-$} on the Determiner tree.}

Together, these equations block introduction of the determiner above the
relative clause.

\section{Other Issues}

\subsection{Reduced Relatives}
The analysis presented above accounts for reduced relatives (which are
commonly treated as derived from relative clauses through deletion of the
relative pronoun and if there is a {\em be}, then deletion of that
also). Reduced relatives are permitted only in cases of subject-extraction.
Past participial reduced relatives are only permitted on passive clauses.
See \ex{1}-\ex{8}.


\enumsentence{
the toy [ $\epsilon$$_{i}$ [ $\epsilon$$_{i}$ playing the banjo ]]]
}
\enumsentence{
*the instrument [ $\epsilon$$_{i}$ [ Amis playing $\epsilon$$_{i}$ ]]]
}
\enumsentence{
*the day [ $\epsilon$$_{w}$ [ Amis playing the banjo ]]]
}
\enumsentence{
the apple [ $\epsilon$$_{i}$ [ $\epsilon$$_{i}$ eaten by Dafna ]]]
}
\enumsentence{
*the child [ $\epsilon$$_{i}$ [ the apple eaten by $\epsilon$$_{i}$ ]]]
}
\enumsentence{
*the day [ $\epsilon$$_{w}$ [ Amis eaten the apple ]]]
}
\enumsentence{
*the apple [ $\epsilon$$_{i}$ [ Dafna eaten $\epsilon$$_{i}$ ]]]
}
\enumsentence{
*the child [ $\epsilon$$_{i}$ [ $\epsilon$$_{i}$ eaten the apple ]]]
}

These restrictions are built into the {\bf $<$mode$>$} specifications
of {\bf S$_r$.t}, as explained in Section~\ref{sec:mode-restriction}.

\subsubsection{Restrictive vs. Non-restrictive relatives}

The English XTAG grammar does not contain any  syntactic distinction between
restrictive and non-restrictive relatives because we believe this to
be a semantic and/or pragmatic difference.



\subsection{Stacking of Complementizers}

Complementizers are prevented from stacking, as in example \ex{1}, just as
in sentential complementation.

\enumsentence{*the book [ $\epsilon$$_w$$_i$ [ that [ that [ Muriel wrote
$\epsilon$$_i$ ]]]]}

\subsection{Adjunction on PRO}
Adjunction on PRO, which would yield the ungrammatical \ex{1} is blocked.

\enumsentence{*I want [[PRO [ who Muriel likes ] to read a book ]].}

This is done by specifying the {\bf $<$case$>$} feature of {\bf NP$_{f}$} to be
{\bf nom/acc}. The {\bf $<$case$>$} feature of PRO is {\bf none}. This
leads to a feature clash and blocks adjunction of relative clauses on to
PRO.

\subsection{Adjunct relative clauses}
\label{sec:adju-RC}
Two types of trees to handle adjunct relative clauses exist in the XTAG
grammar: one in which there is {\bf PP$_{w}$} substitution and one in which
there is a null {\bf NP$_{w}$} built in and a {\bf COMP} adjoins in. There
is no {\bf NP$_{w}$} substitution tree. This is because of the contrast
between \ex{1} and \ex{2}.  

\enumsentence{the day [[on whose predecessor ] [ Muriel left ]]]}
\enumsentence{*the day [[whose predecessor ] [ Muriel left ]]]}

In general, adjunct relatives are not possible with an overt {\bf
NP$_{w}$}.  We do not consider \ex{1} and \ex{2} to be counterexamples
to the above statements because we consider {\em where} and {\em when} to
be exhaustive {\bf PP}s that head a {\bf PP} initial tree.

\enumsentence{the place [ where [ Muriel wrote her first book ]]]}
\enumsentence{the time [ when [ Muriel lived in Bryn Mawr ]]]}

\subsection{ECM}
Cases where {\em for} assigns exceptional case (cf. \ex{1}, \ex{2}) are
handled, again parallel to the way ECM is done in sentential complementation.

\enumsentence{a book [ $\epsilon$$_{w_{i}}$ [ for [ Muriel to read $\epsilon$$_{i}$ ]]]}
\enumsentence{the time [ $\epsilon$$_{w_{i}}$ [ for [ Muriel to leave Haverford ]]]}

The assignment of case by {\em for} is implemented by a combination of the
following equations:

\enumsentence{{\bf S$_{r}$.b:$\langle$assign-case$\rangle$=acc} (in the {\it for} $\beta$COMPs tree)}
\enumsentence{{\bf S$_{r}$.b:$\langle$assign-case$\rangle =$ NP$_{0}$.t:$\langle$case$\rangle$} (in the Relative clause tree)}

\section{Cases not handled}
\subsection{Partial treatment of free-relatives}
Free relatives are only partially handled. All free relatives on non-subject
positions and some free relatives on subject positions 
are handled. The structure assigned 
to free relatives treats the extracted {\em wh}-NP as the head NP of
the relative clause. The remaining relative clause modifies this
extracted {\em wh}-NP (cf. \ex{1}-\ex{3}).

\enumsentence{what(ever) [ $\epsilon$$_{w_{i}}$ [ Mary likes $\epsilon$$_{i}$ ]]]}
\enumsentence{where(ever) [ $\epsilon$$_{w}$ [ Mary lives ]]]}
\enumsentence{who(ever) [ $\epsilon$$_{w_{i}}$ [ Muriel thinks [ $\epsilon$$_{i}$ likes Mary ]]]]}

However, simple subject extractions without further emebedding are not
handled (cf. \ex{1}).

\enumsentence{who(ever) [ $\epsilon$$_{w_{i}}$ [ $\epsilon$$_{i}$ likes Bill ]]]}
This is because \ex{0} is treated exactly like the ungrammatical \ex{1}.
\enumsentence{*the person [ $\epsilon$$_{w_{i}}$ [ $\epsilon$$_{i}$ likes Bill ]]]}


\subsection{Adjunct P-stranding}
The following cases of adjunct preposition stranding are not handled 
(cf. \ex{1}, \ex{2}).

\enumsentence{the pen Muriel wrote this letter with}
\enumsentence{the street Muriel lives on}

Adjuncts are not built into elementary trees in XTAG. So there is no
clean way to represent adjunct preposition stranding. A better
solution might, probably, available if we make use of multi-component
adjunction. 

\subsection{Overgeneration}
The following types of ungrammatical examples are currently accepted by
the XTAG grammar. This is because no clean and conceptually attractive way
of ruling them out is obvious to us.

\subsubsection{{\em how} as {\em wh}-NP}
In standard American English, {\em how} is not acceptable as a 
relative pronoun (cf. \ex{1}).

\enumsentence{*the way [ how [ PRO to solve this problem ]]]}

However, \ex{0} is accepted by the current grammar.
The only way to rule \ex{0} out would be to introduce a special feature
devoted to this purpose. This is unappealing. Further, there exist
speech registers/dialects of English, where \ex{0} is acceptable. 

\subsubsection{Internal head constraint}
Relative clauses in English (and in an overwhelming number of languages)
obey a `no internal head' constraint. This constraint is exemplified in
the contrast between \ex{1} and \ex{2}.

\enumsentence{the person [ who$_{i}$ Muriel likes $\epsilon_i$ ]]}
\enumsentence{*the person [[which person ]$_{i}$ Muriel likes $\epsilon_i$ ]]}

We know of no good way to rule \ex{0} out, while still ruling \ex{1} in.
\enumsentence{the person [[whose mother ]$_{i}$ Muriel likes $\epsilon_i$ ]]}

Dayal (1996) suggests that `full' NPs such as {\em which person} and
{\em whose mother} are R-expressions while {\em who} and {\em whose}
are pronouns. R-expressions, unlike pronouns, are subject to Condition C.
\ex{-2} is, then, ruled out as a violation of Condition C since {\em 
the person} and {\em which person} are co-indexed and {\em the person}
c-commands {\em which person}. If we accept Dayal's argument, we 
have a principled reason for allowing overgeneration of relative clauses
that violate the internal head constraint, the reason being that 
the XTAG grammar does generate binding theory violations.

\subsubsection{Overt COMP constraint on stacked relatives}
Stacked relatives of the kind in \ex{1} are handled.

\enumsentence{ [[the book [ that Bill likes ]] [ which Mary wrote ]]}

However, there is a constraint on stacked relatives: all but the relative
clause closest to the head-NP must have either an overt {\bf COMP} or an
overt {\bf NP$_{w}$}. Thus \ex{1} is ungrammatical.

\enumsentence{*[[the book [ that Bill likes ]] [ Mary wrote ]]}

We currently know of no good way of handling this constraint, and \ex{0} is
incorrectly accepted by XTAG.

\chapter{Adjunct Clauses}
\label{adjunct-cls}
\label{sub-conj}

Adjunct clauses include subordinate clauses (i.e. those with overt
subordinating conjunctions), purpose clauses and participial adjuncts.

Subordinating conjunctions each select four trees, allowing them to
appear in four different positions relative to the matrix clause.  The
positions are (1) before the matrix clause, (2) after the matrix
clause, (3) before the VP, surrounded by two punctuation marks, and
(4) after the matrix clause, separated by a punctuation mark. Each of
these trees is shown in Figure \ref{sub-conj-trees}.

\begin{figure}[htb]
\centering
\begin{tabular}{cccc}
\psfig{figure=ps/sent-adjs-files/Pss.ps,height=2.1in}&
\psfig{figure=ps/sent-adjs-files/vxPNs.ps,height=2.1in}&
\psfig{figure=ps/sent-adjs-files/puPPpuvx.ps,height=2.1in}&
\psfig{figure=ps/sent-adjs-files/spuPs.ps,height=2in}\\
(1) $\beta$Pss & (2) $\beta$vxPNs & (3) $\beta$puPPspuvx & (4) $\beta$spuPs \\
\end{tabular}
\caption{Auxiliary Trees for Subordinating Conjunctions}
\label{sub-conj-trees}
\end{figure}

Sentence-initial adjuncts adjoin at the root S of the matrix clause,
while sentence-final adjuncts adjoin at a VP node. In this, the XTAG
analysis follows the findings on the attachment sites of adjunct
clauses for conditional clauses (\cite{iatridou91}) and for
infinitival clauses (\cite{Browning87}). One compelling argument is
based on Binding Condition C effects.  As can be seen from examples
(\ex{1})-(\ex{3}) below, no Binding Condition violation occurs when
the adjunct is sentence initial, but the subject of the matrix clause
clearly governs the adjunct clause when it is in sentence final
position and co-indexation of the pronoun with the subject of the
adjunct clause is impossible.

\enumsentence{Unless she$_i$ hurries, Mary$_i$ will be late for the meeting.}
\enumsentence{$\ast$She$_i$ will be late for the meeting unless Mary$_i$ hurries.}
\enumsentence{Mary$_i$ will be late for the meeting unless she$_i$ hurries.}

%Tree families with direct objects also contain a pair for the passive trees,
%and the transitive family (Tnx0Vnx1) contains a pair for the ergative
%trees. All of these trees are anchored by the main verb of the adjunct clause,
%and adjoin either at S or VP to the matrix clause.  Subordinating conjunctions
%adjoin to these sentential adjunct trees, as described in section
%\ref{sub-conj} below.  If no conjunction adjoins, only certain modes are
%licensed for the adjunct clause.  These are described immediately below.

We had previously treated subordinating conjunctions as a subclass of
{\em conjunction}, but are now assigning them the POS {\em
preposition}, as there is such clear overlap between words that
function as prepositions (taking NP complements) and subordinating
conjunctions (taking clausal complements). While there are some
prepositions which only take NP complements and some which only take
clausal complements, many take both as shown in examples
(\ex{1})-(\ex{4}), and it seems to be artificial to assign them two
different parts-of-speech.

\enumsentence{Helen left before the party.}
\enumsentence{Helen left before the party began.}
\enumsentence{Since the election, Bill has been elated.}
\enumsentence{Since winning the election, Bill has been elated.}

Each subordinating conjunction selects the values of the {\bf
$<$mode$>$} and {\bf $<$comp$>$} features of the subordinated S. The
{\bf $<$mode$>$} value constrains the types of clauses the
subordinating conjunction may appear with and the {\bf $<$comp$>$}
value constrains the complementizers which may adjoin to that
clause. For instance, indicative subordinate clauses may appear with
the complementizer {\it that} as in (\ex{1}), while participial
clauses may not have any complementizers (\ex{2}).

\enumsentence{Midge left that car so that Sam could drive to work.}
\enumsentence{*Since that seeing the new VW, Midge could think of
nothing else.}

\subsection{Multi-word Subordinating Conjunctions}

We extracted a list of multi-word conjunctions, such as {\it as if},
{\it in order}, and {\it for all (that)}, from \cite{quirk85}. For the
most part, the components of the complex are all anchors, as shown in
Figures~\ref{conjs}(a). In one case, {\it as ADV as}, there is a great
deal of latitude in the choice of adverb, so this is a substitution
site (Figures~\ref{conjs}(b)). This multi-anchor treatment is very
similar to that proposed for idioms in \cite{AS89}, and the analysis
of light verbs in the XTAG grammar (see section~\ref{nx0lVN1-family}).

\begin{figure}[htb]
\centering
\begin{tabular}{ccc}
\psfig{figure=ps/sent-adjs-files/vxPARBPs.ps,height=2.7in}&
\hspace*{0.5in} &
\psfig{figure=ps/sent-adjs-files/vxParbPs.ps,height=2.7in}\\
(a)&\hspace*{0.5in} &(b)\\
\end{tabular}
\caption{Trees Anchored by Subordinating Conjunctions:  $\beta$vxPARBPs and $\beta$vxParbPs}
\label{conjs}
\end{figure}

\section{``Bare'' Adjunct Clauses}

``Bare'' adjunct clauses do not have an overt subordinating
conjunction, but are typically parallel in meaning to clauses with
subordinating conjunctions. For this reason, we have elected to handle
them using the same trees shown above, but with null anchors. They are
selected at the same time and in the same way the {\it PRO} tree is,
as they all have {\it PRO} subjects.  Three values of {\bf $<$mode$>$}
are licensed: {\bf inf} (infinitive), {\bf ger} (gerundive) and {\bf
ppart} (past participal).\footnote{We considered allowing bare
indicative clauses, such as {\it He died that others may live}, but
these were considered too archaic to be worth the additional ambiguity
they would add to the grammar.} They interact with complementizers as
follows:

\begin{itemize}
\item Participial complements do not license any
complementizers:\footnote{While these sound a bit like extraposed
relative clauses (see \cite{kj87}), those move only to the right and
adjoin to S; as these clauses are equally grammatical both
sentence-initially and sentence-finally, we are analyzing them as
adjunct clauses.}

\enumsentence{[Destroyed by the fire], the building still stood.}
\enumsentence{The fire raged for days [destroying the building].}
\enumsentence{$\ast$[That destroyed by the fire], the building
still stood.}
%what about: if destroyed by fire, the building would have been rebuilt?

\begin{figure}[htb]
\begin{tabular}{cc}
\psfig{figure=ps/sent-adjs-files/destroyed-by-fire.ps,height=2.7in}&
\psfig{figure=ps/sent-adjs-files/destroying-the-building.ps,height=2.7in}\\
(a)&(b)
\end{tabular}
\caption{Sample Participial Adjuncts}
\label{destroyed}
\end{figure}

\item Infinitival adjuncts, including purpose clauses, are licensed both with and without the complementizer
{\it for}.
\enumsentence{Harriet bought a Mustang [to impress Eugene].}
\enumsentence{[To impress Harriet], Eugene dyed his hair.}
\enumsentence{Traffic stopped [for Harriet to cross the street].}
\end{itemize}

\section{Discourse Conjunction}

The CONJs auxiliary tree is used to handle `discourse' conjunction,
as in sentence (\ex{1}).  Only the coordinating conjunctions ({\it
and, or} and {\it but}) are allowed to adjoin to the roots of
matrix sentences. Discourse conjunction with {\it and} is shown in the
derived tree in Figure~\ref{seuss-sentence}.

\enumsentence{And Truffula trees are what everyone needs! \cite{seuss71}}

\begin{figure}[htbp]
\centering
\hspace{0in}
\psfig{figure=ps/sent-adjs-files/disc-conj.ps,height=4.5in}
\caption{Example of discourse conjunction, from Seuss' {\it The
Lorax}\protect\nocite{seuss71}}
\label{seuss-sentence}
\end{figure}






 
\chapter{Imperatives} 
\label{imperatives} 
 
\section{Agreement, mode, and the optional subject} 
 
Imperatives in English do not require overt subjects.  The subject in 
imperatives is in general second person, i.e.\ {\it you}, whether it is 
overt or not, as is clear from the verbal agreement and the interpretation. 
The imperatives in which the subject is not overt are handled by the 
imperative trees discussed in this section.  Imperatives with overt 
subjects are not handled currently.  More discussion on imperatives with 
overt subjects is given in Sections \ref{sec:vocative} and 
\ref{sec:overt-subject}. 
 
The imperative trees in each tree family in the English XTAG grammar are 
identical to the declarative tree of that family except that the NP$_{0}$ 
subject position is filled by an $\epsilon$, the NP$_{0}$ {\bf $<$agr~pers$>$} feature is set to the value {\bf 2nd} and the {\bf $<$mode$>$} feature on the root node has the value {\bf imp} (see equations 
\ref{ex:291} -- \ref{ex:292}). Hardwiring the {\bf $<$agr~pers$>$} feature into the 
tree ensures the proper verbal agreement for an imperative.  The {\bf $<$mode$>$} value of {\bf imp} on the root node is recognized as a valid 
mode for a matrix clause.\footnote{% % The other valid {\bf $<$mode$>$} for a matrix clause is {\bf ind}.% % } The {\bf imp} value for {\bf $<$mode$>$} also prevents imperatives from 
appearing as embedded clauses.  Figure \ref{alphaInx0Vnx1} shows the 
imperative tree in the transitive tree family. 
 
\beginsentences
\sitem{{\bf NP$_0$.t:$<$agr~pers$>$ = 2}}\label{ex:291} 
\sitem{{\bf S$_r$.b:$<$mode$>$ = imp}}\label{ex:292} 
\endsentences

 
 
\begin{rawhtml} <p> \end{rawhtml}
\centering{ \begin{tabular}{c} \htmladdimg{ps/imperatives-files/alphaInx0Vnx1.ps.gif} \end{tabular} } 
\begin{rawhtml} <dl> <dt>{Transitive imperative tree: $\alpha$Inx0Vnx1 <p> </dl> \end{rawhtml}
\label{alphaInx0Vnx1} 
\label{2;11,1} 
\begin{rawhtml} <p> \end{rawhtml}
 
 
Moreover, the {\bf $<$mode$>$} feature on the anchor is unspecified, and 
the {\bf $<$mode$>$} feature on the top feature structure associated with 
the VP has the value {\bf base} (see equation in \ref{ex:293}). 
 
\beginsentences
\sitem{{\bf VP.t:$<$mode$>$ = base}}\label{ex:293} 
\endsentences

 
This allows the lexical verb of the imperative to be any type of verbs, as 
long as the left-most verb has {\bf $<$mode$>$ = base}.  For instance, in a 
simple transitive imperative as in \ref{ex:294}, the verb {\it eat}, which is 
specified with {\bf $<$mode$>$ = base}, anchors the imperative tree, 
unifying with {\bf VP.t:$<$mode$>$ = base}.  In an imperative with 
auxiliary {\it be} as in \ref{ex:295}, the verb {\it waiting}, which is specified 
with {\bf $<$mode$>$ = ger}, anchors the imperative tree, and the auxiliary 
{\it be}, which is specified with {\bf $<$mode$>$ = base}, adjoins onto the 
VP, unifying with {\bf VP.t:$<$mode$>$ = base}. 
 
\beginsentences
\sitem{Eat the cake!}\label{ex:294} 
\sitem{Be waiting for me!}\label{ex:295} 
\endsentences

 
 
\section{Negative Imperatives} 
\label{neg-imp} 
 
\subsection{{\it Don't} imperatives} 
 
All Negative imperatives in English require {\it do}-support, even those 
that are formed with {\it be} and auxiliary {\it have}. 
 
\beginsentences
\sitem{Dont' leave!}\label{ex:296} 
\sitem{*Not leave!}\label{ex:297} 
\endsentences

 
\beginsentences
\sitem{Do not open the window!}\label{ex:298} 
\sitem{*Not open the window!}\label{ex:299} 
\endsentences

 
\beginsentences
\sitem{Do not be talking so loud!}\label{ex:300} 
\sitem{*Not be talking so loud!}\label{ex:301} 
\endsentences

 
\beginsentences
\sitem{Don't have eaten everything before the guests arrive!}\label{ex:302} 
\sitem{*Not have eaten everything before the guests arrive!}\label{ex:303} 
\endsentences

 
In English XTAG grammar, negative imperatives receive a similar structural analysis 
to {\it yes-no} questions, as in \cite{potsdamdiss97} and \cite{handiss}. 
That is, {\it do} and {\it don't} in negative imperatives are treated as an 
instance of {\it do}-support and adjoin to a clause.  The crucial 
strucural evidence for our analysis is that when there is an overt subject 
in negative imperatives formed with {\it don't}, the subject must follow 
{\it don't}, just as it does in {\it yes-no} questions. 
 
\beginsentences
\sitem{Don't you worry!}\label{ex:304} 
\sitem{Don't you move!}\label{ex:305} 
\endsentences

 
\beginsentences
\sitem{Don't you like carrots?}\label{ex:306} 
\sitem{Didn't you finish your paper yet?}\label{ex:307} 
\endsentences

 
{\it Do}-support in negative imperatives is handled by the elementary tree 
$\beta$IVs anchored by {\it do} and {\it don't}, as shown in Figure 
\ref{fig:doimp}.  This tree adjoins onto the root node of the imperative 
tree.  The feature {\bf $<$mode$>$ =imp} on the S foot node restricts this 
tree to adjoin only to imperative trees. Furthermore, the S root node of 
$\beta$IVs is specified with {\bf $<$mode$>$ =imp}, which prevents 
imperatives with {\it do}-support from appearing as embedded clauses. 
 
\begin{rawhtml} <p> \end{rawhtml}
\centering 
\begin{tabular}{ccc} 
{\htmladdimg{ps/imperatives-files/betaIVs-do.ps.gif}} & 
{\ } & 
{\htmladdimg{ps/imperatives-files/betaIVs-dont.ps.gif}} \\ 
$\beta$IVs[do] & {\ } & $\beta$IVs[don't] 
\end{tabular} 
\begin{rawhtml} <dl> <dt>{Trees anchored by  do and  don't <p> </dl> \end{rawhtml}
\label{fig:doimp} 
\begin{rawhtml} <p> \end{rawhtml}
 
In negative imperatives formed with {\it don't}, the $\beta$IVs[don't] tree in 
Figure \ref{fig:doimp} adjoins to the root node of the imperative tree. The 
derived tree for the negative imperative {\it Don't leave!} is given in 
Figure \ref{fig:dont-leave}. 
 
\begin{rawhtml} <p> \end{rawhtml}
  \begin{center} \leavevmode \htmladdimg{ps/imperatives-files/dont-leave.ps.gif} 
  \end{center} 
  \begin{rawhtml} <dl> <dt>{Derived tree for  Don't leave! <p> </dl> \end{rawhtml}
\label{fig:dont-leave} 
\begin{rawhtml} <p> \end{rawhtml}
 
\subsection{{\it Do not} imperatives} 
 
In negative imperatives formed with {\it do not}, the $\beta$IVs[do] tree 
in Figure \ref{fig:doimp} adjoins to the root node of the imperative tree 
and the $\beta$NEGvx tree that anchors {\it not} as represented in Figure 
\ref{fig:not} adjoins to the VP node of the imperative tree. 
 
\begin{rawhtml} <p> \end{rawhtml}
  \begin{center} \leavevmode 
\htmladdimg{ps/imperatives-files/betaNEGvx-not.ps.gif} 
  \end{center} 
  \begin{rawhtml} <dl> <dt>{Tree anchored by  not <p> </dl> \end{rawhtml}
\label{fig:not} 
\begin{rawhtml} <p> \end{rawhtml}
 
The  derived tree for the negative imperative {\it Do not eat the cake!} are 
given in Figure \ref{fig:do-not-leave}. 
 
\begin{rawhtml} <p> \end{rawhtml}
\begin{center} \leavevmode 
{\htmladdimg{ps/imperatives-files/do-not-leave1.ps.gif}} 
\end{center} 
\begin{rawhtml} <dl> <dt>{Derived trees for  Do not eat the cake! <p> </dl> \end{rawhtml}
\label{fig:do-not-leave} 
\begin{rawhtml} <p> \end{rawhtml}
 
Note that trees in Figure \ref{fig:dont-leave} and Figure 
\ref{fig:do-not-leave} have an empty verb.  This is due to the feature {\bf $<$displ-const$>$} in $\beta$IVs.\footnote{% % The other possibility of adjoining {\it not} above the VP that projects from the empty verb is ruled out because {\it not} is made to select a VP with the following equation: {\bf $<$displ-const set1=-$>$}.  Since the empty verb tree has {\bf $<$displ-const set1=+$>$}, {\it not} cannot adjoin onto the VP that projects from it.% % } This feature ensures that when $\beta$IVs is adjoined to an elementary 
tree, $\beta$Vvx that anchors an empty verb must also adjoin onto the VP of 
that same elementary tree.  This tree is represented in Figure 
\ref{fig:epsilon}.  The empty verb represents the originating position of 
{\it do} and {\it don't}.  This mechanism is also used in interrogatives 
that have subject-verb inversion to simulate auxiliary verb movement.  For 
more on this, see Chapter~\ref{auxiliaries} on {\it do}-support and 
inversion. 
 
\begin{rawhtml} <p> \end{rawhtml}
  \begin{center} \leavevmode \htmladdimg{ps/imperatives-files/betaVvx-epsilon.ps.gif} 
  \end{center} 
  \begin{rawhtml} <dl> <dt>{$\beta$Vvx[$\epsilon$] <p> </dl> \end{rawhtml}
\label{fig:epsilon} 
\begin{rawhtml} <p> \end{rawhtml}
 
 
If {\it do} in negative imperatives is in the same position as {\it do} in 
{\it yes-no} questions, the fact that an overt subject cannot intervene 
between {\it do} and {\it not} is puzzling. 
 
\beginsentences
\sitem{Do not open the window!}\label{ex:308} 
\sitem{*Do you not open the window!}\label{ex:309} 
\endsentences

 
We adopt the account given in \cite{akmajian84} that this fact is not due 
to syntax but due to an intonational constraint in imperatives.  He argues 
that (i) when an imperative sentence has an overt subject, the subject must be 
the only intonation center preceding the verb phrase and (ii) that in 
negative imperatives with {\it do} and {\it not}, either {\it do} or {\it not} must be the intonation center.  These two contraints conspire to rule 
out {\it do not} imperatives with an overt subject. 
 
\subsection{Negative Imperatives with {\it be} and {\it have}} 
 
Another puzzling fact that needs to be explained  is that in negative 
imperatives even {\it be} and auxiliary {\it have} require {\it do}-support, while it is prohibited in negative declaratives and negative 
questions.  
 
\beginsentences
\sitem{He isn't talking loud.}\label{ex:310} 
\sitem{*He doesn't be talking loud.}\label{ex:311} 
\endsentences

 
\beginsentences
\sitem{Isn't he talking loud?}\label{ex:312} 
\sitem{*Doesn't he be talking loud?}\label{ex:313} 
\endsentences

 
This fact does not pose a problem for the XTAG analysis of negative 
imperatives if we adopt the line of approach given in \cite{handiss}.  She 
points out that while declaratives and questions are tensed, imperatives 
are not, and argues that this is exactly why negative imperatives require 
{\it do}-support even for {\it be} and auxiliary {\it have}.  Assuming a 
clause structure in which CP dominates IP and IP dominates VP (for 
expository purposes), she argues that it is the tense features in I$^0$ 
that attract {\it be} and auxiliary {\it have} in declaratives and 
questions.  In declaratives, {\it be} or auxiliary {\it have} moves to and 
stays in I$^0$, and in questions, once {\it be} or auxiliary {\it have} 
moves to I$^0$, they further move to C$^0$.  Moreover, main verbs cannot 
move at all to I$^0$ in the overt syntax.  Instead, they undergo movement 
at LF.  But negation blocks LF movement and so as a last resort {\it do} is 
inserted in I$^0$ to support INFL.  In imperatives, I$^0$ does not have 
tense features and so it cannot attract {\it be} and auxiliary {\it have}. 
Thus, {\it be} and auxiliary {\it have} as well as main verbs undergo 
movement at LF in imperatives.  And so in negative imperatives, since 
negation blocks LF verb movement, {\it do} is inserted in I$^0$ as a last 
resort device even for {\it be} and auxiliary {\it have} and it further 
moves to C$^0$ in the overt syntax. 
 
\section{Emphatic Imperatives} 
 
Another case where imperatives have {\it do}-support is emphatic 
imperatives. 
 
\beginsentences
\sitem{Do open the window!}\label{ex:314} 
\sitem{Do show up for the lecture!}\label{ex:315} 
\endsentences

 
In English XTAG grammar, {\it do} in emphatic imperatives is treated just 
as {\it do} in negative imperatives.  It is adjoined to an imperative 
clause with an empty subject.  Again, the crucial evidence for this 
analysis comes from word order facts.  When emphatic imperatives have an 
overt subject, it must follow {\it do}. 
 
\beginsentences
\sitem{Do somebody bring me some water!}\label{ex:316} 
\sitem{Do at least some of you show up for the lecture!}\label{ex:317} 
\endsentences

 
\section{Cases not handled} 
 
\subsection{Overt subjects before {\it do/don't}} 
\label{sec:vocative} 
 
Given our analsis of negative imperatives, if the subject precedes {\it do} 
or {\it don't}, we are forced to treat it as a vocative and not a 
sentential subject.  Vocatives are considered to be outside the clause 
structure and does not have any structural relation with any element in the 
clause. 
 
\beginsentences
\sitem{You don't drink the water. = (You! Don't drink the water!)}\label{ex:318} 
\sitem{You do not leave the room. = (You! Do not leave the room!)}\label{ex:319} 
\endsentences

 
Given the fact that the imperatives in \ref{ex:318} and \ref{ex:319} seem to be 
degraded unless there is an intonational break between {\it you} and the 
rest of the sentence, treating {\it you} as a vocative seems to be the 
correct approach.  Currently, XTAG grammar does not handle vocatives. 
 
\subsection{Overt subjects after {\it do/don't}} 
\label{sec:overt-subject} 
 
One remaining task for imperatives is to handle those with overt subjects 
such as \ref{ex:316} and \ref{ex:317}. 
The type of overt subjects allowed in imperatives are restricted: 2nd 
person pronouns and some quantified noun phrases. Currently, the English 
XTAG grammar only has imperative trees with empty subjects.  
 
\subsection{Passive Imperatives} 
 
Passive imperatives like \ref{ex:320} are currently not handled. 
 
\beginsentences
\sitem{Don't be defeated at the race today!}\label{ex:320} 
\endsentences

 
Accounting for them would probably involve making separate passive trees in 
each tree family, as is done for the declaratives and other clause types in 
each family.\footnote{% % A simpler way to allow for passive imperatives would be remove the equation {\bf V.t:$<$passive$>$ = -} from the imperative trees. However, this option may affect the consistency of treatment of the passives. A decision in this respect will have to be made before implementing the imperative passives.% % } 
 
\subsection{Overgeneration} 
 
As was discussed above, imperatives can also be formed with auxiliaries 
like {\it have} and {\it be}, as in \ref{ex:321}, \ref{ex:322} and \ref{ex:323}: 
 
\beginsentences
\sitem{Be waiting for me when I return!}\label{ex:321} 
\sitem{Don't be sleeping while reading your book!}\label{ex:322} 
\sitem{Don't have fallen asleep when I come back!}\label{ex:323} 
\endsentences

 
The auxiliary {\it be} can form affirmative as well as negative 
imperatives. However, {\it have} can only form a negative imperative, as 
can be seen from \ref{ex:323} and the ungrammaticality of \ref{ex:324}: 
 
\beginsentences
\sitem{* have eaten your meal by the time I return!}\label{ex:324} 
\endsentences

 
The current analysis of imperatives, however, does not rule out \ref{ex:324}. 
 
 
 
 
 
 
 
 
 
 

\chapter{Gerund NP's}
\label{gerunds-chapter}

The puzzle over gerunds in the linguistics literature has been that they seem
to have both NP and clausal properties. That is to say they seem to have
certain clausal properties but occur in positions typically occupied by noun
phrases. The bold face portions of examples (\ex{1})-(\ex{3}) show examples of
gerunds as subjects in (\ex{1}) and (\ex{2}), and as the object of a
preposition in (\ex{3}).


\enumsentence{And {\bf avoiding such losses} will take a monumental
effort. (WSJ)}
\enumsentence{{\bf Mr. Nolen's nocturnal wandering} doesn't make him a
weirdo. (WSJ)}
\enumsentence{Is this a case where private markets are approving of
{\bf Washington's bashing of Wall Street}? (WSJ)}


In the English XTAG grammar we adopt a position similar to that of
\cite{Rosenbaum67} and \cite{Emonds70} - that gerunds are NP's exhaustively 
dominating a clause.  In particular, we found that any place an NP is allowed,
a gerundive clause is also allowed, and no cases in which a verb
subcategorized for gerundive clauses, but not NP's.  

\begin{figure}[htb]
\centering
\begin{tabular}{cc}
{\psfig{figure=ps/gerund-files/alphaDnx0Vnx1.ps,height=3.2in}}&
{\psfig{figure=ps/gerund-files/alphaGnx0Vnx1.ps,height=3.2in}}
\\
(a)&(b)\\
\end{tabular}
\caption{Gerund trees from the transitive tree family: $\alpha$Dnx0Vnx1 (a) and
$\alpha$Gnx0Vnx (b)}
\label{gerund-trees}
\label{2;12,1}
\label{2;13,1}
\end{figure}

Our implementation includes at least two gerundive trees in each tree family
(see Figure~\ref{gerund-trees}).  The gerund trees in a tree family have
basically the form of the declarative for that family but have NP as the
category of their top node.  The Determiner Gerund tree in
Figure~\ref{gerund-trees}(a) has an initial DetP and instantiates the direct
object as a PP. It is used for gerunds such as the one in bold face in
sentence~(\ex{1}).

\enumsentence{Some think {\bf the rapid selling of bonds} has a way to go.}

Notice that the modification of {\it selling of bonds} by the adjective {\it
rapid} supports the choice of N as the label for the node dominating V and
PP$_{1}$.

The NP gerund tree in Figure~\ref{gerund-trees}(b) has exactly the same
structure as the declarative transitive tree except for the root node label and
for feature values.  In particular, the verb is required to be {\bf
$<$mode$>$=ger}, and the subject is required to be {\bf
$<$case$>$=acc/none/gen}, i.e. either an accusative, PRO or genitive NP. The
whole NP formed by the gerund can itself have either nominative or accusative
case. The NP gerund tree is used for gerunds such as the one in bold face in
sentence~(\ex{-3}) and (\ex{-2}).

One question that arises with respect to gerunds is whether there is anything
special about their distribution as compared to other types of NP's.  In fact,
it appears that gerund NP's can occur in any NP position.  Some verbs might not
seem to be very accepting of gerund NP arguments, as in (\ex{1}), but we
believe this to be a semantic incompatibility rather than a syntactic problem
since the same structures are fine with other lexical items.

\enumsentence{?[$_{NP}$John's repairing$_{NP}$] ran.}
\enumsentence{[$_{NP}$John's tinkering$_{NP}$] worked.}

By having the root node of gerund trees be NP, the gerunds have the
same distribution as any other NP in the English XTAG grammar without
doing anything exceptional. The clause structure is captured by the
form of the trees and by inclusion in the tree families.

\part{Other Constructions}
\chapter{Determiners and Noun Phrases}
\label{det-comparitives}

{\sc NB: The determiner analysis in the XTAG grammar has changed, and
this section will soon be updated with the current analysis. Briefly,
in the new analysis determiners adjoin to NPs; this means that there
is only one NP tree for all NPs ($\alpha$NXN). The features still work
exactly as described here.}


Previous approaches to syntactic determiner ordering (e.g.\ \cite{quirk85})
have simply divided determiners into subcategories (predet, det, postdet).
This type of approach is inadequate because it allows ungrammatical sequences
like {\it $\ast$all what no}, and misses the finer distinctions among
particular determiners. These finer distinctions are modeled very naturally in
a lexicalized grammar formalism such as FB-LTAG in which pieces of syntactic
structure and features representing linguistic properties are associated with
individual lexical items.

In the English XTAG grammar,\footnote{This chapter is a shortened version of
\cite{HockeyEgedi94}, which contains a more extensive discussion of this 
analysis.} there are two kinds of basic noun phrases (NP), those that take
determiner phrases\footnote{Henceforth DetP's, not to be confused with DP's as
in the DP Hypothesis.} and those that do not.  Nouns that take (or require)
determiners have a DetP substitution site. Complex DetP's are formed by having
determiners adjoin onto each other. There are two basic determiner trees: an
initial tree and an auxiliary tree.  Figure~\ref{det-trees} shows the initial
and auxiliary trees anchored by the determiner {\it these}.  Since any single
determiner can function as a complete DetP,\footnote{By definition.  Our main
criteria in classifying something as a determiner was that it be able to stand
alone with a noun to form an NP.} every determiner selects the initial tree in
Figure~\ref{det-trees}(a).  Determiners that can modify other determiners also
select the auxiliary tree in Figure~\ref{det-trees}(b).

\begin{figure}[hbt]
\centering
\begin{tabular}{ccc}
{\psfig{figure=ps/det-files/alphaD-these.ps,height=12.3cm}} & 
\hspace{1.0in}&
{\psfig{figure=ps/det-files/betaD-these.ps,height=12.3cm}}\\
(a)&&(b)
\end{tabular}
\caption{Determiner Trees with Features: $\alpha$DXD (a) and $\beta$Ddx (b)}
\label{det-trees}
\end{figure}

The current grammar includes a DetP substitution node in the NP but having
determiners adjoin on has also been proposed in the literature
(\cite{Abeille90:TAG}).  The correct ordering of determiners and reasonable
coverage is possible with either approach. In fact, the core of our analysis is
based on the features and would essentially be the same with adjoined
DetP's. We are currently considering whether there would be any compelling
advantages of an adjunction analysis for the XTAG grammar.

In the XTAG grammar, features are crucial to ordering determiners correctly.
We have identified eight features which are sufficient to order the
determiners.  These features are: {\bf definiteness}, {\bf quantity}, {\bf
cardinality}, {\bf genitive}, {\bf decreasing}, {\bf constancy}, {\bf wh} and
{\bf agr}.  These features have all been previously proposed as semantic
properties of determiners.  The semantic definitions underlying the features
are given below.

\begin{description}

\item[Definiteness:] Possible Values [+/--]. \\
A function f is definite iff f is non-trivial and whenever
f(s)~$\neq~\emptyset$ then it is always the intersection of one or
more individuals.  \cite{KeenanStavi86:LP}

\item[Quantity:]  Possible Values [+/--]. \\
If A and B are sets denoting an NP and associated predicate, respectively; E is
a domain in a model M, and F is a bijection from M$_{1}$ to M$_{2}$, then we
say that a determiner satisfies the constraint of quantity if
Det$_{E_{1}}$AB~$\leftrightarrow$~Det$_{E_{2}}$F(A)F(B). \cite{Partee90:BK}

\item[Cardinality:]  Possible Values [+/--]. \\
A determiner D is cardinal iff D $\in$ cardinal numbers~$\geq$~1.

\item[Genitive:]  Possible Values [+/--]. \\
Possessive pronouns and the possessive morpheme ({\it 's}) are marked {\bf
gen+}; all other nouns are {\bf gen--}.

\item[Decreasing:]  Possible Values [+/--]. \\
A set of Q properties is decreasing iff whenever s$\leq$t and t$\in$Q then
s$\in$Q. A function f is decreasing iff for all properties f(s) is a decreasing
set.

A non-trivial NP (one with a Det node) is decreasing iff its denotation in any
model is decreasing. \cite{KeenanStavi86:LP}

\item[Constancy:] Possible Values [+/--]. \\
If A and B are sets denoting an NP and associated predicate, respectively, and
E is a domain, then we say that a determiner displays constancy if
(A$\cup$B)~$\subseteq$~E~$\subseteq$~E$^{\prime}$ then
Det$_{E}$AB~$\leftrightarrow$~Det$_{E^{\prime}}$AB. Modified from
\cite{Partee90:BK}

\item[Wh:]  Possible Values [+/--]. \\
Interrogative determiners are {\bf wh+}; all other determiners are
{\bf wh--}. 

\item[Agreement:] Possible Values [3sg, 3pl, 3sgpl]. \\
Although English does not have the morphological marking of determiners for
case, gender or number, we hold that most determiners in English are
semantically marked for number.

\end{description}

The initial determiner tree in Figure~\ref{det-trees}(a) shows the appropriate
feature values for the determiner {\it these}, while Table~\ref{det-values}
shows the corresponding feature values of several other common determiners.

%\tiny
\begin{table}[hbt]
\centering
\begin{tabular}{|l||c|c|c|c|c|c|c|c|}
\hline
Det&defin&quan&card&gen&wh&decreas&const&agr\\
\hline
\hline
all&$+$&$+$&$-$&$-$&$-$&$-$&$+$&3pl\\
this&$+$&$-$&$-$&$-$&$-$&$-$&$+$&3sg\\
that&$+$&$-$&$-$&$-$&$-$&$-$&$+$&3sg\\
what&$+$&$-$&$-$&$-$&$+$&$-$&$+$&3sgpl\\
the&$+$&$-$&$-$&$-$&$-$&$-$&$+$&3sgpl\\
every&$+$&$+$&$-$&$-$&$-$&$-$&$+$&3sg\\
each&$+$&$+$&$-$&$-$&$-$&$-$&$+$&3sg\\
any&$-$&$+$&$-$&$-$&$-$&$-$&$+$&3sg\\
a&$-$&$+$&$-$&$-$&$-$&$-$&$+$&3sg\\
no&$+$&$+$&$-$&$-$&$-$&$-$&$+$&3sgpl\\
few&$-$&$+$&$-$&$-$&$-$&$+$&$-$&3pl\\
many&$-$&$+$&$-$&$-$&$-$&$-$&$-$&3pl\\
GEN&$+$&$-$&$-$&$+$&$-$&$-$&$+$&\\
CARD&$+$&$+$&$+$&$-$&$-$&$-$&$+$&3pl\footnotemark\ \\
PART&$-$&&$-$&$-$&$-$&$-$&$+$&\\
\hline
\end{tabular}
 \caption{Determiner Features}
\label{det-values}
\end{table}\addtocounter{footnote}{0}\footnotetext{{except {\it one} which is 3sg.}} 

%\normalsize

In addition to the features that represent their own properties, determiners
that select the auxiliary tree have features to represent the selectional
restrictions these determiners impose on the determiners they modify.  The
selectional restriction features of a determiner appear on the DetP foot node
of the auxiliary tree that the determiner anchors.  The DetP$_{f}$ node in the
auxiliary tree in Figure~\ref{det-trees}(b) shows the selectional feature
restriction imposed by {\it these},\footnote{In addition to this tree, {\it
these} would also anchor another auxiliary tree that adjoins onto {\bf
$<$card$>$=+} determiners.} while Table~\ref{det-ordering} shows the
corresponding selectional feature restrictions of several other determiners.

%\tiny
\begin{table}[hbt]
\vspace*{-2mm}
\centering
\begin{tabular}{|l||c|c|c|c|c|c|c|c|}
\hline
Det&defin&quan&card&gen&wh&decreas&const&agr\\
\hline
\hline
all&$+$&$-$&$-$&&$-$&&&\\
&&&$+$&&&&&\\ \hline
this&$-$&&&&$-$&$+$&&\\
&&&$+$&&&&&\\ \hline
that&$-$&&&&$-$&$+$&&\\
&&&$+$&&&&&\\ \hline
what&$-$&&&&$-$&$+$&&\\
&&&$+$&&&&&\\ \hline
the&$-$&&&&$-$&&$-$&\\ 
&&&$+$&&&&&\\ \hline
every&$-$&&&&$-$&$+$&&\\
&&&$+$&&&&&\\ \hline
each&$-$&&&&$-$&$+$&&\\
&&&$+$&&&&&\\ \hline
any&$-$&&&&$-$&$+$&&\\ 
&&&$+$&&&&&\\ \hline
a&$-$&&&&$-$&$+$&&\\ \hline
many&\multicolumn{8}{c|}{only nouns}\\ \hline
no&\multicolumn{8}{c|}{only nouns}\\ \hline
GEN&\multicolumn{8}{c|}{only nouns}\\ \hline
CARD&\multicolumn{8}{c|}{only nouns}\\ \hline
PART&&&&&$-$&&&\\ \hline
\end{tabular}
\caption{Selectional Restrictions Imposed by Determiners}
\label{det-ordering}
\end{table}

%\normalsize


\section{Wh and Agr Features}
\label{agr-section}
A determiner with a {\bf $<$wh$>$=+} feature is always the leftmost
determiner since no determiners can adjoin onto it.  The presence of a wh+
determiner makes the entire NP wh+, so this feature is always passed through to
the NP node, unlike other features which are considered internal to the
determiner system.

The {\bf $<$agr$>$} feature functions differently from most of the features in
the determiner sequencing system.  Notice that in the auxiliary tree in
Figure~\ref{det-trees}(b), the {\bf $<$agr$>$} feature is the only feature not
passed from the D node to the root DetP node, but is passed instead from the
foot DetP to the root DetP.  In the determiner system, the {\bf $<$agr$>$}
feature is generally propagated from the rightmost determiner (i.e.\ the one
closest to the noun), although some adjoining determiners require that they
also agree with that determiner.  This distinction is captured in XTAG by
having each determiner specify in its lexical entry whether or not its
agreement feature is passed to the root DetP (i.e.\ from the D node to the
DetP$_{r}$ node).

\section{Genitive Constructions}

There are two kinds of genitive constructions: genitive pronouns, and
genitive NP's (which have an explicit genitive marker, {\it 's},
associated with them).  It is clear from examples such as {\it her
favorite artist prefers oils\/} vs. {\it $\ast$favorite artist prefers
oils\/} that genitive pronouns function as determiners and as such,
they sequence with the rest of the determiners.  The features for the
genitives are the same as for other determiners, and are given in
Table~\ref{det-values}.  No {\bf $<$agr$>$} is specified however,
since the number and person of the genitive will depend on its
particular form (e.g.\ {\it my} vs. {\it their}).  Genitives are not
required to agree with either the determiners or the nouns that they
modify.

%Figure of alphaDXnxG-features and betanxGdx-features here 

\begin{figure}[hbt]
\centering
\begin{tabular}{ccc}
{\psfig{figure=ps/det-files/alphaDXnxG-features.ps,height=11.3cm}} & 
\hspace{1.0in}&
{\psfig{figure=ps/det-files/betanxGdx-features.ps,height=11.8cm}}\\
(a)&&(b)
\end{tabular}
\caption{Initial: $\alpha$DXnxG (a) and Auxiliary: $\beta$nxGdx (b) Genitive Determiner Trees}
\label{gen-trees}
\end{figure}


Genitive NP's are particularly interesting because they are potentially
recursive structures.  Complex NP's can easily be embedded in a determiner
phrase.

\enumsentence{[[[John]'s friend from high school]'s uncle]'s mother came to
town .}

There are two things to note in sentence~(\ex{0}).  One is that in embedded
NP's, the genitive morpheme comes at the end of the NP phrase, even if the head
of the NP is at the beginning of the phrase.  The other is that the determiner
of an embedded NP can also be a genitive NP, hence the possibility of recursive
structures.

In the XTAG grammar, the genitive marker, {\it 's}, is separated from the
lexical item that it is attached to and given its own category (G).  In this
way, we can allow the full complexity of NP's to come from the existing NP
system, including any recursive structures.  The two trees in
Figure~\ref{gen-trees} demonstrate how easily the complexity of genitive NP's
are captured in XTAG.  As with the standard determiner trees, there are two
trees - one for the determiner that stands alone and one for a determiner that
adjoins onto another.


\section{Partitive Constructions}

Partitive constructions (e.g.\ {\it some kind of}, {\it all of\/}) are
another type of complex determiner construction.  Partitive constructions
interact with other determiners.  Since they can modify the noun itself ({\it
[a certain kind of] machine}), or modify other determiners ({\it [some parts
of] these machines}), the partitive construction has both an initial and
auxiliary tree that are anchored by the preposition {\it of}. The partitive
trees are shown in Figure~\ref{partitive-trees}.

\begin{figure}[hbt]
\centering
\begin{tabular}{ccc}
{\psfig{figure=ps/det-files/alphaDXnxP.ps,height=11.5cm}} & 
\hspace{1.0in}&
{\psfig{figure=ps/det-files/betanxPdx.ps,height=12.0cm}}\\
(a)&&(b) \\
\end{tabular}
\caption{Initial: $\alpha$DXnxP (a) and Auxiliary: $\beta$nxPdx (b) Partitive trees}
\label{partitive-trees}
\end{figure}



\chapter{Modifiers}
\label{modifiers}

This chapter covers various types of modifiers: adverbs, prepositions,
adjectives, and noun modifiers in noun-noun compounds.\footnote{Relative
clauses are discussed in Chapter~\ref{rel_clauses}.}  These categories
optionally modify other lexical items and phrases by adjoining onto them.  In
their modifier function these items are adjuncts; they are not part of the
subcategorization frame of the items they modify.  Examples of some of these
modifiers are shown in (\ex{1})-(\ex{3}).

\enumsentence{[$_{ADV}$ Certainly $_{ADV}$], the October 13 sell-off
didn't settle any stomachs. (WSJ)}

\enumsentence{Mr. Bakes [$_{ADV}$ previously $_{ADV}$] had a turn at running
Continental. (WSJ)}

\enumsentence{Most [$_{ADJ}$ foreign $_{ADJ}$] [$_{N}$ government
$_{N}$] [$_{N}$ bond $_{N}$] [prices] rose [$_{PP}$ in light
trading $_{PP}$]. (WSJ)}

The trees used for the various modifiers are quite similar in form.  The
modifier anchors the tree and the root and foot nodes of the tree are of the
category that the particular anchor modifies. Some modifiers,
e.g. prepositions, have arguments that are also included in the tree.  The foot
node may be to the right or the left of the anchoring modifier (and its
arguments) depending on whether that modifier occurs before or after the
category it modifies. For example, almost all adjectives appear to the left of
the nouns they modify, while prepositions appear to the right when modifying
nouns.


\section{Adjectives}
\label{adj-modifier}

In addition to being modifiers, adjectives in the XTAG English grammar can be
also anchor clauses (see Adjective Small Clauses in
Chapter~\ref{small-clauses}).  There is also one tree family, Intransitive with
Adjective (Tnx0Va1), that has an adjective as an argument and is used for
sentences such as {\it Seth felt happy}. In that tree family the adjective
substitutes into the tree rather than adjoining as is the case for modifiers.


\begin{figure}[htb]
\centering
\begin{tabular}{cc}
{\psfig{figure=ps/modifiers-files/betaAn-features.ps,height=3.5in}}
\end{tabular}\\
\caption {Standard Tree for Adjective modifying a Noun: $\beta$An}
\label {An-tree}
\end{figure}

As modifiers, adjectives anchor the tree shown in Figure~\ref{An-tree}.  The
features of the N node onto which the $\beta$An tree adjoins are passed through
to the top node of the resulting N.  The null adjunction marker (NA) on the N
foot node imposes right binary branching such that each subsequent adjective
must adjoin on top of the leftmost adjective that has already adjoined.  Due to
the NA constraint, a sequence of adjectives will have only one derivation in
the XTAG grammar. The adjective's morphological features such as superlative or
comparative are not currently used in the tree.  At this point, the treatment
of adjectives in the XTAG English grammar does not include selectional or
ordering restrictions. Consequently, any adjective can adjoin onto any noun and
on top of any other adjective already modifying a noun. All of the modified
noun phrases shown in (\ex{1})-(\ex{4}) currently parse with the same structure
shown for {\it colorless green ideas\/} in Figure
\ref{colorless-green-adj}.

\enumsentence{big green bugs}
\enumsentence{big green ideas}
\enumsentence{colorless green ideas}
\enumsentence{$\ast$green big ideas}


\begin{figure}[htb]
\centering
\begin{tabular}{cc}
{\psfig{figure=ps/modifiers-files/colorless-green-ideas.ps,height=2.3in}}
\end{tabular}\\
\caption {Multiple adjectives modifying a noun}
\label {colorless-green-adj}
\end{figure}


While (\ex{-2})-(\ex{0}) are all semantically anomalous, (\ex{0}) also suffers
from an ordering problem that makes it seem ungrammatical as
well.\footnote{This is, in fact, the point of the famous linguistic example in
(\ex{-1}).} We would argue that the grammar should accept (\ex{-3})-(\ex{-1})
but not (\ex{0}).  One of the future goals for the grammar is to develop a
treatment of adjective ordering similar to that developed by
\cite{HockeyEgedi94} for determiners\footnote{See
Chapter~\ref{det-comparitives} or \cite{HockeyEgedi94} for details of the
determiner analysis.}. An adequate implementation of ordering restrictions for
adjectives would rule out (\ex{0}).

Another area in which we plan to have future grammar development is
comparatives.  Comparatives that involve ellipsis will require a general
solution of the problem of representing ellipsis.  Simpler comparatives without
ellipsis, such as {\it fewer than nine\/} in (\ex{1}), should be amenable to
analysis as complex determiners.


\enumsentence{Cats actually have fewer than nine lives.}


\section{Noun-Noun Modifiers}
\label{noun-modifier}

Noun-noun compounding in the English XTAG grammar is very similar to
adjective-noun modification.  The noun modifier tree, shown in
Figure~\ref{noun-compound-tree}, has essentially the same structure as the
adjective modifier tree in Figure~\ref{An-tree}, except for the syntactic
category label of the anchor.  

\begin{figure}[htb]
\centering
\begin{tabular}{c}
{\psfig{figure=ps/modifiers-files/betaNn.ps,height=3.5in}}
\end{tabular}
\caption {Noun-noun compounding tree: $\beta$Nn}
\label {noun-compound-tree}
\end{figure}


Noun compounds have a variety of scope possibilities not available to
adjectives, as illustrated by the single bracketing possibility in (\ex{1}) and
the two possibilities for (\ex{2}).  This ambiguity is manifested in the XTAG
grammar by the two possible adjunction sites in the noun-noun compound tree
itself.  Subsequent modifying nouns can adjoin either onto the N$_r$ node or
onto the N anchor node of that tree, which results in exactly the two
bracketing possibilities shown in (\ex{2}).  This inherent structural ambiguity
results in noun-noun compounds regularly having multiple derivations. However,
the multiple derivations are not a defect in the grammar because they are
necessary to correctly represent the genuine ambiguity of these phrases.

\enumsentence{[$_{N}$ big [$_{N}$ green design $_{N}$]$_{N}$]}

\enumsentence{[$_{N}$ computer [$_{N}$ furniture design $_{N}$]$_{N}$]\\
\/~~[$_{N}$ [$_{N}$ computer furniture $_{N}$] design $_{N}$]}



\section{Prepositions}
\label{prep-modifier}

There are three basic types of prepositional phrases, and three places at
which they can adjoin.  The three types of prepositional phrases are:
Preposition with NP Complement, Preposition with Sentential Complement, and
Exhaustive Preposition.  The three places are to the right of an NP, to the
right of a VP, and to the left of an S.  Each of the three types of PP can
adjoin at each of these three places, for a total of nine PP modifier
trees. Table \ref{prep-summary} gives the tree family names for the
various combinations of type and location.

\begin{table}[htb]
\centering
\begin{tabular}{|l||c|c|c|}
\hline
\multicolumn{1}{|c||}{}&\multicolumn{3}{c|}{position and category modified}\\
\cline{2-4}
\multicolumn{1}{|c||}{}&pre-sentential&post-NP&post-VP\\
\multicolumn{1}{|c||}{Complement type}&S modifier&NP modifier&VP modifier\\
\hline
\hline
S-complement&$\beta$Pss&$\beta$nxPs&$\beta$vxPs\\
\hline
NP-complement&$\beta$Pnxs&$\beta$nxPnx&$\beta$vxPnx\\
\hline
no complement&$\beta$Ps&$\beta$nxP&$\beta$vxP\\
(exhaustive)&&&\\
\hline
\end{tabular}
\caption{Preposition Anchored Modifiers}
\label{prep-summary}
\end{table}

The subset of preposition anchored modifier trees in Figure~\ref{prep-trees}
illustrates the three locations and the three PP types.

\begin{figure}[htb]
\centering
\begin{tabular}{ccccccc}
{\psfig{figure=ps/modifiers-files/betaPss.ps,height=1.5in}}
& \hspace{.5in} &
{\psfig{figure=ps/modifiers-files/betanxPnx.ps,height=1.5in}}
&  \hspace{.5in} &
{\psfig{figure=ps/modifiers-files/betavxP.ps,height=1.5in}}
&  \hspace{.5in} &
{\psfig{figure=ps/betavxPPnx.ps,height=1.75in}}
\\
$\beta$Pss&&$\beta$nxPnx&&$\beta$vxP&&$\beta$vxPPnx\\
\end{tabular}\\
\caption {Selected Prepositional Phrase Modifier trees:
$\beta$Pss, $\beta$nxPnx, $\beta$vxP and $\beta$vxPPnx}
\label {prep-trees}
\end{figure}


Example sentences using the trees in Figure \ref{prep-trees} are shown
in (\ex{1})-(\ex{3}). There are also identical trees which have
multi-word prepositions as anchors. Examples of these are: {\it ahead
of}, {\it contrary to}, {\it at variance with} and {\it in line with}. 

\enumsentence{[$_{PP}$ With Clove healthy $_{PP}$], the veterinarian's
bill will be more affordable. ($\beta$Pss\footnote{{\it Clove healthy} is an adjective small clause})}
\enumsentence{The frisbee [$_{PP}$ in the brambles $_{PP}$] was hidden.
($\beta$nxPnx)}
\enumsentence{Clove played frisbee [$_{PP}$ outside $_{PP}$]. ($\beta$vxP)}
\enumsentence{Clove played frisbee [$_{PP}$ outside of the house
$_{PP}$]. ($\beta$vxPPnx)}

Prepositions that take NP complements assign accusative case to those
complements (see section~\ref{prep-case} for details).  Most prepositions take
NP complements.  There are just a few prepositions that take sentential
complements (see section~\ref{NPA}).


\section{Adverbs}
\label{adv-modifier}

In the English XTAG grammar, VP and S-modifying adverbs anchor the auxiliary
trees $\beta$ARBs, $\beta$sARB, $\beta$vxARB and $\beta$ARBvx,\footnote{In the
naming conventions for the XTAG trees, ARB is used for {\underline
a}dve{\underline {rb}}s.  Because the letters in A, Ad, and Adv are all used
for other parts of speech ({\underline a}djective, {\underline d}eterminer and
{\underline v}erb), ARB was chosen to eliminate ambiguity.
Appendix~\ref{tree-naming} contains a full explanation of naming conventions.}
allowing pre and post modification of S's and VP's.  Besides the VP and
S-modifying adverbs, the grammar includes adverbs that modify other
categories. Examples of adverbs modifying an adjective, an adverb, a
PP and a determiner are
shown in (\ex{1})-(\ex{7}).

\begin{itemize}
\item{Modifying an adjective}
\enumsentence{{\bf extremely} good}
\enumsentence{{\bf rather} tall}
\enumsentence{rich {\bf enough}}

\item{Modifying an adverb}
\enumsentence{oddly {\bf enough}} 
\enumsentence{{\bf very} well}

\item{Modifying a PP}
\enumsentence{{\bf right} through the wall}

\item{Modifying a determiner}
\enumsentence{{\bf exactly} five men}

\end{itemize}

XTAG has separate trees for each of the modified categories and for pre and
post modification where needed.  The kind of treatment given to adverbs here is
very much in line with the base-generation approach proposed by \cite{Ernst84},
which assumes all positions where an adverb can occur to be base-generated, and
that the semantics of the adverb specifies a range of possible positions
occupied by each adverb. While the relevant semantic features of the adverbs
are not currently implemented, implementation of semantic features is scheduled
for future work.  The trees for adverb anchored modifiers are very similar in
form to the adjective anchored modifier trees.  Examples of two of the basic
adverb modifier trees are shown in Figure~\ref{adv-trees}.

\begin{figure}[hb]
\centering
\begin{tabular}{ccc}
{\psfig{figure=ps/modifiers-files/betaARBs.ps,height=4.5in}}&
\hspace*{1.0in}&
{\psfig{figure=ps/modifiers-files/betavxARB.ps,height=4in}}\\
(a)&&(b)\\
\end{tabular}
\caption {Adverb Trees for pre-modification of S: $\beta$ARBs (a) and
post-modification of a VP: $\beta$vxARB (b)}
\label{adv-trees}
\end{figure}

\newpage

Like the adjective anchored trees, these trees also have the NA
constraint on the foot node to restrict the number of derivations
produced for a sequence of adverbs.  Features of the modified category
are passed from the foot node to the root node, reflecting correctly
that these types of properties are unaffected by the adjunction of an
adverb.  A summary of the categories modified and the position of
adverbs is given in Table \ref{adv-summary}.

\begin{table}[h]
\centering
\begin{tabular}{|c||c|c|}
\hline
&\multicolumn{2}{c|}{Position with respect to item modified}\\
\cline{2-3}
Category Modified&Pre&Post\\
\hline
\hline
S&$\beta$ARBs&$\beta$sARB\\
\hline
VP&$\beta$ARBvx&$\beta$vxARB\\
\hline
A&$\beta$ARBa&$\beta$aARB\\
\hline
PP&$\beta$ARBpx&$\beta$pxARB\\
\hline
ADV&$\beta$ARBarb&$\beta$arbARB\\
\hline
N&$\beta$ARBn&\\
\hline
Det&$\beta$ARBdx&\\
\hline
\end{tabular}
\caption{Simple Adverb Anchored Modifiers}
\label{adv-summary}
\end{table}


In the English XTAG grammar, no traces are posited for wh-adverbs, in-line with
the base-generation approach (\cite{Ernst84}) for various positions of
adverbs. Since convincing arguments have been made against traces for adjuncts
of other types (e.g. \cite{Baltin}), and since the reasons for wanting traces
do not seem to apply to adjuncts, we make the general assumption in our grammar
that adjuncts do not leave traces.  Sentence initial wh-adverbs select the same
auxiliary tree used for other sentence initial adverbs ($\beta$ARBs) with the
feature {\bf $<$wh$>$=+}.  Under this treatment, the derived tree for the
sentence {\it How did you fall?} is as in Figure (\ref{how-did-you-fall}), with
no trace for the adverb.


\begin{figure}[h]
\centering
\begin{tabular}{c}
{\psfig{figure=ps/modifiers-files/how-did-you-fall.ps,height=3.5in}}
\end{tabular}
\caption {Derived tree for {\it How did you fall?}}
\label {how-did-you-fall}
\end{figure}


\begin{figure}[h]
\centering
\begin{tabular}{c}
{\psfig{figure=ps/modifiers-files/betaARBarbs.ps,height=6.0in}}
\end{tabular}
\caption {Complex adverb phrase modifier: $\beta$ARBarbs}
\label{weird-adv-tree}
\end{figure}

There is one more adverb modifier tree in the grammar which is not included in
Table \ref{adv-summary}.  This tree, shown in Figure~\ref{weird-adv-tree}, has
a complex adverb phrase and is used for wh+ two-adverb phrases that occur
sentence initially, such as in sentence (\ex{1}).  Since {\it how} is the only
wh+ adverb, it is the only adverb that can anchor this tree.

\enumsentence{How quickly did Srini fix the problem?}












\chapter{Auxiliaries}
\label{auxiliaries}

Although there has been some debate about the lexical category of auxiliaries,
the English XTAG grammar follows \cite{mccawley88}, \cite{haegeman91}, and
others in classifying auxiliaries as verbs. The category of verbs can therefore
be divided into two sets, main or lexical verbs, and auxiliary verbs, which can
cooccur in a verbal sequence.  Only the highest verb in a verbal sequence is
marked for tense and agreement regardless of whether it is a main or auxiliary
verb.  Some auxiliaries ({\it be}, {\it do}, and {\it have}) share with main
verbs the property of having overt morphological marking for tense and
agreement, while the modal auxiliaries do not.  However, all auxiliary verbs
differ from main verbs in several crucial ways.

\begin{itemize}

\item Multiple auxiliaries can occur in a single sentence, while a matrix
sentence may have at most one main verb. 

\item Auxiliary verbs cannot occur as the sole verb in the sentence, but must
be followed by a main verb.

\item All auxiliaries precede the main verb in verbal sequences.

\item Auxiliaries do not subcategorize for any arguments.

\item Auxiliaries impose requirements on the morphological form of the verbs
that immediately follow them.

\item Only auxiliary verbs invert in questions (with the sole exception in 
American English of main verb {\it be}\footnote{Some dialects, particularly
British English, can also invert main verb {\it have} in yes/no questions
(e.g. {\it Have you any Grey Poupon?}).  This is usually attributed to the
influence of auxiliary {\it have}, coupled with the historic fact that English
once allowed this movement for all verbs.\label{have-footnote}}).

\item An auxiliary verb must precede sentential negation (e.g. $\ast${\it John not goes}).

\item Auxiliaries can form contractions with subjects and negation (e.g. {\it
he'll}, {\it won't}).

\end{itemize}


\noindent The restrictions that an auxiliary verb imposes on the succeeding verb limits
the sequence of verbs that can occur.  In English, sequences of up to five
verbs are allowed, as in sentence (\ex{1}).

\enumsentence{The music should have been being played [when the president arrived].}

\noindent 
The required ordering of verb forms when all five verbs are present is:

\begin{quote}
\begin{tabular}{ccl}
& & {\bf modal base perfective progressive passive}
\end{tabular}
\end{quote}

\noindent
The rightmost verb is the main verb of the sentence.  While a main verb
subcategorizes for the arguments that appear in the sentence, the auxiliary
verbs select the particular morphological forms of the verb to follow each of
them.  The auxiliaries included in the English XTAG grammar are listed in Table
\ref{aux-table} by type.  The third column of Table \ref{aux-table} lists the
verb forms that are required to follow each type of auxiliary verb.

\vspace*{0.2in}

\begin{table}[ht]
\centering
\begin{tabular}{|l|c|c|}  
\hline
TYPE&LEX ITEMS&SELECTS FOR\\     
\hline
modals & {\it can}, {\it could}, {\it may}, {\it might}, {\it will}, & base form\footnotemark
\\ & {\it would}, {\it ought}, {\it shall}, {\it should} & e.g. {\it will
go}, {\it might come}\\
\hline
perfective & {\it have} & past participle\\
& & (e.g. {\it has gone})\\  
\hline
progressive & {\it be} & gerund\\
& & (e.g. {\it is going}, {\it was coming})\\  
\hline
passive & {\it be} & past participle\\
& & (e.g. {\it was helped by Jane})\\  
\hline
do support & {\it do} &base form\\
& & (e.g. {\it did go}, {\it does come})\\  
\hline
infinitive to & {\it to} & base form\\
& & (e.g. {\it to go}, {\it to come})\\  
\hline
\end{tabular}
\caption{Auxiliary Verb Properties}
\label{aux-table}
\end{table}

\vspace*{0.2in}

%  This text belong with the table above.  It is put here so that it gets on
%  the right page.
\footnotetext{There are American dialects, particularly in the South, which
allow double modals such as {\it might could} and {\it might should}. These
constructions are not allowed in the XTAG English grammar.}

\section{Non-inverted sentences}
\label{aux-non-inverted}

This section and the sections that follow describe how the English XTAG grammar
accounts for properties of the auxiliary system described above.

In our grammar, auxiliary trees are added to the main verb tree by adjunction.
Figure~\ref{Vvx} shows the adjunction tree for non-inverted
sentences\footnote{We saw this tree briefly in section~\ref{case-for-verbs},
but with most of its features missing.  The full tree is presented here.}.

\begin{figure}[htb]
\centering
\begin{tabular}{c}
\psfig{figure=ps/auxs-files/betaVvx-with-features.ps,height=4.2in}
\end{tabular}
\caption{Auxiliary verb tree for non-inverted sentences: $\beta$Vvx }
\label{Vvx} 
\end{figure}

\begin{figure}[htbp]
\centering
\begin{tabular}{ccc}
{\psfig{figure=ps/auxs-files/betaVvx_should-with-features.ps,height=3.9in}} &
\hspace*{1.0in}&
{\psfig{figure=ps/auxs-files/betaVvx_have-with-features.ps,height=3.9in}} \\
\\
{\psfig{figure=ps/auxs-files/betaVvx_been-with-features.ps,height=3.9in}} &
\hspace*{1.in}&
{\psfig{figure=ps/auxs-files/betaVvx_being-with-features.ps,height=3.9in}} \\
\end{tabular}
\caption{Auxiliary trees for {\it The music should have been being played.}}
\label{anchored-aux-trees}
\end{figure}

The restrictions outlined in column 3 of Table \ref{aux-table} are
implemented through the features {\bf $<$mode$>$}, {\bf
$<$perfective$>$}, {\bf $<$progressive$>$} and {\bf $<$passive$>$}.
The syntactic lexicon entries for the auxiliaries gives values for
these features on the foot node~(VP$_{*}$) in Figure~\ref{Vvx}.  Since
the top features of the foot node must eventually unify with the
bottom features of the node it adjoins onto for the sentence to be
valid, this enforces the restrictions made by the auxiliary node.  In
addition to these feature values, each auxiliary also gives values to
the anchoring node~(V$\diamond$), to be passed up the tree to the root
VP~(VP$_{r}$) node; there they will become the new features for the
top VP node of the sentential tree.  Another auxiliary may now adjoin
on top of it, and so forth.  These feature values thereby ensure the
proper auxiliary sequencing.  Figure~\ref{anchored-aux-trees} shows the auxiliary trees anchored by the four 
auxiliary verbs in sentence (\ex{0}).  Figure~\ref{non-inverted-sentence} shows
the final tree created for this sentence.

\begin{figure}[htbp]
\centering
\begin{tabular}{c}
{\psfig{figure=ps/auxs-files/non-inverted-sentence.ps,height=3.1in}}
\end{tabular}
\caption{{\it The music should have been being played.}}
\label{non-inverted-sentence}
\end{figure}

The general English restriction that matrix clauses must have tense (or be
imperatives) is enforced by requiring the top S-node of a sentence to have {\bf
$<$mode$>$=ind/imp} (indicative or imperative).  Since only the indicative
sentences have tense, non-tensed clauses are restricted to occurring in
embedded environments.

\section{Inverted Sentences}

In inverted sentences, the two trees shown in Figure~\ref{inverted-trees}
adjoin to an S tree anchored by a main verb.  The tree in
Figure~\ref{inverted-trees}(a) is anchored by the auxiliary verb and adjoins to
the S node, while the tree in Figure~\ref{inverted-trees}(b) is anchored by an
empty element and adjoins at the VP node.  Figure~\ref{yes/no-question} shows
these trees (anchored by {\it will}) adjoined to the declarative transitive
tree\footnote{The declarative transitive tree was seen in
section~\ref{nx0Vnx1-family}.} (anchored by main verb {\it buy}).


\begin{figure}[htbp]
\centering
\begin{tabular}{ccc}
{\psfig{figure=ps/auxs-files/betaVs-with-features.ps,height=3.9in}} &
\hspace*{0.5in} &
{\psfig{figure=ps/auxs-files/betaVvx_epsilon-with-features.ps,height=4.25in}} \\
(a) &&(b) \\ 
\end{tabular}
\caption{Trees for auxiliary verb inversion: $\beta$Vs (a) and $\beta$Vvx (b)}
\label{inverted-trees}
\end{figure}

\begin{figure}[htb]
\centering
\begin{tabular}{c}
{\psfig{figure=ps/auxs-files/yes-no-question.ps,height=3.0in}} \\
\end{tabular}
\caption{{\it Will John buy a backpack?}}
\label{yes/no-question}
\end{figure}

The feature {\bf $<$displ-const$>$} ensures that both of the trees in
Figure~\ref{inverted-trees} must adjoin to an elementary tree whenever one of
them does. For more discussion on this mechanism, which simulates tree local
multi-component adjunction, see \cite{hockeysrini93}.  The tree in
Figure~\ref{inverted-trees}(b), anchored by $\epsilon$, represents the
originating position of the inverted auxiliary. Its adjunction blocks the {\bf
$<$assign-case$>$} values of the VP it dominates from being co-indexed with the
{\bf $<$case$>$} value of the subject. Since {\bf $<$assign-case$>$} values
from the VP are blocked, the {\bf $<$case$>$} value of the subject can only be
co-indexed with the {\bf $<$assign-case$>$} value of the inverted auxiliary
(Figure~\ref{inverted-trees}(a)).  Consequently, the inverted auxiliary
functions as the case-assigner for the subject in these inverted structures.
This is in contrast to the situation in uninverted structures where the anchor
of the highest (leftmost) VP assigns case to the subject (see
section~\ref{case-for-verbs} for more on case assignment).  The XTAG analysis
is similar to GB accounts where the inverted auxiliary plus the
$\epsilon$-anchored tree are taken as representing I to C movement.

\section{Do-Support}

It is well-known that English requires a mechanism called `do-support' for
negated sentences and for inverted yes-no questions without auxiliaries.

\enumsentence {John does not want a car.}
\enumsentence {$\ast$John not wants a car.}
\enumsentence {John will not want a car.}
\enumsentence {Do you want to leave home?}
\enumsentence {$\ast$Want you to leave home?}
\enumsentence {Will you want to leave home?}

\subsection{In negated sentences}
\label{do-support-negatives}

The GB analysis of do-support in negated sentences hinges on the separation of
the INFL and VP nodes (see \cite{chomsky65}, \cite{jackendoff72} and
\cite{chomsky86}).  The claim is that the presence of the negative morpheme
blocks the main verb from getting tense from the INFL node, thereby forcing the
addition of a verbal lexeme to carry the inflectional elements.  If an
auxiliary verb is present, then it carries tense, but if not, periphrastic or
`dummy', {\it do} is required.  This seems to indicate that {\it do} and other
auxiliary verbs would not cooccur, and indeed this is the case (see sentences
(\ex{1})-(\ex{2})).  Auxiliary {\it do} is allowed in English when no
negative morpheme is present, but this usage is marked as emphatic.  Emphatic
{\it do} is also not allowed to cooccur with auxiliary verbs (sentences
(\ex{3})-(\ex{6})).

\enumsentence {$\ast$We will have do bought a sleeping bag.}
\enumsentence {$\ast$We do will have bought a sleeping bag.}
\enumsentence {You {\bf do} have a backpack, don't you?}
\enumsentence {I {\bf do} want to go!}
\enumsentence {$\ast$You {\bf do} can have a backpack, don't you?}
\enumsentence {$\ast$I {\bf did} have had a backpack!}

In the XTAG grammar, {\it do} is prevented from cooccurring with
other auxiliary verbs by a requirement that it adjoin only onto main
verbs.  It has indicative mode, so no other auxiliaries can adjoin
above it\footnote{Earlier, we said that indicative mode carries tense
with it.  Since only the topmost auxiliary carries the tense, any
subsequent verbs must {\bf not} have indicative mode.}.  The lexical
item {\it not} is only allowed to adjoin onto an non-indicative (and
therefore non-tensed) verbs.  Since all matrix clauses must be
indicative (or imperative), a negated sentence will fail unless an
auxiliary verb, either {\it do} or another auxiliary, adjoins
somewhere above the negative morpheme, {\it not}. In addition to
forcing adjunction of an auxiliary, this analysis of {\it not} allows
it freedom to move around in the auxiliaries, as seen in the sentences
(\ex{1})-(\ex{4}).

\enumsentence {John will have had a backpack.}
\enumsentence {$\ast$John not will have had a backpack.}
\enumsentence {John will not have had a backpack.}
\enumsentence {John will have not had a backpack.}

\subsection{In inverted yes/no questions}

In inverted yes/no questions, {\it do} is required if there is no auxiliary
verb to invert, as seen in sentences (\ex{-12})-(\ex{-10}), replicated here
as (\ex{1})-(\ex{3}).

\enumsentence {Do you want to leave home?}
\enumsentence {$\ast$Want you to leave home?}
\enumsentence {Will you want to leave home?}
\enumsentence {$\ast$Do you will want to leave home?}

In English, unlike other Germanic languages, the main verb cannot move to the
beginning of a clause, with the exception of main verb {\it be}\footnote{The
inversion of main verb {\it have} in British English was previously noted.}.
In a GB account of inverted yes/no questions, the tense feature is said to be
in C$^{0}$ at the front of the sentence.  Since main verbs cannot move, they
cannot pick up the tense feature, and do-support is again required if there is
no auxiliary verb to perform the role.  Sentence (\ex{0}) shows that {\it do}
does not interact with other auxiliary verbs, even when in the inverted
position.

In XTAG, trees anchored by a main verb that lacks tense are required to have an
auxiliary verb adjoin onto them, whether at the VP node to form a declarative
sentence, or at the S node to form an inverted question.  {\it Do} selects the
inverted auxiliary trees given in Figure~\ref{inverted-trees}, just as other
auxiliaries do, so it is available to adjoin onto a tree at the S node to form
a yes/no question.  The mechanism described in
section~\ref{do-support-negatives} prohibits {\it do} from cooccurring with
other auxiliary verbs, even in the inverted position.


\section{Infinitives}

The infinitive {\it to} is considered an auxiliary verb in the XTAG system, and
selects the auxiliary tree in Figure~\ref{Vvx}.  {\it To}, like {\it do}, does
not interact with the other auxiliary verbs, adjoining only to main verb base
forms, and carrying infinitive mode.  It is used in embedded clauses, both with
and without a complementizer, as in sentences (\ex{1})-(\ex{3}).  Since it
cannot be inverted, it simply does not select the trees in
Figure~\ref{inverted-trees}.

\enumsentence {John wants to have a backpack.}
\enumsentence {John wants Mary to have a backpack.}
\enumsentence {John wants for Mary to have a backpack.}

The usage of inifinitival {\em to} interacts closely with the distribution of
null subjects (PRO), and is described in more detail in
section~\ref{for-complementizer}.







\chapter{Conjunction}
\label{conjunction}

\section{Introduction}

The XTAG system can handle sentences with conjunction of two
constituents of the same syntactic category. The coordinating
conjunctions which select the conjunction trees are {\it and}, {\it
or} and {\it but}.\footnote{We believe that the restriction of {\it
but} to conjoining only two items is a pragmatic one, and our grammars
accepts sequences of any number of elements conjoined by {\it but}.}
There are also multi-word conjunction trees, anchored by {\it either-or}, 
{\it neither-nor}, {\it both-and}, and {\it as well as}.  There are eight
syntactic categories that can be coordinated, and in each case an
auxiliary tree is used to implement the conjunction.  These eight
categories can be considered as four different cases, as described in
the following sections.  In all cases the two constituents are
required to be of the same syntactic category, but there may also be
some additional constraints, as described below.


\section{Adjective, Adverb, Preposition and PP Conjunction}

Each of these four categories has an auxiliary tree that is used for
conjunction of two constituents of that category.  The auxiliary tree
adjoins into the left-hand-side component, and the right-hand-side
component substitutes into the auxiliary tree.  

\begin{figure}[htb]
\centering
\begin{tabular}{ccc}
{\psfig{figure=ps/conj-files/betaA1conjA2.ps,height=2.8in}}&
\hspace*{0.5in}&
{\psfig{figure=ps/conj-files/derived-tree-140291.ps,height=2.8in}}\\
(a) & \hspace*{0.5in}& (b)\\
\end{tabular}
\caption{Tree for adjective conjunction: $\beta$a1CONJa2 and a resulting parse tree}
\label{A1conjA2}
\end{figure}

Figure~\ref{A1conjA2}(a) shows the auxiliary tree for adjective conjunction,
and is used, for example, in the derivation of the parse tree for the noun
phrase {\it the dark and dreary day}, as shown in Figure~\ref{A1conjA2}(b).
The auxiliary tree adjoins onto the node for the left adjective, and the
right adjective substitutes into the right hand side node of the auxiliary
tree. The analysis for adverb, preposition and PP conjunction is exactly the
same and there is a corresponding auxiliary tree for each of these that is
identical to that of Figure~\ref{A1conjA2}(a) except, of course, for the node
labels.


\section{Noun Phrase and Noun Conjunction}

The tree for NP conjunction, shown in Figure~\ref{NP1conjNP2}(a), has
the same basic analysis as in the previous section except that the
{\bf $<$wh$>$} and {\bf $<$case$>$} features are used to force the two
noun phrases to have the same {\bf $<$wh$>$} and {\bf $<$case$>$}
values.  This allows, for example, {\it he and she wrote the book
together} while disallowing {\it $\ast$he and her wrote the book
together.}  Agreement is lexicalized, since the various conjunctions
behave differently. With {\it and}, the root {\bf $<$agr num$>$} value
is {\bf $<$plural$>$}, no matter what the number of the two
conjuncts. With {\it or}, however, the root {\bf $<$agr num$>$} is
co-indexed with the {\bf $<$agr num$>$} feature of the right
conjunct. This ensures that the entire conjunct will bear the number
of both conjuncts if they agree (Figure~\ref{NP1conjNP2}(b)), or of
the most ``recent'' one if they differ ({\it Either the boys or John
is going to help you.}). There is no rule per se on what the
agreement should be here, but people tend to make the verb agree with
the last conjunct (cf. Quirk et. al \shortcite{quirk85}, section 10.41
for discussion). The tree for N conjunction is identical to that for
the NP tree except for the node labels. (The multi-word conjunctions
do not select the N conjunction tree - {\it $^*$the both dogs and
cats}).

\begin{figure}[htb]
\centering
\begin{tabular}{cc}
{\psfig{figure=ps/conj-files/betaCONJnx1CONJnx2.ps,height=6.5in}}
\hspace{0.5cm} &
{\psfig{figure=ps/conj-files/aardvarks-and-emus.ps,height=6.5in}}\\
(a) &  (b)\\
\end{tabular}
\caption{Tree for NP conjunction: $\beta$CONJnx1CONJnx2 and a resulting
parse tree}
\label{NP1conjNP2}
\end{figure}


\section{Determiner Conjunction}

%CDD,10/31/95:We cannot find any reason for this tree to be different,
%so it is now like the other conj trees, with the foot on the left.
%
%The tree for determiner conjunction, shown in Figure~\ref{DX1conjDX2}, is
%unlike the other conjunction trees in that the foot node is on the right.  This
%is because determiner phrases generally build to the left. For the
%same reason, 

In determiner coordination, all of the determiner feature values are
taken from the left determiner, and the only requirement is that the
{\bf $<$wh$>$} feature is the same, while the other features, such as
{\bf $<$card$>$}, are unconstrained.  For example, {\it which and
what} and {\it all but one} are both acceptable determiner
conjunctions, but {\it $\ast$which and all} is not.

\enumsentence{how many and which people camp frequently ?}
\enumsentence{$^*$some or which people enjoy nature .}

\begin{figure}[htbp]
\centering
\begin{tabular}{c}
\psfig{figure=ps/conj-files/betad1CONJd2.ps,height=5.3in}
\end{tabular}
\vspace{-0.25in}
\caption{Tree for determiner conjunction: $\beta$d1CONJd2.ps}
\label{DX1conjDX2}
\end{figure}

\section{Sentential Conjunction}

The tree for sentential conjunction, shown in Figure~\ref{S1conjS2},
is based on the same analysis as the conjunctions in the previous two
sections, with a slight difference in features.  The {\bf $<$mode$>$}
feature\footnote{See section~\ref{s-features} for an explanation of
the {\bf $<$mode$>$} feature.}  is used to constrain the two sentences
being conjoined to have the same mode so that {\it the day is dark and
the phone never rang} is acceptable, but {\it $\ast$the day dark and
the phone never rang} is not. Similarly, the two sentences must agree
in their {\bf $<$wh$>$}, {\bf $<$comp$>$} and {\bf $<$extracted$>$}
features.  Co-indexation of the {\bf $<$comp$>$} feature ensures that
either both conjuncts have the same complementizer, or there is a
single complementizer adjoined to the complete conjoined S.  The {\bf
$<$assign-comp$>$} feature\footnote{See
section~\ref{for-complementizer} for an explanation of the {\bf
$<$assign-comp$>$} feature.} feature is used to allow conjunction of
infinitival sentences, such as {\it to read and to sleep is a good
life}.

\begin{figure}[htb]
\centering
\begin{tabular}{c}
\psfig{figure=ps/conj-files/betaS1conjS2.ps,height=3.5in}
\end{tabular}
\caption{Tree for sentential conjunction: $\beta$s1CONJs2}
\label{S1conjS2}
\end{figure}

\section{Comma as a conjunction}

We treat comma as a conjunction in conjoined lists. It anchors the
same trees as the lexical conjunctions, but is considerably more
restricted in how it combines with them. The trees anchored by commas
are prohibited from adjoining to anything but another comma conjoined
element or a non-coordinate element. (All scope possibilities are
allowed for elements coordinated with lexical conjunctions.) Thus,
structures such as Tree
\ref{Comma-conj}(a) are permitted, with each element stacking
sequentially on top of the first element of the conjunct, while
structures such as Tree \ref{Comma-conj}(b) are blocked. 

\begin{figure}[htb]
\centering
\begin{tabular}{ccc}
{\psfig{figure=ps/conj-files/good-adj-conj.ps,height=2.75in}}&
\hspace*{0.5in}&
{\psfig{figure=ps/conj-files/bad-adj-conj.ps,height=2.75in}}\\
(a) Valid tree with comma conjunction & \hspace*{0.5in}& (b) Invalid tree\\
\end{tabular}
\caption{}
\label{Comma-conj}
\end{figure}

This is accomplished by using the {\bf $<$conj$>$} feature, which has the
values {\bf and/or/but} and {\bf comma} to differentiate the lexical
conjunctions from commas. The {\bf $<$conj$>$} values for a comma-anchored
tree and {\it and}-anchored tree are shown in Figure
\ref{conj-contrast}. The feature {\bf $<$conj$>$ = comma/none} on
A$_1$ in (a) only allows comma conjoined or non-conjoined elements as
the left-adjunct, and {\bf $<$conj$>$ = none} on A in (a) allows
only a non-conjoined element as the right conjunct. We also need the
feature {\bf $<$conj$>$ = and/or/but/none} on the right conjunct of
the trees anchored by lexical conjunctions like (b), to block
comma-conjoined elements from substituting there. Without this
restriction, we would get multiple parses of the NP in Tree
\ref{Comma-conj}; with the restrictions we only get the derivation
with the correct scoping, shown as (a).

Since comma-conjoined lists can appear without a lexical conjunction
between the final two elements, as shown in example (\ex{1}), we cannot
force all comma-conjoined sequences to end with a lexical conjunction.

\enumsentence{So it is too with many other spirits which we all know: the
spirit of Nazism or Communism, school spirit , the spirit of a street
corner gang or a football team, the spirit of Rotary or the Ku Klux
Klan. \hfill [Brown cd01]}


\begin{figure}[htb]
\centering
\begin{tabular}{cc}
{\psfig{figure=ps/conj-files/adj-comma-conj.ps,height=2.5in}}&
{\psfig{figure=ps/conj-files/adj-and-conj.ps,height=2.5in}}\\
\end{tabular}
\caption{$\beta$a1CONJa2 (a) anchored by comma and (b) anchored by {\it and}}
\label{conj-contrast}
\end{figure}


\section{{\it But-not}, {\it not-but}, {\it and-not} and  {\it
$\epsilon$-not}}

We are analyzing conjoined structures such as {\it The women but not
the men} with a multi-anchor conjunction tree anchored by the
conjunction plus the adverb {\it not}. The alternative is to allow
{\it not} to adjoin to any constituent. However, this is the only
construction where {\it not} can freely occur onto a constituent other
than a VP or adjective (cf. $\beta$NEGvx and $\beta$NEGa trees). It
can also adjoin to some determiners, as discussed in Section
\ref{det-comparitives}. We want to allow sentences like (\ex{1}) and
rule out those like (\ex{2}). The tree for the good example is shown
in Figure \ref{but-not}. There are similar trees for {\it and-not} and
{\it $\epsilon$-not}, where $\epsilon$ is interpretable as either {\it
and} or {\it but}, and a tree with {\it not} on the first conjunct for
{\it not-but}.

\enumsentence{Beth grows basil in the house (but) not in the garden .}
\enumsentence{$^*$Beth grows basil (but) not in the garden .}

\begin{figure}[htb]
\centering
\begin{tabular}{c}
\psfig{figure=ps/conj-files/but-not.ps,height=2.5in}
\end{tabular}
\caption{Tree for conjunction with but-not: $\beta$px1CONJARBpx2}
\label{but-not}
\end{figure}

Although these constructions sound a bit odd when the two conjuncts do
not have the same number, they are sometimes possible. The agreement
information for such NPs is always that of the non-negated conjunct:
{\it his sons, and not Bill, are in charge of doing the laundry} or
{\it not Bill, but his sons, are in charge of doing the laundry}
(Some people insist on having the commas here, but they are frequently
absent in corpus data.) The agreement feature from the non-negated
conjunct in passed to the root NP, as shown in Figure
\ref{not-but}. Aside from agreement, these constructions behave just
like their non-negated counterparts.

%"Everyone is going on line and not just the younger generation," said Mal

\begin{figure}[htb]
\centering
\begin{tabular}{c}
\psfig{figure=ps/conj-files/not-but.ps,height=4in}
\end{tabular}
\caption{Tree for conjunction with not-but: $\beta$ARBnx1CONJnx2} 
\label{not-but}
\end{figure}

\section{{\it To} as a Conjunction}

{\it To} can be used as a conjunction for adjectives
(Fig. \ref{to-conj}) and determiners, when they denote points on a
scale:

\enumsentence{two to three degrees}
\enumsentence{high to very high temperatures}

As far as we can tell, when the conjuncts are determiners they must be
cardinal.

\begin{figure}[htb]
\centering
\begin{tabular}{c}
\psfig{figure=ps/conj-files/to.ps,height=3.5in}
\end{tabular}
\caption{Example of conjunction with {\it to}} 
\label{to-conj}
\end{figure}

\section{Predicative Coordination}

This section describes the method for predicative coordination
(including VP coordination of various kinds) used in XTAG. The
description is derived from work described in (\cite{anoopjoshi96}).
It is important to say that this implementation of predicative
coordination is not part of the XTAG release at the moment due massive
parsing ambiguities. This is partly because of the current
implementation and also the inherent ambiguities due to VP
coordination that cause a combinatorial explosion for the parser. We
are trying to remedy both of these limitations using a probability
model for coordination attachments which will be included as part of a
later XTAG release.

This extended domain of locality in a lexicalized Tree Adjoining
Grammar causes problems when we consider the coordination of such
predicates. Consider~(\ex{1}) for instance, the NP {\em the beans that
  I bought from Alice} in the Right-Node Raising (RNR) construction
has to be shared by the two elementary trees (which are anchored by
{\em cooked} and {\em ate} respectively).

\enumsentence{(((Harry cooked) and (Mary ate)) the beans that I bought
  from Alice)}

We use the standard notion of coordination which is shown in
Figure~\ref{fig:conj} which maps two constituents of {\em like type},
but with different interpretations, into a constituent of the same
type.

\begin{figure}[htbp]
  \begin{center}
    \leavevmode
    \psfig{figure=ps/conj-files/conj.ps,scale=110}
  \end{center}
  \caption{Coordination schema}
  \label{fig:conj}
\end{figure}

We add a new operation to the LTAG formalism (in addition to
substitution and adjunction) called {\em conjoin} (later we discuss an
alternative which replaces this operation by the traditional
operations of substitution and adjunction). While substitution and
adjunction take two trees to give a derived tree, {\em conjoin\/}
takes three trees and composes them to give a derived tree.  One of
the trees is always the tree obtained by specializing the schema in
Figure~\ref{fig:conj} for a particular category. The tree obtained
will be a lexicalized tree, with the lexical anchor as the
conjunction: {\em and}, {\em but}, etc.

The conjoin operation then creates a {\em contraction\/} between nodes
in the contraction sets of the trees being coordinated.  The term {\em
  contraction\/} is taken from the graph-theoretic notion of edge
contraction. In a graph, when an edge joining two vertices is
contracted, the nodes are merged and the new vertex retains edges to
the union of the neighbors of the merged vertices. The conjoin
operation supplies a new edge between each corresponding node in the
contraction set and then contracts that edge.

For example, applying {\em conjoin\/} to the trees {\em Conj(and)},
$\alpha(eats)$ and $\alpha(drinks)$ gives us the derivation tree and
derived structure for the constituent in \ex{1} shown in
Figure~\ref{fig:vpc}.

\enumsentence{$\ldots$ eats cookies and drinks beer}

\begin{figure}[htbp]
  \begin{center}
    \leavevmode
    \psfig{figure=ps/conj-files/vpc.ps,scale=110}
  \end{center}
  \caption{An example of the {\em conjoin\/} operation. $\{1\}$
    denotes a shared dependency.}
  \label{fig:vpc}
\end{figure}

Another way of viewing the conjoin operation is as the construction of
an auxiliary structure from an elementary tree. For example, from the
elementary tree $\alpha(drinks)$, the conjoin operation would create
the auxiliary structure $\beta(drinks)$ shown in
Figure~\ref{fig:aux-conj}. The adjunction operation would now be
responsible for creating contractions between nodes in the contraction
sets of the two trees supplied to it. Such an approach is attractive
for two reasons. First, it uses only the traditional operations of
substitution and adjunction. Secondly, it treats {\em conj X} as a
kind of ``modifier'' on the left conjunct {\em X}. This approach
reduces some of the parsing ambiguities introduced by the predicative
coordination trees and forms the basis of the XTAG implementation.

\begin{figure}[htbp]
  \begin{center}
    \leavevmode
    \psfig{figure=ps/conj-files/aux-conj.ps,scale=110}
  \end{center}
  \caption{Coordination as adjunction.}
  \label{fig:aux-conj}
\end{figure}

More information about predicative coordination can be found in
(\cite{anoopjoshi96}), including an extension to handle gapping constructions.

\section{Pseudo-coordination}

The XTAG grammar does handle one sort of verb pseudo-coordination. 
Semi-idiomatic phrases such as 'try and' and 'up and' (as in 'they might 
try and come today') are handled as multi-anchor modifiers 
rather than as true coordination. These items adjoin to a V node, using 
the $\beta$VCONJv tree. This tree adjoins only to verbs in their base 
morphological (non-inflected) form. The verb anchor of the $\beta$VCONJv 
must also be in its base form, as shown in examples 
(\ex{1})-(\ex{3}). This blocks 3rd-person singular derivations, 
which are the only person morphologically marked in the present, except when 
an auxiliary verb is present or the verb is in the infinitive.

\enumsentence{$\ast$He tried and came yesterday.}
\enumsentence{They try and exercise three times a week.}
\enumsentence{He wants to try and sell the puppies.}


%\section{Other Conjunctions}

%The conjunction analysis described in the previous sections is
%designed to handle only the most straightforward cases of conjunction.
%Three types of conjunction that are not handled are:

%\begin{itemize}
%\item {\bf Incomplete Constituents:} The sentence 
%{\it John likes and Bill hates bananas} cannot be handled by the
%current XTAG grammar.  Since {\it likes} anchors a tree that needs
%both a subject noun phrase and an object noun phrase to be substituted
%in, the first conjunct would need have an unfilled substitution node
%after {\it John likes} for the sentence to parse.

%\item {\bf Verb Phrase Conjunction:} Since verbs anchor a tree with a root 
%node of type S and not VP, there is no straightforward way to implement verb
%phrase conjunction.  For example, in the sentence {\it John eats cookies and
%drinks beer}, there is no point in the derivation at which {\it eats cookies}
%and {\it drinks beer} are available as separate trees ready to be conjoined.
%They are both only subtrees in their respective S trees.  This could also be
%considered as a case of incomplete constituents, since {\it drinks beer} is
%missing a noun phrase.

%\item {\bf Gapping:}
%Sentences such as {\it John likes apples and Bill pears} are also not
%handled by the previous analysis.  These could also be considered as a case
%of incomplete constituents.
%\end{itemize}

%One grammar formalism that is capable of handling these types of 
%conjunction is Combinatory Categorial Grammar (CCG) (\cite{steedman90})
%which relies on a nonstandard notion of a constituent in order to accomplish
%this.  Proposals have been made (e.g. \cite{joshischabes91}),
%inspired by the CCG approach, to handle these problematic cases in the
%FB-LTAG formalism.  Unlike the CCG analysis, however, the traditional notion
%of constituents and phrase structure is maintained.  Such proposals are
%as of yet unimplemented.

%%test sentences 
%%I ran and found a Brickel bush
%% you and me and the whole world
%% hook and line and bait




\part{Appendices}
\appendix
\chapter{Future Work}
\label{future-work}

\section{Adjective ordering}

At this point, the treatment of adjectives in the XTAG English grammar does not
include selectional or ordering restrictions\footnote{This section is a repeat
of information found in section~\ref{adj-modifier}.}. Consequently, any
adjective can adjoin onto any noun and on top of any other adjective already
modifying a noun. All of the modified noun phrases shown in (\ex{1})-(\ex{4})
currently parse.

\enumsentence{big green bugs}
\enumsentence{big green ideas}
\enumsentence{colorless green ideas}
\enumsentence{$\ast$green big ideas}

While (\ex{-2})-(\ex{0}) are all semantically anomalous, (\ex{0}) also suffers
from an ordering problem that makes it seem ungrammatical as well.  Since the
XTAG grammar focuses on syntactic constructions, it should accept
(\ex{-3})-(\ex{-1}) but not (\ex{0}).  Both the auxiliary and determiner
ordering systems are structured on the idea that certain types of lexical items
(specified by features) can adjoin onto some types of lexical items, but not
others.  We believe that an analysis of adjectival ordering would follow the
same type of mechanism.



\section{Determiner Adverbs}

There are some apparent adverbs that interact with the NP and determiner system
(\cite{quirk85}), although there is some debate in the literature as to whether
these should be classified as determiners or adverbs.\footnote{This section is
from \cite{HockeyEgedi94}.}

\enumsentence{{\bf Hardly} any attempt was made at restitution.}
\enumsentence{{\bf Only} Albert would say such a thing.}
\enumsentence{{\bf Almost} all the people had left by 5pm.}

Adverbs that modify NP's or determiners have restrictions on what types of NP's
or determiners they can modify. They divide into three classes based on the
pattern of these restrictions.  The adverbs {\it especially}, {\it even}, {\it
just} and {\it only} form a class that can modify any NP that is {\bf
$<$wh$>$=-}, including proper nouns.  A second class, consisting of adverbs
such as {\it hardly}, {\it merely} and {\it simply}, modifies NP's with
determiners that are {\bf $<$definite$>$=-} and {\bf $<$const$>$=+}, or that
are {\bf $<$gen$>$=+}.  This second class of adverbs can also modify NP's with
{\it the} as a determiner, but they do not modify NP's without determiners.
The third class, exemplified by {\it almost}, {\it approximately} and {\it
relatively}, modifies the determiner itself.  These adverbs are restricted to
modifying determiners with the {\bf $<$card$>$=+} feature, as well as {\it
all}, {\it double} and {\it half}.  The distinction between adverbs that modify
NP's and ones that modify determiners can be seen in the NP's in
({\ex{1}})~and~({\ex{2}}).

\enumsentence{[Just][half the people]}
\enumsentence{[Approximately half][the people]}




\section{More work on Determiners}

In addition to the analysis described in section~\ref{det-comparitives}, there
remains work to be done to complete the analysis of determiner constructions in
English\footnote{This section is from \cite{HockeyEgedi94}.}.  Although
constructions such as determiner coordination are easily handled if
overgeneration is allowed, blocking sequences such as {\it one and some} while
allowing sequences such as {\it five or ten} still remains to be worked out.
There are still a handful of determiners that are not currently handled by our
system.  We do not have an analysis to handle {\it most}, {\it such}, {\it
certain}, {\it other} and {\it own}\footnote{The behavior of {\it own} is
sufficiently unlike other determiners that it most likely needs a tree of its
own, adjoining onto the right-hand side of genitive determiners.}.  In
addition, there is a set of lexical items that we consider adjectives ({\it
enough}, {\it less}, {\it more} and {\it much}) that have the property that
they cannot cooccur with determiners.  We feel that a complete analysis of
determiners should be able to account for this phenomena, as well.




\section{Comparatives}

Also included in our future grammar development plans are comparatives.
Comparatives that involve ellipsis would require a general solution of the
problem of representing ellipsis, but simpler comparatives without ellipsis,
such as {\it fewer than nine\/} in (\ex{1}), should be amenable to analysis as
complex determiners, perhaps with trees similar in construction to the
partitive and genitive NP trees.

\enumsentence{Cats have {\bf fewer than} nine lives.}



\section{Time NP's}

Although in general NP's cannot simply adjoin onto sentences, there is a class
of NP's, called Time NP's, that can.  These NP's behave essentially like PP's,
and the XTAG analysis for this is fairly simple, requiring only the creation of
proper NP auxiliary trees.  Only slightly more difficult is the identification
of all possible anchors of these trees.  A {\bf $<$time$>$=+} feature will be
used to ensure that only certain nouns can select the time NP auxiliary trees.

\enumsentence{I went to Kentucky last month/$\ast$big cat.}
\enumsentence{This morning/$\ast$Big cat, we practiced juggling four balls.}



\section{-ing adjectives}

An analysis has already been provided for -ed adjectives (as in sentence~
(\ex{1})), which are restricted to the Transitive Verb family.\footnote{This
analysis may need to be extended to the Transitive Verb particle family as
well.}  A similar analysis needs to take place for the -ing adjectives.  This
type of adjective, however, does not seem to be as restricted as the -ed
adjectives, since verbs in other tree families seem to exhibit this alternation
as well (e.g. sentences~(\ex{2}) and (\ex{3})).

\enumsentence{The murdered man was a doctoral student at UPenn.}
\enumsentence{The man died.}
\enumsentence{The dying man pleaded for his life.}



\section{Punctuation}

We are currently developing an analysis of comma coordination, to
cover sentences such as (\ex{1}) and (\ex{2}).

\enumsentence{The brilliant, funny and timeless comedian had his 99th birthday
today.}
\enumsentence{NP's, PP's and VP's are all adjunction sites.}

Beyond this, we intend to add an analysis of other types of punctuation. We
believe that there are cases where the punctuation will serve as a guide to the
correct parse (e.g. comma following topicalized element), thus reducing
ambiguity.


\section{PRO control}

Within the FB-LTAG formalism, PRO-control is an interesting problem because of
the intrinsic non-local nature of control.\footnote{This section is taken from
\cite{bhatt94}.}  The controller NP and the controlled PRO are always in
different clauses.  In this sense, Control is even more non-local than Binding.

In the literature on Control, two types are often distinguished: obligatory
control, as in sentences~(\ex{1}) and (\ex{2}), and optional control, as in
sentence~(\ex{3}).

\enumsentence{Jan$_i$ promised Maria [PRO$_i$ to go].}
\enumsentence{Jan persuaded Maria$_i$ [PRO$_i$ to go].}
\enumsentence{[PRO$_{arb}$ to dance] is important.}

An analysis for obligatory control has been worked out, although it has yet to
be implemented.  The NP anchored by PRO will have the feature {\bf
$<$control$>$=+}.  The {\bf $<$control$>$} feature is also introduced in trees
that can take sentential arguments.  Depending on the verb, the control
propagation paths in the auxiliary trees are different.  In the case of subject
control (as in sentence~(\ex{-2})), the subject NP and the foot node are
constrained to have the same control features, while for object control
(e.g. sentence~(\ex{-1})), the object NP and the foot node are constrained to
have the same control features. 

Work has also been done on an XTAG analysis for optional control, but this has
not been fully worked out yet.




\section{Verb selectional restrictions}

Although we explicitly do not want to model semantics in the XTAG grammar,
there is some work along the syntax/semantics interface that would help reduce
syntactic ambiguity and thus decrease the number of semantically anomalous
parses.  In particular, verb selectional restrictions, particularly for PP
arguments and adjuncts, would be quite useful.  With the exception of the
required {\it to} in the Ditransitive with PP Shift tree family (Tnx0Vnx1Pnx2),
any preposition is allowed in the tree families that have prepositions as their
arguments.  In addition, there are no restrictions as to which prepositions are
allowed to adjoin onto a given verb.  The sentences in (\ex{1})-(\ex{3}) are
all currently accepted by the XTAG grammar.  Their violations are stronger than
would be expected from purely semantic violations, however, and the presence of
verb selectional restrictions on PP's would keep these sentences from being
accepted.

\enumsentence{\#Survivors walked of the street.}
\enumsentence{\#The man about the earthquake survived.}
\enumsentence{\#The president arranged on a meeting.}




\section{Idioms}

An analysis of idioms has already been worked out (\cite{AS89}), and one idiom
tree family is contained in the English XTAG grammar (Transitive Idioms;
section~\ref{nx0Vdn1-family}).  What remains to be done is a wide-ranging
cataloging of the many English idioms.  The list of idioms must then be divided
into appropriate tree families, based on the construction of the idiom, the
elements that are frozen and must therefore anchor the trees, and the
alternations that the idiom can undergo (e.g. passive, wh-movement, etc).  This
work has not been done.

\chapter{Tree Naming conventions}
\label{tree-naming}

The various trees within the XTAG grammar are named more or less according to
the following tree naming conventions.  Although these naming conventions are
generally followed, there are occasional trees that do not strictly follow
these conventions.

\section{Tree Nodes}
In the description of a tree, nodes are named as a pair $(l,s)$
(also represented as $l_s$ in the graphic representation of its structure
and $l\_s$ in the feature equations description),
where $l$ is the label or grammar symbol assigned to the node and $s$ is a 
subscript whose primary purpose is to make a node name unique for any given 
tree. Typical examples are: $S_r$, $VP$, $NP_0$, $NP_1$. Notice that the
subscript may be empty as in $VP$. 
There are several conventions generally
followed for the use of subscripts, as naming $S_r$ the node immediately 
dominating the subject position in a verb tree. 
We will not exhaustively describe them here,
except a few, which are used for naming the trees. They
should give the user an idea of what the tree is about. 

Anchors are generally assigned a null subscript, unless the tree has more than
on anchor with the same label, in which case each receive numeric 
subscripts $1$, $2$, etc. The main consistency condition here is that 
these subscripts have to match the multi-anchor entries in the syntactic 
lexicon that select the tree.
 
Arguments in verbal trees are assigned subscripts according to their 
thematic roles. The main
idea is that the subscript for a certain argument should be preserved across 
the trees in the grammar, whenever it is seen that these trees are 
transformationally related (the arguments should be at the same position in
the logical form). 
For instance, the subscript for the NP subject of
a passive tree should be the same as for the NP in the corresponding 
declarative tree that has been passivized. 
Additionally, the general convention is that the subscript $0$
is assigned to the underlying subject of a tree, $1$ to the first object,
and so on. Notable exceptions arise when a certain family is also used
as a part of the subcategorization description of a class of verbs. For
instance, dative verbs take two families: one having $NP_1$
and $PP_2$ as arguments (actually a multi-anchor tree where $P$ is an 
anchor); the other being the double object family, for the
dative shift. In order to maintain the relation that the leftmost object
in the dative shift tree is logically related to the $PP_2$, it also
receives the subscript $2$.
Another example is that
subjects in the ergative family have the 
same subscript, $1$, as the object in the base transitive tree.

When a verb argument is also the anchor of a tree, 
as in the predicative families, light verbs and some multi-anchor idioms,
the projection of the argument as well as its own arguments (e.g. PP and NP
for a P anchor,) 
should generally also take the subscript corresponding to its position w.r.t.
to the verb, which will be generally diferent from the anchor that carries
no subscript as a role.
Trees for wh-moved objects have no subscript in the extracted site. 
The original subscript is used for the landing site. 
Relative clause trees on the other hand 
preserve the subscript in the original position and use 
the subscript $w$ for the
landing site. 

\section{Tree Families}
Tree families are named according to the basic declarative tree structure in
the tree family (see section~\ref{family-trees}), but with a T as the first
character instead of an $\alpha$ or $\beta$.

\section{Trees within tree families}
\label{family-trees}

Each tree begins with either an $\alpha$ (alpha) or a $\beta$ (beta) symbol,
indicating whether it is an initial or auxiliary tree, respectively.  Following
an $\alpha$ or a $\beta$ the name may additionally contain one of the 
following depending on the family:

\begin{description}
\item\begin{tabular}{ll}
E& trees in the Ergative family\\
R& trees in a Resultative family\\
RE& trees in a Resultative family for Ergatives\\
X&ECM trees (eXceptional case marking)\\
\end{tabular}
\end{description}

\noindent Next, the name may contain one of the following,
the digit corresponding to the subscript of the node moved in the
tree. In Nc, Npx the absence of digit means relativized adjunct.

\begin{description}
\item\begin{tabular}{ll}
I&imperative\\
W0,W1,W2&wh-NP extraction\\
pW0,pW1,pW2&wh-PP extraction\\
N0,N1,N2&relative clause, NP argument relativized, wh-word \\
Nc,Nc0,Nc1,Nc2&relative clause, NP argument relativized, no wh-word \\
Npx,Npx1,Npx2&relative clause, PP relativized \\
Nby& relative clause, by-clause relativized in passive constructions\\
G&NP gerund\\
D&Determiner gerund\\
Inv&Inverted arguments (for equative BE and It-clefts)
\end{tabular}
\end{description}

% \noindent Numbers are assigned according to the position of the argument in the
% declarative tree, as follows:

% \begin{description}
% \item\begin{tabular}{ll}
% 0&subject position\\
% 1&first argument (e.g. direct object)\\
% 2&second argument (e.g. indirect object)\\
% \end{tabular}
% \end{description}

% \noindent The body of the name consists of a string of the following 
% components, which corresponds to the leaves of the tree.  The anchor(s) of the
% trees is(are) indicated by capitalizing the part of speech corresponding to the
% anchor.

\noindent The rest of the name consists of a string where each component
correspond to one leaf of the tree from the left to right. The formation
of a component is as follows: start with one of the elements in the table
below that corresponds to the leaf being translated: in lower case if the
node is a substitution or foot node;
 or upper case if it is an anchor. Then add 
``x'' if the node is a projection (or ``X'' if an anchor and a projection). 
Finally add the subscript at the node if any.
Notice that empty elements ($\epsilon$) are generally ignored and their
dominating node is used instead, except in the case of $PRO$, which by the
way is capitalized.

\begin{description}
\item\begin{tabular}{ll}
s&sentence\\
a&adjective\\
arb&adverb\\
be&{\it be}\\
% c&relative complementizer\\
% x&phrasal category\\
d&determiner\\
v&verb\\
lv&light verb\\
conj&conjunction\\
comp&complementizer\\
it&{\it it}\\
n&noun\\
p&preposition\\
PRO&a PRO subject \\
% to&{\it to}\\
pl&particle\\
by&{\it by}\\
neg&negation\\
\end{tabular}
\end{description}

\noindent As an example, the transitive declarative tree consists of a subject
NP, followed by a verb (which is the anchor), followed by the object NP.  This
translates into $\alpha$nx0Vnx1.  If the subject NP had been extracted, then
the tree would be $\alpha$W0nx0Vnx1.  A passive tree with the {\it by} phrase
in the same tree family would be $\alpha$nx1Vbynx0.  Note that even though the
object NP has moved to the subject position, it retains the object encoding
(nx1). 

\section{Assorted Initial Trees}

Trees that are not part of the tree families are generally gathered into
several files for convenience.  The various initial trees are located in {\tt
lex.trees}.  All the trees in this file should begin with an $\alpha$,
indicating that they are initial trees.  This is followed by the root category
which follows the naming conventions in the previous section (e.g. n for noun,
x for phrasal category).  The root category is in all capital letters.  After
the root category, the node leaves are named, beginning from the left, with the
anchor of the tree also being capitalized.  As an example, the $\alpha$NXN
tree is rooted by an NP node (NX) and anchored by a noun (N).

\section{Assorted Auxiliary Trees}

The auxiliary trees are mostly located in the buffers {\tt
prepositions.trees}, {\tt conjunctions.trees}, {\tt
determiners.trees}, {\tt advs-adjs.trees}, and {\tt modifiers.trees},
although a couple of other files also contain auxiliary trees.  The
auxiliary trees follow a slightly different naming convention from the
initial trees.  Since the root and foot nodes must be the same for the
auxiliary trees, the root nodes are not explicitly mentioned in the
names of auxiliary trees.  The trees are named according to the leaf
nodes, starting from the left, and capitalizing the anchor node.  All
auxiliary trees begin with a $\beta$, of course.  For example,
$\beta$ARBs, indicates a tree anchored by an adverb (ARB), that
adjoins onto the left of an S node (Note that S must be the foot node,
and therefore also the root node).

% \subsection{Relative Clause Trees}
% For relative clause trees, the following naming conventions have been
% adopted: if the {\em wh}-moved NP is overt, it is not explicitly
% represented. Instead the index of the site of movement
% (0 for subject, 1 for object, 2 for indirect object) is appended to the
% N. So $\beta$N0nx0Vnx1 is a subject
% extraction relative clause with {\bf NP$_{w}$} substitution
% and $\beta$N1nx0Vnx1 is an object extraction
% relative clause. If the {\em wh}-moved NP is covert and Comp substitutes
% in, the Comp node is represented by {\em c} in the tree name and the
% index of the extraction site follows {\em c}. Thus
% $\beta$Nc0nx0Vnx1 is a subject extraction
% relative clause with Comp substitution. Adjunct trees are similar, except
% that since the extracted material is not co-indexed to a trace, no index
% is specified (cf. $\beta$Npxnx0Vnx1, which is an adjunct relative clause with
% PP pied-piping, and $\beta$Ncnx0Vnx1, which is an adjunct relative clause
% with Comp substitution). Cases of pied-piping, in which the pied-piped
% material is part of the anchor have the anchor capitalized or spelled-out
% (cf. $\beta$Nbynx0nx1Vbynx0 which is a relative clause with {\em by}-phrase
% pied-piping and {\bf NP$_{w}$} substitution.).


\chapter{Features}
\label{features}

Table~\ref{feature-table} contains a comprehensive list of the features in the
XTAG grammar and their possible values.

The table is followed by short `biographical' sketches of the various features
currently in use in the Xtag English grammar.

\footnotesize
\begin{table}[hbt]
\centering
\begin{tabular}{|l|l|}
\hline
Feature&Value\\
\hline
\hline
$<$agr 3rdsing$>$&$+,-$\\
$<$agr num$>$&plur,sing\\
$<$agr pers$>$&1,2,3\\
$<$agr gen$>$&fem,masc,neuter\\
$<$assign-case$>$&nom,acc,none\\
$<$assign-comp$>$&that,whether,if,for,ecm,rel,inf\_nil,ind\_nil,ppart\_nil,none\\
$<$card$>$&$+,-$\\
$<$case$>$&nom,acc,gen,none\\
$<$comp$>$&that,whether,if,for,rel,inf\_nil,ind\_nil,nil\\
$<$compar$>$&$+,-$\\
$<$compl$>$&$+,-$\\
$<$conditional$>$&$+,-$\\
$<$conj$>$&and,or,but,comma,scolon,to,nil\\
$<$const$>$&$+,-$\\
$<$contr$>$&$+,-$\\
$<$control$>$&no value, indexing only\\
$<$decreas$>$&$+,-$\\
$<$definite$>$&$+,-$\\
$<$disc-conj$>$&$+, -$\\
$<$displ-const$>$&$+,-$\\
$<$equiv$>$&$+,-$\\
$<$extracted$>$&$+,-$\\
$<$gen$>$&$+,-$\\
$<$gerund$>$&$+,-$\\
$<$inv$>$&$+,-$\\
$<$invlink$>$&no value, indexing only\\
$<$irrealis$>$&$+,-$\\
$<$mainv$>$&$+,-$\\
$<$mode$>$&base,ger,ind,inf,imp,nom,ppart,prep,sbjunt\\
$<$neg$>$&$+,-$\\
$<$passive$>$&$+,-$\\
$<$perfect$>$&$+,-$\\
$<$pred$>$&$+,-$\\
$<$progressive$>$&$+,-$\\
$<$pron$>$&$+,-$\\
$<$punct bal$>$&dquote,squote,paren,nil\\
$<$punct contains colon$>$&$+,-$\\
$<$punct contains dash$>$&$+,-$\\
$<$punct contains dquote$>$&$+,-$\\
$<$punct contains scolon$>$&$+,-$\\
$<$punct contains squote$>$&$+,-$\\
$<$punct struct$>$&comma,dash,colon,scolon,nil\\
$<$punct term$>$&per,qmark,excl,nil\\
$<$quan$>$&$+,-$\\
$<$refl$>$&$+,-$\\
$<$rel-clause$>$&$+,-$\\
$<$rel-pron$>$&ppart,ger,adj-clause\\
$<$select-mode$>$&ind,inf,ppart,ger\\
$<$super$>$&$+,-$\\
$<$tense$>$&pres,past\\
$<$trace$>$&no value, indexing only\\
$<$trans$>$&$+,-$\\
$<$weak$>$&$+,-$\\
$<$wh$>$&$+,-$\\
\hline
\end{tabular}
\caption{List of features and their possible values}
\label{feature-table}
\end{table}

\normalsize


\section{Agreement}
{\bf $\langle$agr$\rangle$} is a complex feature.
It can have as its subfeatures:\\
{\bf $\langle$agr 3rdsing$\rangle$}, possible values: {\bf $+/-$ }\\
{\bf $\langle$agr num$\rangle$}, possible values: {\bf $plur,sing$ }\\
{\bf $\langle$agr pers$\rangle$}, possible values: {\bf $1,2,3$ }\\
{\bf $\langle$agr gen$\rangle$}, possible values: {\bf $masc,fem,neut$ }

These features are used to ensure agreement between a verb and its subject.

Where does it occur:\\ Nouns comes specified from the lexicon with
their {\bf $\langle$agr$\rangle$} features. e.g. {\em books} is {\bf
$\langle$agr 3rdsing$\rangle$:~--}, {\bf $\langle$agr num$\rangle$:~plur}, and {\bf $\langle$agr pers$\rangle$:~3}. Only pronouns use the
{\bf $<$gen$>$} (gender) feature.

The {\bf $\langle$agr$\rangle$} features of a noun are transmitted up the 
NP tree by the following equation:\\
{\bf NP.b:$\langle$agr$\rangle =$ N.t:$\langle$agr$\rangle$}

Agreement between a verb and its subject is mediated by the following feature
equations:

\enumsentence{ {\bf NP$_{subj}$:$\langle$agr$\rangle =$ VP.t:$\langle$agr$\rangle$}}


\enumsentence{ {\bf VP.b:$\langle$agr$\rangle =$ V.t:$\langle$agr$\rangle$}}

Agreement has to be done as a two step process because whether the
verb agrees with the subject or not depends upon whether some auxiliary verb
adjoins in and upon what the {\bf $\langle$agr$\rangle$} specification of 
the verb is. 

Verbs also come specified from the lexicon with their {\bf
$\langle$agr$\rangle$} features, e.g. the {\bf $\langle$agr$\rangle$}
features of the verb {\em sings} are {\bf $\langle$agr
3rdsing$\rangle$:~+}, {\bf $\langle$agr num$\rangle$:~sing}, and {\bf
$\langle$agr pers$\rangle$:~3}; Non-finite forms of the verb {\em
sing} e.g. {\em singing} do not come with an {\bf
$\langle$agr$\rangle$} feature specification.

\subsection{Agreement and Movement}
The {\bf $\langle$agr$\rangle$} features of a moved NP and its trace 
are co-indexed. This captures the fact that movement does not disrupt 
a pre-existing agreement relationship between an NP and a verb.

\enumsentence{ \ [Which boys]$_{i}$ does John think [t$_{i}$ are/*is intelligent]?}



\section{Case}

There are two features responsible for case-assignment:\\
{\bf $\langle$case$\rangle$}, possible values: {\bf nom, acc, gen, none}\\
{\bf $\langle$assign-case$\rangle$}, possible values: {\bf nom, acc, none}

Case assigners (prepositions and verbs) as well as the VP, S and PP
nodes that dominate them have an {\bf $\langle$assign-case$\rangle$}
case feature. Phrases and lexical items that have case i.e. Ns and NPs
have a {\bf $\langle$case$\rangle$} feature.

Case assignment by prepositions involves the following equations:

\enumsentence{ {\bf PP.b:$\langle$assign-case$\rangle =$ P.t:$\langle$case$\rangle$}}


\enumsentence{ {\bf NP.t:$\langle$case$\rangle =$ P.t:$\langle$case$\rangle$}}

Prepositions come specified from the lexicon with their {\bf $\langle$assign-case$\rangle$}
feature.

\enumsentence{ {\bf P.b:$\langle$assign-case$\rangle =$ acc}}


Case assignment by verbs has two parts: assignment of case to the
object(s) and assignment of case to the subject. Assignment of case to
the object is simpler.  English verbs always assign accusative case to
their NP objects (direct or indirect).  Hence this is built into the
tree and not put into the lexical entry of each individual verb.

\enumsentence{ {\bf NP$_{object}$.t:$\langle$case$\rangle =$ acc}}

Assignment of case to the subject involves the following two equations.

\enumsentence{ {\bf NP$_{subj}$:$\langle$case$\rangle =$ VP.t:$\langle$assign-case$\rangle$}}


\enumsentence{ {\bf VP.b:$\langle$assign-case$\rangle =$ V.t:$\langle$assign-case$\rangle$}}

This is a two step process -- the final case assigned to the subject
depends upon the {\bf $\langle$assign-case$\rangle$} feature of the
verb as well as whether an auxiliary verb adjoins in.

Finite verbs like {\em sings} have {\bf nom} as the value of their
{\bf $\langle$assign-case$\rangle$} feature. Non-finite verbs have
{\bf none} as the value of their {\bf $\langle$assign-case$\rangle$}
feature. So if no auxiliary adjoins in, the only subject they can have
is {\bf PRO} which is the only NP with {\bf none} as the value its
{\bf $\langle$case$\rangle$} feature.

\subsection{ECM}
Certain verbs e.g. {\em want, believe, consider} etc. and one complementizer
{\em for} are able to assign case to the subject of their complement clause. 

The complementizer {\em for}, like the preposition {\em for}, has the
{\bf $\langle$assign-case$\rangle$} feature of its complement set to
{\bf acc}. Since the {\bf $\langle$assign-case$\rangle$} feature of
the root S$_{r}$ of the complement tree and the {\bf
$\langle$case$\rangle$} feature of its NP subject are co-indexed, this
leads to the subject being assigned accusative case.

ECM verbs have the {\bf $\langle$assign-case$\rangle$}  feature of their
foot S node set to {\bf acc}. The co-indexation between the 
{\bf $\langle$assign-case$\rangle$} feature of
the root S$_{r}$ and the {\bf $\langle$case$\rangle$} feature of the NP subject
leads to the subject being assigned accusative case.

\subsection{Agreement and Case}
The {\bf $\langle$case$\rangle$} features of a moved NP and its trace 
are co-indexed. This captures the fact that movement does not disrupt 
a pre-existing relationship of case-assignment between a verb and an NP.

\enumsentence{ Her$_{i}$/*She$_{i}$, I think that Odo like t$_{i}$.}


\section{Extraction and Inversion}
{\bf $\langle$extracted$\rangle$}, possible vales are {\bf $+/-$}

All sentential trees with extracted components, with the exception of
relative clauses are marked {\bf S.b$\langle$extracted$\rangle = +$}
at their top S node. The extracted element may be a {\em wh}-NP or a
topicalized NP. The {\bf $\langle$extracted$\rangle$} feature 
is currently used to block embedded topicalizations as exemplified
by the following example.
\enumsentence{ * John wants [Bill$_{i}$ [PRO to leave t$_{i}$]] }

{\bf $\langle$trace$\rangle$}: this feature is not assigned any value and
is used to co-index moved NPs and their traces which are marked by
$\epsilon$.

{\bf $\langle$wh$\rangle$}: possible values are {\bf $+/-$}\\ NPs like
{\em who}, {\em what} etc. come marked from the lexicon with a value
of {\bf $+$} for the feature {\bf $\langle$wh$\rangle$}.  Non {\em
wh}-NPs have {\bf $-$} as the value of their {\bf
$\langle$wh$\rangle$} feature. Note that {\bf $\langle$wh$\rangle$ = +
} NPs are not restricted to occurring in extracted positions, to allow
for the correct treatment of echo questions.

The {\bf $\langle$wh$\rangle$} feature is propagated up by possessives
-- e.g. the $+$ {\bf $\langle$wh$\rangle$} feature of the determiner
{\em which} in {\em which boy} is propagated up to the level of the NP
so that the value of the {\bf $\langle$wh$\rangle$} feature of the
entire NP is $+${\bf $\langle$wh$\rangle$}. This process is recursive
e.g. {\em which boy's mother}, {\em which boy's mother's sister}.

The {\bf $\langle$wh$\rangle$} feature
is also propagated up PPs. Thus the PP {\em to whom} has $+$ as the value of its 
{\bf $\langle$wh$\rangle$} feature. 

In trees with extracted NPs, the {\bf $\langle$wh$\rangle$} feature of the
root node S node is equated with the {\bf $\langle$wh$\rangle$} feature
of the extracted NPs. 

The {\bf $\langle$wh$\rangle$} feature is used to impose
subcategorizational constraints.
Certain verbs like {\em wonder} can
only take interrogative complements, other verbs such as {\em know}
can take both interrogative and non-interrogative complements, and yet
other verbs like {\em think} can only take non-interrogative
complements (cf. the {\bf $\langle$extracted$\rangle$} and {\bf
$\langle$mode$\rangle$} features also play a role in imposing 
subcategorizational constraints).

The {\bf $\langle$wh$\rangle$} feature is also used to get the correct
inversion patterns.


\subsection{Inversion, Part 1}
The following three features are used to ensure the correct pattern of
inversion:\\
{\bf $\langle$wh$\rangle$}: possible values are {\bf $+/-$}\\
{\bf $\langle$inv$\rangle$}: possible values are {\bf $+/-$}\\
{\bf $\langle$invlink$\rangle$}: possible values are {\bf $+/-$}

Facts to be captured:\\
1. No inversion with topicalization\\
2. No inversion with matrix extracted subject {\em wh}-questions\\
3. Inversion with matrix extracted object {\em wh}-questions\\
4. Inversion with all matrix {\em wh}-questions involving extraction from an
embedded clause\\
5. No inversion in embedded questions \\
6. No matrix subject topicalizations.

Consider a tree with object extraction, where NP is extracted. 
The following feature equations are used:\\

\enumsentence{ {\bf S$_{q}$.b:$\langle$wh$\rangle =$ NP.t:$\langle$wh$\rangle$}\label{inv1}}
\enumsentence{ {\bf S$_{q}$.b:$\langle$invlink$\rangle =$  S$_{q}$.b:$\langle$inv$\rangle$}\label{inv2}}
\enumsentence{ {\bf S$_{q}$.b:$\langle$inv$\rangle =$  S$_{r}$.t:$\langle$inv$\rangle$}\label{inv3}}
\enumsentence{ {\bf S$_{r}$.b:$\langle$inv$\rangle = -$}\label{inv4}}

{\bf Root restriction}: A restriction is imposed on the final root
node of any XTAG derivation of a tensed sentence which equates the
{\bf $\langle$wh$\rangle$} feature and the {\bf
$\langle$invlink$\rangle$} feature of the final root node.

If the extracted NP is not a {\em wh}-word i.e. its {\bf
$\langle$wh$\rangle$} feature has the value $-$, at the end of the
derivation, {\bf S$_{q}$.b:$\langle$wh$\rangle$} will also have the
value $-$. Because of the root constraint {\bf
S$_{q}$.b:$\langle$wh$\rangle$} will be equated to {\bf
S$_{q}$.b:$\langle$invlink$\rangle$} which will also come to have the
value $-$. Then, by (\ref{inv3}), {\bf
S$_{r}$.t:$\langle$inv$\rangle$} will acquire the value $-$. This will
unify with {\bf S$_{r}$.b:$\langle$inv$\rangle$} which has the value
$-$ (cf. \ref{inv4}). Consequently, no auxiliary verb adjunction will
be forced. Hence, there will never be inversion in topicalization.

If the extracted NP is a {\em wh}-word i.e. its {\bf $\langle$wh$\rangle$} 
feature has the value $+$, at the end of the derivation, 
{\bf S$_{q}$.b:$\langle$wh$\rangle$} will also have the value $+$. Because of
the root constraint {\bf S$_{q}$.b:$\langle$wh$\rangle$} will be equated 
to {\bf S$_{q}$.b:$\langle$invlink$\rangle$} which will also come to have
the value $+$. Then, by (\ref{inv3}), {\bf S$_{r}$.t:$\langle$inv$\rangle$} 
will acquire the value $+$. This will not unify with {\bf S$_{r}$.b:$\langle$inv$\rangle$}
which has the value $+$ (cf. \ref{inv4}). Consequently, the adjunction
of an inverted auxiliary verb is required for the derivation to succeed.

Inversion will still take place even if the extraction is from an embedded
clause.

\enumsentence{ Who$_{i}$ does Loida think [Miguel likes t$_{i}$]}

This is because the adjoined tree's root node will also have its 
{\bf S$_{r}$.b:$\langle$inv$\rangle$} set to $-$. 


Note that inversion is only forced upon us because S$_{q}$ is the
final root node and the {\bf Root restriction} applies. In embedded
environments, the root restriction would not apply and the feature
clash that forces adjunction would not take place.

The {\bf $\langle$invlink$\rangle$} feature is not present in subject
extractions.  Consequently there is no inversion in subject questions.

Subject topicalizations are blocked by setting the 
{\bf $\langle$wh$\rangle$} feature of the extracted NP to $+$ i.e. only
{\em wh}-phrases can go in this location. 

\subsection{Inversion, Part 2}

{\bf $\langle$displ-const$\rangle$}:\\ Possible values: {\bf [set1:
+], [set1: --]}\\ In the previous section, we saw how inversion is
triggered using the {\bf $\langle$invlink$\rangle$}, {\bf
$\langle$inv$\rangle$}, {\bf $\langle$wh$\rangle$} features. Inversion
involves movement of the verb from V to C. This movement process is
represented using the {\bf $\langle$displ-const$\rangle$} feature
which is used to simulate Multi-Component TAGs.\footnote{The {\bf
$\langle$displ-const$\rangle$} feature is also used in the ECM
analysis.} The sub-value {\bf set1} indicates the inversion
multi-component set; while there are not currently any other uses of
this mechanism, it could be expanded with other sets receiving
different {\bf set} values.

The {\bf $\langle$displ-const$\rangle$} feature is used to ensure
adjunction of two trees, which in this case are the auxiliary
tree corresponding to the moved verb (S adjunct) and the auxiliary tree
corresponding to the trace of the moved verb (VP adjunct). The following
equations are used:

\enumsentence{  {\bf S$_{r}$.b:$\langle$displ-const set1$\rangle = -$}\label{dis1}}
\enumsentence{  {\bf S.t:$\langle$displ-const set1$\rangle = +$}\label{dis2}}
\enumsentence{  {\bf VP.b:$\langle$displ-const set1$\rangle =$
          V.t:$\langle$displ-const set1$\rangle$}\label{dis3}}
\enumsentence{  {\bf V.b:$\langle$displ-const set1$\rangle = +$}\label{dis4}}
\enumsentence{  {\bf S$_{r}$.b:$\langle$displ-const set1$\rangle =$ 
          VP.t:$\langle$displ-const set1$\rangle$}\label{dis5}}


\section{Clause Type}
There are several features that mark clause type.\footnote{We have
already seen one instance of a feature that marks clause-type: {\bf
$\langle$extracted$\rangle$}, which marks whether a certain S involves
extraction or not.} They are:\\ {\bf $\langle$mode$\rangle$}\\ {\bf
$\langle$passive$\rangle$}: possible values are {\bf +/--}


{\bf $\langle$mode$\rangle$}: possible values are 
{\bf base, ger, ind, inf, imp, nom, ppart, prep, sbjnct}\\
The {\bf $\langle$mode$\rangle$} feature of a verb in its root form is
{\bf base}. The {\bf $\langle$mode$\rangle$} feature of a verb in its past 
participial form is {\bf ppart}, the {\bf $\langle$mode$\rangle$} feature of a 
verb in its progressive/gerundive form is {\bf ger}, 
the {\bf $\langle$mode$\rangle$} feature of a tensed verb is {\bf ind},
and the {\bf $\langle$mode$\rangle$} feature of a verb in the imperative 
is {\bf imp}. 

{\bf nom} is the {\bf $\langle$mode$\rangle$} value of AP/NP
predicative trees headed by a null copula.  {\bf prep} is the {\bf
$\langle$mode$\rangle$} value of PP predicative trees headed by a null
copula.  Only the copula auxiliary tree, some sentential complement
verbs (such as {\it consider} and raising verb auxiliary trees have
{\bf nom/prep} as the {\bf $\langle$mode$\rangle$} feature
specification of their foot node. This allow them, and only them, to
adjoin onto AP/NP/PP predicative trees with null copulas.

\subsection{Auxiliary Selection}
The {\bf $\langle$mode$\rangle$} feature is also used to state the
subcategorizational constraints between an auxiliary verb and its
complement. We model the following constraints:\\
{\em have} takes past participial complements\\
passive {\em be} takes past participial complements\\
active {\em be} takes progressive complements\\
modal verbs, {\em do}, and {\em to} take VPs headed by verbs in their
base form as their complements. 

An auxiliary verb transmits its own mode to its root and imposes its
subcategorizational restrictions on its complement i.e. on its foot node.
e.g. the auxiliary {\em have} in its infinitival form involves the
following equations:

\enumsentence{ {\bf VP$_{r}$.b:$\langle$mode$\rangle =$ 
          V.t:$\langle$mode$\rangle$}\label{mode1}}
\enumsentence{ {\bf  V.t:$\langle$mode$\rangle =$ base}\label{mode2}}
\enumsentence{ {\bf VP.b:$\langle$mode$\rangle =$ ppart}\label{mode3}}


{\bf $\langle$passive$\rangle$}: This feature is used to ensure that
passives only have {\em be} as their auxiliary. Passive trees start
out with their {\bf $\langle$passive$\rangle$} feature as {\bf +}.
This feature starts out at the level of the verb and is percolated up
to the level of the VP. This ensures that only auxiliary verbs whose
foot node has {\bf +} as their {\bf $\langle$passive$\rangle$} feature
can adjoin on a passive. Passive trees have {\bf ppart} as the value
of their {\bf $\langle$mode$\rangle$} feature. So the only auxiliary
trees that we really have to worry about blocking are trees whose foot
nodes have {\bf ppart} as the value of their {\bf
$\langle$mode$\rangle$} feature. There are two such trees -- the {\em
be} tree and the {\em have} tree. The {\em be} tree is fine because
its foot node has {\bf +} as its {\bf $\langle$passive$\rangle$}
feature, so both the {\bf $\langle$passive$\rangle$} and {\bf
$\langle$mode$\rangle$} values unify; the {\em have} tree is blocked
because its foot node has {\bf --} as its {\bf
$\langle$passive$\rangle$} feature.

\section{Relative Clauses}
Features that are peculiar to the relative clause system are:\\
{\bf $\langle$select-mode$\rangle$}, possible values are {\bf ind, inf, ppart, ger}\\
{\bf $\langle$rel-pron$\rangle$}, possible values are {\bf ppart, ger, adj-clause}\\
{\bf $\langle$rel-clause$\rangle$}, possible values are {\bf +/--}

{\bf $\langle$select-mode$\rangle$}:\\
Comps are lexically specified for {\bf $\langle$select-mode$\rangle$}.
In addition, the {\bf $\langle$select-mode$\rangle$} feature of a Comp
is equated to the {\bf $\langle$mode$\rangle$} feature of its
sister S node by the following equation:

\enumsentence{ {\bf Comp.t:$\langle$select-mode$\rangle =$ S$_{t}$.t:$\langle$mode$\rangle$}}


The lexical specifications of the Comps are shown below:
\begin{itemize}
\item $\epsilon$$_{C}$, {\bf Comp.t:$\langle$select-mode$\rangle
=$ind/inf/ger/ppart}
\item {\em that}, {\bf Comp.t:$\langle$select-mode$\rangle =$ind}
\item {\em for}, {\bf Comp.t:$\langle$select-mode$\rangle =$inf}
\end{itemize}

{\bf $\langle$rel-pron$\rangle$}:\\
There are additional constraints on where the null Comp $\epsilon$$_{C}$
can occur. The null Comp is not permitted in cases of subject
extraction unless there is an intervening clause or or
the relative clause is a reduced relative ({\bf mode = ppart/ger}).

To model this paradigm, the feature {\bf $\langle$rel-pron$\rangle$} is used in
conjunction with the following equations.


\enumsentence{
{\bf S$_{r}$.t:$\langle$rel-pron$\rangle =$ Comp.t:$\langle$rel-pron$\rangle$}}
\enumsentence{
{\bf S$_{r}$.b:$\langle$rel-pron$\rangle =$ S$_{r}$.b:$\langle$mode$\rangle$}}
\enumsentence{
{\bf Comp.b:$\langle$rel-pron$\rangle =$ppart/ger/adj-clause}
(for $\epsilon$$_{C}$)}

The full set of the equations above is only present in Comp
substitution trees involving subject extraction. So the following will
not be ruled out.

\enumsentence{
the toy [$\epsilon$$_{i}$ [$\epsilon$$_{C}$ [ Dafna likes t$_{i}$ ]]] }


The feature mismatch induced by the above equations
is not remedied by adjunction of just any S-adjunct
because all other S-adjuncts
are transparent to the {\bf $\langle$rel-pron$\rangle$} feature
because of the following equation:

\enumsentence{
{\bf S$_{m}$.b:$\langle$rel-pron$\rangle =$ S$_{f}$.t:$\langle$rel-pron$\rangle$}}


{\bf $\langle$rel-clause$\rangle$}:\\ The XTAG analysis forces the
adjunction of the determiner below the relative clause. This is done
by using the {\bf $\langle$rel-clause$\rangle$} feature. The relevant
equations are:

\enumsentence{ On the root of the RC: {\bf NP$_{r}$.b:$\langle$rel-clause$\rangle = +$}}
\enumsentence{ On the foot node of the 
Determiner tree: {\bf NP$_{f}$.t:$\langle$rel-clause$\rangle = -$}}




\section{Complementizer Selection}
The following features are used to ensure the appropriate distribution
of complementizers:
\\
{\bf $\langle$comp$\rangle$}, possible values: {\bf that, if, whether,
for, rel, inf\_nil, ind\_nil, nil}\\
{\bf $\langle$assign-comp$\rangle$}, possible values: {\bf that, if,
whether, for, ecm, rel, ind\_nil, inf\_nil, none}\\
{\bf $\langle$mode$\rangle$}, possible values: {\bf ind, inf, sbjnct, ger, base, ppart, 
nom, prep}\\
{\bf $\langle$wh$\rangle$}, possible values: {\bf +, --}

The value of the {\bf $\langle$comp$\rangle$} feature tells us what complementizer we 
are dealing with. The trees which introduce complementizers come 
specified from the lexicon with their 
{\bf $\langle$comp$\rangle$} feature and {\bf $\langle$assign-comp$\rangle$} 
feature. The {\bf $\langle$comp$\rangle$} of the Comp tree regulates 
what kind of tree goes above the Comp tree, while the 
{\bf $\langle$assign-comp$\rangle$} feature regulates what kind of tree
goes below.
e.g.
the following equations are used for {\em that}:

\enumsentence{ {\bf S$_{c}$.b:$\langle$comp$\rangle =$ Comp.t:$\langle$comp$\rangle$} }
\enumsentence{ {\bf S$_{c}$.b:$\langle$wh$\rangle =$ Comp.t:$\langle$wh$\rangle$}}
\enumsentence{ {\bf S$_{c}$.b:$\langle$mode$\rangle =$ ind/sbjnct}}
\enumsentence{ {\bf S$_{r}$.t:$\langle$assign-comp$\rangle =$ Comp.t:$\langle$comp$\rangle$}}
\enumsentence{ {\bf S$_{r}$.b:$\langle$comp$\rangle =$ nil}}

By specifying {\bf S$_{r}$.b:$\langle$comp$\rangle =$ nil}, we ensure that
complementizers do not adjoin onto other complementizers. The root node
of a complementizer tree always has its {\bf $\langle$comp$\rangle$} feature
set to a value other than {\bf nil}.

Trees that take clausal complements specify with the {\bf $\langle$comp$\rangle$} feature
on their foot node what kind of complementizer(s) they can take. 
The {\bf $\langle$assign-comp$\rangle$} feature of an S node is determined 
by the highest VP below the S node and the syntactic configuration
the S node is in. 

\subsection{Verbs with object sentential complements}
Finite sentential complements:

\enumsentence{ {\bf S$_{1}$.t:$\langle$comp$\rangle =$ that/whether/if/nil}}
\enumsentence{{\bf S$_{1}$.t:$\langle$mode$\rangle =$ ind/sbjnct} or {\bf S$_{1}$.t:$\langle$mode$\rangle =$ ind}}
\enumsentence{ {\bf S$_{1}$.t:$\langle$assign-comp$\rangle =$ ind\_nil/inf\_nil}}

The presence of an overt complementizer is optional.

Non-finite sentential complements, do not permit {\em for}:

\enumsentence{ {\bf S$_{1}$.t:$\langle$comp$\rangle =$ nil}}
\enumsentence{ {\bf S$_{1}$.t:$\langle$mode$\rangle =$ inf}}
\enumsentence{ {\bf S$_{1}$.t:$\langle$assign-comp$\rangle =$ ind\_nil/inf\_nil}
}

Non-finite sentential complements, permit {\em for}:

\enumsentence{ {\bf S$_{1}$.t:$\langle$comp$\rangle =$ for/nil}}
\enumsentence{ {\bf S$_{1}$.t:$\langle$mode$\rangle =$ inf}}
\enumsentence{ {\bf S$_{1}$.t:$\langle$assign-comp$\rangle =$ ind\_nil/inf\_nil}}

Cases like `*I want for to win' are independently ruled out due to a 
case feature clash between the {\bf acc} assigned by {\em for} and the
intrinsic case feature {\bf none} on the PRO.

Non-finite sentential complements, ECM:

\enumsentence{ {\bf S$_{1}$.t:$\langle$comp$\rangle =$ nil}}
\enumsentence{ {\bf S$_{1}$.t:$\langle$mode$\rangle =$ inf}}
\enumsentence{ {\bf S$_{1}$.t:$\langle$assign-comp$\rangle =$ ecm}}


\subsection{Verbs with sentential subjects}
The following contrast involving complementizers surfaces with sentential
subjects:

\enumsentence{ *(That) John is crazy is likely.}

Indicative sentential subjects obligatorily have complementizers while
infinitival sentential subjects may or may not have a complementizer. 
Also {\em if} is possible as the complementizer of an object clause
but not as the complementizer of a sentential subject. 

\enumsentence{ {\bf S$_{0}$.t:$\langle$comp$\rangle =$ that/whether/for/nil}}
\enumsentence{ {\bf S$_{0}$.t:$\langle$mode$\rangle =$ inf/ind}}
\enumsentence{ {\bf S$_{0}$.t:$\langle$assign-comp$\rangle =$ inf\_nil}}

If the sentential subject is finite and a complementizer does
not adjoin in, the {\bf $\langle$assign-comp$\rangle$} feature of the 
S$_{0}$ node of the embedding clause and the root node of the
embedded clause will fail to unify. If a complementizer adjoins in,
there will be no feature-mismatch because the root of the
complementizer tree is not specified for the {\bf $\langle$assign-comp$\rangle$} feature.

The {\bf $\langle$comp$\rangle$} feature {\bf nil} is split into two
{\bf $\langle$assign-comp$\rangle$} features {\bf ind\_nil} and
{\bf inf\_nil} to capture the fact that there are certain configurations in
which it is acceptable for an infinitival clause to lack a complementizer
but not acceptable for an indicative clause to lack a complementizer. 

\subsection{{\em That}-trace and {\em for}-trace effects}

\enumsentence{ Who$_{i}$ do you think (*that) t$_{i}$ ate the apple?}

{\em That} trace violations are blocked by the presence of the following
equation:

\enumsentence{ {\bf S$_{r}$.b:$\langle$assign-comp$\rangle =$ inf\_nil/ind\_nil/ecm}}

on the bottom of the S$_{r}$ nodes of trees with extracted subjects (W0). 
The {\bf ind\_nil} feature specification permits the above example
while the {\bf inf\_nil/ecm} feature specification allows the
following examples to be derived:

\enumsentence{ Who$_{i}$ do you want [ t$_{i}$ to win the World Cup]?}
\enumsentence{ Who$_{i}$ do you consider [ t$_{i}$ intelligent]?}

The feature equation that ruled out the {\em that}-trace filter violations
will also serve to rule out the {\em for}-trace violations above.

\section{Determiner ordering}
{\bf $\langle$card$\rangle$}, possible values are {\bf +, --}\\
{\bf $\langle$compl$\rangle$}, possible values are {\bf +, --}\\
{\bf $\langle$const$\rangle$}, possible values are {\bf +, --}\\
{\bf $\langle$decreas$\rangle$}, possible values are {\bf +, --}\\
{\bf $\langle$definite$\rangle$}, possible values are {\bf +, --}\\
{\bf $\langle$gen$\rangle$}, possible values are {\bf +, --}\\
{\bf $\langle$quan$\rangle$}, possible values are {\bf +, --}

For detailed discussion see Chapter \ref{det-comparitives}.

\section{Punctuation}
{\bf $\langle$punct$\rangle$} is a complex feature. It has the following
as its subfeatures:\\
{\bf $\langle$punct bal$\rangle$}, possible values are {\bf dquote,
squote, paren, nil}\\
{\bf $\langle$punct contains colon$\rangle$}, possible values are {\bf +, --}\\
{\bf $\langle$punct contains dash$\rangle$}, possible values are {\bf +, --}\\
{\bf $\langle$punct contains dquote$\rangle$}, possible values are {\bf +, --}\\
{\bf $\langle$punct contains scolon$\rangle$}, possible values are {\bf +, --}\\
{\bf $\langle$punct contains squote$\rangle$}, possible values are {\bf +, --}\\
{\bf $\langle$punct struct$\rangle$}, possible values are {\bf comma,
dash, colon, scolon, none, nil}\\
{\bf $\langle$punct term$\rangle$}, possible values are {\bf per, qmark, excl, 
none, nil}

For detailed discussion see Chapter~\ref{punct-chapt}.


\section{Conjunction}
{\bf $\langle$conj$\rangle$}, possible values are {\bf but, and, or,
comma, scolon, to, nil}\\
The {\bf $\langle$conj$\rangle$} feature is specified in the lexicon
for each conjunction and is passed up to the root node 
of the conjunction tree. If the conjunction is {\em and}, the 
root {\bf $\langle$agr num$\rangle$} is {\bf $\langle$plural$\rangle$}, no
matter what the number of the two conjuncts. With {\em or}, the
the root {\bf $\langle$agr num$\rangle$} is equated to the
{\bf $\langle$agr num$\rangle$} feature of the right conjunct. 


{\bf $\langle$disc-conj$\rangle$}, possible values are {\bf +, --}\\
This feature is only used in the $\beta$CONJs tree.
It blocks the adjunction of one $\beta$CONJs tree on another.
The following equations are used (note that S$_{r}$ is the foot node
and S$_{c}$ is the root node):
\enumsentence{ S$_{r}$.t:$\langle$disc-conj$\rangle$ = -}
\enumsentence{ S$_{c}$.b:$\langle$disc-conj$\rangle$ = +}


\section{Comparatives}
{\bf $\langle$compar$\rangle$}, possible values are {\bf +, --}\\
{\bf $\langle$equiv$\rangle$}, possible values are {\bf +, --}\\
{\bf $\langle$super$\rangle$}, possible values are {\bf +, --}

For detailed discussion see Chapter~\ref{compars-chapter}.

\section{Control}
{\bf $\langle$control$\rangle$} has no value and is used only for indexing
purposes.  The root node of every clausal tree has its {\bf
$\langle$control$\rangle$} feature coindexed with the control feature of
its subject.  This allows adjunct control to take place. In addition,
clauses that take infinitival clausal complements have the control feature
of their subject/object coindexed with the control feature of their
complement clause S, depending upon whether they are subject control verbs
or object control verbs respectively.


\section{Other Features}
{\bf $\langle$neg$\rangle$}, possible values are {\bf +, --}\\
Used for controlling the interaction of negation and auxiliary verbs.

{\bf $\langle$pred$\rangle$}, possible values are {\bf +, --}\\
The {\bf $\langle$pred$\rangle$} feature is used in the following tree
families: Tnx0N1.trees and Tnx0nx1ARB.trees.
In the Tnx0N1.trees family, the following equations are used:\\
for $\alpha$W1nx0N1:

\enumsentence{ NP$_{1}$.t:$\langle$pred$\rangle$ = +}
\enumsentence{ NP$_{1}$.b:$\langle$pred$\rangle$ = +}
\enumsentence{ NP.t:$\langle$pred$\rangle$ = +}
\enumsentence{ N.t:$\langle$pred$\rangle$ = NP.b:$\langle$pred$\rangle$}

This is the only tree in this tree family to use the 
{\bf $\langle$pred$\rangle$} feature.

The other tree family where the {\bf $\langle$pred$\rangle$} feature is
used is Tnx0nx1ARB.trees.  Within this family, this feature (and the
following equations) are used only in the $\alpha$W1nx0nx1ARB tree.

\enumsentence{ AdvP$_{1}$.t:$\langle$pred$\rangle$ = +}
\enumsentence{ AdvP$_{1}$.b:$\langle$pred$\rangle$ = +}
\enumsentence{ NP.t:$\langle$pred$\rangle$ = +}
\enumsentence{ AdvP.b:$\langle$pred$\rangle$ = NP.t:$\langle$pred$\rangle$}


{\bf $\langle$pron$\rangle$}, possible values are {\bf +, --}\\
This feature indicates whether a particular NP is a pronoun or not. 
Certain constructions which do not permit pronouns use this 
feature to block pronouns.

{\bf $\langle$tense$\rangle$}, possible values are {\bf pres, past}\\
It does not seem to be the case that the {\bf $\langle$tense$\rangle$}
feature interacts with other features/syntactic processes. It 
comes from the lexicon with the verb and is transmitted up the
tree in such a way that the root S node ends up with the
tense feature of the highest verb in the tree. The equations
used for this purpose are:

\enumsentence{ {\bf S$_{r}$.b:$\langle$tense$\rangle$ = VP.t:$\langle$tense$\rangle$}}
\enumsentence{ {\bf VP.b:$\langle$tense$\rangle$ = V.t:$\langle$tense$\rangle$}}


{\bf $\langle$trans$\rangle$}, possible values are {\bf +, --}\\
Many but not all English verbs can anchor both transitive and intransitive trees.

\enumsentence{ The sun melted the ice cream.}
\enumsentence{ The ice cream melted.}
\enumsentence{ Elmo borrowed a book.}
\enumsentence{ * A book borrowed.}

Transitive trees have the {\bf $\langle$trans$\rangle$} feature of their
anchor set to {\ +} and intransitive trees have the 
{\bf $\langle$trans$\rangle$} feature of their
anchor set to {\ --}. Verbs such as {\em melt} which can occur 
in both transitive and intransitive trees come unspecified for the 
{\bf $\langle$trans$\rangle$} feature from the lexicon. Verbs which 
can only occur in transitive trees e.g. {\em borrow} have their
{\bf $\langle$trans$\rangle$} feature 
specified in the lexicon as {\bf +} thus blocking their anchoring of 
an intransitive tree.



\chapter{Evaluation and Results} 
\label{evaluation} 
 
In this appendix we describe various evaluations done of the XTAG 
grammar. Some of these evaluations were done on an earlier version of 
the XTAG grammar (the 1995 release), while other were done more 
recently. We will try to indicate in each section which version was 
used. 
 
\section{Parsing Corpora} 
 
In the XTAG project, we have used corpus analysis in two main ways: 
(1) to measure the performance of the English grammar on a given genre 
and (2) to identify gaps in the grammar. 
 
The second type of evaluation involves performing detailed error 
analysis on the sentences rejected by the parser, and we have done 
this several times on WSJ and Brown data. 
%Lifted directly from TAG+ paper 
Based on the results of such analysis, we prioritize upcoming grammar 
development efforts. The results of a recent error analysis are shown 
in Table \ref{errors}.  The table does not show errors in parsing due 
to mistakes made by the POS tagger which contributed the largest 
number of errors: 32. At this point, we have added a treatment of 
punctuation to handle \#1, an analysis of time NPs (\#2), a large 
number of multi-word prepositions (part of \#3), gapless relative 
clauses (\#7), bare infinitives (\#14) and have added the missing 
subcategorization (\#3) and missing lexical entry (\#12).  We are in 
the process of extending the parser to handle VP coordination (\#9) 
(See Section~\ref{conjunction} on recent work to handle VP and other 
predicative coordination). We find that this method of error analysis 
is very useful in focusing grammar development in a productive 
direction. 
 
\begin{table}[htb] 
\centering 
\begin{tabular}{|l|l|l|} \hline 
Rank & No of errors & Category of error \\ \hline 
\#1  & 11  &    Parentheticals and appositives \\ \hline 
\#2  & 8     &  Time NP \\ \hline 
\#3  & 8  &     Missing subcat \\ \hline 
\#4  & 7 &      Multi-word construction \\ \hline 
\#5  & 6 &       Ellipsis \\ \hline 
\#6  & 6  &      Not sentences \\ \hline 
\#7  & 3  &      Relative clause with no gap \\ \hline 
\#8  & 2  &      Funny coordination \\ \hline 
\#9  & 2  &      VP coordination \\ \hline 
\#10  & 2  &      Inverted predication \\ \hline 
\#11  & 2  &      Who knows \\ \hline 
\#12  & 1  &      Missing entry \\ \hline 
\#13  & 1   &     Comparative? \\ \hline 
\#14  & 1    &    Bare infinitive \\ \hline 
\end{tabular} 
\begin{rawhtml} <dl> <dt>{Results of Corpus Based Error Analysis <p> </dl> \end{rawhtml}
\label{errors} 
\end{table} 
 
To ensure that we are not losing coverage of certain phenomena as we 
extend the grammar, we have a benchmark set of grammatical and 
ungrammatical sentences from this technical report. We parse these 
sentences periodically to ensure that in adding new features and 
constructions to the grammar, we are not blocking previous analyses. 
There are approximately $590$ example sentences in this set. 
 
\section{TSNLP} 
 
In addition to corpus-based evaluation, we have also run the English 
Grammar on the Test Suites for Natural Language Processing (TSNLP) 
English corpus \cite{Lehmann96}. The corpus is intended to be a 
systematic collection of English grammatical phenomena, including 
complementation, agreement, modification, diathesis, modality, tense 
and aspect, sentence and clause types, coordination, and negation. It 
contains 1409 grammatical sentences and phrases and 3036 ungrammatical 
ones. 
 
\begin{table*}[htb] 
\centering 
\begin{tabular}{|l|c|c|} 
\hline 
Error Class & \% & Example \\ \hline 
POS Tag &  19.7\% & She adds  to/V it , He noises/N him abroad \\ \hline 
Missing lex item & 43.3\% & {\it used} as an auxiliary V, {\it calm NP down} \\ \hline 
Missing tree & 21.2\% & {\it should've}, {\it bet NP NP S}, {\it regard NP as Adj} \\ \hline 
Feature clashes & 3\% & {\it My every firm}, {\it All money} \\ \hline 
Rest&12.8\% & {\it approx}, {\it e.g.} \\ 
\hline 
\end{tabular} 
\begin{rawhtml} <dl> <dt>{Breakdown of TSNLP Errors <p> </dl> \end{rawhtml}
\label{tsnlp-table} 
\end{table*} 
 
%Before parsing the grammatical subset of the TSNLP data, we made a few 
%tokenization changes: we changed contractions from two tokens to one, 
%downcased the first words of sentences, changed a pair of square 
%brackets to parentheses and changed quotes to pairs of opens and 
%closes. 
There were 42 examples which we judged ungrammatical, and removed from 
the test corpus. These were sentences with conjoined subject pronouns, 
where one or both were accusative, e.g. {\it Her and him succeed.} 
Overall, we parsed 61.4\% of the 1367 remaining sentences and 
phrases. The errors were of various types, broken down in 
Table~\ref{tsnlp-table}. As with the error analysis described above, 
we used this information to help direct our grammar development 
efforts. It also highlighted the fact 
that our grammar is heavily slanted toward American English---our 
grammar did not handle {\it dare} or {\it need} as auxiliary verbs, 
and there were a number of very British particle constructions, 
e.g. {\it She misses him out}. 
 
One general problem with the test-suite is that it uses a very 
restricted lexicon, and if there is one problematic lexical item it is 
likely to appear a large number of times and cause a disproportionate 
amount of grief. {\it Used to} appears 33 times and we got all 33 
wrong. However, it must be noted that the XTAG grammar has analyses 
for syntactic phenomena that were not represented in the TSNLP test 
suite such as sentential subjects and subordinating clauses among 
others. This effort was, therefore, useful in highlighting some 
deficiencies in our grammar, but did not provide the same sort of 
general evaluation as parsing corpus data. 
 
\section{Chunking and Dependencies in XTAG Derivations} 
 
We evaluated the XTAG parser for the text chunking 
task~\cite{abney91}. In particular, we compared NP chunks and verb 
group (VG) chunks\footnote{We treat a sequence of verbs and verbal   modifiers, including auxiliaries, adverbs, modals as constituting a   verb group.}  produced by the XTAG parser with the NP and VG chunks 
from the Penn Treebank~\cite{marcus93}. The test involved $940$ 
sentences of length $15$ words or less from sections $17$ to $23$ of 
the Penn Treebank, parsed using the XTAG English grammar. The results 
are given in Table~\ref{chunking-results}. 
 
\begin{table*}[htb] 
\centering 
\begin{tabular}{|l|c|c|} 
\hline 
& NP Chunking & VG Chunking \\ \hline 
Recall & 82.15\% & 74.51\% \\ \hline 
Precision & 83.94\%  & 76.43\% \\ \hline 
\end{tabular} 
\begin{rawhtml} <dl> <dt>{Text Chunking performance of the XTAG parser <p> </dl> \end{rawhtml}
\label{chunking-results} 
\end{table*} 
 
\begin{table*}[htb] 
\centering 
\begin{tabular}{|c|c|c|c|} \hline 
System & Training Size & Recall & Precision  \\ \hline \hline 
Ramshaw \& Marcus & Baseline & 81.9\% & 78.2\% \\ \hline 
Ramshaw \& Marcus & 200,000 & 90.7\% & 90.5\% \\ 
(without lexical information) & & & \\ \hline 
Ramshaw \& Marcus & 200,000 & 92.3\% & 91.8\% \\ 
(with lexical information) & & & \\ \hline \hline 
Supertags & Baseline & 74.0\% & 58.4\% \\ \hline 
Supertags & 200,000 & 93.0\% & 91.8\% \\ \hline 
Supertags & 1,000,000 & 93.8\% & 92.5\% \\ \hline 
\end{tabular} 
\begin{rawhtml} <dl> <dt>{Performance comparison of the transformation based noun chunker and the supertag based noun chunker <p> </dl> \end{rawhtml}
\label{nc-compare} 
\end{table*} 
 
As described earlier, the results cannot be directly compared with 
other results in chunking such as in~\cite{lance&mitch95} since we do 
not train from the Treebank before testing. However, in earlier work, 
text chunking was done using a technique called 
supertagging~\cite{srini97iwpt} (which uses the XTAG English grammar) 
which can be used to train from the Treebank.  The comparative results 
of text chunking between supertagging and other methods of chunking is 
shown in Figure~\ref{nc-compare}.\footnote{It is important to note in   this comparison that the supertagger uses lexical information on a   per word basis only to pick an initial set of supertags for a given   word.} 
 
We also performed experiments to determine the accuracy of the 
derivation structures produced by XTAG on WSJ text, where the 
derivation tree produced after parsing XTAG is interpreted as a 
dependency parse. We took sentences that were $15$ words or less from 
the Penn Treebank~\cite{marcus93}. The sentences were collected from 
sections $17$--$23$ of the Treebank. $9891$ of these sentences were 
given at least one parse by the XTAG system. Since XTAG typically 
produces several derivations for each sentence we simply picked a 
single derivation from the list for this evaluation. Better results 
might be achieved by ranking the output of the parser using the sort 
of approach described in~\cite{srinietal95}. 
 
There were some striking differences in the dependencies implicit in 
the Treebank and those given by XTAG derivations. For instance, often 
a subject NP in the Treebank is linked with the first auxiliary verb 
in the tree, either a modal or a copular verb, whereas in the XTAG 
derivation, the same NP will be linked to the main verb. Also XTAG 
produces some dependencies within an NP, while a large number of words 
in NPs in the Treebank are directly dependent on the verb. To 
normalize for these facts, we took the output of the NP and VG chunker 
described above and accepted as correct any dependencies that were 
completely contained within a single chunk. 
 
For example, for the sentence {\em Borrowed shares on the Amex rose to another record}, the XTAG and Treebank chunks are shown below. 
 
\begin{verbatim} 
XTAG chunks:     
 [Borrowed shares] [on the Amex] [rose] 
    [to another record] 
Treebank chunks: 
 [Borrowed shares on the Amex] [rose] 
    [to another record] 
\end{verbatim} 
 
Using these chunks, we can normalize for the fact that in the 
dependencies produced by XTAG {\em borrowed} is dependent on {\em shares} (i.e. in the same chunk) while in the Treebank {\em borrowed} 
is directly dependent on the verb {\em rose}. That is to say, we are 
looking at links between \underline{chunks}, not between 
\underline{words}. The dependencies for the sentence are given below. 
 
\begin{verbatim} 
XTAG dependency    Treebank dependency 
 
Borrowed::shares   Borrowed::rose 
shares::rose       shares::rose 
on::shares         on::shares 
the::Amex          the::Amex 
Amex::on           Amex::on 
rose::NIL          rose::NIL 
to::rose           to::rose 
another::record    another::record 
record::to         record::to 
\end{verbatim} 
 
After this normalization, testing simply consisted of counting how 
many of the dependency links produced by XTAG matched the Treebank 
dependency links. Due to some tokenization and subsequent alignment 
problems we could only test on $835$ of the original $9891$ parsed 
sentences. There were a total of $6135$ dependency links extracted 
from the Treebank. The XTAG parses also produced $6135$ dependency 
links for the same sentences. Of the dependencies produced by the XTAG 
parser, $5165$ were correct giving us an accuracy of $84.2\%$. 
 
%\section{Performance on various corpora} 
 
%XTAG was used to parse Wall Street Journal (sentences of length $\leq$ 
%15 words), IBM manual, and ATIS corpora as a means of evaluating the 
%coverage and correctness of XTAG parses. For this evaluation, a 
%sentence is considered to have parsed if the correct parse is among 
%the parses generated by XTAG. Verifying the presence of the correct 
%parse among the generated parses is done manually at present by random 
%sampling. The results are shown in Table~\ref{results}. 
 
%\begin{table}[ht] 
%\begin{center} 
%\begin{tabular}{|l|c|c|c|} \hline 
%& \# of & & Av. \# of\\ 
%Corpus & Sentences & \% Parsed & Parses/Sent\\ \hline 
 
%WSJ & 18,730 & 41.22 \% & 7.46 \\\hline 
 
%IBM Manual & 2040 & 75.42\% & 6.12\\ \hline 
 
%ATIS & 524 & 88.35\% & 6.0 \\ \hline 
%\end{tabular} 
%\end{center} 
 
%\begin{rawhtml} <dl> <dt>{Performance of XTAG on various corpora <p> </dl> \end{rawhtml}
%\label{results} 
%\end{table} 
 
%Performance on the WSJ corpus is lower relative to IBM and ATIS due 
%to the wide-variety of syntactic constructions used. Even grammars 
%induced on the partially bracketed WSJ corpus have fairly low 
%performance (e.g. 57.1\% sentence accuracy for \cite{schabes93}).  
 
\section{Comparison with IBM} 
 
The evaluation in this section was done with the earlier 1995 release 
of the grammar. This section describes an experiment to measure the 
crossing bracket accuracy of the XTAG-parsed IBM-manual sentences.  In 
this experiment, XTAG parses of 1100 IBM-manual sentences have been 
ranked using certain heuristics. The ranked parses have been 
compared\footnote{We used the parseval program written by Phil Harison   (phil@atc.boeing.com).}  against the bracketing given in the 
Lancaster Treebank of IBM-manual sentences\footnote{The Treebank was   obtained through Salim Roukos (roukos@watson.ibm.com) at IBM.}. 
Table~\ref{ibm-results} shows the results of XTAG obtained in this 
experiment, which used the highest ranked parse for each system. It 
also shows the results of the latest IBM statistical grammar 
(\cite{jelineketal94}) on the same genre of sentences. Only the 
highest-ranked parse of both systems was used for this evaluation. 
Crossing Brackets is the percentage of sentences with no pairs of 
brackets crossing the Treebank bracketing (i.e.  (~(~a~b~)~c~) has a 
crossing bracket measure of one if compared to (~a~(~b~c~)~)~). Recall 
is the ratio of the number of constituents in the XTAG parse to the 
number of constituents in the corresponding Treebank sentence. 
Precision is the ratio of the number of correct constituents to the 
total number of constituents in the XTAG parse. 
 
\begin{table}[ht] 
\centering 
\begin{tabular}{|l|c|c|c|c|} \hline 
System & \# of & Crossing Bracket & Recall & Precision \\ 
& sentences & Accuracy & & \\ \hline 
XTAG & 1100 & 81.29\% & 82.34\% & 55.37\% \\ \hline 
IBM Statistical & 1100 & 86.20\% & 86.00\% & 85.00\% \\ 
grammar & &  &  &\\ \hline 
\end{tabular} 
 
\vspace{0.1in} 
 
\begin{rawhtml} <dl> <dt>{Performance of XTAG on IBM-manual sentences <p> </dl> \end{rawhtml}
\label{ibm-results} 
 
\end{table} 
 
As can be seen from Table~\ref{ibm-results}, the precision figure for 
the XTAG system is considerably lower than that for IBM. For the 
purposes of comparative evaluation against other systems, we had to 
use the same crossing-brackets metric though we believe that the 
crossing-brackets measure is inadequate for evaluating a grammar like 
XTAG. There are two reasons for the inadequacy. First, the parse 
generated by XTAG is much richer in its representation of the internal 
structure of certain phrases than those present in manually created 
treebanks (e.g. IBM: [$_N$ your personal computer], XTAG: [$_{NP}$ 
[$_G$ your] [$_N$ [$_N$ personal] [$_N$ computer]]]). This is 
reflected in the number of constituents per sentence, shown in the 
last column of Table~\ref{const-no}.\footnote{We are aware of the fact   that increasing the number of constituents also increases the recall   percentage. However we believe that this a legitimate gain.} 
 
\begin{table}[ht] 
\centering 
\begin{tabular}{|l|c|c|c|c|} \hline 
System & Sent. & \# of & Av. \# of & Av. \# of \\ 
& Length & sent & words/sent & Constituents/sent \\ \hline 
XTAG & 1-10 & 654 & 7.45 & 22.03  \\ \cline{2-5} 
& 1-15 & 978 & 9.13 & 30.56 \\ \hline 
IBM Stat. & 1-10 & 447 & 7.50 & 4.60 \\ \cline{2-5} 
Grammar & 1-15 & 883 & 10.30 & 6.40 \\ \hline 
\end{tabular} 
\begin{rawhtml} <dl> <dt>{Constituents in XTAG parse and IBM parse <p> </dl> \end{rawhtml}
\label{const-no} 
\end{table} 
 
A second reason for considering the crossing bracket measure 
inadequate for evaluating XTAG is that the primary structure in XTAG 
is the derivation tree from which the bracketed tree is derived. Two 
identical bracketings for a sentence can have completely different 
derivation trees (e.g. {\it kick the bucket} as an idiom vs. a 
compositional use). A more direct measure of the performance of XTAG 
would evaluate the derivation structure, which captures the 
dependencies between words. 
 
\section{Comparison with Alvey} 
 
The evaluation in this section was done with the earlier 1995 release 
of the grammar. This section compares XTAG to the Alvey Natural 
Language Tools (ANLT) Grammar. We parsed the set of LDOCE Noun Phrases 
presented in Appendix B of the technical report (\cite{Carroll93}) 
using XTAG.  Table~\ref{Alvey-xtag} summarizes the results of this 
experiment.  A total of 143 noun phrases were parsed. The NPs which 
did not have a correct parse in the top three derivations were 
considered failures for either system. The maximum and average number 
of derivations columns show the highest and the average number of 
derivations produced for the NPs that have a correct derivation in the 
top three.  We show the performance of XTAG both with and without the 
tagger since the performance of the POS tagger is significantly 
degraded on the NPs because the NPs are usually shorter than the 
sentences on which it was trained. It would be interesting to see if 
the two systems performed similarly on a wider range of data. 
 
\begin{table}[ht] 
\centering 
\begin{tabular}{|l|c|c|c|c|c|}  \hline 
% This table will compare the performance of the XTAG with Alvey. 
System & \# of & \# parsed & \% parsed & Maximum & Average \\ 
& NPs &&& derivations & derivations \\ \hline 
ANLT Parser & 143 & 127 & 88.81\% & 32 & 4.57 \\ \hline 
XTAG Parser with & 143 & 93 & 65.03\% & 28 & 3.45 \\ 
POS tagger & & & & & \\ \hline 
XTAG Parser without & 143 & 120 & 83.91\% & 28 & 4.14\\ 
POS tagger & & & & & \\ \hline 
\end{tabular} \\ 
 
\vspace{0.1in} 
 
\begin{rawhtml} <dl> <dt>{Comparison of XTAG and ANLT Parser <p> </dl> \end{rawhtml}
\label{Alvey-xtag} 
\end{table} 
 
%In \cite{Carroll93}, only the LDOCE NPs 
%are annotated with the number of derivations; we are interested 
%in getting more data annotated with this information, so that we can 
%make further comparisons. 
 
\section{Comparison with CLARE} 
 
The evaluation in this section was done with the earlier 1995 release 
of the grammar. This section compares the performance of XTAG against 
that of the CLARE-2 system (\cite{clare-report92}) on the ATIS corpus. 
Table~\ref{clare-results} shows the performance results. The 
percentage parsed column for both systems represents the percentage of 
sentences that produced any parse.  It must be noted that the 
performance result shown for CLARE-2 is without any tuning of the 
grammar for the ATIS domain. The performance of CLARE-3, a later 
version of the CLARE system, is estimated to be 10\% higher than that 
of the CLARE-2 system.\footnote{When CLARE-3 is tuned to the ATIS   domain, performance increases to 90\%. However XTAG has not been   tuned to the ATIS domain.} 
 
\begin{table}[ht] 
\centering 
\begin{tabular}{|l|c|c|}  \hline 
% This table will compare the performance of the XTAG with CLARE-2 
System & Mean length & \% parsed \\ \hline 
CLARE-2  & 6.53 & 68.50\% \\ \hline 
XTAG  & 7.62 & 88.35\% \\ \hline 
\end{tabular} 
\begin{rawhtml} <dl> <dt>{Performance of CLARE-2 and XTAG on the ATIS corpus <p> </dl> \end{rawhtml}
\label{clare-results} 
\end{table} 
 
In an attempt to compare the performance of the two systems on a wider 
range of sentences (from similar genres), we provide in 
Table~\ref{clare-results1} the performance of CLARE-2 on LOB corpus 
and the performance of XTAG on the WSJ corpus. The performance was 
measured on sentences of up to 10 words for both systems. 
\begin{table}[ht] 
\centering 
\begin{tabular}{|c|c|c|c|}  \hline 
% This table will compare the performance of the XTAG with CLARE-2 
System & Corpus & Mean length & \% parsed \\ \hline 
CLARE-2 & LOB & 5.95 & 53.40\% \\ \hline 
XTAG & WSJ & 6.00 & 55.58\% \\ \hline 
\end{tabular} 
\begin{rawhtml} <dl> <dt>{Performance of CLARE-2 and XTAG on LOB and WSJ corpus respectively <p> </dl> \end{rawhtml}
\label{clare-results1} 
\end{table} 
 

\bibliographystyle{aaai-named}
\bibliography{xtag}

\end{document}



