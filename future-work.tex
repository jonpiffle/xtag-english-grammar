%\appendix
\chapter{Future Work}
\label{future-work}

\section{Adjective ordering}

At this point, the treatment of adjectives in the XTAG English grammar does not
include selectional or ordering restrictions.\footnote{This section is a repeat
of information found in section~\ref{adj-modifier}.} Consequently, any
adjective can adjoin onto any noun and on top of any other adjective already
modifying a noun. All of the modified noun phrases shown in (\ex{1})-(\ex{4})
currently parse.

\enumsentence{big green bugs}
\enumsentence{big green ideas}
\enumsentence{colorless green ideas}
\enumsentence{$\ast$green big ideas}

While (\ex{-2})-(\ex{0}) are all semantically anomalous, (\ex{0}) also suffers
from an ordering problem that makes it seem ungrammatical as well.  Since the
XTAG grammar focuses on syntactic constructions, it should accept
(\ex{-3})-(\ex{-1}) but not (\ex{0}).  Both the auxiliary and determiner
ordering systems are structured on the idea that certain types of lexical items
(specified by features) can adjoin onto some types of lexical items, but not
others.  We believe that an analysis of adjectival ordering would follow the
same type of mechanism.

\section{More work on Determiners}

In addition to the analysis described in Chapter~\ref{det-comparitives}, there
remains work to be done to complete the analysis of determiner constructions in
English.\footnote{This section is from \cite{ircs:det98}.}  Although
constructions such as determiner coordination are easily handled if
overgeneration is allowed, blocking sequences such as {\it one and some} while
allowing sequences such as {\it five or ten} still remains to be worked out.
There are still a handful of determiners that are not currently handled by our
system.  We do not have an analysis to handle {\it most}, {\it such}, {\it
certain}, {\it other} and {\it own}\footnote{The behavior of {\it own} is
sufficiently unlike other determiners that it most likely needs a tree of its
own, adjoining onto the right-hand side of genitive determiners.}.  In
addition, there is a set of lexical items that we consider adjectives ({\it
enough}, {\it less}, {\it more} and {\it much}) that have the property that
they cannot cooccur with determiners.  We feel that a complete analysis of
determiners should be able to account for this phenomenon, as well.

%\section{{\it -ing} adjectives}

%An analysis has already been provided for past participal ({\it -ed})
%adjectives (as in sentence~ (\ex{1})), which are restricted to the
%Transitive Verb family.\footnote{This analysis may need to be extended
%to the Transitive Verb particle family as well.}  A similar analysis
%needs to take place for the present participle~({\it -ing}) used as a
%pre-nominal modifier.  This type of adjective, however, does not seem
%to be as restricted as the~{\it -ed} adjectives, since verbs in other
%tree families seem to exhibit this alternation as well
%(e.g. sentences~(\ex{2}) and (\ex{3})).

%\enumsentence{The murdered man was a doctoral student at UPenn .}
%\enumsentence{The man died .}
%\enumsentence{The dying man pleaded for his life .}

\section{Removal of empty elements as anchors}

In the current version of the grammar, we dispensed with some empty
elements that were lexicalizing trees in the grammar. One was the null COMP
used in relative clauses, the second was the empty verb anchoring the
auxiliary verb tree used for multi-component adjunction, and the third was
the non-lexical PRO which anchored the initial NP tree and substituted into
the subject NP nodes of non-ECM infinitival clauses. However, there is
still one more empty element remaining in the grammar which we also need to
remove. This is the empty prepositional element anchoring the tree for
subordinating conjunctions (see Chapter~\ref{Adjunct Clauses}). These
non-lexicalized trees are used to handle bare adjunct clauses like:

\enumsentence{The fire raged for days, destroying the building .}

\section{Verb selectional restrictions}

Although we explicitly do not want to model semantics in the XTAG grammar,
there is some work along the syntax/semantics interface that would help reduce
syntactic ambiguity and thus decrease the number of semantically anomalous
parses.  In particular, verb selectional restrictions, particularly for PP
arguments and adjuncts, would be quite useful.  With the exception of the
required {\it to} in the Ditransitive with PP Shift tree family (Tnx0Vnx1Pnx2),
any preposition is allowed in the tree families that have prepositions as their
arguments.  In addition, there are no restrictions as to which prepositions are
allowed to adjoin onto a given verb.  The sentences in (\ex{1})-(\ex{3}) are
all currently accepted by the XTAG grammar.  Their violations are stronger than
would be expected from purely semantic violations, however, and the presence of
verb selectional restrictions on PP's would keep these sentences from being
accepted.

\enumsentence{\#survivors walked of the street .}
\enumsentence{\#The man about the earthquake survived .}
\enumsentence{\#The president arranged on a meeting .}

\section{Thematic Roles}

Elementary trees in TAGs capture several notions of locality, with the most
primary of these being locality of $\theta$-role assignment.  Each elementary
tree has associated with it the $\theta$-roles assigned by the anchor of that
elementary tree.  In the current XTAG system, while the notion of locality of
$\theta$-role assignment within an elementary tree has been implicit, the
$\theta$-roles assigned by a head have not been explicitly represented in the
elementary tree. Incorporating $\theta$-role information will make the
elementary trees more informative and will enable efficient pruning of spurious
derivations when embedded into a specific context.  In the case of a
Synchronous TAG, $\theta$-roles can also be used to automatically establish
links between two elementary trees, one in the object language and one in the
target language.





















