\section{Introduction}
The English grammar described in this report is based on the TAG formalism
developed in Joshi, Levy, and Takahashi \cite{joshi75}, which has been extended
to include lexicalization \cite{schabes88}, and unification-based feature
structures \cite{vijay91}. Tree Adjoining Languages (TALs) fall into the class
of mildly context-sensitive languages, and as such are more powerful than
context free languages.  The TAG formalism in general, and lexicalized TAGs in
particular, are well-suited for linguistic applications.  As first shown by
\cite{joshi85} and \cite{kj87}, the properties of TAGs permit us to encapsulate
diverse syntactic phenomena in a very natural way.  For example, TAG's extended
domain of locality and its factoring of recursion from local dependencies lead,
among other things, to a localization of so-called unbounded dependencies.

\subsection{TAG formalism}

The primitive elements of the standard TAG formalism are known as {\sc
elementary trees}.  Elementary trees are of two types: {\sc initial trees} and
{\sc auxiliary trees} (see Figure \ref{elem-fig}).  In describing natural
language, {\sc initial trees} are minimal linguistic structures that contain no
recursion, i.e. trees containing the phrasal structure of simple sentences,
NPs, PPs, and so forth.  Initial trees are characterized by the following: 1)
all internal nodes are labeled by non-terminals, 2) all leaf nodes are labeled
by terminals, or by non-terminal nodes marked for substitution. An initial tree
is called an X-type initial tree if its root is labeled with type X.  

\begin{figure}[ht]
\centering
\rule[.1in]{\textwidth}{0.01in} 
\psfig{figure=ps/intro-files/schematic-elem-trees.ps,height=2.0in}
\caption{Elementary trees in TAG}
\rule[.1in]{\textwidth}{0.01in} 
\label{elem-fig}
\end{figure}

Recursive structures are represented by {\sc auxiliary trees}, which represent
constituents that are adjuncts to basic structures (e.g. adverbials).  {\sc
Auxiliary trees} are characterized as follows: 1) all internal nodes are
labeled by non-terminals, 2) all leaf nodes are labeled by terminals, or by
non-terminal nodes marked for substitution, except for exactly one non-terminal
node, called the foot node, which can only be used to adjoin the tree to
another node\footnote{A null adjunction constraint (NA) is systematically put
on the footnode of an auxiliary tree. This disallows adjunction of a tree onto
a footnode itself.}, 3) the foot node has the same label as the root node of
the tree.

There are two operations defined in the TAG formalism,
substitution\footnote{Technically, substitution is a specialized version of
adjunction, but it is useful to make a distinction between the two.} and
adjunction.  In the substitution operation, the root node on an initial tree is
merged into a non-terminal leaf node marked for substitution in another initial
tree, producing a new tree.  The root node and the substitution node must have
the same name.  Figure \ref{proto-subst} shows two initial trees and the tree
resulting from the substitution of one tree into the other.

\begin{figure}[ht]
\centering
\rule[.1in]{\textwidth}{0.01in} 
\psfig{figure=ps/intro-files/schematic-subst2.ps,height=2.0in}
\caption{Substitution in TAG}
\rule[.1in]{\textwidth}{0.01in} 
\label{proto-subst}
\end{figure}

In an adjunction operation, an auxiliary tree is grafted onto a non-terminal
node anywhere in an initial tree.  The root and foot nodes of the auxiliary
tree must match the node at which the auxiliary tree adjoins.  Figure
\ref{proto-adjunction} shows an auxiliary tree and an initial tree, and the
tree resulting from an adjunction operation.

\begin{figure}[ht]
\centering
\rule[.1in]{\textwidth}{0.01in} 
\psfig{figure=ps/intro-files/schematic-adjunction2.ps,height=2.0in}
\caption{Adjunction in TAG}
\rule[.1in]{\textwidth}{0.01in} 
\label{proto-adjunction}
\end{figure}

The tree set of a TAG $G$, ${\cal T}(G)$ is defined to be the set
of all derived trees starting from S-type initial trees in $I$ whose frontier
consists of terminal nodes (all substitution nodes having been filled). The
string language generated by a TAG, ${\cal L}(G)$, is defined to be the
set of all terminal strings on the frontier of the trees in ${\cal T}(G)$.

\subsection{Lexicalization}

`Lexicalized' grammars systematically associate each elementary structure with
a lexical anchor. This means that in each structure there is a lexical item
that is realized.  It does not mean simply adding feature structures (such as
head) and unification equations to the rules of the formalism.  These resultant
elementary structures specify extended domains of locality (as compared to
CFGs) over which constraints can be stated.

Following \cite{schabes88} we say that a grammar is `lexicalized' if it
consists of 1) a finite set of structures each associated with a lexical item,
and 2) an operation or operations for composing the structures.  Each lexical
item will be called the {\it anchor} of the corresponding structure, which
defines the domain of locality over which constraints are specified.  Note
then, that constraints are local with respect to their anchor.

Not every grammar is in a `lexicalized' form.\footnote{Notice the similarity of
the definition of `lexicalized' grammar with the off line parsibility
constraint \cite{kaplan83}. As consequences of our definition, each structure
has at least one lexical item (its anchor) attached to it and all sentences are
finitely ambiguous.} In the process of lexicalizing a grammar, the
`lexicalized' grammar is required to be strongly equivalent to the original
grammar, i.e., it must produce not only the same language, but the same
structures or tree set as well.

\begin{figure*}[ht]
\centering
\rule[.1in]{\textwidth}{0.01in} 
\begin{tabular}{cccc}
{{\psfig{figure=ps/intro-files/john.ps,height=1.0in}}\label{fig1a}}  &
{{\psfig{figure=ps/intro-files/walked.ps,height=1.4in}}\label{fig1b}}  & 
{{\psfig{figure=ps/intro-files/to.ps,height=1.7in}} \label{fig1c} }  & 
{{\psfig{figure=ps/intro-files/philly.ps,height=1.0in}} \label{fig1d}} \\
(a)&(b)&(c)&(d)\\
\end{tabular}\\
\caption {Lexicalized Elementary Trees}
\rule[.1in]{\textwidth}{0.01in} 
\label {lex-elem-trees}
\end{figure*}

In Figure \ref{lex-elem-trees}, which shows sample initial and auxiliary trees,
substitution sites are marked by a ($\downarrow$), and foot nodes are marked by
an ($\ast$).  This notation is standard and is followed in the rest of this
report.


\subsection{Unification-based features}

In a unification framework, a feature structure is associated with each node in
an elementary tree.  This feature structure contains information about how the
node interacts with other nodes in the tree.  It consists of a top part, which
generally contains information relating to the supernode, and a bottom part,
which generally contains information relating to the subnode.  Substitution
nodes, however, have only the top features, since the tree substiuting in must
logically carry the bottom features.

The notions of substitution and adjunction must be augmented to fit within this
new framework.  The feature structure of a new node created by substitution
inherits the union of the features of the original nodes.  The top feature of
the new node is the union of the top features of the two original nodes, while
the bottom feature of the new node is simply the bottom feature of the top node
of the subsituting tree (since the substitution node has no bottom feature).
Feature \ref{subst-fig} shows this more clearly.

\begin{figure}[ht]
\centering
\rule[.1in]{\textwidth}{0.01in} 
\psfig{figure=ps/intro-files/schematic-feat-subst.ps,height=2.0in}
\caption{Substitution in FB-LTAG}
\rule[.1in]{\textwidth}{0.01in} 
\label{subst-fig}
\end{figure}

Adjunction is only slightly more complicated.  The node being adjoined into
splits, and its top feature unifies with the top feature of the root
adjoining node, while its bottom feature unifies with the bottom feature of the
foot adjoining node.  Again, this is easier shown graphically, as in Figure
\ref{adjunct-fig}.

\begin{figure}[ht]
\centering
\rule[.1in]{\textwidth}{0.01in} 
\psfig{figure=ps/intro-files/schematic-feat-adjunction.ps,height=2.0in}
\caption{Adjunction in FB-LTAG}
\label{adjunct-fig}
\rule[.1in]{\textwidth}{0.01in} 
\end{figure}


The embedding of the TAG formalism in a unification framework allows
us to dynamically specify local constraints that would have otherwise
had to have been made statically within the trees.  Constraints that
verbs make on their complements, for instance, can be implemented
through the feature structures.  The notions of Obligatory and
Selective Adjunction, crucial to the formation of lexicalized
grammars, can also be handled through the use of
features.\footnote{The remaining constraint, Null Adjunction (NA),
must still be specified directly on a node.} Perhaps more important to
developing a grammar, though, is that the trees can serve as a
schemata to be instantiated with lexical-specific features when an
anchor is associated with the tree.  To illustrate this, Figure
\ref{lex-with-features} shows the same tree lexicalized with two
different verbs, each of which instantiates the features of the tree
according to its lexical semantics.

\begin{figure*}[ht]
\centering
\begin{tabular}{cc}
{\psfig{figure=ps/intro-files/think-feat.ps,height=5.0in}}  &
{\psfig{figure=ps/intro-files/want-feat.ps,height=5.0in}} \\
{\it think} tree&{\it want} tree\\
\end{tabular}\\
\caption {Lexicalized Elementary Trees with Features}
\label {lex-with-features}
\end{figure*}

In Figure \ref{lex-with-features}, the lexical item {\it thinks} takes an
indicative sentential complement, as you might find in a sentence such as {\it
John thinks that Mary loves Sally}.  {\it Want} takes a sentential complement
as well, but an infinitive one, as in {\it John wants to love Mary}.  This
distinction is easily captured in the features and passed to other nodes to
constrain which trees this tree can adjoin into, both cutting down the number
of separate trees needed and enforcing conceptual Selective Adjunctions (SA).


