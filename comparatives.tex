\chapter{Comparatives}
\label{compars-chapter}

\section{Introduction}

Comparatives in English can manifest themselves in many ways, acting
on many different grammatical categories and often involving ellipsis.
A distinction must be made at the outset between two very different
sorts of comparatives---those which make a comparison between two
propositions and those which compare the extent to which an entity has
one property to a greater or lesser extent than another property.  The
former, which we will refer to as {\it propositional} comparatives, is
exemplified in (\ex{1}), while the latter, which we will call {\it
metalinguistic} comparatives (following Hellan 1981), is seen in
(\ex{2}):

\enumsentence{Ronaldo is more angry than Romario.}
\enumsentence{Ronaldo is more angry than upset.}

\noindent In (\ex{-1}), the extent to which Ronaldo is angry is greater
than the extent to which Romario is angry.  Sentence (\ex{0})
indicates that the extent to which Ronaldo is angry is greater than
the extent to which he is upset.

Apart from certain of the elliptical cases, both kinds of comparatives
can be handled straightforwardly in the XTAG system.  Elliptical cases
which are not presently covered include those exemplified by the
following sentences, which would presumably be handled in the same way
as other sorts of VP ellipsis would.

\enumsentence{Ronaldo is more angry than Romario is.}
\enumsentence{Bill eats more broccoli than George eats.}
\enumsentence{Bill eats more broccoli than George does.}

We turn to the analysis of metalinguistic comparatives first.

\section{Metalinguistic Comparatives}

A metalinguistic comparison can be performed on basically all of the
predicational categories---adjectives, verb phrases, prepositional
phrases, and nouns---as in the following examples:

\enumsentence{The table is more long than wide. (AP)}
\enumsentence{Clark more makes the rules than follows them. (VP)}
\enumsentence{Calvin is more in the living room than in the kitchen. (PP)}
\enumsentence{That unindentified amphibian in the bush is more frog
than toad, I would say. (NP)}

\noindent At present, we only deal with the adjectival metalinguistic
comparatives as in (\ex{-3}).  The analysis given here for these can
be easily extended to prepositional phrases and nominal comparatives of
the metalinguistic sort, but, as with coordination in XTAG, verb
phrases will prove more difficult.

Adjectival comparatives appear to distribute with simple adjectives,
as in the following examples:

\enumsentence{Herbert is more livid than angry.}
\enumsentence{Herbert is more livid and furious than angry.}
\enumsentence{The more innovative than conventional medication cured
everyone in the sick ward.}
\enumsentence{The elephant, more wobbly than steady, fell from the
circus ball.}

\begin{figure}[htb]
\centering
\begin{tabular}{cc}
{\psfig{figure=ps/comparatives-files/betaARBaPa.ps,height=2.0in}}
\end{tabular}\\
\caption {Tree for Metalinguistic Adjective Comparative: $\beta$ARBaPa}
\label {ARBaPa-tree}
\end{figure}

This patterning indicates that we can give these comparatives a tree
that adjoins quite freely onto adjectives, as in
Figure~\ref{ARBaPa-tree}.  This tree is anchored by {\it more/less -
than}.  To avoid grammatically incorrect comparisons such as {\it more
brighter than dark}, the feature {\bf compar} is used to block this
tree from adjoining onto morphologically comparative adjectives.  The
foot node is {\bf compar-}, while {\it brighter} and its comparative
siblings are {\bf compar+}\footnote {The analysis given later for
adjectival propositional comparatives produces aggregated {\bf
compar+} adjectives such as {\it more bright}, which will also be
incompatible (as desired) with $\beta$ARBaPa.}.  We also wish to block
strings like {\it more brightest than dark}, which is accomplished
with the feature {\bf super}, indicating superlatives.  This feature
is negative at the foot node so that $\beta$ARBaPa cannot adjoin to
superlatives like {\it nicest}, which are specified as {\bf super+}
from the morphology.  Furthermore, the root node is {\bf
super+} so that $\beta$ARBaPa cannot adjoin onto itself and produce
monstrosities such as (\ex{1}):

\enumsentence{*Herbert is more less livid than angry than furious.}

\noindent Thus, the use of the {\bf super} feature is less to indicate
superlativeness specifically, but rather to indicate that the subtree
below a {\bf super+} node contains a full-fleshed comparison.  In the
case of lexical superlatives, the comparison is against everything,
implicitly.

A benefit of the multiple-anchor approach here is that we will never
allow sentences such as (\ex{1}), which would be permissible if we
split the comparative component and the {\it than} component of
metalinguistic comparatives into two separate trees.

\enumsentence{*Ronaldo is angrier than upset.}

We also see another variety of adjectival comparatives of the form {\it
more/less than X}, which indicates some property which is more or less
extreme than the property {\it X}.  In a sentence such as (\ex{1}),
some property is being said to hold of Francis such that it is of a
kind with {\it stupid} and that it exceeds {\it stupid} on some scale
(intelligence, for example).  Quirk et al. also note that these
constructions remark on the inadequacy of the lexical item.  Thus, in
(\ex{0}), it could be that {\it stupid} is a starting point from which
the speaker makes an approximation for some property which the speaker
feels is beyond the range of the English lexicon, but which expresses
the supreme lack of intellect of the individual it is predicated of.

\enumsentence{Francis is more than stupid.}
\enumsentence{Romario is more than just upset.}

Taking our inspiration from $\beta$ARBaPa, we can handle these
comparatives, which have the same distribution but contain an empty
adjective, by using the tree shown in Figure~\ref{ARBPa-tree}.

\begin{figure}[htb]
\centering
\begin{tabular}{cc}
{\psfig{figure=ps/comparatives-files/betaARBPa.ps,height=2.0in}}
\end{tabular}\\
\caption {Tree for Adjective-Extreme Comparative: $\beta$ARBPa}
\label {ARBPa-tree}
\end{figure}

This sort of metalinguistic comparative also occurs with the verb
phrase, prepositional phrase, and noun varieties.

\enumsentence{Clark more than makes the rules. (VP)}
\enumsentence{Calvin's hands are more than near the cookie jar. (PP)}
\enumsentence{That stuff on her face is more than mud. (NP)}

\noindent Presumably, the analysis for these would parallel that for
adjectives, though it has not yet been implemented.


\section{Propositional Comparatives}

\subsection{Nominal Comparatives}\label{nom-comparatives-section}

Nominal comparatives are considered here to be those which compare the
cardinality of two sets of entities denoted by nominal phrases.  The
following data lay out a basic distribution of these comparatives.

\enumsentence{More vikings than mongols eat spam.}
\enumsentence{*More the vikings than mongols eat spam.}
\enumsentence{Vikings eat less spaghetti than spam.}
\enumsentence{More men that walk to the store than women who despise
spam enjoyed the football game.}
\enumsentence{More men than James hate scotch on the rocks.}
\enumsentence{James knows fewer bunnies than rabbits.}

Looking at these examples, we are tempted to produce a tree for this
construction that is similar to $\beta$ARBaPa.  However, it is quite
common for the {\it than} portion of these comparatives to be left
out, as in the following sentences:

\enumsentence{More vikings eat spam.}
\enumsentence{Mongols eat less spam.}

\noindent Furthermore, {\it than NP} cannot occur without {\it more}.
These facts indicate that we can and should build up nominal
comparatives with two separate trees.  The first, which allows a
comparative adverb to adjoin to a noun, is given in
Figure~\ref{nom-compar}(a). The second is the noun-phrase modifying
prepositional tree.  The tree $\beta$CARBn is anchored by {\it
more/less/fewer} and $\beta$CnxPnx is anchored by {\it than}.  The
feature {\bf compar} is used to ensure that only one $\beta$CARBn tree
can adjoin to any given noun---its foot node is {\bf compar-} and the
root node is {\bf compar+}.  All nouns are {\bf compar-}, and the {\bf
compar} value is passed up through all trees which adjoin to N or NP.
In order to ensure that we do not allow sentences like *{\it Vikings
than mongols eat spam}, the {\bf compar} feature is used.  The NP foot
node of $\beta$CnxPnx is {\bf compar+}; thus, $\beta$CnxPnx will
adjoin only to NP's which have been already modified by $\beta$CARBn
(and thereby comparativized).  In this way, we capture sentences like
(\ex{-1}) en route to deriving sentences like (\ex{-7}), in a
principled and simple manner.

\begin{figure}[htbp]
\centering
\begin{tabular}{ccc}
{\psfig{figure=ps/comparatives-files/betaCARBn.ps,height=2.1in}}  &
\hspace{0.6in}
{\psfig{figure=ps/comparatives-files/betanxPnx.ps,height=1.7in}} \\
(a) $\beta$CARBn tree& \qquad(b) $\beta$CnxPnx tree \\
\end{tabular}\\
\caption {Nominal comparative trees}
\label {nom-compar}
\end{figure}

Further evidence for this approach comes from comparative clauses
which are missing the noun phrase which is being compared against
something, as in the following:

\enumsentence{The vikings ate more.\footnote{We ignore here the
interpretation in which the comparison covers the eating event,
focussing only on the one which the comparison involves the stuff
being eaten.}}
\enumsentence{The vikings ate more than a boar.\footnote{This
sentence differs from the metalinguistic comparison {\it That stuff on
her face is more than mud} in that it involves a comment on the
quantity and/or type of the compared NP, whereas the other expresses
that the property denoted by the compared noun is an inadequate
characterization of the thing being described.}}

\noindent Sometimes the missing noun refers to an entity or set
available in the prior discourse, while at other times it is a
reference to some anonymous, unspecified set.  The former is
exemplified in a mini-discourse such as the following: \\

\noindent Calvin: ``The mongols ate spam.''\\
\noindent Hobbes: ``The vikings ate more.''  \\

\noindent The latter can be seen in the following example: \\

\noindent Calvin: ``The vikings ate a a boar.''\\
\noindent Hobbes: ``Indeed. But in fact, the vikings ate more than a boar.'' \\

Since the lone comparatives {\it more/less/fewer} have the same basic
distribution as noun phrases, the tree in Figure~\ref{lone-compar} is
employed to capture this fact. The root node of $\alpha$CARB is {\bf
compar+}.  Not only does this accord with our intuitions about what
the {\bf compar} feature is supposed to indicate, it also permits
$\beta$nxPnx to adjoin, giving us strings such as {\it more than NP}
for free.

\begin{figure}[htb]
\centering
\begin{tabular}{cc}
{\psfig{figure=ps/comparatives-files/alphaCARB.ps,height=2.0in}}
\end{tabular}\\
\caption {Tree for Lone Comparatives: $\alpha$CARB}
\label {lone-compar}
\end{figure}

Thus, by splitting nominal comparatives into multiple trees, we make
correct predictions about their distribution with a minimal number of
simple trees.  Furthermore, we now also get certain comparative
coordinations for free, once we place the requirement that nouns and
noun phrases must match for {\bf compar} if they are to be
coordinated.  This yields strings such as the following:

\enumsentence{James eats more grapes and fewer boars than avocados.}
\enumsentence{Were there more or less than fifty people (at the party)?}

\noindent The structures are given in Figure~\ref{comparconjs}. Also, 
it will block strings like {\it more men and women than
children} under the (impossible) interpretation that there are more
men than children but the comparison of the quantity of women to
children is not performed.  Unfortunately, it will permit comparative
clauses such as {\it more grapes and fewer than avocados} under the
interpretation in which there are more grapes than avocados and fewer
of some unspecified thing than avocados (see Figure~\ref{badcomparconj}).

\begin{figure}[htb]
\centering
\begin{tabular}{cc}
{\psfig{figure=ps/comparatives-files/moregrapes.ps,height=3.0in}}\\
{\psfig{figure=ps/comparatives-files/fiftypeople.ps,height=3.0in}}
\end{tabular}\\
\caption {Comparative conjunctions.}
\label{comparconjs}
\end{figure}

\begin{figure}[htb]
\centering
\begin{tabular}{cc}
{\psfig{figure=ps/comparatives-files/fewerthanavocados.ps,height=3.0in}}
\end{tabular}\\
\caption {Comparative conjunctions.}
\label{badcomparconj}
\end{figure}

One aspect of this analysis is that it handles the elliptical
comparatives such as the following:

\enumsentence{Arnold kills more bad guys than Steven.}

\noindent In a sense, this is actually only simulating the ellipsis of
these constructions indirectly.  However, consider the following
sentences:

\enumsentence{Arnold kills more bad guys than I do.}
\enumsentence{Arnold kills more bad guys than I.}
\enumsentence{Arnold kills more bad guys than me.}

\noindent The first of these has a {\it pro}-verb phrase which has a
nominative subject.  If we totally drop the second verb phrase, we
find that the second NP can be in either the nominative or the
accusative case.  Prescriptive grammars disallow accusative case, but
it actually is more common to find accusative case---use of the
nominative in conversation tends to sound rather stiff and unnatural.
This accords with the present analysis in which the second noun phrase
in these comparatives is the complement of {\it than} in $\beta$nxPnx,
and receives its case-marking from {\it than}.  This does mean that
the grammar will not currently accept (\ex{-1}), and indeed such
sentences will only be covered by an analysis which really deals with
the ellipsis.  Yet the fact that most speakers produce (\ex{0})
indicates that some sort of restructuring has occured that results in
the kind of structure the present analysis offers.

There is yet another distributional fact which falls out of this
analysis.  When comparative or comparativized adjectives modify a noun
phrase, they can stand alone or occur with a {\it than} phrase;
furthermore, they are obligatory when a {\it than}-phrase is present.

\enumsentence{Hobbes is a better teacher.}
\enumsentence{Hobbes is a better teacher than Bill.}
\enumsentence{A more exquisite horse launched onto the racetrack.} 
\enumsentence{A more exquisite horse than Black Beauty launched onto
the racetrack.} 
\enumsentence{*Hobbes is a teacher than Bill.}

\noindent Comparative adjectives such as {\it better} 
come from the lexicon as {\bf compar+}.  By having trees such as
$\beta$An transmit the {\bf compar} value of the A node to the root
N node, we can signal to $\beta$CnxPnx that it may adjoin when a
comparative adjective has adjoined.  An example of such an adjunction
is given in Figure~\ref{better-teacher-than-Bill}. Of course, if no
comparative element is present in the lower part of the noun phrase,
$\beta$nxPnx will not be able to adjoin since nouns themselves are
{\bf compar-}.  In order to capture the fact that a comparative
element blocks further modification to N, $\beta$An must only adjoin
to N nodes which are {\bf compar-} in their lower feature matrix.

\begin{figure}[htb]
\centering
\begin{tabular}{cc}
{\psfig{figure=ps/comparatives-files/better_teacher_than_Bill_nf.ps,height=3.0in}}
\end{tabular}\\
\caption {Adjunction of $\beta$nxPnx to NP modified by comparative adjective.}
\label {better-teacher-than-Bill}
\end{figure}

In order to obtain this result for phrases like {\it more exquisite
horse}, we need to provide a way for {\it more} and {\em less} to
modify adjectives without a {\it than}-clause as we have with
$\beta$ARBaPa.  Actually, we need this ability independently for
comparative adjectival phrases, as discussed in the next section.

\subsection{Adjectival Comparatives}

With nominal comparatives, we saw that a single analysis was amenable
to both ``pure'' comparatives and elliptical comparatives.  This is
not possible for adjectival comparatives, as the following examples
demonstrate: 

\enumsentence{The dog is less stupid.}
\enumsentence{The dog is less stupid than the cat.}
\enumsentence{John is as stupid.}
\enumsentence{John is as stupid as Mary.}
\enumsentence{The less stupid dog waited quietly for its master.}
\enumsentence{*The less stupid than the cat dog waited eagerly for
its master.}

\noindent The last example shows that comparative adjectival phrases cannot
distribute quite as freely as comparative nominals.

The analysis of elliptical comparative adjectives follows closely to
that of comparative nominals.  We build them up by first adjoining the
comparative element to the A node, which then signals to the AP node,
via the {\bf compar} feature, that it may allow a {\it than}-clause to
adjoin.  The relevant trees are given in
Figure~\ref{ellip-adj-compar}.  $\beta$CARBa is anchored by {\it more,
less} and {\it as}, and $\beta$axPnx is anchored by both {\it than}
and {\it as}.

\begin{figure}[htbp]
\centering
\begin{tabular}{ccc}
{\psfig{figure=ps/comparatives-files/betaCARBa.ps,height=1.2in}}  &
\hspace{0.6in}
{\psfig{figure=ps/comparatives-files/betaaxPnx.ps,height=2.0in}} \\
(a) $\beta$CARBa tree& \qquad(b) $\beta$axPnx tree \\
\end{tabular}\\
\caption {Elliptical adjectival comparative trees}
\label {ellip-adj-compar}
\end{figure}

The advantages of this analysis are many.  We capture the
distribution exhibited in the examples given in (\ex{-5}) - (\ex{0}).
With $\beta$CARBa, comparative elements may modify adjectives wherever
they occur.  However, {\it than} clauses for adjectives have a more
restricted distribution which coincides nicely with the distribution
of AP's in the XTAG grammar.  Thus, by making them adjoin to AP rather
than A, ill-formed sentences like (\ex{0}) are not allowed.

\begin{figure}[htb]
\centering
\begin{tabular}{cc}
{\psfig{figure=ps/comparatives-files/black_beauty.ps,height=4.0in}}
\end{tabular}\\
\caption {Comparativized adjective triggering $\beta$CnxPnx.}
\label {black_beauty}
\end{figure}


There are two further advantages to this analysis.  One is that
$\beta$CARBa interacts with $\beta$nxPnx to produce sequences like
{\it more exquisite horse than Black Beauty}, a result alluded to at
the end of Section~\ref{nom-comparatives-section}.  We achieve this by
ensuring that the comparativeness of an adjective is controlled by a
comparative adverb which adjoins to it.  A sample derivation is given
in Figure~\ref{black_beauty}.  The second advantage is that we get
sentences such as (\ex{1}) for free.

\enumsentence{Hobbes is better than Bill.}

\noindent Since {\it better} comes from the lexicon as {\bf compar+}
and this value is passed up to the AP node, $\beta$axPnx can adjoin as
desired, giving us the derivation given in
Figure~\ref{better-than-Bill}.

\begin{figure}[htb]
\centering
\begin{tabular}{cc}
{\psfig{figure=ps/comparatives-files/better_than_Bill_f.ps,height=4.0in}}
\end{tabular}\\
\caption {Adjunction of $\beta$axPnx to comparative adjective.}
\label {better-than-Bill}
\end{figure}

Notice that the root AP node of Figure~\ref{better-than-Bill} is {\bf
compar-}, so we are basically saying that strings such as {\it better
than Bill} are not ``comparative.''  This accords with our use of the
{\bf compar} feature---a positive value for {\bf compar} signals that
the clause beneath it is to {\bf be} compared against something else.
In the case of {\it better than Bill}, the comparison has been
fulfilled, so we do not want it to signal for further comparisons.  A
nice result which follows is that $\beta$axPnx cannot adjoin more than
once to any given AP spine, and we have no need for the NA constraint
on the tree's root node.  Also, this treatment of the comparativeness
of various strings proves important in getting the coordination of
comparative constructions to work properly.

A note needs to be made about the analysis regarding the interaction
of the equivalence comparative construction {\it as ... as} and the
inequivalence comparative construction {\it more/less ... than}.  In
the grammar, {\it more, less}, and {\it as} all anchor $\beta$CARBa, and
both {\it than} and {\it as} anchor $\beta$axPnx.  Without further
modifications, this of course will give us sentences such as the
following:

\enumsentence{*?Hobbes is as patient than Bill.}
\enumsentence{*?Hobbes is more patient as Bill.}

\noindent Such cases are blocked with the feature {\bf equiv}:  {\it
more, less, fewer} and {\it than} are {\bf equiv-} while {\it as} (in
both adverbial and prepositional uses) is {\bf equiv+}.  The
prepositional trees then require that their P node and the node to
which they are adjoining match for {\bf equiv}.

An interesting phenomena in which comparisons seem to be paired with
an inappropriate {\it as/than}-clause is exhibited in (\ex{1}) and
(\ex{2}).

\enumsentence{Hobbes is as patient or more patient than Bill.}
\enumsentence{Hobbes is more patient or as patient as Bill.}

\noindent Though prescriptive grammars disfavor these sentences, these
are perfectly acceptable.  We can capture the fact that the {\it
as/than}-clause shares the {\bf equiv} value with the latter of the
comparison phrases by passing the {\bf equiv} value for the second
element to the root of the coordination tree.


\subsection{Adverbial Comparatives}

The analysis of adverbial comparatives encouragingly parallels the
analysis for nominal and elliptical adjectival comparatives---with,
however, some interesting differences.  Some examples of adverbial
comparatives and their distribution are given in the following:

\enumsentence{Albert works more quickly.}
\enumsentence{Albert works more quickly than Richard.}
\enumsentence{Albert works more.}
\enumsentence{*Albert more works.}
\enumsentence{Albert works more than Richard.}
\enumsentence{Hobbes eats his supper more quickly than Calvin.}
\enumsentence{Hobbes more quickly eats his supper than Calvin.}
\enumsentence{*Hobbes more quickly than Calvin eats his supper.}

\noindent When {\it more} is used alone as an adverb, it must also
occur after the verb phrase. Also, it appears that adverbs modified by
{\it more} and {\it less} have the same distribution as when they are
not modified.  However, the {\it than} portion of an adverbial
comparative is restricted to post verb phrase positions.

The first observation can be captured by having {\it more} and {\it
less} select only $\beta$vxARB from the set of adverb trees.
Comparativization of adverbs looks very similar to that of other
categories, and we follow this trend by giving the tree in
Figure~\ref{more-adv-mod}(a), which parallels the adjectival and
nominal trees, for these instances.  This handles the quite free
distribution of adverbs which have been comparativized, while the tree
in Figure~\ref{more-adv-mod}(b), $\beta$vxPnx, allows the {\it than}
portion of an adverbial comparative to occur only after the verb
phrase, blocking examples such as (\ex{0}). 

\begin{figure}[htbp]
\centering
\begin{tabular}{ccc}
{\psfig{figure=ps/comparatives-files/betaCARBarb.ps,height=1.2in}}  &
\hspace{0.6in}
{\psfig{figure=ps/comparatives-files/betavxPnx.ps,height=2.0in}} \\
(a) $\beta$CARBarb tree& \qquad(b) $\beta$vxPnx tree \\
\end{tabular}\\
\caption {Adverbial comparative trees}
\label {more-adv-mod}
\end{figure}

The usage of the {\bf compar} feature parallels that of the adjectives
and nominals; however, trees which adjoin to VP are {\bf compar-} on
their root VP node.  In this way, $\beta$vxPnx anchored by {\it than}
or {\it as} (which must adjoin to a {\bf compar+} VP) can only adjoin
immediately above a comparative or comparativized adverb.  This avoids
extra parses in which the comparative adverb adjoins at a VP node
lower than the {\it than}-clause.

A final note is that {\it as} may anchor $\beta$vxPnx
non-comparatively, as in sentence (\ex{1}). This means that
there will be two parses for sentences such as (\ex{2}).

\enumsentence{John works as a carpenter.}
\enumsentence{John works as quickly as a carpenter.}

\noindent This appears to be a legitimate ambiguity.  One is that John
works as quickly as a carpenter (works quickly), and the other is that
John works quickly when he is acting as a carpenter (but maybe he is
slow when he acting as a plumber).

\section{Future Work}

\begin{itemize}
\item Interaction with determiner sequencing (e.g., {\it several more
men than women} but not {\it *every more men than women}).

\item Handle sentential complement comparisons (e.g., {\it Bill eats
more pasta than Angus drinks beer}).

\item Add partitives.

\item Deal with constructions like {\it as many} and {\it as much}.

\item Look at {\it so...as} construction.

\end{itemize}




