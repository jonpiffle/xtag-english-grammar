\section{Extraction}
The discussion in this section covers constructions that are analyzed
as having wh-movement in GB.  In particular, wh-questions and
topicalization. Relative clauses which could also be considered
extractions are discussed in Section
\ref{rel_clauses}

Extraction involves a constituent appearing in a linear position to
the left of the clause with which it is interpreted. One clause
arguement position is empty. For example in the position filled by
{\it frisbee} in the declarative in (\ex{2}) is empty in (\ex{1}). The
wh-item {\it what} in (\ex{1}) is of the same syntactic category as
{\it frisbee} in (\ex{2}) and fills the same role with respect to the
subcategorization.

\enumsentence{What$_{i}$ did Clove catch $\epsilon_{i}$} 
\enumsentence{Clove caught a frisbee}

The tree that is used to derive (\ex{-1}) in the English LTAG grammar  is
shown in figure \ref{alphaW1nx0Vnx1}

\begin{figure}[htbp]
\center{
\psfig{figure=/mnt/linc/extra/xtag/work/doc/tech-rept/ps/extraction-files/alphaW1nx0Vnx1.ps,height=10.0cm}
\caption{ \label{alphaW1nx0Vnx1} Tree:  $\alpha$W1nx0Vnx1}
}
\end{figure}

As can be seen in figure \ref{alphaW1nx0Vnx1} (like GB) the English
LTAG grammar represents the connection between the extracted element
and the empty position with coindexing.  The {\bf $<$trace$>$} feature
is used to implement the coindexing.  In extraction trees in the
English LTAG grammar the ``empty'' position is filled with an
$\epsilon$.  The extracted item always appears in these trees as a
sister to the the S$_{r}$ tree, with both dominated by a S$_{q}$ root
node.  The S$_{r}$ subtrees in extraction trees have the same
structure as the declarative tree in the same tree family.  The
additional structure in extraction trees of the S$_{q}$ and NP nodes,
roughly corresponds to the CP and Spec of CP positions in GB.



\subsection{Topicalization and the value of the {\bf $<$inv$>$} feature}
Our analysis of topicalization uses the same trees as wh-extraction.
For any NP complement position a single tree is used for wh-questions
and for topicalization on that position. Wh-questions have
subject-auxiliary inversion and topicalizations do not.  This
difference between the constructions is captured by equating the
values of the S$_{r}$'s {\bf $<$inv$>$} feature and the extracted NP's
{\bf $<$wh$>$} feature.  This means that if the extracted item is a
wh-expression, as in wh-questions, the value of {\bf $<$inv$>$} will
be $+$ and an inverted auxiliary will be forced to adjoin. If the
extracted item is a non-wh, {\bf $<$inv$>$} will be $-$ and no
auxiliary adjunction will occur. An additional complication is that
inversion only occurs in matrix clauses, so the values of {\bf
$<$inv$>$} and {\bf $<$wh$>$} should only be equated in matrix clauses
and not in embedded clauses.  In the English LTAG Grammar appropriate
equating of {\bf $<$inv$>$} and {\bf $<$wh$>$} features is
accomplished using the {\bf $<$inv-link$>$} feature and a restriction
imposed on the root S of a derivation. In particular, in extraction
trees that are used for both wh-questions and topicalization, the
value of the {\bf $<$inv$>$} feature for the top of the S$_{r}$ node
is coindexed to the value of the {\bf $<$inv$>$} feature on the bottom
of the S$_{q}$ node.  On the bottom of the S$_{q}$ node the {\bf
$<$inv$>$} feature is coindexed to the {\bf $<$inv-link$>$} feature.
The {\bf $<$wh$>$} feature of the extracted NP node is coindexed to
the value of the {\bf $<$wh$>$} feature on the bottom of S$_{q}$. The
linking between the value of the S$_{q}$ {\bf $<$wh$>$} and the {\bf
$<$inv-link$>$} features is imposed by 
A condition on the final root node of a derivation (i.e. the top S
node of a matrix clause) requires that {\bf $<$inv-link$>=<$wh$>$}.
For example, the tree $\alpha$W1nx0Vnx1 in figure
(\ref{alphaW1nx0Vnx1}) is used to derive both (\ex{1}) and (\ex{2})


\enumsentence{John, I like.}
\enumsentence{Who do you like?}

For the question (\ex{0}) {\it Who do you like?}, the extracted item {\it who}
has $<$wh$>=+$  so the value of the {\bf $<$inv$>$} feature on VP is also $+$ and
an auxiliary, in this case {\it do}, is forced to adjoin.  For the topicalization (\ex{-1}) the values
for {\it John}`s wh feature and for S$_{q}$'s inv feature are both $-$ and no
auxiliary adjoins. Topicalization of PP complements is handled in a
similar way using the same trees as wh-questions on PP complement
positions. 





\subsection{Wh-moved NP complement}
Wh-questions can be formed on every NP object or indirect object that
appears in the declarative tree or in the passive trees.  A tree
family will contain one tree for each of these possible NP complement positions
. A tree of this type for a
Di-transitive tree family is shown in figure (\ref{alphaW1nx0Vnx1nx2}) with wh-extraction of the
direct object. 

\begin{figure}[htbp]
\center{
\psfig{figure=/mnt/linc/extra/xtag/work/doc/tech-rept/ps/extraction-files/alphaW1nx0Vnx1nx2.ps,height=10.0cm}
\caption{ \label{alphaW1nx0Vnx1nx2} Tree:  $\alpha$W1nx0Vnx1nx2}
}
\end{figure}

An extraction from  a passive in the
same tree family is shown in (\ref{alphaW2nx1Vnx2bynx0}).  

\begin{figure}[htbp]
\center{
\psfig{figure=/mnt/linc/extra/xtag/work/doc/tech-rept/ps/extraction-files/alphaW0nx1Vpnx2bynx0.ps,height=10.0cm}
\caption{ \label{alphaW0nx1Vpnx2bynx0} Tree:  $\alpha$W0nx1Vpnx2bynx0}
}
\end{figure}

The important features
of this type of tree are:
\begin{enumerate}
\item The subtree that has S$_{r}$ as its root is identical to the
declarative tree or a non-extracted passive tree, except for having
one NP position in the VP filled by $\epsilon$.
\item The root S node is S$_{q}$ which dominates NP and S$_{r}$
\item The trace feature of the $\epsilon$ filled NP is coindexed with
the trace feature of the NP daughter of S$_{q}$.
\item The inv feature of S$_{r}$ is coindexed to the wh feature of NP
in order to force subject-auxiliary inversion where needed (see the
section on Topicalization for more discussion of the {\bf
$<$inv$>$}-{\bf$<$wh$>$} coindexing
and the use of these trees for topicalization).
\end{enumerate}




\subsection{Wh-moved object of a P}
Wh-questions can be formed on the NP object of a complement PP as in
(\ex{1}).

\enumsentence{$[$Which dog$]_{i}$ did Beth Ann give a bone to $\epsilon_{i}$?}

The by-phrases of passives behave like complements and can undergo the
same type of extraction (\ex{1}).

\enumsentence{Which dog was the frisbee caught by?}

Tree structures for this type of sentence are very similar to those
for the wh-extraction of NP complements discussed in Section \ref and
have the identical important features related to tree structure and
trace and inversion features.

The tree $\alpha$W2nx0Vnx1pnx2 in figure \ref{alphaW2nx0Vnx1pnx2} is
an example of this type of tree.

\begin{figure}[htbp]
\center{
\psfig{figure=/mnt/linc/extra/xtag/work/doc/tech-rept/ps/extraction-files/alphaW2nx0Vnx1pnx2.ps,height=10.0cm}
\caption{ \label{alphaW2nx0Vnx1pnx2} Tree:  $\alpha$W2nx0Vnx1pnx2}
}
\end{figure}



\subsection{Extracted subjects}

The extracted subject trees provide for sentences such as {\it Who
left?}, {\it Who wrote the paper?}, and {\it Who was happy?},
depending on the tree family with which it is associated.
Wh-questions on subjects differ from other arguement extractions in
not having subject-auxiliary inversion.  This means that in subject
wh-questions the linear order of the constituents is the same as in
declaratives so it is difficult to tell whether the subject has moved
out of position or not (see \ref{heycock/kroch93-gagl} for arguments
for and against moved subject). The same problem holds for
topicalization. Clearly the value passing between the {\bf $<$wh$>$}
and {\bf $<$inv$>$} features that is used in other extractions will
not give the correct result for subject extractions.

The English LTAG treatment of subject extractions assumes the
following:

\begin{enumerate}
\item Syntactic subject topicalizations don't exist; and 
\item Subjects in wh-questions are extracted rather than insitu.
\end{enumerate}

The assumption that there is no syntactic subject topicalization is
reasonable in English since there is no convincing syntactic evidence
and since the interpretability of subjects as topics seems to be
mainly affected by discourse and intonational factors rather than
syntactic structure. As for the
assumption that wh-question subjects are extracted, these questions
seem to have more similarities to other extractions than to the two
cases in English that have been considered in situ wh:  mulitple wh
questions and echo questions. In multiple wh
questions such as (\ex{1}) one of the wh-items is blocked from moving
sentence initially because the first wh-item already occupies the
location to which it would move.

\enumsentence{Who ate what?}

This type of ``blocking'' account is not applicable to
subject wh-questions because there is no obvious candidate to do the
blocking.  Similarity between subject wh-questions and echo questions
is also lacking.  At least one account of echo questions
\cite{hockey94} argues that echo questions are not ordinary
wh-questions at all, but rather focus constructions in which the
wh-item is the focus. Clearly, this is not applicable to subject
wh-questions. So it seems that treating subject wh-questions similarly
to other wh-extractions is more justified than an in situ treatment. 

Given these assumptions, there must be separate trees in each tree
family for subject extractions. The declarative tree cannot be used
even though the linear order is the same because the structure is
different. Since topicalizations are not allowed, the {\bf $<$wh$>$}
feature for the extracted NP node is set in these trees to $+$. Since
subject-auxiliary inversion does not occur the {\bf $<$inv$>$} feature
is set in these trees to $-$.  Like other wh-extractions, the S$_{q}$ node is
marked {\bf $<$extracted$>$ = +} to constrain the occurance of these
trees in embedded sentences. The tree $\alpha$W0nx0V in figure
\ref{alphaW0nx0V} is an example of a subject wh-question tree.
%check how the extracted feature works with topicalizations.

\begin{figure}[htbp]
\center{
\psfig{figure=/mnt/linc/extra/xtag/work/doc/tech-rept/ps/extraction-files/alphaW0nx0V.ps,height=10.0cm}
\caption{ \label{alphaW0nx0V} Tree:  $\alpha$W0nx0V}
}
\end{figure}

\subsection{Wh-moved S complement}
Except for the node label on the extracted position, the trees for
wh-questions on S complements look exactly like the trees for
wh-questions on NP's in the same positions. This is because there is
no separate wh-lexical item for clauses in English, so the item {\it
what} is ambiguous between representing a clause or an NP.  To
illustrate this ambiguity notice that the question in (\ex{1}) could
be answered by either a clause as in (\ex{2}) or an NP as in (\ex{3}).

\enumsentence{what does Clove want?}
\enumsentence{Beth Ann to play frisbee with her}
\enumsentence{a biscuit}

Figure (\ref{wh-s-extr}) is an example of a tree for a wh-question on
a S complement from the tree family Tnx0Vs1.

\begin{figure}[htbp]
\center{
\psfig{figure=/mnt/linc/extra/xtag/work/doc/tech-rept/ps/extraction-files/betaW1nx0Vs1.ps,height=10.0cm}
\caption{ \label{betaW1nx0Vs1} Tree:  $\beta$W1nx0Vs1}
}
\end{figure}


\subsection{Wh-moved PP}
Like NP complements PP complements can be extracted to form
wh-questions as in (\ex{1}).

\enumsentence{[To which dog]$_{i}$ did Beth Ann throw the frisbee $\epsilon_{i}$?}

As can be seen in the tree $\alpha$pW2nx0Vnx1pnx2 in figure
\ref{alphapW2nx0Vnx1pnx2} extraction of PP complements is very similar
to extraction of NP complements from the same positions.  

\begin{figure}[htbp]
\center{
\psfig{figure=/mnt/linc/extra/xtag/work/doc/tech-rept/ps/extraction-files/alphapW2nx0Vnx1pnx2.ps,height=10.0cm}
\caption{ \label{alphapW2nx0Vnx1pnx2} Tree:  $\alpha$pW2nx0Vnx1pnx2}
}
\end{figure}


The PP extraction trees differ from NP extraction trees in having a PP
rather than an NP left daughter node under S$-{q}$ and in having the
$\epsilon$ fill a PP rather than an NP position in the VP. In other
respects these PP extraction structures behave like the NP extractions
including being used for Topicalization.

\subsection{Wh-moved Adjective complement}
In subcategorizations that select an adjective complement that
complement can be questioned in a wh-question such as (\ex{1}).

\enumsentence{how$_{i}$ did he feel $\epsilon_{i}$}

The tree families with adjective complements include trees for such
adjective extractions that are very similar to the wh-extraction trees
for other categories of complements.  The adjective position in the VP
is filled by an $\epsilon$ and the trace feature of the adjective
complement and of the adjective daughter of S$_{q}$ are coindexed.  An
example of this type of tree is shown in Figure (\ref{wh-adj-extr})

\begin{figure}[htbp]
\center{
\psfig{figure=/mnt/linc/extra/xtag/work/doc/tech-rept/ps/extraction-files/alphaWA1nx0Va1.ps,height=10.0cm}
\caption{ \label{alphaWA1nx0Va1} Tree:  $\alpha$WA1nx0Va1}
}
\end{figure}







