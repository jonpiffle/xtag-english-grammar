\chapter{Extraction}
\label{extraction}

The discussion in this chapter covers constructions that are analyzed
as having wh-movement in GB, in particular, wh-questions and
topicalization. Relative clauses, which could also be considered
extractions, are discussed in Chapter~\ref{rel_clauses}.

Extraction involves a constituent appearing in a linear position to the left of
the clause with which it is interpreted. One clause argument position is
empty. For example, the position filled by {\it frisbee} in the declarative in
sentence~(\ex{1}) is empty in sentence~(\ex{2}). The wh-item {\it what} in
sentence~(\ex{2}) is of the same syntactic category as {\it frisbee} in
sentence~(\ex{1}) and fills the same role with respect to the
subcategorization.

\enumsentence{Clove caught a frisbee.}
\enumsentence{What$_{i}$ did Clove catch $\epsilon_{i}$?} 


The English XTAG grammar represents the connection between the extracted
element and the empty position with co-indexing (as does GB).  The {\bf
$<$trace$>$} feature is used to implement the co-indexing.  In extraction trees
in XTAG, the `empty' position is filled with an {\it $\epsilon$}.  The
extracted item always appears in these trees as a sister to the S$_{r}$
tree, with both dominated by a S$_{q}$ root node.  The S$_{r}$ subtrees in
extraction trees have the same structure as the declarative tree in the same
tree family.  The additional structure in extraction trees of the S$_{q}$ and
NP nodes roughly corresponds to the CP and Spec of CP positions in GB.

All sentential trees with extracted components (this does not include relative
clause trees) are marked {\bf $<$extracted$>$=+} at the top S node, while
sentential trees with no extracted components are marked {\bf
$<$extracted$>$=--}.  Items that take embedded sentences, such as nouns, verbs
and some prepositions can place restrictions on whether the embedded sentence
is allowed to be extracted or not.  For instance, sentential subjects and
sentential complements of nouns and prepositions are not allowed to be
extracted, while certain verbs may allow extracted sentential complements and
others may not (e.g. sentences (\ex{1})-(\ex{4})).

\enumsentence{The jury wondered [who killed Nicole].}
\enumsentence{The jury wondered [who Simpson killed].}
\enumsentence{The jury thought [Simpson killed Nicole].}
\enumsentence{$\ast$The jury thought [who did Simpson kill]?}
The {\bf $<$extracted$>$} feature is also used to block embedded topicalization
in infinitival complement clauses as exemplified in (\ex{1}). 
\enumsentence{* John wants [ Bill$_{i}$ [PRO to see t$_{i}$]]}
Verbs such as {\em want} that take non-{\em wh} infinitival complements
specify that the {\bf $<$extracted$>$} feature of their complement clause
(i.e. of the foot S node)
is {\bf --}. Clauses that involve topicalization have {\bf +} as the value
of their {\bf $<$extracted$>$} feature (i.e. of the root S node). 
Sentences like (\ex{0}) are thus ruled out.  

\begin{figure}[htb]
\centering
\mbox{}
\psfig{figure=ps/extraction-files/alphaW1nx0Vnx1.ps,height=10.0cm}
\caption{Transitive tree with object extraction: $\alpha$W1nx0Vnx1}
\label{alphaW1nx0Vnx1}
\label{2;5,1}
\end{figure} 


The tree that is used to derive the embedded sentence in (\ex{-2}) in
the English XTAG grammar is shown in
Figure~\ref{alphaW1nx0Vnx1}\footnote{Features not pertaining to this
  discussion have been taken out to improve readability.}.  The
important features of extracted trees are:

\begin{itemize}
\item The subtree that has S$_{r}$ as its root is identical to the
  declarative tree or a non-extracted passive tree, except for having
  one NP position in the VP filled by $\epsilon$.

\item The root S node is S$_{q}$, which dominates NP and S$_{r}$.
  
\item The {\bf $<$trace$>$} feature of the $\epsilon$ filled NP is
  co-indexed with the {\bf $<$trace$>$} feature of the NP daughter of
  S$_{q}$.

\item The {\bf $<$case$>$} and {\bf $<$agr$>$} features are passed
  from the empty NP to the extracted NP.  This is particularly
  important for extractions from subject NP's, since {\bf $<$case$>$}
  can continue to be assigned from the verb to the subject NP
  position, and from there be passed to the extracted NP.
  
\item The {\bf $<$inv$>$} feature of S$_{r}$ is co-indexed to the {\bf
    $<$wh$>$} feature of NP through the use of the {\bf $<$invlink$>$}
  feature in order to force subject-auxiliary inversion where needed
  (see section~\ref{topicalization} for more discussion of the {\bf
    $<$inv$>$}/{\bf$<$wh$>$} co-indexing and the use of these trees
  for topicalization).

\end{itemize}



\section{Topicalization and the value of the {\bf $<$inv$>$} feature}
\label{topicalization}

Our analysis of topicalization uses the same trees as wh-extraction.  For any
NP complement position a single tree is used for both wh-questions and for
topicalization from that position. Wh-questions have subject-auxiliary
inversion and topicalizations do not.  This difference between the
constructions is captured by equating the values of the S$_{r}$'s {\bf
$<$inv$>$} feature and the extracted NP's {\bf $<$wh$>$} feature.  This means
that if the extracted item is a wh-expression, as in wh-questions, the value of
{\bf $<$inv$>$} will be {\bf +} and an inverted auxiliary will be forced to
adjoin. If the extracted item is a non-wh, {\bf $<$inv$>$} will be {\bf --}
and no auxiliary adjunction will occur. An additional complication is that
inversion only occurs in matrix clauses, so the values of {\bf $<$inv$>$} and
{\bf $<$wh$>$} should only be equated in matrix clauses and not in embedded
clauses.  In the English XTAG grammar, appropriate equating of the {\bf
$<$inv$>$} and {\bf $<$wh$>$} features is accomplished using the {\bf
$<$invlink$>$} feature and a restriction imposed on the root S of a
derivation. In particular, in extraction trees that are used for both
wh-questions and topicalizations, the value of the {\bf $<$inv$>$} feature for
the top of the S$_{r}$ node is co-indexed to the value of the {\bf $<$inv$>$}
feature on the bottom of the S$_{q}$ node.  On the bottom of the S$_{q}$ node
the {\bf $<$inv$>$} feature is co-indexed to the {\bf $<$invlink$>$} feature.
The {\bf $<$wh$>$} feature of the extracted NP node is co-indexed to the value
of the {\bf $<$wh$>$} feature on the bottom of S$_{q}$. The linking between the
value of the S$_{q}$ {\bf $<$wh$>$} and the {\bf $<$invlink$>$} features is
imposed by a condition on the final root node of a derivation (i.e. the top S
node of a matrix clause) requires that {\bf $<$invlink$>$=$<$wh$>$}.  For
example, the tree in Figure~\ref{alphaW1nx0Vnx1} is used to
derive both (\ex{1}) and (\ex{2}).


\enumsentence{John, I like.}
\enumsentence{Who do you like?}

For the question in (\ex{0}), the extracted item {\it who} has the feature
value {\bf $<$wh$>$=+}, so the value of the {\bf $<$inv$>$} feature on VP is
also $+$ and an auxiliary, in this case {\it do}, is forced to adjoin.  For the
topicalization (\ex{-1}) the values for {\it John}'s {\bf $<$wh$>$} feature and
for S$_{q}$'s {\bf $<$inv$>$} feature are both {\bf --} and no auxiliary
adjoins.



\section{Extracted subjects}
\label{subject-extraction}

The extracted subject trees provide for sentences like (\ex{1})-(\ex{3}),
depending on the tree family with which it is associated.

\enumsentence{Who left?}
\enumsentence{Who wrote the paper?}
\enumsentence{Who was happy?}

Wh-questions on subjects differ from other argument extractions in
not having subject-auxiliary inversion.  This means that in subject
wh-questions the linear order of the constituents is the same as in
declaratives so it is difficult to tell whether the subject has moved
out of position or not (see \cite{heycock/kroch93gagl} for arguments
for and against moved subject). 

The English XTAG treatment of subject extractions assumes the
following:

\begin{itemize}
\item Syntactic subject topicalizations don't exist; and 
\item Subjects in wh-questions are extracted rather than in situ.
\end{itemize}

The assumption that there is no syntactic subject topicalization is reasonable
in English since there is no convincing syntactic evidence and since the
interpretability of subjects as topics seems to be mainly affected by discourse
and intonational factors rather than syntactic structure. As for the assumption
that wh-question subjects are extracted, these questions seem to have more
similarities to other extractions than to the two cases in English that have
been considered in situ wh: multiple wh questions and echo questions. In
multiple wh questions such as sentence~(\ex{1}), one of the wh-items is blocked
from moving sentence initially because the first wh-item already occupies the
location to which it would move.

\enumsentence{Who ate what?}

This type of `blocking' account is not applicable to
subject wh-questions because there is no obvious candidate to do the
blocking.  Similarity between subject wh-questions and echo questions
is also lacking.  At least one account of echo questions
(\cite{hockey94}) argues that echo questions are not ordinary
wh-questions at all, but rather focus constructions in which the
wh-item is the focus. Clearly, this is not applicable to subject
wh-questions. So it seems that treating subject wh-questions similarly
to other wh-extractions is more justified than an in situ treatment. 

Given these assumptions, there must be separate trees in each tree family for
subject extractions. The declarative tree cannot be used even though the linear
order is the same because the structure is different. Since topicalizations are
not allowed, the {\bf $<$wh$>$} feature for the extracted NP node is set in
these trees to {\bf +}.  The lack of subject-auxiliary inversion is handled
by the absence of the {\bf $<$invlink$>$} feature.  Without the presence of
this feature, the {\bf $<$wh$>$} and {\bf $<$inv$>$} are never linked, so
inversion can not occur.  Like other wh-extractions, the S$_{q}$ node is marked
{\bf $<$extracted$>$=+} to constrain the occurrence of these trees in
embedded sentences. The tree in Figure~\ref{alphaW0nx0V} is an example of a
subject wh-question tree.

\begin{figure}[htb]
\centering
\begin{tabular}{c}
\psfig{figure=ps/extraction-files/alphaW0nx0V.ps,height=10.3cm}
\end{tabular}
\caption{Intransitive tree with subject extraction: $\alpha$W0nx0V}
\label{alphaW0nx0V}
\label{1;4,13} 
\end{figure}



\section{Wh-moved NP complement}
\label{NP-extr}

Wh-questions can be formed on every NP object or indirect object that appears
in the declarative tree or in the passive trees, as seen in sentences
(\ex{1})-(\ex{6}).  A tree family will contain one tree for
each of these possible NP complement positions.
Figure~\ref{ditrans-extractions} shows the two extraction trees from the
ditransitive tree family for the extraction of the direct
(Figure~\ref{ditrans-extractions}(a)) and indirect object
(Figure~\ref{ditrans-extractions}(b)).

\enumsentence{Dania asked Beth a question.}
\enumsentence{Who$_{i}$ did Dania ask $\epsilon_{i}$ a question?}
\enumsentence{What$_{i}$ did Dania ask Beth $\epsilon_{i}$?}
\enumsentence{Beth was asked a question by Dania.}
\enumsentence{Who$_{i}$ was Beth asked a question by $\epsilon_{i}$??}
\enumsentence{What$_{i}$ was Beth asked $\epsilon_{i}$? by Dania?}

\begin{figure}[htb]
\centering
\begin{tabular}{ccc}
\psfig{figure=ps/extraction-files/alphaW1nx0Vnx2nx1.ps,height=6.0cm}&
\hspace{1.0in}&
\psfig{figure=ps/extraction-files/alphaW2nx0Vnx2nx1.ps,height=6.0cm}\\
(a)&&(b)
\end{tabular}
\caption{Ditransitive trees with direct object: $\alpha$W1nx0Vnx2nx1 (a) and
indirect object extraction: $\alpha$W2nx0Vnx2nx1 (b)}
\label{ditrans-extractions}
\label{2;5,3}
\end{figure}

%%\begin{figure}[htb]
%%\centering
%%\mbox{}
%%\psfig{figure=ps/extraction-files/alphaW0nx1Vpnx2bynx0.ps,height=8.0cm}
%%\caption{Tree:  $\alpha$W0nx1Vpnx2bynx0}
%%\label{alphaW0nx1Vpnx2bynx0} 
%%\label{2;21,4}
%%\end{figure}


\section{Wh-moved object of a P}
Wh-questions can be formed on the NP object of a complement PP as in
sentence~(\ex{1}).

\enumsentence{$[$Which dog$]_{i}$ did Beth Ann give a bone to $\epsilon_{i}$?}

The {\it by} phrases of passives behave like complements and can undergo the
same type of extraction, as in (\ex{1}).

\enumsentence{$[$Which dog$]_{i}$ was the frisbee caught by $\epsilon_{i}$?}

Tree structures for this type of sentence are very similar to those for the
wh-extraction of NP complements discussed in section~\ref{NP-extr} and have the
identical important features related to tree structure and trace and inversion
features.  The tree in Figure~\ref{alphaW2nx0Vnx1pnx2} is an example of this
type of tree.  Topicalization of NP objects of prepositions is handled the same
way as topicalization of complement NP's.

\begin{figure}[htb]
\centering
\mbox{}
\psfig{figure=ps/extraction-files/alphaW2nx0Vnx1pnx2.ps,height=6.0cm}
\caption{Ditransitive with PP tree with the object of the PP extracted: $\alpha$W2nx0Vnx1pnx2}
\label{alphaW2nx0Vnx1pnx2}
\label{2;8,4}
\end{figure}



\section{Wh-moved PP}
Like NP complements, PP complements can be extracted to form
wh-questions, as in sentence (\ex{1}).

\enumsentence{[To which dog]$_{i}$ did Beth Ann throw the frisbee $\epsilon_{i}$?}

As can be seen in the tree in Figure~\ref{alphapW2nx0Vnx1pnx2}, extraction of
PP complements is very similar to extraction of NP complements from the same
positions.

\begin{figure}[htb]
\centering
\mbox{}
\psfig{figure=ps/extraction-files/alphapW2nx0Vnx1pnx2.ps,height=6.0cm}
\caption{Ditransitive with PP with PP extraction tree: $\alpha$pW2nx0Vnx1pnx2}
\label{alphapW2nx0Vnx1pnx2} 
\label{2;9,4}
\end{figure}


The PP extraction trees differ from NP extraction trees in having a PP
rather than an NP left daughter node under S$_{q}$ and in having the
$\epsilon$ fill a PP rather than an NP position in the VP. In other
respects these PP extraction structures behave like the NP extractions,
including being used for topicalization.



\section{Wh-moved S complement}

Except for the node label on the extracted position, the trees for wh-questions
on S complements look exactly like the trees for wh-questions on NP's in the
same positions.  This is because there is no separate wh-lexical item for
clauses in English, so the item {\it what} is ambiguous between representing a
clause or an NP.  To illustrate this ambiguity notice that the question in
(\ex{1}) could be answered by either a clause as in (\ex{2}) or an NP as in
(\ex{3}).  The extracted NP in these trees is constrained to be {\bf
$<$wh$>$=+}, since sentential complements can not be topicalized.

\enumsentence{What does Clove want?}
\enumsentence{for Beth Ann to play frisbee with her}
\enumsentence{a biscuit}

%%\begin{figure}[htb]
%%\centering
%%\mbox{}
%%\psfig{figure=ps/extraction-files/betaW1nx0Vs1.ps,height=20cm}
%%\caption{Tree:  $\beta$W1nx0Vs1}
%%\label{wh-s-extr} 
%%\label{2;6,10}
%%\end{figure}



\section{Wh-moved Adjective complement}
In subcategorizations that select an adjective complement, that
complement can be questioned in a wh-question, as in sentence~(\ex{1}).

\enumsentence{How$_{i}$ did he feel $\epsilon_{i}$?}

\begin{figure}[htb]
\centering
\mbox{}
\psfig{figure=ps/extraction-files/alphaWA1nx0Vax1.ps,height=6.0cm}
\caption{Predicative Adjective tree with extracted adjective: $\alpha$WA1nx0Vax1}
\label{wh-adj-extr} 
\label{1;7,14}
\end{figure}


The tree families with adjective complements include trees for such adjective
extractions that are very similar to the wh-extraction trees for other
categories of complements.  The adjective position in the VP is filled by an
{\it $\epsilon$} and the trace feature of the adjective complement and of the
adjective daughter of S$_{q}$ are co-indexed.  The extracted adjective is
required to be {\bf $<$wh$>$=+}\footnote{{\it How} is the only {\bf
$<$wh$>$=+} adjective currently in the XTAG English grammar.}, so no
topicalizations are allowed.  An example of this type of tree is shown in
Figure~\ref{wh-adj-extr}.









