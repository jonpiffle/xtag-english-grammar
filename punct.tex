
\chapter{Punctuation Marks}

Many parsers require that punctuation be stripped out of the
input. Since punctuation is often optional, this sometimes has no
effect. However, there are a number of constructions which must
obligatorily contain punctuation and adding analyses of these to the
grammar without the punctuation would lead to severe
overgeneration. An especially common example is noun
appositives. Without access to punctuation, one would have to allow
every combinatorial possibility of NPs in noun sequences, which is
clearly undesirable (especially since there is already unavoidable
noun-noun compounding ambiguity). Aside from coverage issues, it is
also preferable to take input ``as is'' and do as little editing as
possible. With the addition of punctuation to the XTAG grammar, we
need only do/assume the conversion of certain sequences of punctuation
into the ``British'' order (this is discussed in more detail below in
Section \ref{bal}).

The XTAG POS tagger currently tags every punctuation mark as
itself. These tags are all converted to the POS tag "Punct" before
parsing. This allows us to treat the punctuation marks as a single POS
class. They then have features which distinguish amongst them.
Wherever possible, we have the punctuation marks as anchors, to
facilitate early filtering.

The full set of punctuation marks are separated into three classes:
balanced, separating and terminal. The balanced punctuation marks are
quotes and parentheses, separating are commas, dashes, semi-colons and
colons, and terminal are periods, exclamation points and question
marks. Thus, the {\bf $<$punct$>$} feature is complex (like the {\bf
$<$agr$>$} feature), yielding feature equations like {\bf $<$Punct bal
= paren$>$} or {\bf $<$Punct term = excl$>$}. Separating and terminal
punctuation marks do not occur adjacent to other members of the same
class, but may occasionally occur adjacent to members of the other
class, e.g. a question mark on a clause which is separated by a dash
from a second clause. Balanced punctuation marks are sometimes adjacent
to one another, e.g. quotes immediately inside of parentheses. The
{\bf $<$punct$>$} feature allows us to control these local
interactions.

We also need to control non-local interaction of punctuation
marks. Two cases of this are so-called quote alternation, wherein
embedded quotation marks must alternate between single and double, and
the impossibility of embedding an item containing a colon inside of
another item containing a colon. Thus, we have a fourth value for {\bf
$<$punct$>$}, {\bf $<$contains colon/dquote/etc. +/-$>$}, which
indicates whether or not a constituent contains a particular
punctuation mark. This feature is percolated through all auxiliary
trees.  Things which may not embed are: colons under colons,
semi-colons, dashes or commas; semi-colons under semi-colon or commas;
NOT SURE about dash separated things.  Although it is rare,
parentheses may appear inside of parentheses, say with an academic
reference inside a parenthesized sentence.
 
%Embedded appositives are fine:`noted Simon Briscoe, UK economist for Midland
%Montagu, a unit of Midland Bank PLC.'

%if we want to be absolutely meticulous it should be
%percolated everywhere, since, for instance, an argument (say NP) might
%be enclosed in single quotes and the larger constituent (say S) be
%enclosed in a second set of quotes. We would currently not be able to
%force the outer quotes to be double.

\section{Appositives, parentheticals and vocatives}

These trees handle constructions where additional lexical material is
only licensed in conjunction with particular punctuation marks. Since
the lexical material is unconstrained (virtually any noun can occur as
an appositive), the punctuation marks are anchors and the other nodes
are substitution sites. There are cases where the lexical material is
restricted, as with parenthetical adverbs like {\it however}, and in
those cases we have the adverb as the anchor and the punctuation marks
as substitution sites.

When these constructions can appear inside of clauses
(non-peripherally), they must be separated by punctuation marks on
both sides. However, when they occur peripherally they have either a
preceeding or following punctuation mark. We handle this by having
both peripheral and non-peripheral trees for the relevant
constructions. The alternative is to insert the second (following)
punctuation mark in the tokenization process (i.e. insert a comma
before the period when an appositive appears on the last NP of a
sentence). However, this is very difficult to do accurately.

\subsection{$\beta$nxPUnxPU}

The symmetric (non-peripheral) tree for NP appositives, anchored by:
comma, dash or parentheses. It is shown in Figure \ref{nxPUnxPU} anchored by
parentheses. 

\enumsentence{The music here , Russell Smith's ``Tetrameron '' ,
sounded good . [cc09]}
\enumsentence{...cost 2 million pounds (3 million dollars)}
\enumsentence{Sen. David Boren (D., Okla.)...}

\enumsentence{
...some analysts believe the two recent natural
disasters -- Hurrican Hugo and the San Francisco earthquake -- will
carry economic ramifications.... [wsj]
}

\begin{figure}[hbt]
\centering
\hspace{0.0in}
\psfig{figure=/mnt/linc/xtag/work/doc/tech-rept/ps/punct-files/nxPUnxPU.ps,height=3.0in}
\caption{The $\beta$nxPUnxPU tree, anchored by parentheses}
\label{nxPUnxPU}
\end{figure}


The punctuation marks are the anchors and the appositive NP is
substituted. The appositive can be conjoined, but only with a lexical
conjunction (not with a comma). Appositives with commas or dashes
cannot be pronouns, although they may be conjuncts containing
pronouns; likewise, they cannot modify pronouns.  When used with
parentheses this tree actually presents an alternative rather than an
appositive, so a pronoun is possible. Finally, the appositive position
is restricted to having nominative or accusative case to block PRO
from appearing here.

Appositives can be embedded, as in (\ex{1}), but do not seem to be
able to stack on a single NP. In this they are more like restrictive
relatives than appositive relatives, which typically can stack.

\enumsentence{...noted Simon Briscoe, UK economist for Midland
Montagu, a unit of Midland Bank PLC.}

\subsection{$\beta$nPUnxPU}

The symmetric (non-peripheral) tree for N-level NP appositives, is
anchored by comma. The modifier is typically an address.  It is clear
from examples such as (\ex{1}) that these are attached at N, rather
than NP. {\it Carrier } is not an appositive on {\it Menlo Park}, as it
would be if these were simply stacked appositives. Rather, {\it
Calif.} modifies {\it Menlo Park}, and that entire complex is
compounded with {\it carrier}, as shown in the correct derivation in
Figure \ref{nPUnx}. Because this distinction is less clear when the
modifier is peripheral (e.g. ends the sentence), and it would be
difficult to distinguish between NP and N attachment, we do not
currently allow a peripheral N-level attachment.

\enumsentence{An official at Consolidated Freightways Inc., a Menlo
Park, Calif., less-than-truckload carrier , said...}
\enumsentence{Rep. Ronnie Flippo (D., Ala.), of the delegation,
says...}

\begin{figure}[hbt]
\centering
\hspace{0.0in}
\psfig{figure=/mnt/linc/xtag/work/doc/tech-rept/ps/punct-files/nPUnx.ps,height=3.0in}
\caption{An N-level modifier, using the $\beta$nPUnx tree}
\label{nPUnx}
\end{figure}

\subsection{$\beta$nxPUnx}

This tree, which can be anchored by a comma, dash or colon, handles
asymmetric (peripheral) NP appositives and NP colon expansions of
NPs. Figure \ref{nxPUnx} shows this tree anchored by a dash and a
colon. Like the symmetric appostive tree, $\beta$nxPUnxpu, the
asymmetric appositive cannot be a pronoun, while the colon expansion
can. Thus, this constraint comes from the syntactic entry in both
cases rather than being built into the tree.

\begin{figure}[hbt]
\centering
\hspace{0.0in}
\begin{tabular}{cc}
\psfig{figure=/mnt/linc/xtag/work/doc/tech-rept/ps/punct-files/nxPUnx.ps,height=3.0in}
& \hspace{0.5in}
insert tree
%\psfig{figure=/mnt/linc/xtag/work/doc/tech-rept/ps/punct-files/nxPUnx-colon.ps,height=3.0in}
\\
(a) & (b) \\
\end{tabular}
\caption{The derived trees for an NP with (a) an peripheral, dash-separated
appositive and (b) an NP colon expansion}
\label{nxPUnx}
\end{figure}

\enumsentence{the bank's 90\% shareholder -- Petroliam Nasional Bhd. [brown]}

\enumsentence{...said Chris Dillow, senior U.K. economist at Nomura
Research Institute .}

\enumsentence{...qualities that are seldom found in one work: Scrupulous
scholarship, a fund of personal experience,... [cc06]}
\enumsentence{I had eyes for only one person : him .}

The colon expansion cannot itself contain a colon, so the foot $S$ has the
feature NP.t:$<punct contains colon> = -$. 

\subsection{$\beta$PUpxPUvx}

Tree for pre-VP parenthetical PP, anchored by commas or dashes - 

\enumsentence{John , in a fit of anger , broke the vase}
\enumsentence{Mary , just within the last year , has totalled two cars}

\noindent
These are clearly not NP modifiers. 

Figures \ref{betaPUpxPUvx} and \ref{PUpxPUvx-anger} show this tree
alone and as part of the parse for (\ex{-1}).

\begin{figure}[hbt]
\centering
\hspace{0.0in}
\psfig{figure=/mnt/linc/xtag/work/doc/tech-rept/ps/punct-files/betaPUpxPUvx.ps,height=2.0in}
\caption{The $\beta$PUpxPUvx tree, anchored by commas}
\label{betaPUpxPUvx}
\end{figure}

\begin{figure}[hbt]
\centering
\hspace{0.0in}
\psfig{figure=/mnt/linc/xtag/work/doc/tech-rept/ps/punct-files/betaPUpxPUvx-anger.ps,height=3.0in}
\caption{Tree illustrating the use of $\beta$PUpxPUvx}
\label{PUpxPUvx-anger}
\end{figure}

\subsection{$\beta$puARBpuvx}
\label{par-adverb}

Parenthetical adverbs - {\it however}, {\it though}, etc. Since the
class of adverbs is highly restricted, this tree is anchored by the
adverb and the punctuation marks substitute.  The punctuation marks
may be either commas or dashes.  Like the parenthetical PP above,
these are clearly not NP modifiers.

\enumsentence{The new argument over the notification
guideline , however , could sour any atmosphere of cooperation that
existed . \hfill [WSJ]}

\subsection{$\beta$sPUnx}

Sentence final vocative, anchored by comma:  
 
\enumsentence{You were there , Stanley/my boy .}

Also, when anchored by colon, NP expansion on S. These often appear to
be extraposed modifiers of some internal NP. The NP must be quite
heavy, and is usually a list:

\enumsentence{Of the major expansions in 1960, three were financed
under the R. I. Industrial Building Authority's 100% guaranteed
mortgage plan: Collyer Wire, Leesona Corporation, and American Tube \&
Controls.}

A simplified version of this sentence is shown in figure
\ref{sPUnx}. The NP cannot be a pronoun in either of these cases.
Both vocatives and colon expansions are restricted to appear on tensed
clauses (indicative or imperative).

\begin{figure}[hbt]
\centering
\hspace{0.0in}
\psfig{figure=/mnt/linc/xtag/work/doc/tech-rept/ps/punct-files/sPUnx-colon.ps,height=3.0in}
\caption{A tree illustrating the use of sPUnx for a colon expansion
attached at S.}
\label{sPUnx}
\end{figure}


\subsection{$\beta$nxPUs}

Tree for sentence initial vocatives, anchored by a comma: 

\enumsentence{Stanley/my boy , you were there}

The noun phrase may be anything but a pronoun, although it is most
commonly a proper noun. The clause adjoined to must be indicative or
imperative.

\section{Bracketing punctuation}
\label{bal}

Trees: $\beta$PUsPU, $\beta$PUnxPU, $\beta$PUnPU, $\beta$PUvxPU, $\beta$PUvPU,
$\beta$PUarbPU, $\beta$PUaPU, $\beta$PUdPU, $\beta$PUpxPU,
$\beta$PUpPU

\noindent
These trees are selected by parentheses and quotes and can adjoin onto
any node type, whether a head or a phrasal constituent.  This handles
things in parentheses or quotes which are syntactically integrated
into the surrounding context. Figure \ref{bal-trees} shows the $\beta$PUsPU
anchored by parentheses, and this tree along with $\beta$PUnxPU in a
derived tree.

\enumsentence{Dick Carroll and his accordion (which we now refer to
as ``Freida'') held over at Bahia Cabana where ``Sir'' Judson
Smith brings in his calypso capers Oct. 13 .\hfill [brown:ca31]}

\enumsentence{...noted that the term ``teacher-employee'' (as
opposed to, e.g., ``maintenance employee'') was a not inapt
description. \hfill [brown:ca35]}

\begin{figure}[hbt]
\centering
\hspace{0.0in}
\begin{tabular}{cc}
\psfig{figure=/mnt/linc/xtag/work/doc/tech-rept/ps/punct-files/PUsPU-paren.ps,height=2.0in}
& \hspace{0.5in}
\psfig{figure=/mnt/linc/xtag/work/doc/tech-rept/ps/punct-files/bal-parse.ps,height=3.5in}
\\
(a) & (b) \\
\end{tabular}
\caption{$\beta$PUsPU anchored by parentheses, and in a derivation,
along with  $\beta$PUnxPU}
\label{bal-trees}
\end{figure}

There is a convention in English that quotes embedded in quotes
alternate between single and double; in American English the outermost
are double quotes, while in British English they are single.  The {\bf
contains} feature is used to control this alternation. The trees
anchored by double quotation marks have the feature {\bf punct
contains dquote = -} on the foot node and the feature {\bf punct
contains dquote = +} on the root. All adjunction trees are transparent
to the the {\bf contains} feature, so if any tree below the double
quote is itself enclosed in double quotes the derivation will
fail. Likewise with the trees anchored by single quotes. The quote
trees in effect ``toggle'' the {\bf contains Xquote}
feature. Immediate proximity is handled by the {\bf punct balanced}
feature, which allows quotes inside of parentheses, but not vice-versa.

In addition, American English typically places/moves periods (and
commas) inside of quotation marks when they would logically occur
outside, as in example
\ex{1}. The comma in the first part of the quote is not part of the
quote, but rather part of the parenthetical quoting clause. However,
by convention it is shifted inside the quote, as is the final
period. British English does not do this. We assume here that the
input has already been tokenized into the ``British'' format.

\enumsentence{``You can't do this to us ,'' Diane screamed .  ``We are
Americans.''}

The $\beta$PUsPU can handle quotation marks around multiple sentences,
since the sPUs tree allows us to join two sentences with a period,
exclamation point or question mark. Currently, however, we cannot
handle the style where only an open quote appears at the beginning of
a paragraph when the quotation extends over multiple
paragraphs. We could allow a lone open quote to select the $\beta$PUs
tree, if this is deemed desirable.

Also, the $\beta$PUsPU is selected by a pair of commas to handle
non-peripheral appositive relative clauses, such as
\ex{1}. Restrictive and appositive relative clauses are not
syntactically differentiated in the XTAG grammar
(cf. \ref{rel_clauses}).

\enumsentence{This news , announced by Jerome Toobin , the orchestra's
administrative director , brought applause ...  [brown:cc09]}

The trees discussed in this section will only allow balanced
punctuation marks to adjoin to constituents. We will not get them
around non-constituents, as in (\ex{1}).

\enumsentence{Mary asked him to leave (and  he left)}

\section{Punctuation trees containing no lexical material}

\subsection{$\alpha$PU}

This is the elementary tree for substitution of punctuation
marks. This tree is used in the quoted speech trees, where including
the punctuation mark as an anchor along with the verb of saying would
require a new entry for every tree selecting the relevant tree
families. It is also used in the tree for parethetical adverbs
($\beta$puARBpuvx), and for S-adjoined PPs and adverbs ($\beta$spuARB
and $\beta$spuPnx).

\subsection{$\beta$PUs}

Anchored by comma: allows comma-separated clause initial adjuncts,
(\ex{1}-\ex{2}).

\enumsentence{Here , as in ``Journal'' , Mr. Louis has given himself
the lion's share of the dancing... \hfill [cc09]}

\enumsentence{Choreographed by Mr. Nagrin, the work filled the second
half of a program}

To keep this tree from appearing on root Ss (i.e. {\it , sentence}),
we have a root constraint that {\bf $<$punct struct = nil$>$} (similar
to the requirement that root Ss be tensed, i.e. {\bf $<$mode =
ind/imp$>$}).  The {\bf $<$punct struct$>$ = nil} feature on the foot
blocks stacking of multiple punctuation marks. This feature is shown
in the tree in Figure \ref{PUs}.

\begin{figure}[hbt]
\centering
\hspace{0.0in}
\psfig{figure=/mnt/linc/xtag/work/doc/tech-rept/ps/punct-files/PUs.ps,height=3.5in}
\caption{$\beta$PUs, with features displayed}
\label{PUs}
\end{figure}

This tree can be also used by adjuncts on embedded clauses:

\enumsentence{One might expect that in a poetic career of seventy-odd
years, some changes in style and method would have occurred, some
development taken place. \hfill [cj65]}

These adjucts sometimes have commas on both sides of the adjunct, or,
like (\ex{0}), only have them at the end of the adjunct.

Finally, this tree is also used for peripheral appositive relative
clauses.

\enumsentence{Interest may remain limited into tomorrow's U.K. trade figures,
which the market will be watching closely to see if there is any
improvement after disappointing numbers in the previous two months.}

%Don't need to block BELOW comps: Mary said that, in her experience,
%cats sleep a lot.  Don't want comp=nil on bottom of foot, because
%this would block "his explanation was this: that he couldn't stay in
%his present job\", or sub-conj=nil there bec it blocks \"Mary
%sneezed, because Bill blew pepper in her face\".

\subsection{$\beta$sPUs}
\label{sPUs}

This tree handles clausal ``coordination'' with comma, dash, colon,
semi-colon or any of the terminal punctuation marks. The first clause
must be either indicative or imperative. The second may also be
infinitival with the separating punctuation marks, but must be
indicative or imperative with the terminal marks; with a comma, it may
only be indicative. The two clauses need not share the same mode. NB:
Allowing the terminal punctuation marks to anchor this tree allows us
to parse sequences of multiple sentences. This is not the usual mode
of parsing; if it were, this sort of sequencing might be better
handled by a higher level of processing.

\enumsentence{For critics , Hardy has had no poetic periods -- one does
not speak of early Hardy or late Hardy , or of the London or Max Gate
period....}

\enumsentence{Then there was exercise , boating and hiking , which was
not only good for you but also made you more virile : the thought of
strenuous activity left him exhausted.}

This construction is one of the few where two non-bracketing
punctuation marks can be adjacent. It is possible (if rare) for the
first clause to end with a question mark or exclamation point, when
the two clauses are conjoined with a semi-colon, colon or
dash. Features on the foot node, as shown in Figure \ref{sPUs-tree}, control
this interaction.

\begin{figure}[hbt]
\centering
\hspace{0.0in}
\psfig{figure=/mnt/linc/xtag/work/doc/tech-rept/ps/punct-files/sPUs.ps,height=3.5in}
\caption{$\beta$sPUs, with features displayed}
\label{sPUs-tree}
\end{figure}

%This will need to change with the reanal of sub-conjs.
Complementizers are not permitted on either conjunct. Subordinating
conjunctions sometimes appear on the right conjunct, but seem to be
impossible on the left:
	
\enumsentence{Killpath would just have to go out and drag Gun back by
the heels once an hour ; because he'd be damned if he was going to
be a mid-watch pencil-pusher . [Brown, cl17]}

\enumsentence{The best rule of thumb for detecting corked wine
(provided the eye has not already spotted it) is to smell the wet end
of the cork after pulling it : if it smells of wine , the bottle is
probably all right ; if it smells of cork , one has grounds for
suspicion.	[cf27]}

%	?Max left -- he was very tired -- he needed some sleep .
%	Max left ; he was very tired ; he needed some sleep .
%	Max left ; (after work) he was very tired (today) ; he needed
%		some sleep .

%**Do I want to allow ambiguous scopes here? Or force right-branching?

\subsection{$\beta$sPU}

This tree handles the sentence final punctuation marks when selected
by a question mark, exclamation point or period. One could also
require a final punctuation mark for all clauses, but such an approach
would not allow non-periods to occur internally, for instance before a
semi-colon or dash as noted above in Section \ref{sPUs}. This tree
currently only adjoins to indicative or imperative (root) clauses.

\enumsentence{He left !}
\enumsentence{Get lost .}
\enumsentence{Get lost ?}

The feature {\bf punct bal= nil} on the foot node ensures that this
tree only adjoins inside of parentheses or quotes completely enclosing
a sentence (\ex{1}), but does not restrict it from adjoining to clause
which ends with balanced punctuation if only the end of the clause is
contained in the parentheses or quotes (\ex{2}).

\enumsentence{(John then left .)} 
\enumsentence{(John then left) .}
\enumsentence{Mary asked him to leave (immediately) .}

This tree is also selected by the colon to handle a colon expansion
after adjunct clause --

\enumsentence{Expressed differently : if the price for becoming a
faithful follower... \hfill [cd02]}

\enumsentence{Expressing it differently : if the price for becoming a
faithful follower... }

\enumsentence{To express it differently : if the price for becoming a
faithful follower... \hfill [cd02]}

This tree is only used after adjunct (untensed) clauses, which adjoin
to the tensed clause using the adjunct clause trees (cf Section
\ref{adjunct-cls} ); the {\bf mode} of the complete clause is that of
the matrix rather than the adjunct. Indicative or imperatative
(i.e. root) clauses separated by a colon use the $\beta$spus tree
(Section \ref{sPUs}).

\subsection{$\beta$vPU}

%PROB: get adjunction at e in pred trees. Laura will fix.

This tree is anchored by a colon or a dash, and occurs between a verb
and its complement. These typically are lists.

\enumsentence{Printed material Available , on request , from U.S. Department of
Agriculture , Washington 25, D.C. , are : Cooperative Farm Credit Can
Assist......\hfill [Brown ch01]}

\subsection{$\beta$pPU}

This tree is anchored by a colon or a dash, and occurs between a
preposition and its complement. It typically occurs with a sequence
of complements. As with the tree above, this typically occurs with a
conjoined complement.

\enumsentence{...and utilization such as : (A) the protection of forage...}
\enumsentence{...can be represented as : Af.}

\section{Other trees}

\subsection{$\beta$spuARB}
\label{post-adverb}

In general, we attach post-clausal modifiers at the VP node, as you
typically get scope ambiguity effects with negation ({\it John didn't
leave today} - did he leave or not?). However, with post-sentential,
comma-separated adverbs, there is no ambiguity - in {\it John didn't
leave, today} he definitely did not leave. Since this tree is only
selected by a subset of the adverbs (namely, those which can appear
pre-sententially, without a punctuation mark), it is anchored by the
adverb.

\enumsentence{The names of some of these products don't suggest the 
risk involved in buying them , either . \hfill [wsj]}

%It should not be forgotten that wines mature fastest in half-bottles,
%less fast in full bottles, slowly in Magnums -- and slower yet in
%Tregnums, double Magnums, Jeroboams, Methuselahs, and Imperiales,
%respectively.
%There is an interesting issue here about where 'respectively'
%attaches - is it to the lowest point of the sequence, or the highest
%(S)? Currently we will only get S, but there is an interesting
%research issue here.

\subsection{$\beta$spuPnx}
\label{post-PP}

Clause-final PP separated by a comma. Like the adverbs described
above, these differ from VP adjoined PPs in taking widest scope.

\enumsentence{...gold for current delivery settled at \$367.30 an
ounce , up 20 cents .} 

\enumsentence{It increases employee commitment to the company , with
all that means for efficiency and quality control .}

\subsection{$\beta$nxPUa}

Anchored by colon or dash, allows for post-modification of NPs by
adjectives.
	
\enumsentence{Make no mistake , this Gorky Studio drama is a
respectable import -- aptly grave , carefully written , performed and
directed . }

%Vocatives:Quirk discusses these, says they can go where adverbs
%can - initial, medial or final.

%STILL DON'T HANDLE: parentheticals - of course

%The Usage Panel now has respected linguist  Geoffrey 
% Nunberg --not the fervent Edwin Newman of television fame--as its
%chair, and its composition, they proudly tell us, is closer to
%mainstream America: 112 men, 61 women, an average age of 61 (as
%opposed to 68 in previous editions).  From:The American Heritage
%Dictionary of the English Language, 3d ed.-book reviews, The Christian
%Century, 5/12/1993, Vol 11,no 16

%Add entry for PUpxPU w/commas and dashes, appositive use (like
%relative clauses)
 
