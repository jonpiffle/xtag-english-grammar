\chapter{Getting Around}

This technical report presents the English XTAG grammar as implemented by the
XTAG Research Group at the University of Pennsylvania.  The technical report is
organized into four parts, plus a set of appendices.  Part 1 contains general
information about the XTAG system and some of the underlying mechanisms that
help shape the grammar.  Chapter~\ref{intro-FBLTAG} contains an introduction to
the formalism behind the grammar and parser, while Chapter~\ref{overview}
contains information about the entire XTAG system.  Linguists interested solely
in the grammar of the XTAG system may safely skip Chapters~\ref{intro-FBLTAG}
and \ref{overview}.  Chapter~\ref{underview} contains information on some of
the linguistic principles that underlie the XTAG grammar, including the
distinction between complements and adjuncts, and how case is handled.

The actual description of the grammar begins with Part 2, and is contained in
the following three parts.  Parts 2 and 3 contains information on the verb
classes and the types of trees allowed within the verb classes, respectively,
while Part 4 contains information on trees not included in the verb classes
(e.g.  NP's, PP's, various modifiers, etc).  Chapter~\ref{table-intro} of Part
2 contains a table that attempts to provide an overview of the verb classes and
tree types by providing a graphical indication of which tree types are allowed
in which verb classes.  This has been cross-indexed to tree figures shown in
the tech report.  Chapter~\ref{verb-classes} contains an overview of all of the
verb classes in the XTAG grammar.  The rest of Part 2 contains more details on
several of the more interesting verb classes, including ergatives, sentential
subjects, sentential complements, small classes, ditransitives, and it-clefts.

Part 3 contains information on some of the tree types that are available within
the verb classes.  These tree types correspond to what would be transformations
in a movement based approach.  Not all of these types of trees are contained in
all of the verb classes.  The table (previously mentioned) in Part 2 contains a
list of the tree types and indicates which verb classes each occurs in.  

Part 4 focuses on the non-verb class trees in the grammar.  NP's and
determiners are presented in Chapter~\ref{det-comparitives}, while the various
modifier trees are presented in Chapter~\ref{modifiers}.  Auxiliary verbs,
which are classed separate from the verb classes, are presented in
Chapter~\ref{auxiliaries}, while certain types of conjunction are shown in
Chapter~\ref{conjunction}.  The XTAG treatment of comparatives is
presented in Chapter~\ref{compars-chapter}, and our treatment of
punctuation is discussed in Chapter~\ref{punct-chapt}.

Throughout the technical report, mention is occasionally made of
changes or analyses that we hope to incorporate in the future.
Appendix~\ref{future-work} details a list of these and other future
work.  The appendices also contain information on some of the nitty
gritty details of the XTAG grammar, including a system of metarules
which can be used for grammar development and maintenance in
Appendix~\ref{metarules}, a system for the organization of the grammar
in terms of an inheritance hierarchy is in Appendix~\ref{lexorg}, the
tree naming conventions used in XTAG are explained in detail in
Appendix~\ref{tree-naming}, and a comprehensive list of the features
used in the grammar is given in Appendix~{\ref{features}.
  Appendix~\ref{evaluation} contains an evaluation of the XTAG
  grammar, including comparisons with other wide coverage grammars.


