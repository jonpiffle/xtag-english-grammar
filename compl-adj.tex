\chapter{Underview}
\label{underview}

The morphology, syntactic, and tree databases together comprise the
English grammar.  A lexical item that is not in the databases receives
a default tree selection and features for its part of speech and
morphology.  In designing the grammar, a decision was made early on to
err on the side of acceptance whenever there are conflicting opinions
as to whether or not a construction is grammatical.  In this sense,
the XTAG English grammar is intended to function primarily as an
acceptor rather than a generator of English sentences.  The range of
syntactic phenomena that can be handled is large and includes
auxiliaries (including inversion), copula, raising and small clause
constructions, topicalization, relative clauses, infinitives, gerunds,
passives, adjuncts, it-clefts, wh-clefts, PRO constructions, noun-noun
modifications, extraposition, determiner sequences, genitives,
negation, noun-verb contractions, clausal adjuncts and imperatives.

\section{Subcategorization Frames}
\label{subcat-frames}

Elementary trees for non-auxiliary verbs are used to represent the linguistic
notion of subcategorization frames.  The anchor of the elementary tree
subcategorizes for the other elements that appear in the tree, forming a
clausal or sentential structure.  Tree families group together trees belonging
to the same subcategorization frame.  Consider the following uses of the verb
{\it buy}:

\enumsentence{Srini bought a book.}
\enumsentence{Srini bought Beth a book.}

In sentence (\ex{-1}), the verb {\it buy} subcategorizes for a direct
object NP.  The elementary tree anchored by {\it buy} is shown in
Figure~\ref{subcat-trees}(a) and includes nodes for the NP complement
of {\it buy} and for the NP subject.  In addition to this declarative
tree structure, the tree family also contains the trees that would be
related to each other transformationally in a movement based approach,
i.e passivization, imperatives, wh-questions, relative clauses, and so
forth.  Sentence (\ex{0}) shows that {\it buy} also subcategorizes for
a double NP object.  This means that {\it buy} also selects the double
NP object subcategorization frame, or tree family, with its own set of
transformationally related sentence structures.
Figure~\ref{subcat-trees}(b) shows the declarative structure for this
set of sentence structures.

\begin{figure}[ht]
\centering
\begin{tabular}{ccc}
{\psfig{figure=ps/compl-adj-files/alphanx0Vnx1_bought.ps,height=1.8in}} & 
\hspace*{0.5in} & 
{\psfig{figure=ps/compl-adj-files/alphanx0Vnx1nx2_bought.ps,height=1.8in}}\\
(a) & \hspace*{0.5in} & (b) \\ 
\end{tabular}
\caption{Different subcategorization frames for the verb {\it buy}}
\label{subcat-trees}
\end{figure}


\section{Complements and Adjuncts}
\label{compl-adj}

Complements and adjuncts have very different structures in the XTAG grammar.
Complements are included in the elementary tree anchored by the verb that
selects them, while adjuncts do not originate in the same elementary tree as
the verb anchoring the sentence, but are instead added to a structure by
adjunction.  The contrasts between complements and adjuncts have been
extensively discussed in the linguistics literature and the classification of a
given element as one or the other remains a matter of debate (see
\cite{rizzi90},
\cite{larson88}, \cite{jackendoff90}, \cite{larson90}, \cite{cinque90}, 
\cite{obernauer84}, \cite{lasnik-saito84}, and \cite{chomsky86}).  The guiding
rule used in developing the XTAG grammar is whether or not the sentence is
ungrammatical without the questioned structure.\footnote{Iteration of a
structure can also be used as a diagnostic: {\it Srini bought a book at the
bookstore on Walnut Street for a friend}.} Consider the following
sentences:

\enumsentence{Srini bought a book.}
\enumsentence{Srini bought a book at the bookstore.}
\enumsentence{Srini arranged for a ride.}
\enumsentence{$\ast$Srini arranged.}

Prepositional phrases frequently occur as adjuncts, and when they are
used as adjuncts they have a tree structure such as that shown in
Figure~\ref{compl-adjunct}(a).  This adjunction tree would adjoin into
the tree shown in Figure~\ref{subcat-trees}(a) to generate sentence
(\ex{-2}).  There are verbs, however, such as {\it arrange}, {\it
hunger} and {\it differentiate}, that take prepositional phrases as
complements.  Sentences (\ex{-1}) and (\ex{0}) clearly show that the
prepositional phrase are not optional for {\it arrange}.  For these
sentences, the prepositional phrase will be an initial tree (as shown
in Figure~\ref{compl-adjunct}(b)) that substitutes into an elementary
tree, such as the one anchored by the verb {\it arrange} in
Figure~\ref{compl-adjunct}(c).

\begin{figure}[ht]
\centering
\begin{tabular}{ccccc}
{\psfig{figure=ps/compl-adj-files/betavxPnx_at.ps,height=1.8in}} &
\hspace*{0.5in} &
{\psfig{figure=ps/compl-adj-files/alphaPXPnx_for.ps,height=1.3in}} &
\hspace*{0.5in} & 
{\psfig{figure=ps/compl-adj-files/alphanx0Vpnx1_arranged.ps,height=1.8in}}\\
(a) & \hspace*{0.5in} & (b) & \hspace*{0.5in} & (c) \\ 
\end{tabular}
\caption{Trees illustrating the difference between Complements and Adjuncts}
\label{compl-adjunct}
\label{2;1,9}
\end{figure}


Virtually all parts of speech, except for main verbs, function as both
complements and adjuncts in the grammar.  More information is available in this
report on various parts of speech as complements: adjectives (e.g. section
\ref{nx0Vax1-family}), nouns (e.g.  section~\ref{nx0Vnx1-family}), and
prepositions (e.g. section~\ref{nx0Vpnx1-family}); and as adjuncts: adjectives
(section~\ref{adj-modifier}), adverbs (section~\ref{adv-modifier}), nouns
(section~\ref{noun-modifier}), and prepositions (section~\ref{prep-modifier}).

\section{Non-S constituents}

Although sentential trees are generally considered to be special cases in any
grammar, insofar as they make up a `starting category', it is the case that any
initial tree constitutes a phrasal constituent.  These initial trees may have
substitution nodes that need to be filled (by other initial trees), and may be
modified by adjunct trees, exactly as the trees rooted in S.  Although grouping
is possible according to the heads or anchors of these trees, we have not found
any classification similar to the subcategorization frames for verbs that can
be used by a lexical entry to `group select' a set of trees.  These trees are
selected one by one by each lexical item, according to each lexical item's
idiosyncrasies.  The grammar described by this technical report places them
into several files for ease of use, but these files do not constitute tree
families in the way that the subcategorization frames do.

\section{Case Assignment}
\label{case-assignment}
\subsection{Approaches to Case}
\subsubsection{Case in GB theory}

GB (Government and Binding) theory proposes the following ``case filter'' as a
requirement on S-structure.\footnote{There are certain problems with applying
the case filter as a requirement at the level of S-structure.  These issues are
not crucial to the discussion of the English LTAG implementation of case and so
will not be discussed here.  Interested readers are referred to
\cite{lasnik-uriagereka88}.}

\begin{verse}
\underline{Case Filter}
Every overt NP must be assigned abstract case.
\end{verse}

Abstract case is taken to be universal.  Languages with rich morphological case
marking, such as Latin, and languages with very limited morphological case
marking, like English, are all presumed to have full systems of abstract case
that differ only in the extent of morphological realization.

In GB, abstract case is assigned to NPs by various case assigners, namely
verbs, prepositions, and INFL.  Verbs and prepositions are said to assign
accusative case to NPs that they govern, and INFL assigns nominative case to
NPs that it governs.  These governing categories are constrained in where they
can assign case by means of `barriers' based on `minimality conditions',
although these are relaxed in `exceptional case marking' situations.  The
details of the GB analysis are beyond the scope of this technical report, but
see \cite{chomsky86} for the original analysis or \cite{haegeman91} for an
overview.  Let it suffice for us to say that the notion of abstract case and
the case filter are useful in accounting for a number of phenomenon including
the distribution of nominative and accusative case, and the distribution of
overt NPs and empty categories (such as PRO).

\subsubsection{Minimalism and Case} 

A major conceptual difference between GB theories and minimalism is that in
minimalism, lexical items carry their features with them rather than being
assigned their features based on the nodes that they end up at.  For nouns,
this means that they carry case with them, and that case is 'checked' by
AGR$_s$ or AGR$_o$, which then disappears \cite{chomsky92}.

\subsection{Case in XTAG}

The English XTAG grammar adopts the notion of case and the case filter for
many of the same reasons argued in the GB literature.  However, the English
XTAG grammar implementation of case more closely resembles the treatment in
Chomsky's minimalism framework \cite{chomsky92} than the system outlined in the
GB literature \cite{chomsky86}.  As in minimalism, nouns in the XTAG approach
carry case with them, which is eventually 'checked' against the case values
assigned by the verb during the unification of the feature structures.  Unlike
Chomsky's minimalism, there is no separate AGR nodes; the case checking comes
from the verbs directly.

Most nouns in English do not have separate forms for nominative and accusative
case, and so they are ambiguous between the two.  Pronouns, of course, are
morphologically marked for case, and each carries the appropriate case in its
feature.  Figures \ref{nouns-with-case}a and \ref{nouns-with-case}b show the NP
tree anchored by a noun and a pronoun, respectively, along with the feature
values associated with each word.

\begin{figure*}[ht]
\centering
\rule[.1in]{3.5in}{0.01in} \\
\begin{tabular}{cc}
{\psfig{figure=ps/case-files/alphaNXN_books.ps,height=3.0in}}  &
{\psfig{figure=ps/case-files/alphaNXN_she.ps,height=3.2in}} \\
(a)&(b)\\
\end{tabular}\\
\caption {Lexicalized NP trees with case markings}
\rule[.1in]{3.5in}{0.01in}
\label {nouns-with-case}
\end{figure*}

\subsection{Case Assigners}

\subsubsection{Prepositions}
\label{prep-case}

Case is assigned in the XTAG English grammar by two components - verbs and
prepositions\footnote{{\it For} also assigns case as a complementizer.  See
section \ref{for-complementizer} for more details.}.  Prepositions assign
accusative case ({\bf acc})through their {\bf assign-case} feature, which is
linked directly to the {\bf case} feature of their objects.  Figure
\ref{PXPnx-with-case}a shows a lexicalized preposition tree, while
\ref{PXPnx-with-case}b shows the same tree with the NP tree from
\ref{nouns-with-case}a substituted into the NP position.  Figure
\ref{PXPnx-with-case}c is the tree \ref{PXPnx-with-case}b after unification has
taken place.  Note that the case ambiguity of {\it books} has been resolved to
accusative case.

\begin{figure*}[ht]
\centering
\rule[.1in]{6.0in}{0.01in}
\begin{tabular}{ccc}
{\psfig{figure=ps/case-files/alphaPXPnx_of.ps,height=1.7in}}  &
{\psfig{figure=ps/case-files/NXN-substituted-into-PXPnx.ps,height=3.5in}}  &
{\psfig{figure=ps/case-files/NXN-substituted-into-PXPnx-unified.ps,height=2.8in}} \\
(a)&(b)&(c)\\
\end{tabular}\\
\caption {Assigning case in prepositional phrases}
\rule[.1in]{6.0in}{0.01in}
\label {PXPnx-with-case}
\end{figure*}

\subsubsection{Verbs}
\label{case-for-verbs}
Verbs are the other part of speech in XTAG that can assign case.  Because
XTAG does not distinguish INFL and VP nodes\footnote{See section
\ref{VP-INFL-collapse} for an explanation of how this was done.}, verbs must
provide case assignment on the subject position in addition to the
case assigned to their NP complements.

Assigning case to NP complements is handled by building the case values of the
complements directly into the tree that the case assigner (the verb) anchors.
Figures \ref{S-tree-with-case}a and \ref{S-tree-with-case}b show an S
tree\footnote{Features not pertaining to this discussion have been taken out to
improve readability and to make the trees easier to fit onto the page.} that
would be anchored\footnote{The diamond marker ($\diamond$) indicates the
anchor(s) of a structure if the tree has not yet been lexicalized.} by a
transitive and ditransitive verb, respectively.  Note that the case assignments
for the NP complements are already in the tree, even though there is not yet a
lexical item anchoring the tree.  Since every verb that selects these trees
(and other trees in each respective subcategorization frame) assigns the same
case to the complements, building case features into the tree has exactly the
same result as putting the case feature value in each verb's lexical entry.

\begin{figure*}[ht]
\centering
\rule[.1in]{5.0in}{0.01in}
\begin{tabular}{cc}
{\psfig{figure=ps/case-files/alphanx0Vnx1-case-features.ps,height=2.0in}}  &
{\psfig{figure=ps/case-files/alphanx0Vnx1nx2-case-features.ps,height=2.0in}} \\
(a)&(b)\\
\end{tabular}\\
\caption {Case assignment to NP complements}
\rule[.1in]{5.0in}{0.01in}
\label {S-tree-with-case}
\end{figure*}

The case assigned to the subject position varies with verb form.  Since the
XTAG grammar treats the inflected verb as a single unit rather than dividing
it into INFL and V nodes, case, along with tense and agreement, is expressed in
the features of verbs, and must be passed in the appropriate manner.  The trees
in Figure \ref {S-tree-with-case} show the path of linkages that joins the {\bf
assign-case} feature of the V to the {\bf case} feature of the subject NP.  The
morphological form of the verb determines the value of the {\bf assign-case}
feature.  Figures \ref{lexicalized-S-tree-with-case}a and
\ref{lexicalized-S-tree-with-case}b show the same tree anchored by different
morphological forms of the verb {\it sing}, which give different values for the
assign-case feature\footnote{Again, the feature structures shown have been
restricted to those that pertain to the V/NP interaction.}.

\begin{figure*}[ht]
\centering
\rule[.1in]{5.0in}{0.01in}
\begin{tabular}{cc}
{\psfig{figure=ps/case-files/alphanx0Vnx1_sings-case-features.ps,height=3.2in}}  &
{\psfig{figure=ps/case-files/alphanx0Vnx1_singing-case-features.ps,height=2.9in}} \\
(a)&(b)\\
\end{tabular}\\
\caption {Assigning case according to verb form}
\rule[.1in]{5.0in}{0.01in}
\label {lexicalized-S-tree-with-case}
\end{figure*}

The adjunction of an auxiliary verb onto the VP node breaks the {\bf
assign-case} link from the main V and substitutes a link from the auxiliary
verb instead\footnote{see section \ref{aux-non-inverted} for a more complete
explanation of how this relinking occurs.}. The progressive form of the verb in
Figure \ref{lexicalized-S-tree-with-case}b assigns case {\bf none}, but this is
overridden by the adjunction of the appropriate form of the auxiliary word {\it
be}.  Figure \ref{Vvx-with-case}a shows the lexicalized auxiliary tree, while
\ref{Vvx-with-case}b shows it adjoined into the transitive tree shown in Figure
\ref{lexicalized-S-tree-with-case}b.  The case value passed to the NP is now
{\bf nom} (nominative).

\begin{figure*}[ht]
\centering
\rule[.1in]{5.0in}{0.01in}
\begin{tabular}{cc}
{\psfig{figure=ps/case-files/betaVvx_is-with-case.ps,height=2.1in}}  &
{\psfig{figure=ps/case-files/betaVvx_is-adjoined-into-nx0Vnx1_singing.ps,height=3.5in}} \\
(a)&(b)\\
\end{tabular}\\
\caption {Proper case assignment with auxiliary verbs}
\rule[.1in]{5.0in}{0.01in}
\label {Vvx-with-case}
\end{figure*}


\subsection{PRO in a unification based framework}

Most forms of a verb assign nominative case, although some forms, such as past
participle, assign no case whatsoever.  This is different than assigning case
{\bf none}, as the progressive form of the verb {\it sing} does in Figure
\ref{lexicalized-S-tree-with-case}b.  The distinction of a case {\bf none} from
no case is indicative of a divergence from the standard GB theory.  In GB
theory, the absence of case on an NP means that only PRO can fill that NP.  In
XTAG, the absence of case on an NP means that *any* NP can fill it,
regardless of its case.  This is due to the mechanism of unification, in which
if something is unspecified, it can unify with anything.  Thus we have a
specific case {\bf none} to handle verb forms that in GB theory do not assign
case.  PRO is the only NP with case {\bf none}.  Verbs forms that assign no
case, as the past participle mentioned above, can do so because they cannot
occur without an auxiliary verb which takes care of the case assignment.  Note
that although we are drawn to this treatment by our use of unification for
feature manipulation, \cite{watanabe93} proposes a very similar approach within
Chomsky's minimalist framework for entirely different reasons.

