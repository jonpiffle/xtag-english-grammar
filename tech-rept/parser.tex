Words in a lexicalized grammar are associated with structures that
they anchor. This property naturally suggests a two pass parsing
strategy for lexicalized grammars. Given an input sentence to parse,
the parser, in its first pass, needs to only select the relevant
subset of the grammar; the structures associated with the lexical
items of the sentence; instead of the entire grammar. This first pass
also allows the parser to utilize bottom-up information such as the
position of a lexical item in the input, to filter the selected subset
of the grammar. The second pass then combines the selected structures
using the operations of the formalism. It must be noted that the fewer
the selected structures by the first pass, the faster the second pass
performs.

The XTAG system uses a similar two-pass parsing strategy. The first
pass, tree-selection pass, uses the syntactic database entry for each
lexical item in the sentence to select a set of elementary trees from
the tree database. The second pass, the tree-grafting pass combines
the selected trees using substitution and adjunction operations and
performs unification of features associated with the trees to obtain
the parse of the sentence. It uses an Earley-style parsing algorithm
that has been extended to handle feature structures \cite{schabes90}.
The parser produces two structures -- a {\bf derived} tree and a {\bf
derivation} tree. The derived tree represents the surface constituent
structure, while the derivation tree represents the derivation history
of the parse.

%The nodes of the
%derivation tree are the names of the elementary trees that are
%anchored by the lexical items of the input.  The arcs of the
%derivation tree indicate the type of the combining operation; a dashed
%line for substitution and a bold line for adjunction.  The number
%beside each tree name is the address of the node at which the
%operation took place.  The derivation tree can also be interpreted as
%a dependency graph with unlabeled arcs between words of the sentence.

\subsection{Tree Selection}

- selection of trees from the db.
- heuristics to prune the set.

\subsection{Tree Grafting}

In this section, we present the working of the Earley-style parsing
using an example. A more formal presentation is given in
\cite{schabes90}.

Consider the sentence {\it John left quickly}. Figure~\ref{blah} shows
the trees anchored by the three words to be combined for the parse of
the sentence.

\begin{figure}

Insert 3 trees with 
\end{figure}

-information in the parser state.

The parser begins 