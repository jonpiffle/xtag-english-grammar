\section{Verb Transformations}
\label{verb-transformations}

The verbs classes described in Section \ref{verb-classes} each have a number of
trees associated with them.  These trees are the various related forms
(transformations) of sentences that verbs in that class can undergo.  This
section describes the basic structure and functionality of each of these
transformations.  Example trees will be given, although they will, of course,
need to be extrapolated for any other tree family.

\subsection{declarative}
The declarative tree is in some sense the most basic. It has no
extracted positions. It also has what we treat as the cannonical order
of constituents for a given subcategorization. For example, in a
declarative tree the subject will be called NP$_{0}$ or S$_{0}$, the
indirect object NP$_{1}$, the indirect object NP$_{2}$ etc. Other
constructions such as passive are treated as having reordered
constituents relative to the declarative. The declarative tree is used for
simple declarative sentences such as (\ex{1})-(\ex{2}) and for simple embedded
sentences such as (\ex{3})-(\ex{4}). 

\enumsentence{Beth Ann gave Clove a treat.}
\enumsentence{Dogs howl.}
\enumsentence{Jim thinks Beth Ann gave Clove a treat.}
\enumsentence{Everyone knows that dogs howl.}

The declarative tree is also used in forming yes-no questions (see
section?? below and Section~\ref{yes/no-question} for more details) and 
wh-questions on adjuncts (See Section~\ref{adv-modifier}). Figure (\ref{alphanx0Vnx1}) is the
declarative tree for the transitive tree family.

\begin{figure}[htbp]
\center{
\psfig{figure=/mnt/linc/extra/xtag/work/doc/tech-rept/ps/alphanx0Vnx1.ps,height=10.0cm}
\caption{ \label{alphanx0Vnx1} Tree:  $\alpha$nx0Vnx1}
}
\end{figure}



\subsection{Passive (with and without by-phrase)}

In passive constructions such as (\ex{1} ) the subject NP is
interpreted as having the same role as the direct object NP in the
corresponding declarative (e.g. (\ex{2}).

\enumsentence{{\bf An airline buy-out bill} was approved by the House.
(WSJ)}
\enumsentence{The House approved {\bf an airline buy-out bill}.}

 In a movement analysis, the direct object is said to have moved to
the subject position.  The original declarative subject is either
absent in the passive or is in a {\it by}-headed
PP (by-phrase). Passive trees are found in all tree families that have
a direct object in the declarative tree. Each of these tree families
has at least a passive tree with and one without the
by-phrase. Variations in the location of the by-phrase may require
additional trees.  For example the tree family Tnx0Vnx1s2 will have
the passive trees shown in figures (\ref{passive1})-
(\ref{passive3}). Figure (\ref{passive1}) shows typical features for
the passive. The V node is required to have {\bf $<$passive$>=+$} which
has the effect of forcing adjunction of the passive {\it be}.  The
other feature requirement on the V is that {\bf $,$mode$>=$ppart}
which has the effect of requiring passive morphology.  Figures
(\ref{passive2}) and (\ref{passive3}) show variants in the location of
the by-phrase.

 
\begin{figure}[htbp]
\center{
\psfig{figure=/mnt/linc/extra/xtag/work/doc/tech-rept/ps/betanx1Vs2.ps,height=10.0cm}
\caption{ \label{passive1} Tree:  $\beta$nx1Vs2}
}
\end{figure}

\begin{figure}[htbp]
\center{
\psfig{figure=/mnt/linc/extra/xtag/work/doc/tech-rept/ps/betanx1Vbynx0s2.ps,height=10.0cm}
\caption{ \label{passive2} Tree:  $\beta$nx1Vbynx0s2}
}
\end{figure}


\begin{figure}[htbp]
\center{
\psfig{figure=/mnt/linc/extra/xtag/work/doc/tech-rept/ps/betanx1Vs2bynx0.ps,height=10.0cm}
\caption{ \label{passive3} Tree:  $\beta$nx1Vs2bynx0}
}
\end{figure}


\subsection{Yes/No questions}

Yes/No questions are formed by the adjunction of an auxiliary verb to the left
side of the S node of the declarative or passive trees.  This mechanism is
described in more detail in the section on auxiliary verbs
(Section~\ref{yes/no-question}).

\subsection{Wh-moved subject}

The wh-moved subject tree provides for sentences such as {\it Who left?}, {\it
Who wrote the paper?}, and {\it Who was happy?}, depending on the tree family
with which it is associated.  Because wh+ subject does not require an auxiliary
verb, it is difficult to tell whether the subject has moved out of position or
not (see \ref{heycock/kroch93-gagl} for arguments for and against moved subject).  However,
we find that the wh+ subjects pattern like other wh+ elements that explicitly
move to the beginning of the sentence, and unlike wh+ elements that stay at
their place of origin (as in echo-questions, such as {\it John likes
*who*?}\footnote{The word surrounded by the *'s is intended to be read as
highly stressed.}).  This becomes apparent when looking at embedded sentences,
which are generally not allowed to have wh-moved elements ($\ast${\it John
thought that who$_i$ Mary loved t$_i$}), with the exception of specific verbs
which require it (John wondered who Mary loved).  This pattern switches with
regard to echo questions ({\it John thought that Mary loved *who*?} vs
$\ast${\it John wondered *who* Mary loved}).  Wh+ subjects pattern like other
wh-moved elements ($\ast$ {\it John thought that who$_i$ T$_i$ loved
Mary}\footnote{No emphasis on {\it who}, which would change it into an
echo-question reading} {\it John wondered who loved Mary}.  Accordingly, the
tree for the wh+ subject has the NP node extracted to a higher level, and is
contrained to be {\bf $<$wh$>$ = +}.  In addition, the highest S node is marked
{\bf $<$extracted$>$ = +}.  It is this extracted value that is used to regulate
the occurance of these trees in embedded sentences.




\subsection{Wh-moved NP complement}
Wh-questions can be formed on every NP object or indirect object that
appears in the declarative tree or in the passive trees.  A tree
family will contain one tree for each of these possible NP complement positions
. A tree of this type for a
Di-transitive tree family is shown in figure (\ref{alphaW1nx0Vnx1nx2}) with wh-extraction of the
direct object. 

\begin{figure}[htbp]
\center{
\psfig{figure=/mnt/linc/extra/xtag/work/doc/tech-rept/ps/alphaW1nx0Vnx1nx2.ps,height=10.0cm}
\caption{ \label{alphaW1nx0Vnx1nx2} Tree:  $\alpha$W1nx0Vnx1nx2}
}
\end{figure}

An extraction from  a passive in the
same tree family is shown in (\ref{alphaW2nx1Vnx2bynx0}).  

\begin{figure}[htbp]
\center{
\psfig{figure=/mnt/linc/extra/xtag/work/doc/tech-rept/ps/alphaW0nx1Vpnx2bynx0.ps,height=10.0cm}
\caption{ \label{{alphaW0nx1Vpnx2bynx0} Tree:  $\alpha$W0nx1Vpnx2bynx0}
}
\end{figure}

The important features
of this type of tree are:
\begin{enumerate}
\item The subtree that has S$_{r}$ as its root is identical to the
declarative tree or a non-extracted passive tree, except for having
one NP position in the VP filled by $\epsilon$.
\item The root S node is S$_{q}$ which dominates NP and S$_{r}$
\item The trace feature of the $\epsilon$ filled NP is coindexed with
the trace feature of the NP daughter of S$_{q}$.
\item The inv feature of S$_{r}$ is coindexed to the wh feature of NP
in order to force subject-auxiliary inversion where needed (see the
section on Topicalization for more discussion of the inv-wh coindexing
and the use of these trees for topicalization).
\end{enumerate}

\subsection{Wh-moved S complement}
Except for the node label on the extracted position, the trees for
wh-questions on S complements look exactly like the trees for
wh-questions on NP's in the same positions. This is because there is
no separate wh-lexical item for clauses in English, so the item {\it
what} is ambiguous between representing a clause or an NP.  To
illustrate this ambiguity notice that the question in (\ex{1}) could
be answered by either a clause as in (\ex{2}) or an NP as in (\ex{3}).

\enumsentence{what does Clove want?}
\enumsentence{Beth Ann to play frisbee with her}
\enumsentence{a biscuit}

Figure (\ref{wh-s-extr}) is an example of a tree for a wh-question on
a S complement from the tree family Tnx0Vs1.

\begin{figure}[htbp]
\center{
\psfig{figure=/mnt/linc/extra/xtag/work/doc/tech-rept/ps/betaW1nx0Vs1.ps,height=10.0cm}
\caption{ \label{betaW1nx0Vs1} Tree:  $\beta$W1nx0Vs1}
}
\end{figure}


\subsection{Wh-moved Adjective complement}
In subcategorizations that select an adjective complement that
complement can be questioned in a wh-question such as (\ex{1}).

\enumsentence{how$_{i}$ did he feel $\epsilon_{i}$}

The tree families with adjective complements include trees for such
adjective extractions that are very similar to the wh-extraction trees
for other categories of complements.  The adjective position in the VP
is filled by an $\epsilon$ and the trace feature of the adjective
complement and of the adjective daughter of S$_{q}$ are coindexed.  An
example of this type of tree is shown in Figure (\ref{wh-adj-extr})

\begin{figure}[htbp]
\center{
\psfig{figure=/mnt/linc/extra/xtag/work/doc/tech-rept/ps/betaWA1nx0Va1.ps,height=10.0cm}
\caption{ \label{betaWA1nx0Va1} Tree:  $\beta$WA1nx0Va1}
}
\end{figure}

\subsection{Wh-moved object of a P}
Wh-questions can be formed on the NP object of a complement PP as in
(\ex{1}).

\enumsentence{[Which dog]$_{i}$ did Beth Ann give a bone to $\epsilon_{i}$?

The by-phrases of passives behave like complements and can undergo the
same type of extraction (\ex{1}).

\enumsentence{Which dog was the frisbee caught by?}

Tree structures for this type of sentence are very similar to those
for the wh-extraction of NP complements discussed in Section \ref and
have the identical important features related to tree structure and
trace and inversion features.

The tree $\alpha$W2nx0Vnx1pnx2 in figure \ref{alphaW2nx0Vnx1pnx2} is
an example of this type of tree.

\begin{figure}[htbp]
\center{
\psfig{figure=/mnt/linc/extra/xtag/work/doc/tech-rept/ps/alphaW2nx0Vnx1pnx2.ps,height=10.0cm}
\caption{ \label{{alphaW2nx0Vnx1pnx2} Tree:  $\alpha$W2nx0Vnx1pnx2}
}
\end{figure}

\subsection{Wh-moved PP}
Like NP complements PP complements can be extracted to form
wh-questions as in (\ex{1}).

\enumsentence{[To which dog]$_{i}$ did Beth Ann throw the frisbee $\epsilon_{i}$?}

As can be seen in the tree $\alpha$pW2nx0Vnx1pnx2 in figure
\ref{alphapW2nx0Vnx1pnx2} extraction of PP complements is very similar
to extraction of NP complements from the same positions.  

\begin{figure}[htbp]
\center{
\psfig{figure=/mnt/linc/extra/xtag/work/doc/tech-rept/ps/alphapW2nx0Vnx1pnx2.ps,height=10.0cm}
\caption{ \label{{alphapW2nx0Vnx1pnx2} Tree:  $\alpha$pW2nx0Vnx1pnx2}
}
\end{figure}


The PP extraction trees differ from NP extraction trees in having a PP
rather than an NP left daughter node under S$-{q}$ and in having the
$\epsilon$ fill a PP rather than an NP position in the VP. In other
respects these PP extraction structures behave like the NP extractions
including being used for Topicalization.

\subsection{Topicalized NP complement}
Our analysis of topicalization uses the same trees as wh-extraction.
For any NP complement position a single tree is used for wh-questions
and for topicalization on that position. Wh-questions have
subject-auxiliary inversion and topicalizations do not.  This
difference between the constructions is captured by equating the
values of the S$_{r}$'s inv feature and the extracted NP's wh feature.
This means that if the extracted item is a wh-expression, as in
wh-questions, the value of inv will be $+$ and an inverted auxiliary
will be forced to adjoin. If the extracted item is a non-wh, inv will
be $-$ and no auxiliary adjunction will be required. For example, the
tree $\alpha$W1nx0Vnx1 in figure (\ref{2alphaW1nx0Vnx1}) is used to derive both (\ex{1]) and (\ex{2})

\begin{figure}[htbp]
\center{
\psfig{figure=/mnt/linc/extra/xtag/work/doc/tech-rept/ps/alphaW1nx0Vnx1.ps,height=10.0cm}
\caption{ \label{{2alphaW1nx0Vnx1} Tree:  $\alpha$W1nx0Vnx1}
}
\end{figure}

\ennumsentence{John, I like.}
\ennumsentence{Who do you like?}

For the question (\ex{0}) {\it Who do you like?}, the extracted item {\it who}
has $<$wh$>=+$  so the value of the inv feature on VP is also $+$ and
{\it do} is forced to adjoin.  For the topicalization (\ex{-1}) the values
for {\it John}`s wh feature and for S$_[q}$'s inv feature are both $-$ and no
auxiliary adjoins. Topicalization of PP complements is handled in a
similar way using the same trees as wh-questions on PP complement
positions. 

\subsection{Imperative}
Imperatives in English do not have overt subjects, and the subject
role is always interpreted as being second person, i.e. {\it you}.
The imperative trees in English LTAG grammar are identical to the
declarative tree except that the NP$_{0}$ subject position is filled by an
$\epsilon$ and the NP$_{0}$ agr-pers feature is set in the tree to the
value 2. Figure \ref{alphaInx0Vnx1} is the imperative tree for the transitive tree
family.

\begin{figure}[htbp]
\center{
\psfig{figure=/mnt/linc/extra/xtag/work/doc/tech-rept/ps/alphaInx0Vnx1.ps,height=10.0cm}
\caption{ \label{{alphaInx0Vnx1} Tree:  $\alpha$Inx0Vnx1}
}
\end{figure}}

\subsection{Determiner gerund}
\subsection{NP gerund}
\subsection{Relative clause on subject}

\enumsentence{the dog that caught the frisbee}
\enumsentence{dealers trading for their own account (WSJ)}

\subsection{Relative clause on NP complement}

\enumsentence{frisbees dogs like}
\enumsentence{political prisoners who had incurred his displeasure
(Brown Corpus)}
\enumsentence{less good news to report (WSJ)}


\subsection{Passive with wh-moved subject (with and without {\it by}
phrase)}


\subsection{Passive with wh-moved indirect object (with and without {\it by} phrase)}
\subsection{Passive with wh-moved object of the {\it by} phrase}
\subsection{Passive with wh-moved {\it by} phrase}





%TODO:
% for each section:
%	1) description
%	2) examples
%	3) trees
%Try latexing


