\section{Determiners}
\subsection{Introduction}

  We set out to provide an implementable syntactic theory of determiners over
the entire class.  We place emphasis on the word syntactic, since much work
has been done on the semantics of a small set of determiners, specifically
quantifier scoping.  By concentrating on the syntactic form, we hope to be able
to present a theory on how the class of determiners order with respect to each
other, thereby forcing semantics (and scoping operations) to conform to the
realities of the language.  Even segmented in this way, it is not an easy task,
and we have not completely covered all the phenomena to be found in the very
rich determiner system of English.  We do, however, cover a large portion of
the system both in terms of phenomena handled and especially in terms of corpus
coverage.

%Dania, can we get some kind of statistics on corpus coverage to present?

\subsection{Types of Determiners}

It is generally accepted that there are several subclasses of determiners.  We
recognize three subclasses of full determiners : predeterminers, determiners,
and postdeterminers.  By 'full' determiner, we mean that each word in this
class can modify a 3rdsing non-mass noun that requires a determiner to make a
full Noun Phrase.  All three subclasses are considered to be full determiners,
In addition to the full determiners, we also recognize a class of words that we
call determiner adverbs.  These are words which are generally considered to be
determiners, but which can not stand alone as a determiner phrase.  They do,
however, interact with other determiners to form Determiner Phrases.  We also
consider in our theory larger determiner constructions: genitive and partitive
constructions, and numbers.

\subsection{Predeterminers, determiners, and postdeterminers}

\subsubsection{Subclasses}
Since the class of determiners is closed, it is possible to give a list of the
determiners that we place in each subclass.  The subclasses are
purely syntactically motivated.  Pre-determiners precede determiners, which
preceded postdeterminers.

\begin{tabular}{lll}

\underline{Predeterminers}&\underline{Determiners}&\underline{Postdeterminers}\\
&&\\
both&these&few\\
all&those&many\\
&them&several\\
&this&various\\
&that&sundry\\
&the&some\\
&what&no\\
&which&neither\\
&whatever&wheen\\
&whichever&yon\\
&every&yond\\
&each&yonder\\
&a&PARTIVE\\
&an&CARDINAL NUMBERS\\
&some&\\
&any&\\
&another&\\
&GENITIVES&\\
\end{tabular}
			
\vspace{.5cm}
Although these classes order as shown above, it is not the case that one can
blindly choose any word from each of the above categories and expect that to be
a valid determiner phrase.  Each of the words above has both features
associated with the word itself, and features that determine what determiners
it can modify.

\subsubsection{Trees}

In our TAG grammar, the $\alpha$NXdxN tree is one of two basic NP trees.  It is the
tree used when a determiner is required or provided.  There are other
constructions in which the $\alpha$NXdxN tree does not substitute in, but is already
built into the tree, but in these cases, the structure looks the same.

\begin{figure}[ht]
\centering
\mbox{}
\psfig{figure=ps/determiners-files/alphaNXdxN.ps,height=3.0cm}
\caption{ Tree:  $\alpha$NXdxN}
\end{figure}

For all of these uses of Determiners Phrases, the DetP substitutes in.  It is
also necessary, though, to have the determiners build up by having them adjoin
onto each other.  Because of this, we have two basic determiner trees for the
full determiners, one that substitutes and one that adjoins.  Since any one of
the full determiners can function as a determiner alone, each of the
determiners listed above will get the substitution tree.  The predeterminer and
determiner subclasses will also get the auxiliary tree, since they can modify
other determiners.  Since the postdeterminers only go onto nouns, none of the
postdeterminers get the auxiliary tree.


\begin{figure}[ht]
\centering
\mbox{}
\begin{tabular}{ccc}
{\psfig{figure=ps/determiners-files/alphaDXD.ps}} & \hspace*{1.5in} & {\psfig{figure=ps/determiners-files/betaDdx.ps}}\\
%{\psfig{figure=ps/determiners-files/alphaDXD.ps}} & {\psfig{figure=ps/determiners-files/betaDds.ps}} \\
\end{tabular}
\caption{Trees: $\alpha$DXD and $\beta$Ddx}
\end{figure}


\subsubsection{Features}

\vspace{0.5in}
\noindent
{\bf Internal Features}
\vspace{0.25in}

The features are the most critical part in getting the determiner ordering
correct.  We have identified 8 features as being sufficient to order the
determiners that we have looked at.  These features and their range of values
are given in table \ref{det-features}:

\vspace*{0.2in}
\begin{center}
\begin{tabular}{lll}
\underline{feature}&\underline{values}&\underline{property}\\
predet&$+-$&Predeterminer\\
definite&$+-$&Definite\\
quan&$+-$&Quantifier\\
card&$+-$&Cardinal number\\
gen&$+-$&Genitive\\
wh&$+-$&wh element\\
decreasing&$+-$&Decreasing\\
agr&3sg, 3pl, 3sgpl&Agreement\\
\end{tabular}
\label{det-features}
\end{center}

%Beth, perhaps we should give a brief explanation of each feature here and
%why each is linguistically motivated (and where they 'came from')?  

We do forward ordering on the determiners, which means that each determiner, in
addition to having features describing its own values, also has features
describing what kinds of determiners it can modify.  Note that the
postdeterminers, since they select only the substitution tree, have only the
self-describing features.  We also looked at the possibility of doing
backward ordering, so that each determiner described what kinds of determiners
could modify it, but we found that this required making distinctions that were
not linguistically motivated.

The following chart gives the features for each determiner and the features for
the determiners that it can go on.

\vspace*{0.3in}

\begin{tabular}{lll}
\underline{Determiner}&\underline{Features}&\underline{Modifies}\\
both&predet+,definite+,quan-,card-,gen-,3pl&[predet-,definite+,quan-,card-,wh-]\\
all&predet+,definite+,quan-,card-,gen-,3pl&[predet-,definite+,quan-,card-,wh- ] or card+\\
&&\\
these&predet-,definite+,quan-,card-,gen-,3pl&card+ or[quan+,decreasing+,definite-,wh-]\\
those&predet-,definite+,quan-,card-,gen-,3pl&card+ or [quan+,decreasing+,definite-,wh-]\\
them&predet-,definite+,quan-,card-,gen-,3pl&card+ or [quan+,decreasing+,definite-,wh-]\\
this&predet-,definite+,quan-,card-,gen-,3sg&card+ or [quan+,decreasing+,definite-,wh-]\\
that&predet-,definite+,quan-,card-,gen-,3sg&card+ or [quan+,decreasing+,definite-,wh-]\\
&&\\
yon&predet-,definite+,quan-,card-,gen-,3sg&only nouns\\
yond&predet-,definite+,quan-,card-,gen-,3sg&only nouns\\
yonder&predet-,definite+,quan-,card-,gen-,3pl&only nouns\\
&&\\
what&predet-,definite+,quan-,card-,gen-,wh+,3sgpl&card+ or [quan+,decreasing+,definite-,wh-]\\
whatever&predet-,definite+,quan-,card-,gen-,wh+,3sgpl&card+ or [quan+,decreasing+,definite-,wh-]\\
which&predet-,definite+,quan-,card-,gen-,wh+,3sgpl&card+ or [quan+,decreasing+,definite-,wh-]\\
whichever&predet-,definite+,quan-,card-,gen-,wh+,3sgpl&card+ or [quan+,decreasing+,definite-,wh-]\\
&&\\
the&predet-,definite+,quan-,card-,gen-,3sgpl&card+ or [quan+,definite-,wh-]\\
&&\\
every&predet-,definite+,quan+,card-,gen-,3sg&card+ or [quan+,decreasing+,definite-,wh-]\\
each&predet-,definite+,quan+,card-,gen-,3sg&card+ or [quan+,decreasing+,definite-,wh-]\\
any&predet-,definite-,quan+,card-,gen-,3sgpl&card+ or [quan+,decreasing+,definite-,wh-]\\
another&predet-,definite-,quan+,card-,gen-,3sg&card+ or [quan+,decreasing+,definite-,wh-]\\
&&\\
GEN&predet-,definite+,quan-,card-,gen+,unspec.&card+ or [quan+,definite-,wh-]\\
&&\\
a&predet-,definite-,quan-,card-,gen-,3sg&[quan+,decreasing+,definite-,wh-]\\
an&predet-,definite-,quan-,card-,gen-,3sg&[quan+,decreasing+,definite-,wh-]\\
some&predet-,definite-,quan-,card-,gen-,3sgpl&[quan+,decreasing+,definite-,wh-]\\
&&\\
CARD&predet-,definite+,quan-,card+,gen-,one=3sg;rest=3pl&only nouns\\
&&\\
PART&predet-,definite-,quan-,card-,gen-,unspec.&anything wh- \\
&&\\
few&predet-,definite-,quan+,card-,decreasing+,3pl&only nouns\\
wheen&predet-,definite-,quan+,card-,decreasing+,3pl&only nouns\\
many&predet-,definite-,quan+,card-,decreasing-,3pl&only nouns\\
several&predet-,definite-,quan+,card-,decreasing-,3pl&only nouns\\
various&predet-,definite-,quan+,card-,decreasing-,3pl&only nouns\\
sundry&predet-,definite-,quan+,card-,decreasing-,3pl&only nouns\\
some&predet-,definite-,quan+,card-,decreasing-,3pl&only nouns\\
no&predet-,definite+,quan+,card-,decreasing+,3sgpl&only nouns\\
neither&predet-,definite+,quan+,card-,decreasing+,3sgpl&only nouns\\
\end{tabular}


\vspace*{0.3in}

The passing of the features within the trees is fairly simple.  The top DetP
node inherits the feature of the leftmost DetP, with the exception of the agr
feature, which is discussed in the next section.  The lexical entry for each
determiner instantiates the features which are passed up to the root DetP for
that tree, as well as specifying certain values which must be unified with in
the foot node of the adjoining tree.


%Figure of alphaDXD-features here

\begin{figure}[ht]
\centering
\mbox{}
\psfig{figure=ps/determiners-files/alphaDXD.ps,height=3.0cm}
\caption{Tree:  $\alpha$DXD}
\label{DXD} 
\end{figure}

%Figure of betaDdx-features here

\begin{figure}[ht]
\centering
\mbox{}
\psfig{figure=ps/determiners-files/betaDdx.ps,height=3.0cm}
\caption{\label{Ddx-features} Tree:  $\beta$Ddx}
\end{figure}


%Figure of betaDdx_both_.ps here

\begin{figure}[ht]
\centering
\mbox{}
\psfig{figure=ps/determiners-files/betaDdx_both_.ps,height=10.0cm}
\caption{ \label{Ddx_both} Tree:  $\beta$Ddx}
\end{figure}


The {\bf wh} feature does not do anything special with the determiner system, since
no predeterminers can adjoin onto it, it is always the left-most determiner.
Thereby, as described above, the entire DetP wil inherit its wh value, even if
it occurs with other determiners (i.e {\it which few men}).  The {\bf agr} feature, on
the other hand, is a little more complicated.  You may notice in Figure \ref{Ddx-features} that the agr feature is the only feature not passed from
the D node to the root DetP, but is instead passed from the foot DetP to the
root DetP.  In the determiner system, the {\bf agr} feature generally comes from the
rightmost determiner, although some adjoining determiners also require that
they agree with the rightmost determiner.  

%Beth - is this part of the scoping problem.  

We have captured this distinction easily be simply having each
lexical item specify whether they are require to agree or not by allowing each
determiner to explicitly pass (or not) its agreement feature to the root DetP.
In Figure \ref{Ddx_both}, the {\it both} determiner requires that it
agree with the determiner it modifies (both these men, *both this man).  In
comparison, the {\it another} determiner does not require that it agree with the
determiner that it modified (e.g. {\it another few men, another man}).  

%Figure of alphaDdx_another_.ps here 

\begin{figure}[ht]
\centering
\mbox{}
\psfig{figure=ps/determiners-files/betaDdx_another_.ps,height=10.0cm}
\caption{Tree:$\alpha$Ddx}
\label{Ddx}
\end{figure}

\subsection{Adverbial Determiners}

There are a number of words that have generally been considered to be in the
class of determiners, but which do not function as determiners on their own.
This class includes words such as hardly, even, most, virtually, scarcely,
such, almost, and only.  Although we know that we want these words to adjoin
into Determiner Phrases, we have not built a theory as to exactly what
determiners the various adverbial determiner should adjoin onto.  Because of
this incompleteness, we have not implemented this part at all.  We anticipate
that the tree for these determiners will look similar to the Ddx tree in Figure\ref{Ddx}  we may find it necessary to have a tree that adjoins onto the
right side of a determiner as well.  We are including what we have worked out in Table \ref{det-adv}
in anticipation of this section being implemented in the near future.

\vspace*{0.3in}

\begin{tabular}{lll}[ht]
hardly&NA&a/any/another/every\\
only&NA&[predet-,quan-,wh-]\\
even&NA&[predet-,quan-,wh-]\\
almost&NA&any/every/a/an\\
most&NA&any/every\\
virtually&NA&all/any/every/the/that/this\\
scarely&NA&any/another/a/an\\
\end{tabular}
\label{det-adv}

\vspace*{0.3in}

\subsection{Genitive Constructions}

The genitive construction functions as a full determiner, and interacts with
other determiners in the system.  The features for the genitive are the same as
for full determiners (except that all genitives are marked gen$+$, as opposed to
all other determiners), so we will not repeat that information here.  We will
mention, in accordance with the discussion of the {\bf agr} feature in section
**3.3.2**, that gentive's are not required to agree with either the
determiners or the nouns that they modify.  

There are two kinds of genitive contructions: genitive pronouns, which carry
the genitive marker internally, and genitive NPs, which have an explicit
genitive marker ('s) associated with them.

\subsubsection{Genitive Pronouns}

It is clear the genitive pronouns function as determiners as seen from examples
such as 'her dog was run over' vs '*dog was run over'.  In our determiner
system, they interact fully with the rest of the determiners, as can be seen in
Chart **X**.  Accordingly, there are two trees associated with the genitive
pronouns that correspond to the two determiner trees.  They are identical to
the trees for the full determiners, except for the category of the anchor.


%Figure of alphaDXG-features here 

\begin{figure}[ht]
\centering
\mbox{}
\psfig{figure=ps/determiners-files/alphaDXG-features.ps,height=8.0cm}
\caption{Tree:  }
\end{figure}

%Figure of betaGdx-features here 

\begin{figure}[ht]
\centering
\mbox{}
\psfig{figure=ps/determiners-files/betaGdx-features.ps,height=8.0cm}
\caption{Tree:  }
\end{figure}

\subsubsection{Genitive NPs}

Genitive NP's are interesting in that they are potentially recursive
structures.  Here, entire structures of NP can be embedded in a determiner
phrase.  Note the following examples:

  
\enumsentence{[The man]'s mother came to town.}
\enumsentence{[The man with the telescope]'s mother came to town.}
\enumsentence{[[John]'s friend]'s mother came to town.}
\enumsentence{[[[John]'s friend]'s uncle]'s mother came to town.}

There are two things of note in the above examples.  One is that in embedded
NP's, the genitive marker comes at the end of the NP phrase, even if the head
of the NP is earlier.  The other thing to note is that the determiner of the NP
can also be a genitive NP, hence the possibility of recursive structures.  

In Xtag, the genitive marker is separated out from the lexical item that it is
attached to and given its own category (G).  In this way, we can allow the full
complexity of NP's to come from the existing NP system, including any recursive
structures.  The two trees below demonstrate how easily the complexity of
genitive NP's can be captured in TAGs.  The two trees, again, correspond to the two
determiner trees presented above.

%Figure of alphaDXnxG-features here

%Figure of betanxGdx-features here 


\subsection{Partitive Constructions}

Partitive constructions (e.g. {\it  some kind of}, {\it all of\/}) were placed in the category
of determiner constructions due to the agreement features.  If the partitive
constructions were analyzed as an NP with a PP adjoined on, then we would
expect to get agreement with the head of the NP (the left N).  If, on the other hand, we
analyze them as a determiner construction on a N, then we would expect to get
agreement with the right N.

\enumsentence{[a certain kind of] these machines are prone to failure.}
\enumsentence{$\ast$[a certain kind of] these machines is prone to failure.}

This is certainly not to say that one can't get a (NP PP) reading out of these
constructions, just that the partitive construction does exist, and that the
distinguishing caracteristic is with which noun the agreement occurs.  This,
of course, concurs with intuition of which NP should head the clause.

\enumsentence{some parts [of this machine] are worn-out.}
\enumsentence{$\ast$some parts [of this machine] is worn-out.}

As a full determiner construction, partitive constructions interact with other
determiners.  They can modify the noun itself (a certain kind of machine), or
modify other determiners, as seen in the examples above.  Accordingly, these
partive construction has the two trees as we have seen in all the other full
determiner constructions.

%Figure of alphaDXnxP-features here

\begin{figure}[ht]
\centering
\psfig{figure=ps/determiners-files/alphaDXnxP-features.ps,height=8.0cm}
\caption{ Tree: }
\end{figure}

%Figure of betanxPdx-features here 
\begin{figure}[ht]
\centering
\psfig{figure=ps/determiners-files/betanxPdx-features.ps,height=8.0cm}
\caption{ Tree:  }
\end{figure}

Note that the P (Preposition) anchors these trees.  There is only 1 Preposition
({\it of}) that selects for these trees.  It should be noted as well that the NP in
these constructions should be restricted semantically to a small class of
nouns, such as {\it sort}, {\it form}, {\it kind}, and {\it type}.  They do not function as
full NP as in the genitive constructions, although since the embedded NP can
have its own determiner, recursive structures are available here as well.


\subsection{Comparative Constructions}

We beleive that comparative constructions such as {\it more than\/} or
{\it other than\/}
will fit into the determiner phrase category, although some very
similar constructions may be better analyzed  as adjective phrases.

\enumsentence{[More than one] person was chosen.	(DetP?)}
\enumsentence{John was [bigger than a mountain].  (AdjP?)}
    
In any case, we do not currently have an analysis for these kinds of
determiners.

\subsection{Conjunction of determiners}

Determiner conjunction in English seems to be  fairly unconstrained in English.  It allows conjunction of 'unlike'
determiners, such as {\it one and all}, where {\it one\/} is a post-determiner
and {\it all\/}
is a predeterminer.  Determiners are not required to agree with each other,
except in the case of the wh feature ({\it who and which}, {\it $\ast$who and some}).  We
take the determiner features of the leftmost determiner as the values for the
root DetP.

%Figure beataDX1conjDX2-features.ps
\begin{figure}[ht]
\centering
\psfig{figure=ps/determiners-files/betaDX1conjDX2-features.ps,height=8.0cm}
\caption{ Tree:  }
\end{figure}

\subsection{Numbers}

The number system has a complexity all its own outside of its interaction with
the determiner system.  Number lists, such as address or telephone numbers, are
easily handled, albeit outside of the determiner system.  In part, this is
because we have found no evidence that they are actually functioning as
determiners in these situations.  The set of numbers, in addition to being
determiners, are also nouns, and we handle noun noun compounds quite easily.
In addition to handling address and telephone numbers, this also allows us to
handle dates and other compounds where numbers are interspersed with other nouns.

More difficult is to get the correct behavior for number sequences
such as {\it one
thousand four hundred and twenty two}.  In contrast to the number lists, these
do seem to function as determiners (e.g. {\it twenty two men\/}), and as such, we would like
to handle them. 

The basic determiner system does handle some of these numbers already.  While
the one through cardinal numbers get only the standard determiner substitution
tree, the other numbers (such as twenty, thirty, etc), get the auxiliary tree
as well, but they are constrained to only go on other cardinal numbers.  This
handles determiners such as {\it thirty-one}.  

To handle numbers such as {\it one thousand four hundred and twenty two}, we need
to introduce two new determiner trees.  These trees are selected for by the
cardinal numbers ({\it one\/} through {\it ten}, {\it twenty}, {\it thirty}, etc).  They take a D
tree as a subsitution node, and that D tree corresponds to the set of lexical
items such as {\it hundred}, {\it thousand}, {\it million\/} etc.  The two types of tree
should be familiar by now, one is the subsitution tree, and the other the
auxiliary tree.  Note the this auxiliary tree can only go onto card+
determiners.

%Figure of alpha-DXDd here
\begin{figure}[ht]
\centering
\psfig{figure=ps/determiners-files/alphaDXDd-features.ps,height=8.0cm}
\caption{ Tree:  }
\end{figure}

%Figure of beta-Dddx here 
\begin{figure}[ht]
\centering
\psfig{figure=ps/determiners-files/betaDddx-features.ps,height=8.0cm}
\caption{ Tree: }
\end{figure}

%%\subsection{Special and Problem determiners???}
%%
%%Beth,  could you say something about 'some' somewhere in here?  We have it
%%twice, because it functions differently, but I don't remember exactly how that
%%worked, and you seemed pretty clear on it when it came up.  It doesn't have to
%%be in this section.  Perhaps when you discuss the features in the second
%%section?

			


