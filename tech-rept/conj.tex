\chapter{Conjunction}
\label{conjunction}

\section{Introduction}

The XTAG system can handle sentences with conjunction of two
constituents of the same syntactic category. The coordinating
conjunctions which select the conjunction trees are {\it and}, {\it
or} and {\it but}.\footnote{We believe that the restriction of {\it
but} to conjoining only two items is a pragmatic one, and our grammars
accepts sequences of any number of elements conjoined by {\it but}.}
There are also multi-word conjunction trees, anchored by {\it either-or}, 
{\it neither-nor}, {\it both-and}, and {\it as well as}.  There are eight
syntactic categories that can be coordinated, and in each case an
auxiliary tree is used to implement the conjunction.  These eight
categories can be considered as four different cases, as described in
the following sections.  In all cases the two constituents are
required to be of the same syntactic category, but there may also be
some additional constraints, as described below.


\section{Adjective, Adverb, Preposition and PP Conjunction}

Each of these four categories has an auxiliary tree that is used for
conjunction of two constituents of that category.  The auxiliary tree
adjoins into the left-hand-side component, and the right-hand-side
component substitutes into the auxiliary tree.  

\begin{figure}[htb]
\centering
\begin{tabular}{ccc}
{\psfig{figure=ps/conj-files/betaA1conjA2.ps,height=1in}}&
\hspace*{0.5in}&
{\psfig{figure=ps/conj-files/derived-tree-140291.ps,height=2.8in}}\\
(a) & \hspace*{0.5in}& (b)\\
\end{tabular}
\caption{Tree for adjective conjunction: $\beta$a1CONJa2 and a resulting parse tree}
\label{A1conjA2}
\end{figure}

Figure~\ref{A1conjA2}(a) shows the auxiliary tree for adjective conjunction,
and is used, for example, in the derivation of the parse tree for the noun
phrase {\it the dark and dreary day}, as shown in Figure~\ref{A1conjA2}(b).
The auxiliary tree adjoins onto the node for the left adjective, and the
right adjective substitutes into the right hand side node of the auxiliary
tree. The analysis for adverb, preposition and PP conjunction is exactly the
same and there is a corresponding auxiliary tree for each of these that is
identical to that of Figure~\ref{A1conjA2}(a) except, of course, for the node
labels.


\section{Noun Phrase and Noun Conjunction}

The tree for NP conjunction, shown in Figure~\ref{NP1conjNP2}(a), has
the same basic analysis as in the previous section except that the
{\bf $<$wh$>$} and {\bf $<$case$>$} features are used to force the two
noun phrases to have the same {\bf $<$wh$>$} and {\bf $<$case$>$}
values.  This allows, for example, {\it he and she wrote the book
together} while disallowing {\it $\ast$he and her wrote the book
together.}  Agreement is lexicalized, since the various conjunctions
behave differently. With {\it and}, the root {\bf $<$agr num$>$} value
is {\bf $<$plural$>$}, no matter what the number of the two
conjuncts. With {\it or}, however, the root {\bf $<$agr num$>$} is
co-indexed with the {\bf $<$agr num$>$} feature of the right
conjunct. This ensures that the entire conjunct will bear the number
of both conjuncts if they agree (Figure~\ref{NP1conjNP2}(b)), or of
the most ``recent'' one if they differ ({\it Either the boys or John
is going to help you.}). There is no rule per se on what the
agreement should be here, but people tend to make the verb agree with
the last conjunct (cf. \cite{quirk85}, section 10.41
for discussion). The tree for N conjunction is identical to that for
the NP tree except for the node labels. (The multi-word conjunctions
do not select the N conjunction tree - {\it $^*$the both dogs and
cats}).

\begin{figure}[htb]
\centering
\begin{tabular}{cc}
{\psfig{figure=ps/conj-files/betaCONJnx1CONJnx2.ps,height=4in}}
\hspace{0.5cm} &
{\psfig{figure=ps/conj-files/aardvarks-and-emus.ps,height=4in}}\\
(a) &  (b)\\
\end{tabular}
\caption{Tree for NP conjunction: $\beta$CONJnx1CONJnx2 and a resulting
parse tree}
\label{NP1conjNP2}
\end{figure}


\section{Determiner Conjunction}

%CDD,10/31/95:We cannot find any reason for this tree to be different,
%so it is now like the other conj trees, with the foot on the left.
%
%The tree for determiner conjunction, shown in Figure~\ref{DX1conjDX2}, is
%unlike the other conjunction trees in that the foot node is on the right.  This
%is because determiner phrases generally build to the left. For the
%same reason, 

In determiner coordination, all of the determiner feature values are
taken from the left determiner, and the only requirement is that the
{\bf $<$wh$>$} feature is the same, while the other features, such as
{\bf $<$card$>$}, are unconstrained.  For example, {\it which and
what} and {\it all but one} are both acceptable determiner
conjunctions, but {\it $\ast$which and all} is not.

\enumsentence{how many and which people camp frequently ?}
\enumsentence{$^*$some or which people enjoy nature .}

\begin{figure}[htbp]
\centering
\begin{tabular}{c}
\psfig{figure=ps/conj-files/betad1CONJd2.ps,height=5.3in}
\end{tabular}
\vspace{-0.25in}
\caption{Tree for determiner conjunction: $\beta$d1CONJd2.ps}
\label{DX1conjDX2}
\end{figure}

\section{Sentential Conjunction}

The tree for sentential conjunction, shown in Figure~\ref{S1conjS2},
is based on the same analysis as the conjunctions in the previous two
sections, with a slight difference in features.  The {\bf $<$mode$>$}
feature\footnote{See section~\ref{s-features} for an explanation of
the {\bf $<$mode$>$} feature.}  is used to constrain the two sentences
being conjoined to have the same mode so that {\it the day is dark and
the phone never rang} is acceptable, but {\it $\ast$the day dark and
the phone never rang} is not. Similarly, the two sentences must agree
in their {\bf $<$wh$>$}, {\bf $<$comp$>$} and {\bf $<$extracted$>$}
features.  Co-indexation of the {\bf $<$comp$>$} feature ensures that
either both conjuncts have the same complementizer, or there is a
single complementizer adjoined to the complete conjoined S.  The {\bf
$<$assign-comp$>$} feature\footnote{See
section~\ref{for-complementizer} for an explanation of the {\bf
$<$assign-comp$>$} feature.} feature is used to allow conjunction of
infinitival sentences, such as {\it to read and to sleep is a good
life}.

\begin{figure}[htb]
\centering
\begin{tabular}{c}
\psfig{figure=ps/conj-files/betaS1conjS2.ps,height=3.5in}
\end{tabular}
\caption{Tree for sentential conjunction: $\beta$s1CONJs2}
\label{S1conjS2}
\end{figure}

\section{Comma as a conjunction}

We treat comma as a conjunction in conjoined lists. It anchors the
same trees as the lexical conjunctions, but is considerably more
restricted in how it combines with them. The trees anchored by commas
are prohibited from adjoining to anything but another comma conjoined
element or a non-coordinate element. (All scope possibilities are
allowed for elements coordinated with lexical conjunctions.) Thus,
structures such as Tree
\ref{Comma-conj}(a) are permitted, with each element stacking
sequentially on top of the first element of the conjunct, while
structures such as Tree \ref{Comma-conj}(b) are blocked. 

\begin{figure}[htb]
\centering
\begin{tabular}{ccc}
{\psfig{figure=ps/conj-files/good-adj-conj.ps,height=2.75in}}&
\hspace*{0.5in}&
{\psfig{figure=ps/conj-files/bad-adj-conj.ps,height=2.75in}}\\
(a) Valid tree with comma conjunction & \hspace*{0.5in}& (b) Invalid tree\\
\end{tabular}
\caption{}
\label{Comma-conj}
\end{figure}

This is accomplished by using the {\bf $<$conj$>$} feature, which has the
values {\bf and/or/but} and {\bf comma} to differentiate the lexical
conjunctions from commas. The {\bf $<$conj$>$} values for a comma-anchored
tree and {\it and}-anchored tree are shown in Figure
\ref{conj-contrast}. The feature {\bf $<$conj$>$ = comma/none} on
A$_1$ in (a) only allows comma conjoined or non-conjoined elements as
the left-adjunct, and {\bf $<$conj$>$ = none} on A in (a) allows
only a non-conjoined element as the right conjunct. We also need the
feature {\bf $<$conj$>$ = and/or/but/none} on the right conjunct of
the trees anchored by lexical conjunctions like (b), to block
comma-conjoined elements from substituting there. Without this
restriction, we would get multiple parses of the NP in Tree
\ref{Comma-conj}; with the restrictions we only get the derivation
with the correct scoping, shown as (a).

Since comma-conjoined lists can appear without a lexical conjunction
between the final two elements, as shown in example (\ex{1}), we cannot
force all comma-conjoined sequences to end with a lexical conjunction.

\enumsentence{So it is too with many other spirits which we all know: the
spirit of Nazism or Communism, school spirit , the spirit of a street
corner gang or a football team, the spirit of Rotary or the Ku Klux
Klan. \hfill [Brown cd01]}


\begin{figure}[htb]
\centering
\begin{tabular}{cc}
{\psfig{figure=ps/conj-files/adj-comma-conj.ps,height=2.5in}}&
{\psfig{figure=ps/conj-files/adj-and-conj.ps,height=2.5in}}\\
\end{tabular}
\caption{$\beta$a1CONJa2 (a) anchored by comma and (b) anchored by {\it and}}
\label{conj-contrast}
\end{figure}


\section{{\it But-not}, {\it not-but}, {\it and-not} and  {\it
$\epsilon$-not}}

We are analyzing conjoined structures such as {\it The women but not
the men} with a multi-anchor conjunction tree anchored by the
conjunction plus the adverb {\it not}. The alternative is to allow
{\it not} to adjoin to any constituent. However, this is the only
construction where {\it not} can freely occur onto a constituent other
than a VP or adjective (cf. $\beta$NEGvx and $\beta$NEGa trees). It
can also adjoin to some determiners, as discussed in Section
\ref{det-comparitives}. We want to allow sentences like (\ex{1}) and
rule out those like (\ex{2}). The tree for the good example is shown
in Figure \ref{but-not}. There are similar trees for {\it and-not} and
{\it $\epsilon$-not}, where $\epsilon$ is interpretable as either {\it
and} or {\it but}, and a tree with {\it not} on the first conjunct for
{\it not-but}.

\enumsentence{Beth grows basil in the house (but) not in the garden .}
\enumsentence{$^*$Beth grows basil (but) not in the garden .}

\begin{figure}[htb]
\centering
\begin{tabular}{c}
\psfig{figure=ps/conj-files/but-not.ps,height=2.5in}
\end{tabular}
\caption{Tree for conjunction with but-not: $\beta$px1CONJARBpx2}
\label{but-not}
\end{figure}

Although these constructions sound a bit odd when the two conjuncts do
not have the same number, they are sometimes possible. The agreement
information for such NPs is always that of the non-negated conjunct:
{\it his sons, and not Bill, are in charge of doing the laundry} or
{\it not Bill, but his sons, are in charge of doing the laundry}
(Some people insist on having the commas here, but they are frequently
absent in corpus data.) The agreement feature from the non-negated
conjunct in passed to the root NP, as shown in Figure
\ref{not-but}. Aside from agreement, these constructions behave just
like their non-negated counterparts.

%"Everyone is going on line and not just the younger generation," said Mal

\begin{figure}[htb]
\centering
\begin{tabular}{c}
\psfig{figure=ps/conj-files/not-but.ps,height=4in}
\end{tabular}
\caption{Tree for conjunction with not-but: $\beta$ARBnx1CONJnx2} 
\label{not-but}
\end{figure}

\section{{\it To} as a Conjunction}

{\it To} can be used as a conjunction for adjectives
(Fig. \ref{to-conj}) and determiners, when they denote points on a
scale:

\enumsentence{two to three degrees}
\enumsentence{high to very high temperatures}

As far as we can tell, when the conjuncts are determiners they must be
cardinal.

\begin{figure}[htb]
\centering
\begin{tabular}{c}
\psfig{figure=ps/conj-files/to.ps,height=3.5in}
\end{tabular}
\caption{Example of conjunction with {\it to}} 
\label{to-conj}
\end{figure}

\section{Predicative Coordination}

This section describes the method for predicative coordination
(including VP coordination of various kinds) used in XTAG. The
description is derived from work described in (\cite{anoopjoshi96}).
It is important to say that this implementation of predicative
coordination is not part of the XTAG release at the moment due massive
parsing ambiguities. This is partly because of the current
implementation and also the inherent ambiguities due to VP
coordination that cause a combinatorial explosion for the parser. We
are trying to remedy both of these limitations using a probability
model for coordination attachments which will be included as part of a
later XTAG release.

This extended domain of locality in a lexicalized Tree Adjoining
Grammar causes problems when we consider the coordination of such
predicates. Consider~(\ex{1}) for instance, the NP {\em the beans that
  I bought from Alice} in the Right-Node Raising (RNR) construction
has to be shared by the two elementary trees (which are anchored by
{\em cooked} and {\em ate} respectively).

\enumsentence{(((Harry cooked) and (Mary ate)) the beans that I bought
  from Alice)}

We use the standard notion of coordination which is shown in
Figure~\ref{fig:conj} which maps two constituents of {\em like type},
but with different interpretations, into a constituent of the same
type.

\begin{figure}[htbp]
  \begin{center}
    \leavevmode
    \psfig{figure=ps/conj-files/conj.ps,height=1.0in}
  \end{center}
  \caption{Coordination schema}
  \label{fig:conj}
\end{figure}

We add a new operation to the LTAG formalism (in addition to
substitution and adjunction) called {\em conjoin} (later we discuss an
alternative which replaces this operation by the traditional
operations of substitution and adjunction). While substitution and
adjunction take two trees to give a derived tree, {\em conjoin\/}
takes three trees and composes them to give a derived tree.  One of
the trees is always the tree obtained by specializing the schema in
Figure~\ref{fig:conj} for a particular category. The tree obtained
will be a lexicalized tree, with the lexical anchor as the
conjunction: {\em and}, {\em but}, etc.

The conjoin operation then creates a {\em contraction\/} between nodes
in the contraction sets of the trees being coordinated.  The term {\em
  contraction\/} is taken from the graph-theoretic notion of edge
contraction. In a graph, when an edge joining two vertices is
contracted, the nodes are merged and the new vertex retains edges to
the union of the neighbors of the merged vertices. The conjoin
operation supplies a new edge between each corresponding node in the
contraction set and then contracts that edge.

For example, applying {\em conjoin\/} to the trees {\em Conj(and)},
$\alpha(eats)$ and $\alpha(drinks)$ gives us the derivation tree and
derived structure for the constituent in \ex{1} shown in
Figure~\ref{fig:vpc}.

\enumsentence{$\ldots$ eats cookies and drinks beer}

\begin{figure}[htbp]
  \begin{center}
    \leavevmode
    \psfig{figure=ps/conj-files/vpc.ps,height=2.0in}
  \end{center}
  \caption{An example of the {\em conjoin\/} operation. $\{1\}$
    denotes a shared dependency.}
  \label{fig:vpc}
\end{figure}

Another way of viewing the conjoin operation is as the construction of
an auxiliary structure from an elementary tree. For example, from the
elementary tree $\alpha(drinks)$, the conjoin operation would create
the auxiliary structure $\beta(drinks)$ shown in
Figure~\ref{fig:aux-conj}. The adjunction operation would now be
responsible for creating contractions between nodes in the contraction
sets of the two trees supplied to it. Such an approach is attractive
for two reasons. First, it uses only the traditional operations of
substitution and adjunction. Secondly, it treats {\em conj X} as a
kind of ``modifier'' on the left conjunct {\em X}. This approach
reduces some of the parsing ambiguities introduced by the predicative
coordination trees and forms the basis of the XTAG implementation.

\begin{figure}[htbp]
  \begin{center}
    \leavevmode
    \psfig{figure=ps/conj-files/aux-conj.ps,height=3.0in}
  \end{center}
  \caption{Coordination as adjunction.}
  \label{fig:aux-conj}
\end{figure}

More information about predicative coordination can be found in
(\cite{anoopjoshi96}), including an extension to handle gapping constructions.

\section{Pseudo-coordination}

The XTAG grammar does handle one sort of verb pseudo-coordination. 
Semi-idiomatic phrases such as 'try and' and 'up and' (as in 'they might 
try and come today') are handled as multi-anchor modifiers 
rather than as true coordination. These items adjoin to a V node, using 
the $\beta$VCONJv tree. This tree adjoins only to verbs in their base 
morphological (non-inflected) form. The verb anchor of the $\beta$VCONJv 
must also be in its base form, as shown in examples 
(\ex{1})-(\ex{3}). This blocks 3rd-person singular derivations, 
which are the only person morphologically marked in the present, except when 
an auxiliary verb is present or the verb is in the infinitive.

\enumsentence{$\ast$He tried and came yesterday.}
\enumsentence{They try and exercise three times a week.}
\enumsentence{He wants to try and sell the puppies.}


%\section{Other Conjunctions}

%The conjunction analysis described in the previous sections is
%designed to handle only the most straightforward cases of conjunction.
%Three types of conjunction that are not handled are:

%\begin{itemize}
%\item {\bf Incomplete Constituents:} The sentence 
%{\it John likes and Bill hates bananas} cannot be handled by the
%current XTAG grammar.  Since {\it likes} anchors a tree that needs
%both a subject noun phrase and an object noun phrase to be substituted
%in, the first conjunct would need have an unfilled substitution node
%after {\it John likes} for the sentence to parse.

%\item {\bf Verb Phrase Conjunction:} Since verbs anchor a tree with a root 
%node of type S and not VP, there is no straightforward way to implement verb
%phrase conjunction.  For example, in the sentence {\it John eats cookies and
%drinks beer}, there is no point in the derivation at which {\it eats cookies}
%and {\it drinks beer} are available as separate trees ready to be conjoined.
%They are both only subtrees in their respective S trees.  This could also be
%considered as a case of incomplete constituents, since {\it drinks beer} is
%missing a noun phrase.

%\item {\bf Gapping:}
%Sentences such as {\it John likes apples and Bill pears} are also not
%handled by the previous analysis.  These could also be considered as a case
%of incomplete constituents.
%\end{itemize}

%One grammar formalism that is capable of handling these types of 
%conjunction is Combinatory Categorial Grammar (CCG) (\cite{steedman90})
%which relies on a nonstandard notion of a constituent in order to accomplish
%this.  Proposals have been made (e.g. \cite{joshischabes91}),
%inspired by the CCG approach, to handle these problematic cases in the
%FB-LTAG formalism.  Unlike the CCG analysis, however, the traditional notion
%of constituents and phrase structure is maintained.  Such proposals are
%as of yet unimplemented.

%%test sentences 
%%I ran and found a Brickel bush
%% you and me and the whole world
%% hook and line and bait



