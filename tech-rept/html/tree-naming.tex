\chapter{Tree Naming conventions} 
\label{tree-naming} 
 
The various trees within the XTAG grammar are named more or less according to 
the following tree naming conventions.  Although these naming conventions are 
generally followed, there are occasional trees that do not strictly follow 
these conventions. 
 
\section{Tree Nodes} 
In the description of a tree, nodes are named as a pair $(l,s)$ 
(also represented as $l_s$ in the graphic representation of its structure 
and $l\_s$ in the feature equations description), 
where $l$ is the label or grammar symbol assigned to the node and $s$ is a 
subscript whose primary purpose is to make a node name unique for any given 
tree. Typical examples are: $S_r$, $VP$, $NP_0$, $NP_1$. Notice that the 
subscript may be empty as in $VP$. 
There are several conventions generally 
followed for the use of subscripts, as naming $S_r$ the node immediately 
dominating the subject position in a verb tree. 
We will not exhaustively describe them here, 
except a few, which are used for naming the trees. They 
should give the user an idea of what the tree is about. 
 
Anchors are generally assigned a null subscript, unless the tree has more than 
on anchor with the same label, in which case each receive numeric 
subscripts $1$, $2$, etc. The main consistency condition here is that 
these subscripts have to match the multi-anchor entries in the syntactic 
lexicon that select the tree. 
 
Arguments in verbal trees are assigned subscripts according to their 
thematic roles. The main 
idea is that the subscript for a certain argument should be preserved across 
the trees in the grammar, whenever it is seen that these trees are 
transformationally related (the arguments should be at the same position in 
the logical form). 
For instance, the subscript for the NP subject of 
a passive tree should be the same as for the NP in the corresponding 
declarative tree that has been passivized. 
Additionally, the general convention is that the subscript $0$ 
is assigned to the underlying subject of a tree, $1$ to the first object, 
and so on. Notable exceptions arise when a certain family is also used 
as a part of the subcategorization description of a class of verbs. For 
instance, dative verbs take two families: one having $NP_1$ 
and $PP_2$ as arguments (actually a multi-anchor tree where $P$ is an 
anchor); the other being the double object family, for the 
dative shift. In order to maintain the relation that the leftmost object 
in the dative shift tree is logically related to the $PP_2$, it also 
receives the subscript $2$. 
Another example is that 
subjects in the ergative family have the 
same subscript, $1$, as the object in the base transitive tree. 
 
When a verb argument is also the anchor of a tree, 
as in the predicative families, light verbs and some multi-anchor idioms, 
the projection of the argument as well as its own arguments (e.g. PP and NP 
for a P anchor,) 
should generally also take the subscript corresponding to its position w.r.t. 
to the verb, which will be generally diferent from the anchor that carries 
no subscript as a role. 
Trees for wh-moved objects have no subscript in the extracted site. 
The original subscript is used for the landing site. 
Relative clause trees on the other hand 
preserve the subscript in the original position and use 
the subscript $w$ for the 
landing site. 
 
\section{Tree Families} 
Tree families are named according to the basic declarative tree structure in 
the tree family (see section~\ref{family-trees}), but with a T as the first 
character instead of an $\alpha$ or $\beta$. 
 
\section{Trees within tree families} 
\label{family-trees} 
 
Each tree begins with either an $\alpha$ (alpha) or a $\beta$ (beta) symbol, 
indicating whether it is an initial or auxiliary tree, respectively.  Following 
an $\alpha$ or a $\beta$ the name may additionally contain one of the 
following depending on the family: 
 
\begin{description} 
\item\begin{tabular}{ll} 
E& trees in the Ergative family\\ 
R& trees in a Resultative family\\ 
RE& trees in a Resultative family for Ergatives\\ 
X&ECM trees (eXceptional case marking)\\ 
\end{tabular} 
\end{description} 
 
\noindent Next, the name may contain one of the following, 
the digit corresponding to the subscript of the node moved in the 
tree. In Nc, Npx the absence of digit means relativized adjunct. 
 
\begin{description} 
\item\begin{tabular}{ll} 
I&imperative\\ 
W0,W1,W2&wh-NP extraction\\ 
pW0,pW1,pW2&wh-PP extraction\\ 
N0,N1,N2&relative clause, NP argument relativized, wh-word \\ 
Nc,Nc0,Nc1,Nc2&relative clause, NP argument relativized, no wh-word \\ 
Npx,Npx1,Npx2&relative clause, PP relativized \\ 
Nby& relative clause, by-clause relativized in passive constructions\\ 
G&NP gerund\\ 
D&Determiner gerund\\ 
Inv&Inverted arguments (for equative BE and It-clefts) 
\end{tabular} 
\end{description} 
 
% \noindent Numbers are assigned according to the position of the argument in the 
% declarative tree, as follows: 
 
% \begin{description} 
% \item\begin{tabular}{ll} 
% 0&subject position\\ 
% 1&first argument (e.g. direct object)\\ 
% 2&second argument (e.g. indirect object)\\ 
% \end{tabular} 
% \end{description} 
 
% \noindent The body of the name consists of a string of the following 
% components, which corresponds to the leaves of the tree.  The anchor(s) of the 
% trees is(are) indicated by capitalizing the part of speech corresponding to the 
% anchor. 
 
\noindent The rest of the name consists of a string where each component 
correspond to one leaf of the tree from the left to right. The formation 
of a component is as follows: start with one of the elements in the table 
below that corresponds to the leaf being translated: in lower case if the 
node is a substitution or foot node; 
 or upper case if it is an anchor. Then add 
``x'' if the node is a projection (or ``X'' if an anchor and a projection). 
Finally add the subscript at the node if any. 
Notice that empty elements ($\epsilon$) are generally ignored and their 
dominating node is used instead, except in the case of $PRO$, which by the 
way is capitalized. 
 
\begin{description} 
\item\begin{tabular}{ll} 
s&sentence\\ 
a&adjective\\ 
arb&adverb\\ 
be&{\it be}\\ 
% c&relative complementizer\\ 
% x&phrasal category\\ 
d&determiner\\ 
v&verb\\ 
lv&light verb\\ 
conj&conjunction\\ 
comp&complementizer\\ 
it&{\it it}\\ 
n&noun\\ 
p&preposition\\ 
PRO&a PRO subject \\ 
% to&{\it to}\\ 
pl&particle\\ 
by&{\it by}\\ 
neg&negation\\ 
\end{tabular} 
\end{description} 
 
\noindent As an example, the transitive declarative tree consists of a subject 
NP, followed by a verb (which is the anchor), followed by the object NP.  This 
translates into $\alpha$nx0Vnx1.  If the subject NP had been extracted, then 
the tree would be $\alpha$W0nx0Vnx1.  A passive tree with the {\it by} phrase 
in the same tree family would be $\alpha$nx1Vbynx0.  Note that even though the 
object NP has moved to the subject position, it retains the object encoding 
(nx1). 
 
\section{Assorted Initial Trees} 
 
Trees that are not part of the tree families are generally gathered into 
several files for convenience.  The various initial trees are located in {\tt lex.trees}.  All the trees in this file should begin with an $\alpha$, 
indicating that they are initial trees.  This is followed by the root category 
which follows the naming conventions in the previous section (e.g. n for noun, 
x for phrasal category).  The root category is in all capital letters.  After 
the root category, the node leaves are named, beginning from the left, with the 
anchor of the tree also being capitalized.  As an example, the $\alpha$NXN 
tree is rooted by an NP node (NX) and anchored by a noun (N). 
 
\section{Assorted Auxiliary Trees} 
 
The auxiliary trees are mostly located in the buffers {\tt prepositions.trees}, {\tt conjunctions.trees}, {\tt determiners.trees}, {\tt advs-adjs.trees}, and {\tt modifiers.trees}, 
although a couple of other files also contain auxiliary trees.  The 
auxiliary trees follow a slightly different naming convention from the 
initial trees.  Since the root and foot nodes must be the same for the 
auxiliary trees, the root nodes are not explicitly mentioned in the 
names of auxiliary trees.  The trees are named according to the leaf 
nodes, starting from the left, and capitalizing the anchor node.  All 
auxiliary trees begin with a $\beta$, of course.  For example, 
$\beta$ARBs, indicates a tree anchored by an adverb (ARB), that 
adjoins onto the left of an S node (Note that S must be the foot node, 
and therefore also the root node). 
 
% \subsection{Relative Clause Trees} 
% For relative clause trees, the following naming conventions have been 
% adopted: if the {\em wh}-moved NP is overt, it is not explicitly 
% represented. Instead the index of the site of movement 
% (0 for subject, 1 for object, 2 for indirect object) is appended to the 
% N. So $\beta$N0nx0Vnx1 is a subject 
% extraction relative clause with {\bf NP$_{w}$} substitution 
% and $\beta$N1nx0Vnx1 is an object extraction 
% relative clause. If the {\em wh}-moved NP is covert and Comp substitutes 
% in, the Comp node is represented by {\em c} in the tree name and the 
% index of the extraction site follows {\em c}. Thus 
% $\beta$Nc0nx0Vnx1 is a subject extraction 
% relative clause with Comp substitution. Adjunct trees are similar, except 
% that since the extracted material is not co-indexed to a trace, no index 
% is specified (cf. $\beta$Npxnx0Vnx1, which is an adjunct relative clause with 
% PP pied-piping, and $\beta$Ncnx0Vnx1, which is an adjunct relative clause 
% with Comp substitution). Cases of pied-piping, in which the pied-piped 
% material is part of the anchor have the anchor capitalized or spelled-out 
% (cf. $\beta$Nbynx0nx1Vbynx0 which is a relative clause with {\em by}-phrase 
% pied-piping and {\bf NP$_{w}$} substitution.). 
 
