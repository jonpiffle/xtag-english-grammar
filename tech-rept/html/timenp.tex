\section{Time Noun Phrases} 
\label{timenps} 
 
%% \subsection{Introduction} 
 
 
Although in general NPs cannot modify clauses, there is a class of 
NPs, with meanings that relate to time, that can do so.\footnote{   There may be other classes of NPs, such as directional phrases, such   as {\em north, south} etc., which behave similarly. We have not yet   analysed these phrases.} We call this class of NPs ``Time~NPs''. 
Time~NPs behave essentially like PPs. Like PPs, time~NPs can adjoin at 
four places: to the right of an NP, to the right and left of a VP, and 
to the left of an~S. 
 
Time~NPs may include determiners, as in {\em this month} in example 
(\ref{ex:630}), or may be single lexical items as in {\em today} in example 
(\ref{ex:631}).  Like other NPs, time~NPs can also include adjectives, as in 
example (\ref{ex:635}). 
 
%% \enumsentence{I go there {\ul every month}} 
%% \enumsentence{I am free {\ul today}} 
\beginsentences
\sitem{Elvis left the building \underline{this week}}\label{ex:630} 
\sitem{Elvis left the building \underline{today}}\label{ex:631} 
\sitem{It has no bearing on our work force \underline{today} (WSJ)}\label{ex:632} 
\sitem{The fire \underline{yesterday} claimed two lives}\label{ex:633} 
\endsentences

%% In early trading in Tokyo Tuesday, the Nikkei index rose 35.28 
%% points to 35452.72. 
%% \enumsentence{It has no bearing \underline{today} on our work force} 
\beginsentences
\sitem{\underline{Today} it has no bearing on our work force}\label{ex:634} 
\sitem{Michael \underline{late yesterday} announced a buy-back program}\label{ex:635} 
\endsentences

%% Michael late yesterday announced a $ 3.8 million stock buy-back program . 
 
The XTAG analysis for time~NPs is fairly simple, requiring only the 
creation of proper NP auxiliary trees.  Only nouns that can be part of 
time~NPs will select the relevant auxiliary trees, and so only these 
type of NPs will behave like PPs under the XTAG analysis. 
Currently, about 60 words select Time~NP trees, but since these 
words can form NPs that include determiners and adjectives, a large 
number of phrases are covered by this class of modifiers. 
 
Corresponding to the four positions listed above, time~NPs 
can select one of the four trees shown in Figure~\ref{timenp-trees}. 
 
\begin{rawhtml} <p> \end{rawhtml}
\centering 
\begin{tabular}{ccccccc} 
{\htmladdimg{ps/timenp-files/betaNs.ps.gif}} 
& \hspace{.5in} & 
{\htmladdimg{ps/timenp-files/betaNvx.ps.gif}} 
&  \hspace{.5in} & 
{\htmladdimg{ps/timenp-files/betavxN.ps.gif}} 
&  \hspace{.5in} & 
{\htmladdimg{ps/timenp-files//betanxN.ps.gif}}\\ 
$\beta$Ns&&$\beta$Nvx&&$\beta$vxN&&$\beta$nxN\\ 
\end{tabular} 
\begin{rawhtml} <dl> <dt>{Time Phrase Modifier trees: $\beta$Ns, $\beta$Nvx, $\beta$vxN, $\beta$nxN <p> </dl> \end{rawhtml}
\label{timenp-trees} 
\begin{rawhtml} <p> \end{rawhtml}
 
Determiners can be added to time~NPs by adjunction in 
the same way that they are added to NPs in other 
positions. The trees in Figure~\ref{everymonth} show that the 
structures of examples (\ref{ex:630}) and (\ref{ex:631}) differ only in the 
adjunction of  {\em this} to the time~NP in example (\ref{ex:630}). 
 
\begin{rawhtml} <p> \end{rawhtml}
\centering 
\begin{tabular}{ccc} 
\htmladdimg{ps/timenp-files/elvis-thisweek.ps.gif} 
& \hspace{.5in} & 
\htmladdimg{ps/timenp-files/elvis-today.ps.gif} \\ 
\end{tabular}\\ 
\begin{rawhtml} <dl> <dt>{Time~NPs with and without a determiner <p> </dl> \end{rawhtml}
\label{everymonth} 
\begin{rawhtml} <p> \end{rawhtml}
 
\newpage 
 
The sentence 
\beginsentences
\sitem{Esso said the Whiting field started production Tuesday (WSJ)}\label{ex:636} 
\endsentences

has (at least) two different interpretations, depending on whether 
{\em Tuesday} attaches to {\em said} or to {\em started}. 
Valid time~NP analyses are available for both these interpretations and 
are shown in Figure~\ref{esso}. 
 
\begin{rawhtml} <p> \end{rawhtml}
{\htmladdimg{ps/timenp-files/EssoSaidTuesday.ps.gif}} & \hspace{.5in} & 
{\htmladdimg{ps/timenp-files/EssoStartedTuesday.ps.gif}} \\ \end{tabular}\\ 
\begin{rawhtml} <dl> <dt>{Time~NP trees: Two different attachments \label{esso <p> </dl> \end{rawhtml}
\begin{rawhtml} <p> \end{rawhtml}
 
Example (\ref{ex:636}) shows that there are cases of genuine ambiguity that will 
be properly represented with multiple valid parses under our 
analysis. 
 
Derived tree structures for examples (\ref{ex:632}) -- (\ref{ex:635}), which 
show the four possible time~NP positions are shown in 
Figures~\ref{bearingtrees} and \ref{lateyesterday}.  The derivation 
tree for example (\ref{ex:635}) is also shown in 
Figure~\ref{lateyesterday}. 
 
\begin{rawhtml} <p> \end{rawhtml}
{\htmladdimg{ps/timenp-files/bearingENDtoday.ps.gif}} & \hspace{.1in} & 
{\htmladdimg{ps/timenp-files/thefireyesterday.ps.gif}} &  \hspace{.1in} & 
%% {\htmladdimg{ps/timenp-files/bearingtoday.ps.gif}} &  \hspace{.1in} & 
{\htmladdimg{ps/timenp-files/todaybearing.ps.gif}} \\ \end{tabular}\\ 
\begin{rawhtml} <dl> <dt>{Time~NPs in different positions ($\beta$vxN, $\beta$nxN and $\beta$Ns) \label {bearingtrees <p> </dl> \end{rawhtml}
\begin{rawhtml} <p> \end{rawhtml}
 
\begin{rawhtml} <p> \end{rawhtml}
\centering 
\begin{tabular}{ccc} 
\htmladdimg{ps/timenp-files/lateyesterday.ps.gif} & 
\hspace{.02in} & 
\htmladdimg{ps/timenp-files/DERIVlateyesterday.ps.gif} \\ 
\end{tabular} 
\begin{rawhtml} <dl> <dt>{Time~NPs: Derived tree and Derivation ($\beta$Nvx position) <p> </dl> \end{rawhtml}
\label{lateyesterday} 
\begin{rawhtml} <p> \end{rawhtml}
 
