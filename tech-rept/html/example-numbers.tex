%%% Robert Rubinoff - Mar 3 1986 
%%% sentences - an environment for producing numbered sentence example lists. 
%%% you enter it with \beginsentences 
%%% finish with \endsentences 
\beginsentences
%%%     (1) This kind is produced by \sitem 
\endsentences

%%%     (2)i. This kind is produced by \smainitem 
%%%       ii. This kind (i.e. subsequent subitems) is produced by \ssubitem 
%%%  The sentences are numbered using \sentencectr and \sentencesubctr 
%%% 
%%% some formatting control provided via: 
%%%     \smainform - controls format of main number - default is \arabic 
%%%     \ssubform  - controls format of sub number - default is \roman 
%%%     \ssubpunc  - controls punctuation after sub number - default is "." 
%%% these can be changed via \renewcommand 
%%% 
%%% you can also generate cross-reference labels via \label; this gives 
%%% you the main counter and (when appropriate) the subcounter.  To get 
%%% a label of just the main item number when in a \smainitem or \ssubitem, 
%%% use \smainlabel; to get just the sublabel, use \ssublabel 
 
\newcounter{sentencectr} 
\newcounter{sentencesubctr} 
 
\renewcommand{\thesentencectr}{(\smainform{sentencectr})} 
\renewcommand{\thesentencesubctr}{\thesentencectr\ssubform{sentencesubctr}} 
 
\newcommand{\smainform}{\arabic} 
\newcommand{\ssubform}{\alph} 
\newcommand{\ssubpunc}{.{}} 
 
\newcommand{\beginsentences}{ % \pagebreak[3] % \begin{list}{(\thesentencectr)}    {\usecounter{sentencesubctr} %The next line controls how much space to put around examples.  The %value I have here is for tight spaces.  Take it out when things %aren't so desparate. %    \setlength{\topsep}{0ex}			 %This following line is 0.5ex by default.  The 1ex spacing looks good in single %spaced examples,  How about in space and a half?     \setlength{\topsep}{1ex}			     \setlength{\itemsep}{0 in}     \setlength{\labelwidth}{0.5 in} %Previous line increases width so we don't get problem with indented %subitems once we have 2 digit example numbers. -- bf %    \addtolength{\leftmargin}{25 pt} %    \setlength{\leftmargin}{.6in} % This next line makes indentation of examples the same as in LI which % looks pretty good, I think. %    \addtolength{\leftmargin}{8ex}     \addtolength{\leftmargin}{4ex}     \setlength{\labelsep}{.05in}     \setlength{\parsep}{0 in}}} 
\def\endsentences{\end{list}} 
 
\beginsentences
\newcommand{\sitem}{\renewcommand{\thesentencesubctr}{(\smainform{sentencectr}}                     \refstepcounter{sentencectr}      \item[(\smainform{sentencectr})\hfill]} 
\endsentences

\newcommand{\smainitem}{\renewcommand{\thesentencesubctr                                     }{\thesentencectr\ssubform{sentencesubctr}}                         \setcounter{sentencesubctr}{0}                         \refstepcounter{sentencectr}                         \refstepcounter{sentencesubctr}      \item[\thesentencectr\hfill\ssubform{sentencesubctr}\ssubpunc]} 
\newcommand{\ssubitem}{\refstepcounter{sentencesubctr}      \item[\hfill\ssubform{sentencesubctr}\ssubpunc]} 
 
\makeatletter            
\newcommand{\smainlabel}[1]{{%  the extra braces make the change local \renewcommand{\@currentlabel}{\thesentencectr}\label{#1}}} 
 
\newcommand{\ssublabel}[1]{{%  the extra braces make the change local \renewcommand{\@currentlabel}{\ssubform{sentencesubctr}}\label{#1}}} 
\makeatother 
 
 
 
\makeatother 
 
