\chapter{PP Complement Verbs} 
\label{pp-complement-chapter} 
 
Just as some verbs require noun phrase complements, others require 
preposition phrase complements.  For an intransitive verb like {\it sleep}, a PP which follows it is an adjunct, as can be seen in 
(\ref{ex:381}) and (\ref{ex:382}), but for a verb such as {\it venture}, the PP is 
required, as the ungrammaticality of (\ref{ex:384}) attests. 
 
\beginsentences
\sitem{The spelunker slept in the cave .}\label{ex:381} 
\sitem{The spelunker slept .}\label{ex:382} 
\sitem{The spelunker ventured into the cave .}\label{ex:383} 
\sitem{*The spelunker ventured .}\label{ex:384} 
\endsentences

 
We will call verbs such as {\it venture} transitive PP 
complement verbs.  There are also ditransitive PP complement 
verbs, such as {\it put}. 
 
\beginsentences
\sitem{Bill put the cigar on the table .}\label{ex:385} 
\sitem{*Bill put the cigar .}\label{ex:386} 
\endsentences

 
Since the PP in examples like (\ref{ex:383}) and (\ref{ex:385}) is an argument of the 
verb, PP complement verbs such as {\it venture} and {\it put} must provide 
a substitution site for the PP. There are two possible structures for 
allowing this in the grammar: the PP substitution node structure (shown in 
Figure~\ref{two-pp-comp-analyses}a) and the expanded PP structure (shown in 
Figure~\ref{two-pp-comp-analyses}b). 
 
\begin{rawhtml} <p> \end{rawhtml}
\centering 
\begin{tabular}{ccc} 
{\htmladdimg{ps/pp-complement-files/PP-subst.ps.gif}}  & 
\hspace{0.6in} 
{\htmladdimg{ps/pp-complement-files/PP-expanded.ps.gif}} \\ 
(a) PP substitition& \qquad(b) Expanded PP \\ 
\end{tabular}\\ 
\begin{rawhtml} <dl> <dt>{Two analyses of the PP complement <p> </dl> \end{rawhtml}
\label{two-pp-comp-analyses} 
\begin{rawhtml} <p> \end{rawhtml}
 
The structure chosen is that of the expanded PP in 
Figure~\ref{two-pp-comp-analyses}b. The choice was motivated by two 
reasons. The first is that it is possible to extract the NP object of the 
P, as shown in (\ref{ex:387}), and expanding the PP allows the NP node to be 
available to metarules which create the trees for extraction. 
 
\beginsentences
\sitem{What did Bill put the cigar on ?}\label{ex:387} 
\endsentences

 
The second reason has to do with the inherent semantics of the elementary 
trees of such verbs. Currently, we treat the verb as taking the NP, the 
object of the preposition, as its argument rather than the whole PP. In 
order to have access to this NP for the semantic representation, it must be 
explicitly represented in the elementary tree. 
 
However, the current XTAG grammar does not parse sentences like (\ref{ex:388}) in 
which the verb takes exhaustive PP's like {\it there} (and also {\it somewhere}, {\it where} etc.) as its argument. 
 
\beginsentences
\sitem{Bill put the cigar there .}\label{ex:388} 
\endsentences

 
Since the existing analysis does not allow an exhaustive PP substitution, 
these cases are currently a problem. 
 
There is another class of PP complement verbs which make it necessary to 
expand the PP structure. These are the verbs which require a particular 
preposition to head their PP complement, as illustrated in the following 
examples: 
 
\beginsentences
\sitem{Calvin depends on Hobbes .}\label{ex:389} 
\sitem{*Calvin depends with Hobbes .}\label{ex:390} 
\sitem{Calvin thought of the transmogrifier .}\label{ex:391} 
\sitem{*Calvin thought on the transmogrifier .\footnote{This is fine as an adjunct, but not as a complement.}}\label{ex:392} 
\sitem{The expedition leader geared the hikers for the trip .}\label{ex:393} 
\sitem{*The expedition leader geared the hikers into the trip .}\label{ex:394} 
\sitem{The speaker subjected the audience to his dull wit .}\label{ex:395} 
\endsentences

 
For such verbs, the preposition is intrinsic to the construction.  Note 
that sentence (\ref{ex:391}) is ambiguous between two interpretations, one in 
which the PP is an adjunct and the other in which it is a complement.  The 
adjunct interpretation of this sentence is something like ``Calvin was 
thinking and the topic of this thought was the transmogrifier'' while the 
complement interpretation is ``Calvin created the idea for the 
transmogrifier''.  The preposition correlated with the verb thus conveys a 
noncompositional meaning, which motivates us to analyze these constructions 
as being multiply-anchored by a verb and a corresponding preposition. 
Having the P as one of the anchors necessitates the expansion of the PP 
complement for these constructions. 
 
We have thus determined one important motivation for the suggested basic 
structure of PP complement verbs.  Another important aspect of these verbs 
is that, contrary to cases where verbs take an NP complement, certain 
adjuncts are found to occur relatively freely between the verb and the PP 
complement. NP complements cannot be separated from the verb by modifiers 
unless they are heavy, as shown in examples (\ref{ex:396}-\ref{ex:399}). 
 
\beginsentences
\sitem{*Bill borrowed quickly the car .}\label{ex:396} 
\sitem{*Bill borrowed for many days the car .}\label{ex:397} 
\sitem{Bill borrowed for many days the car which his father had bought from his uncle two weeks ago .}\label{ex:398} 
\sitem{*John gave Mary often the book .}\label{ex:399} 
\endsentences

 
PP complements, however, allow several types of adjuncts in this position. 
For example, adverbs (\ref{ex:400}-\ref{ex:402}), certain prepositional phrases 
(\ref{ex:403}-\ref{ex:405}), and nominal adjuncts (\ref{ex:406}-\ref{ex:408}) are able to intercede 
the verb and the preposition: 
 
\beginsentences
\sitem{John depends heavily on his staff .}\label{ex:400} 
\sitem{John thinks often of new ideas .}\label{ex:401} 
\sitem{John bases his opinions extensively on public attitudes .}\label{ex:402} 
\endsentences

 
\beginsentences
\sitem{John depends in many ways on his staff .}\label{ex:403} 
\sitem{John thought for many hours of new ideas .}\label{ex:404} 
\sitem{John based his opinions for more than ten years on public attitudes .}\label{ex:405} 
\endsentences

 
\beginsentences
\sitem{John depends quite a bit on his staff .}\label{ex:406} 
\sitem{John thought all the time of new ideas .}\label{ex:407} 
\sitem{John bases his opinions a good deal on public attitudes .}\label{ex:408} 
\endsentences

 
Some restrictions, however, do occur. The locative PP's in \ref{ex:409}-\ref{ex:411} 
and the adjunct clauses in \ref{ex:412}-\ref{ex:414} cannot intervene between the verb 
and its PP complement. 
 
\beginsentences
\sitem{*John depends from Honolulu on his staff .}\label{ex:409} 
\sitem{*John thought in the kitchen of new ideas .}\label{ex:410} 
\sitem{*John based his opinions in Honolulu on public attitudes .}\label{ex:411} 
\endsentences

 
\beginsentences
\sitem{*John depends because he needs help on his wonderful and caring staff .}\label{ex:412} 
\sitem{*John thought because he was so creative of new ideas that provided the basis of many present technologies .}\label{ex:413} 
\sitem{*John bases his opinions because he cannot think for himself on the public attitudes that prevail today .}\label{ex:414} 
\endsentences

 
There are several ways in which we could handle these adjuncts.  One of 
these is to allow only certain adjuncts to adjoin to V.  However, this is 
unappealing because it would cause right-edge ambiguities (V vs. VP 
adjunction), and also because it cannot handle examples like (\ref{ex:402}), 
(\ref{ex:405}), (\ref{ex:408}), and (\ref{ex:411}) where there is an NP between the verb 
and the preposition. 
 
Another option is to make the claim that the adjuncts get into the 
intervening position because of heavy-shift. With such an analysis, the 
structures of the trees for prepositional complements would not provide a 
site for adjunction between the verb and the preposition.  However, there 
is no sense of heaviness in the examples (\ref{ex:400}-\ref{ex:411}), and we would 
expect (\ref{ex:412}),(\ref{ex:413}), and (\ref{ex:414}) to be fine since the PP's are 
quite heavy here. 
 
The analysis we provide is structural, and is given in 
Figure~\ref{pp-comp-trees} for the multi-anchor PP complement verbs.  The 
analysis suggests short verb movement, and is reminiscent of Larsonian 
shells. 
%Tonia: add a reference 
However, the most important aspect of these trees for the purposes 
of the English XTAG grammar is that it provides a site for left adjunction 
to VP that intervenes between the verb and the preposition.  This 
immediately rules out the adjunct clauses as desired, since such clauses 
only right adjoin to VP in the grammar.  Most prepositional phrases are not 
good in the intervening position either, and this fits in nicely with the 
fact that the grammar does not have left VP adjoining prepositional 
phrases.  For the restricted class of prepositions which are permitted in 
the intervening position, the grammar currently does not parse them, but 
entries which allow such left adjuction can be easily added later. 
Adverbs, however, already left-adjoin to VP's in the current grammar, so 
the structure accomodates this without need of any further modification. 
 
The comparatives also provide motivation for the structure that we have 
adopted. Comparative adverb phrases such as {\it more heavily} may either 
left- or right-adjoin, and can thus adjoin in the 
intervening position.  However, the {\it than}-clause portion of a 
comparative can only right adjoin. Together, these two adjoining 
behaviors work with our PP complement analysis to correctly capture the 
distinctions seen in the following examples: 
 
\beginsentences
\sitem{John depends more heavily on his staff than Mary .}\label{ex:415} 
\sitem{John depends on his staff more heavily than Mary .}\label{ex:416} 
\sitem{*John depends more heavily than Mary on his staff .}\label{ex:417} 
\endsentences

 
In (\ref{ex:415}), the comparative adverb phrase has left-adjoined to the lower VP 
and the {\it than}-clause has right adjoined to it.  In (\ref{ex:416}), both are 
right adjoined.  The ungrammaticality of (\ref{ex:417}) is captured by the fact that 
the {\it than}-clause can only right-adjoin, and there is no right-adjoinable 
VP node available between the verb and the preposition. 
 
\begin{rawhtml} <p> \end{rawhtml}
\centering 
\begin{tabular}{ccc} 
{\htmladdimg{ps/pp-complement-files/alphanx0VPnx1.ps.gif}}  & 
\hspace{0.6in} 
{\htmladdimg{ps/pp-complement-files/alphanx0Vnx1Pnx2.ps.gif}} \\ 
(a) $\alpha$nx0VPnx1 & \qquad(b) $\alpha$nx0Vnx1Pnx2\\ 
\end{tabular}\\ 
\begin{rawhtml} <dl> <dt>{Declarative trees for multi-anchor PP complement families <p> </dl> \end{rawhtml}
\label{pp-comp-trees} 
\begin{rawhtml} <p> \end{rawhtml}
 
This basic schema of short verb movement for PP complement families 
captures the adjunction possibilities directly and appears to partition 
the different types of adjoining elements in the appropriate way, with 
very little modification to the grammar. The analysis is operative 
in the families Tnx0Vpnx1, Tnx0Vnx1pnx2, Tnx0VPnx1, and Tnx0Vnx1Pnx2. 
 
Notice that having two VP nodes along the right spine of the trees will 
engender right edge ambiguities. In other structures where this ambiguity 
arises (for example, sentences with auxiliaries), VP right-adjoining modifiers 
are restricted by the feature {\bf mainv = +} on their footnodes to avoid 
right-edge ambiguities. This restriction is extended to the present case where 
right VP adjunction is allowed to occur only on the higher VP by giving the 
lower VP the feature {\bf mainv = -}. 
 
 
