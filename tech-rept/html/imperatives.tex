 
\chapter{Imperatives} 
\label{imperatives} 
 
\section{Agreement, mode, and the optional subject} 
 
Imperatives in English do not require overt subjects.  The subject in 
imperatives is in general second person, i.e.\ {\it you}, whether it is 
overt or not, as is clear from the verbal agreement and the interpretation. 
The imperatives in which the subject is not overt are handled by the 
imperative trees discussed in this section.  Imperatives with overt 
subjects are not handled currently.  More discussion on imperatives with 
overt subjects is given in Sections \ref{sec:vocative} and 
\ref{sec:overt-subject}. 
 
The imperative trees in each tree family in the English XTAG grammar are 
identical to the declarative tree of that family except that the NP$_{0}$ 
subject position is filled by an $\epsilon$, the NP$_{0}$ {\bf $<$agr~pers$>$} feature is set to the value {\bf 2nd} and the {\bf $<$mode$>$} feature on the root node has the value {\bf imp} (see equations 
\ref{ex:291} -- \ref{ex:292}). Hardwiring the {\bf $<$agr~pers$>$} feature into the 
tree ensures the proper verbal agreement for an imperative.  The {\bf $<$mode$>$} value of {\bf imp} on the root node is recognized as a valid 
mode for a matrix clause.\footnote{% % The other valid {\bf $<$mode$>$} for a matrix clause is {\bf ind}.% % } The {\bf imp} value for {\bf $<$mode$>$} also prevents imperatives from 
appearing as embedded clauses.  Figure \ref{alphaInx0Vnx1} shows the 
imperative tree in the transitive tree family. 
 
\beginsentences
\sitem{{\bf NP$_0$.t:$<$agr~pers$>$ = 2}}\label{ex:291} 
\sitem{{\bf S$_r$.b:$<$mode$>$ = imp}}\label{ex:292} 
\endsentences

 
 
\begin{rawhtml} <p> \end{rawhtml}
\centering{ \begin{tabular}{c} \htmladdimg{ps/imperatives-files/alphaInx0Vnx1.ps.gif} \end{tabular} } 
\begin{rawhtml} <dl> <dt>{Transitive imperative tree: $\alpha$Inx0Vnx1 <p> </dl> \end{rawhtml}
\label{alphaInx0Vnx1} 
\label{2;11,1} 
\begin{rawhtml} <p> \end{rawhtml}
 
 
Moreover, the {\bf $<$mode$>$} feature on the anchor is unspecified, and 
the {\bf $<$mode$>$} feature on the top feature structure associated with 
the VP has the value {\bf base} (see equation in \ref{ex:293}). 
 
\beginsentences
\sitem{{\bf VP.t:$<$mode$>$ = base}}\label{ex:293} 
\endsentences

 
This allows the lexical verb of the imperative to be any type of verbs, as 
long as the left-most verb has {\bf $<$mode$>$ = base}.  For instance, in a 
simple transitive imperative as in \ref{ex:294}, the verb {\it eat}, which is 
specified with {\bf $<$mode$>$ = base}, anchors the imperative tree, 
unifying with {\bf VP.t:$<$mode$>$ = base}.  In an imperative with 
auxiliary {\it be} as in \ref{ex:295}, the verb {\it waiting}, which is specified 
with {\bf $<$mode$>$ = ger}, anchors the imperative tree, and the auxiliary 
{\it be}, which is specified with {\bf $<$mode$>$ = base}, adjoins onto the 
VP, unifying with {\bf VP.t:$<$mode$>$ = base}. 
 
\beginsentences
\sitem{Eat the cake!}\label{ex:294} 
\sitem{Be waiting for me!}\label{ex:295} 
\endsentences

 
 
\section{Negative Imperatives} 
\label{neg-imp} 
 
\subsection{{\it Don't} imperatives} 
 
All Negative imperatives in English require {\it do}-support, even those 
that are formed with {\it be} and auxiliary {\it have}. 
 
\beginsentences
\sitem{Dont' leave!}\label{ex:296} 
\sitem{*Not leave!}\label{ex:297} 
\endsentences

 
\beginsentences
\sitem{Do not open the window!}\label{ex:298} 
\sitem{*Not open the window!}\label{ex:299} 
\endsentences

 
\beginsentences
\sitem{Do not be talking so loud!}\label{ex:300} 
\sitem{*Not be talking so loud!}\label{ex:301} 
\endsentences

 
\beginsentences
\sitem{Don't have eaten everything before the guests arrive!}\label{ex:302} 
\sitem{*Not have eaten everything before the guests arrive!}\label{ex:303} 
\endsentences

 
In English XTAG grammar, negative imperatives receive a similar structural analysis 
to {\it yes-no} questions, as in \cite{potsdamdiss97} and \cite{handiss}. 
That is, {\it do} and {\it don't} in negative imperatives are treated as an 
instance of {\it do}-support and adjoin to a clause.  The crucial 
strucural evidence for our analysis is that when there is an overt subject 
in negative imperatives formed with {\it don't}, the subject must follow 
{\it don't}, just as it does in {\it yes-no} questions. 
 
\beginsentences
\sitem{Don't you worry!}\label{ex:304} 
\sitem{Don't you move!}\label{ex:305} 
\endsentences

 
\beginsentences
\sitem{Don't you like carrots?}\label{ex:306} 
\sitem{Didn't you finish your paper yet?}\label{ex:307} 
\endsentences

 
{\it Do}-support in negative imperatives is handled by the elementary tree 
$\beta$IVs anchored by {\it do} and {\it don't}, as shown in Figure 
\ref{fig:doimp}.  This tree adjoins onto the root node of the imperative 
tree.  The feature {\bf $<$mode$>$ =imp} on the S foot node restricts this 
tree to adjoin only to imperative trees. Furthermore, the S root node of 
$\beta$IVs is specified with {\bf $<$mode$>$ =imp}, which prevents 
imperatives with {\it do}-support from appearing as embedded clauses. 
 
\begin{rawhtml} <p> \end{rawhtml}
\centering 
\begin{tabular}{ccc} 
{\htmladdimg{ps/imperatives-files/betaIVs-do.ps.gif}} & 
{\ } & 
{\htmladdimg{ps/imperatives-files/betaIVs-dont.ps.gif}} \\ 
$\beta$IVs[do] & {\ } & $\beta$IVs[don't] 
\end{tabular} 
\begin{rawhtml} <dl> <dt>{Trees anchored by  do and  don't <p> </dl> \end{rawhtml}
\label{fig:doimp} 
\begin{rawhtml} <p> \end{rawhtml}
 
In negative imperatives formed with {\it don't}, the $\beta$IVs[don't] tree in 
Figure \ref{fig:doimp} adjoins to the root node of the imperative tree. The 
derived tree for the negative imperative {\it Don't leave!} is given in 
Figure \ref{fig:dont-leave}. 
 
\begin{rawhtml} <p> \end{rawhtml}
  \begin{center} \leavevmode \htmladdimg{ps/imperatives-files/dont-leave.ps.gif} 
  \end{center} 
  \begin{rawhtml} <dl> <dt>{Derived tree for  Don't leave! <p> </dl> \end{rawhtml}
\label{fig:dont-leave} 
\begin{rawhtml} <p> \end{rawhtml}
 
\subsection{{\it Do not} imperatives} 
 
In negative imperatives formed with {\it do not}, the $\beta$IVs[do] tree 
in Figure \ref{fig:doimp} adjoins to the root node of the imperative tree 
and the $\beta$NEGvx tree that anchors {\it not} as represented in Figure 
\ref{fig:not} adjoins to the VP node of the imperative tree. 
 
\begin{rawhtml} <p> \end{rawhtml}
  \begin{center} \leavevmode 
\htmladdimg{ps/imperatives-files/betaNEGvx-not.ps.gif} 
  \end{center} 
  \begin{rawhtml} <dl> <dt>{Tree anchored by  not <p> </dl> \end{rawhtml}
\label{fig:not} 
\begin{rawhtml} <p> \end{rawhtml}
 
The  derived tree for the negative imperative {\it Do not eat the cake!} are 
given in Figure \ref{fig:do-not-leave}. 
 
\begin{rawhtml} <p> \end{rawhtml}
\begin{center} \leavevmode 
{\htmladdimg{ps/imperatives-files/do-not-leave1.ps.gif}} 
\end{center} 
\begin{rawhtml} <dl> <dt>{Derived trees for  Do not eat the cake! <p> </dl> \end{rawhtml}
\label{fig:do-not-leave} 
\begin{rawhtml} <p> \end{rawhtml}
 
Note that trees in Figure \ref{fig:dont-leave} and Figure 
\ref{fig:do-not-leave} have an empty verb.  This is due to the feature {\bf $<$displ-const$>$} in $\beta$IVs.\footnote{% % The other possibility of adjoining {\it not} above the VP that projects from the empty verb is ruled out because {\it not} is made to select a VP with the following equation: {\bf $<$displ-const set1=-$>$}.  Since the empty verb tree has {\bf $<$displ-const set1=+$>$}, {\it not} cannot adjoin onto the VP that projects from it.% % } This feature ensures that when $\beta$IVs is adjoined to an elementary 
tree, $\beta$Vvx that anchors an empty verb must also adjoin onto the VP of 
that same elementary tree.  This tree is represented in Figure 
\ref{fig:epsilon}.  The empty verb represents the originating position of 
{\it do} and {\it don't}.  This mechanism is also used in interrogatives 
that have subject-verb inversion to simulate auxiliary verb movement.  For 
more on this, see Chapter~\ref{auxiliaries} on {\it do}-support and 
inversion. 
 
\begin{rawhtml} <p> \end{rawhtml}
  \begin{center} \leavevmode \htmladdimg{ps/imperatives-files/betaVvx-epsilon.ps.gif} 
  \end{center} 
  \begin{rawhtml} <dl> <dt>{$\beta$Vvx[$\epsilon$] <p> </dl> \end{rawhtml}
\label{fig:epsilon} 
\begin{rawhtml} <p> \end{rawhtml}
 
 
If {\it do} in negative imperatives is in the same position as {\it do} in 
{\it yes-no} questions, the fact that an overt subject cannot intervene 
between {\it do} and {\it not} is puzzling. 
 
\beginsentences
\sitem{Do not open the window!}\label{ex:308} 
\sitem{*Do you not open the window!}\label{ex:309} 
\endsentences

 
We adopt the account given in \cite{akmajian84} that this fact is not due 
to syntax but due to an intonational constraint in imperatives.  He argues 
that (i) when an imperative sentence has an overt subject, the subject must be 
the only intonation center preceding the verb phrase and (ii) that in 
negative imperatives with {\it do} and {\it not}, either {\it do} or {\it not} must be the intonation center.  These two contraints conspire to rule 
out {\it do not} imperatives with an overt subject. 
 
\subsection{Negative Imperatives with {\it be} and {\it have}} 
 
Another puzzling fact that needs to be explained  is that in negative 
imperatives even {\it be} and auxiliary {\it have} require {\it do}-support, while it is prohibited in negative declaratives and negative 
questions.  
 
\beginsentences
\sitem{He isn't talking loud.}\label{ex:310} 
\sitem{*He doesn't be talking loud.}\label{ex:311} 
\endsentences

 
\beginsentences
\sitem{Isn't he talking loud?}\label{ex:312} 
\sitem{*Doesn't he be talking loud?}\label{ex:313} 
\endsentences

 
This fact does not pose a problem for the XTAG analysis of negative 
imperatives if we adopt the line of approach given in \cite{handiss}.  She 
points out that while declaratives and questions are tensed, imperatives 
are not, and argues that this is exactly why negative imperatives require 
{\it do}-support even for {\it be} and auxiliary {\it have}.  Assuming a 
clause structure in which CP dominates IP and IP dominates VP (for 
expository purposes), she argues that it is the tense features in I$^0$ 
that attract {\it be} and auxiliary {\it have} in declaratives and 
questions.  In declaratives, {\it be} or auxiliary {\it have} moves to and 
stays in I$^0$, and in questions, once {\it be} or auxiliary {\it have} 
moves to I$^0$, they further move to C$^0$.  Moreover, main verbs cannot 
move at all to I$^0$ in the overt syntax.  Instead, they undergo movement 
at LF.  But negation blocks LF movement and so as a last resort {\it do} is 
inserted in I$^0$ to support INFL.  In imperatives, I$^0$ does not have 
tense features and so it cannot attract {\it be} and auxiliary {\it have}. 
Thus, {\it be} and auxiliary {\it have} as well as main verbs undergo 
movement at LF in imperatives.  And so in negative imperatives, since 
negation blocks LF verb movement, {\it do} is inserted in I$^0$ as a last 
resort device even for {\it be} and auxiliary {\it have} and it further 
moves to C$^0$ in the overt syntax. 
 
\section{Emphatic Imperatives} 
 
Another case where imperatives have {\it do}-support is emphatic 
imperatives. 
 
\beginsentences
\sitem{Do open the window!}\label{ex:314} 
\sitem{Do show up for the lecture!}\label{ex:315} 
\endsentences

 
In English XTAG grammar, {\it do} in emphatic imperatives is treated just 
as {\it do} in negative imperatives.  It is adjoined to an imperative 
clause with an empty subject.  Again, the crucial evidence for this 
analysis comes from word order facts.  When emphatic imperatives have an 
overt subject, it must follow {\it do}. 
 
\beginsentences
\sitem{Do somebody bring me some water!}\label{ex:316} 
\sitem{Do at least some of you show up for the lecture!}\label{ex:317} 
\endsentences

 
\section{Cases not handled} 
 
\subsection{Overt subjects before {\it do/don't}} 
\label{sec:vocative} 
 
Given our analsis of negative imperatives, if the subject precedes {\it do} 
or {\it don't}, we are forced to treat it as a vocative and not a 
sentential subject.  Vocatives are considered to be outside the clause 
structure and does not have any structural relation with any element in the 
clause. 
 
\beginsentences
\sitem{You don't drink the water. = (You! Don't drink the water!)}\label{ex:318} 
\sitem{You do not leave the room. = (You! Do not leave the room!)}\label{ex:319} 
\endsentences

 
Given the fact that the imperatives in \ref{ex:318} and \ref{ex:319} seem to be 
degraded unless there is an intonational break between {\it you} and the 
rest of the sentence, treating {\it you} as a vocative seems to be the 
correct approach.  Currently, XTAG grammar does not handle vocatives. 
 
\subsection{Overt subjects after {\it do/don't}} 
\label{sec:overt-subject} 
 
One remaining task for imperatives is to handle those with overt subjects 
such as \ref{ex:316} and \ref{ex:317}. 
The type of overt subjects allowed in imperatives are restricted: 2nd 
person pronouns and some quantified noun phrases. Currently, the English 
XTAG grammar only has imperative trees with empty subjects.  
 
\subsection{Passive Imperatives} 
 
Passive imperatives like \ref{ex:320} are currently not handled. 
 
\beginsentences
\sitem{Don't be defeated at the race today!}\label{ex:320} 
\endsentences

 
Accounting for them would probably involve making separate passive trees in 
each tree family, as is done for the declaratives and other clause types in 
each family.\footnote{% % A simpler way to allow for passive imperatives would be remove the equation {\bf V.t:$<$passive$>$ = -} from the imperative trees. However, this option may affect the consistency of treatment of the passives. A decision in this respect will have to be made before implementing the imperative passives.% % } 
 
\subsection{Overgeneration} 
 
As was discussed above, imperatives can also be formed with auxiliaries 
like {\it have} and {\it be}, as in \ref{ex:321}, \ref{ex:322} and \ref{ex:323}: 
 
\beginsentences
\sitem{Be waiting for me when I return!}\label{ex:321} 
\sitem{Don't be sleeping while reading your book!}\label{ex:322} 
\sitem{Don't have fallen asleep when I come back!}\label{ex:323} 
\endsentences

 
The auxiliary {\it be} can form affirmative as well as negative 
imperatives. However, {\it have} can only form a negative imperative, as 
can be seen from \ref{ex:323} and the ungrammaticality of \ref{ex:324}: 
 
\beginsentences
\sitem{* have eaten your meal by the time I return!}\label{ex:324} 
\endsentences

 
The current analysis of imperatives, however, does not rule out \ref{ex:324}. 
 
 
 
 
 
 
 
 
 
 
