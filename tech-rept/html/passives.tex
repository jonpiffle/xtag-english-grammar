\chapter{Passives} 
\label{passives} 
In passive constructions such as (\ref{ex:379}), the subject NP is 
interpreted as having the same role as the direct object NP in the 
corresponding active declarative (\ref{ex:380}). 
 
\beginsentences
\sitem{{\bf An airline buy-out bill} was approved by the House. (WSJ)}\label{ex:379} 
\sitem{The House approved {\bf an airline buy-out bill}.}\label{ex:380} 
\endsentences

 
\begin{rawhtml} <p> \end{rawhtml}
\centering 
\begin{tabular}{ccccc} 
\htmladdimg{ps/passives-files/betanx1Vs2-reduced-features.ps.gif}& 
\hspace{1.0in}& 
\htmladdimg{ps/passives-files/betanx1Vbynx0s2.ps.gif}& 
\hspace{1.0in}& 
\htmladdimg{ps/passives-files/betanx1Vs2bynx0.ps.gif}\\ 
(a)&&(b)&&(c) 
\end{tabular} 
\begin{rawhtml} <dl> <dt>{Passive trees in the Sentential Complement with NP tree family: $\beta$nx1Vs2 (a), $\beta$nx1Vbynx0s2 (b) and $\beta$nx1Vs2bynx0 (c) <p> </dl> \end{rawhtml}
\label{passive-trees} 
\label{2;2,5} 
\begin{rawhtml} <p> \end{rawhtml}
 
In a movement analysis, the direct object is said to have moved to the subject 
position.  The original declarative subject is either absent in the passive or 
is in a {\it by} headed PP ({\it by} phrase). In the English XTAG grammar, 
passive constructions are handled by having separate trees within the 
appropriate tree families.  Passive trees are found in most tree families that 
have a direct object in the declarative tree (the light verb tree families, for 
instance, do not contain passive trees).  Passive trees occur in pairs - one 
tree with the {\it by} phrase, and another without it.  Variations in the 
location of the {\it by} phrase are possible if a subcategorization includes 
other arguments such as a PP or an indirect object. Additional trees are 
required for these variations.  For example, the Sentential Complement with NP 
tree family has three passive trees, shown in Figure~\ref{passive-trees}: one 
without the {\it by}-phrase (Figure~\ref{passive-trees}(a)), one with the {\it by} phrase before the sentential complement (Figure~\ref{passive-trees}(b)), 
and one with the {\it by} phrase after the sentential complement 
(Figure~\ref{passive-trees}(c)). 
 
Figure~\ref{passive-trees}(a) also shows the feature restrictions imposed on 
the anchor\footnote{A reduced set of features are shown for readability.}. Only 
verbs with {\bf $<$mode$>$=ppart} (i.e. verbs with passive morphology) can 
anchor this tree.  The {\bf $<$mode$>$} feature is also responsible for 
requiring that passive {\it be} adjoin into the tree to create a matrix 
sentence.  Since a requirement is imposed that all matrix sentences must have 
{\bf $<$mode$>$=ind/imp}, an auxiliary verb that selects {\bf $<$mode$>$=ppart} and {\bf $<$passive$>$=+} (such as {\it was}) must adjoin 
(see Chapter~\ref{auxiliaries} for more information on the auxiliary verb 
system). 
 
 
 
 
 
 
 
 
 
