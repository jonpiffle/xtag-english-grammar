\section{Multicomponent Anchors}

\subsection{Light Verbs}

Light verbs are one example of a construction that requires a
multi-component anchor analysis. Light verbs occur in complex
predicates such as {\it take a walk} and {\it give a nod} in which the
interpretation is non-compositional and the noun contributes arguement
structure to the predicate. The interpretation of {\it take a walk}
is close to the interpretation of the verb {\it walk} rather than a
compositional interpretation of {\it take a walk} with {\it take} as
the main predicate. Comparison of (\ex{1}) and (\ex{2}) illustrates
this difference.

\enumsentence{ The man took a walk.}
\enumsentence{The man took a radio.}

Light verb constructions also differ syntactically from other
constructions with the same verbs. In examples (\ex{1}) and (\ex{2}) from \cite{Cattell84} {\it offer} allows for the dative alternation that {\it make} would not otherwise allow.

\enumsentence{They made an offer of money to the police.}
\enumsentence{They made the police an offer of money.}


The English LTAG grammar light verb analysis includes constructions with {\it do}, {\it
give}, {\it have}, {\it make}, and {\it take}.  Marginal candidates such as
{\it get} (as in {\it get a hold of NP} or {\it get a view of NP}) and {\it
put} (as in {\it put NP in writing} or {\it put the blame on NP}) are
excluded.  We have a separate analysis for predicative {\it be}, which in
the literature is often classed along with the light verbs.
\footnote{The (putative) light verb construction with {\it put} requires
two objects and does not undergo dative shift.  This does not depart from
the normal behavior of {\it put}, and therefore we see no need to include
it in the TAG analysis as a light verb.}

We treat as light verb constructions only those constructions that fit
the simple transitive (V-NP) or double object (V-NP1-P-NP2 or
V-NP2-NP1) patterns.  According to the literature, the ``noun''
following the light verb is (usually) a bare infinitive form and
(usually) occurs with {\it a(n)}.  However, we include deverbal
nominals ({\it take a bath}, {\it give a demonstration}) as well as
constructions with bare infinitives ({\it have a good cry}).
Constructions with nouns that do not contribute an argument structure
({\it have a cigarette}, {\it give NP a black eye}) are analyzed as
idioms in the English LTAG grammar\footnote{The grammar includes as many light verb
constructions as possible, even if they do not occur in AmEng (e.g., {\it
have a think}, {\it have a bathe}).}.

Verb-adjective constructions such as {\it make NP angry}, although they
alternate with expressions like {\it anger NP}, are not included in the
light verb analysis.  Also, it is not clear whether the first verb in
serial verb constructions (e.g., {\it go hiking}, {\it come running}, {\it
keep nodding}) is more like a light verb or an aspectual auxiliary.  At
present we don't have an analysis for serial verbs.

English light verbs have partially specified argument structures which
combine with the argument structure of the associated noun to form a
complex predicate.  For example, {\it take a walk} means essentially what
{\it walk} means, but {\it take} imposes additional restrictions on the
types of entities that may be in subject position (for example, the subject
must be [+human]).

There are also semantic restrictions on the type of noun that may appear in
the predicate, but these are not useful to us as they are distinctions that
are not coded for in our lexicon.  (These restrictions have to do with the
aspectual nature of the construction and include space vs. time, length of
activity, whether or not the action is goal-directed, repeatability, etc.)

Light verb trees differ from normal transitive trees in that both the verb
and the noun in direct object position are marked as anchors.

\subsubsection{Syntactic Phenomena}

{\bf Extraction.}

The noun associated with the light verb (nx1, marked as an anchor) does not
extract.  Extraction may take place from any other position, including
extraction from within nx1:

\enumsentence{What did John take?  *A jog }
\enumsentence{Who did an analysis?  John }
\enumsentence{Who did John make a promise to?  His mother }
\enumsentence{What did John do an analysis of?}
\enumsentence{  (cf. *What did John challenge an analysis of?)}


Extraction from complex NPs is blocked, except when the NP in question is
the noun anchor of a light verb construction (nx1).

\enumsentence{*Who did John dislike an imitation of? Bush }
\enumsentence{Who did John do an imitation of? Bush }
\enumsentence{*What did John reject an offer of?  Real estate.}
\enumsentence{What did John make an offer of?  Real estate.}


In addition to the generalization that anchors do not extract, it appears
that indirect objects may not extract from post-verbal position in the
dative-shifted form.

\enumsentence{The man made a promise to his mother }
\enumsentence{Who did the man make a promise to t?  His mother. }
\enumsentence{The man made his mother a promise }
\enumsentence{??Who did the man make t a promise?  His mother}

However, judgements on this are not firm and vary from speaker to speaker
and so we allow for extraction from indirect object position in the
dative-shifted form.

\vspace{0.5in} \noindent
{\bf Relativization.} \vspace{0.25in}

In the first set of examples the relativized noun and its relative clause
constitute a complex noun that is only part of a light verb construction.
In the second set the noun is relativized out of a light verb construction.
\begin{verse}
{The horse that won the race gave his groom a kick.} \\
{I finally got to take the nap that I'd needed all morning.} \\
{The horse gave the groom a kick he'd never forget.}
\end{verse}

\begin{verse} 
{The nap that I took this afternoon was refreshing.} \\
{The kick that I gave him was stunning.} \\
{The promises he made were outlandish.}
\end{verse}
In examples of the type in the first set, the relative clause adjoins onto
the NP anchor that appears in the initial tree.  So for example, {\it I
finally got to take the nap that I'd needed all morning} would start out
as {\it I finally got to take the nap}.  Then {\it that I needed all
morning} should adjoin onto [np the nap np], resulting in [np [np the nap
np] [s that [I needed all morning ]] np]

However, relativization of nx1 poses a problem for the TAG implementation
in examples of the second type.  For example, to get a correct parse for
{\it the nap that I took this afternoon}, the noun {\it the nap} must
combine with the tree for {\it that I took} $\epsilon$ {\it this
afternoon}.  But there is the conflicting requirement that as an anchor it
must appear with the initial S tree for {\it take-nap}.  The solution is to
exclude constructions containing what look like relativized anchors from
the light verb analysis.  That is, examples of the second type will not
receive a light verb parse, while the first type will.

Whether there is independent evidence to suggest that such constructions
are in fact not light verb constructions is a difficult question.  Whatever
the facts, this approach is consistent with other observations about the
constraints on wh-extraction and passivization as far as the noun anchor is
concerned.

\vspace{0.5in} \noindent {\bf  Dative shift.} \vspace{0.25in}

At first glance, it appears that ditransitives with {\it give} have only
the dative-shifted form: {\it give NP a kick} vs. *{\it give a kick to NP}.
However, this restriction appears also to interact with something like
Heavy NP Shift:

\enumsentence{The horse would give a kick *(to) anyone who came too close to him and
didn't offer sugar cubes.}

where the heavy NP moves to the end and appears as the indirect (dative)
object.  Light verb constructions with ditransitive {\it give} behave like
non-light verb constructions with distransitive {\it give} in this respect.
There remains, however, the problem of the nonextractibility of the anchor:

\enumsentence{I gave the horse a kick. }
\enumsentence{What did you give the horse?  *A kick.\footnote{Allowing
extraction from within complex nx1 anchors is not a problem with
ditransitive {\it give}.  What was once the direct object in
constructions such as {\it pull NP}, {\it kiss NP} now shows up as nx2
in light verb constructions such as {\it give NP a pull} or {\it give NP a
kiss}, and not as a complement of the anchor, which is the case in
(\ex{1}-(\ex{2}) below.}}

\enumsentence{imitate Bush				V nx }
\enumsentence{do an imitation of Bush		V Nx1 Pnx }
\enumsentence{kick the horse			V nx }
\enumsentence{give the horse a kick		V nx2 Nx1}

Ditransitives with {\it make} also allow dative shift:

\enumsentence{Make an offer to the police.}
\enumsentence{Make the police an offer of money.}
\enumsentence{Make a promise to one's mother.}
\enumsentence{Make one's mother a promise.}
\enumsentence{Make a payment of \$500 to your agent.}
\enumsentence{Make your agent a payment of \$500.}



\vspace{0.5in} \noindent {\bf  Possessives.} \vspace{0.25in}

Light verbs often allow a possessive on the noun in the predicate.  The
person and number of this possessive must match features of the subject:

\enumsentence{She gave him her usual look.}
\enumsentence{*She gave him his usual look.}
\enumsentence{I took my nap.}
\enumsentence{*I took your nap.}


This subject-identity requirement on possessives has not been implemented
since there are too many exceptions.  For instance, in a case where there
is a limited amount of hot water and John was planning to take a shower but
Mary gets there first, you could say {\it Mary took John's shower}.

\vspace{0.5in} \noindent {\bf  Passives.} \vspace{0.25in}

As expected, the noun associated with the light verb construction (marked
as an anchor) does not passivize, except with ditransitive {\it make}
constructions.

\enumsentence{*A laugh was given by John.}
\enumsentence{The groom was given a kick by the horse}
\enumsentence{*A kick was given to the groom by the horse}
\enumsentence{An offer was made to John by a large conglomerate}



\enumsentence{A laugh was given.}
\enumsentence{The groom was given a kick.}
\enumsentence{A kick was given to the groom.}
\enumsentence{After months of interviewing, an offer was finally made to John, but
Bill still hasn't heard a word.}

We include trees for passivization (with and without {\it by}) of the
direct object when it is not the anchor noun, with the exception of
ditransitives with {\it make}.

\subsection{Particle verbs}
Particle verbs are another construction that is analyzed in the
English LTAG grammar with multicomponent anchors. The verb and particle act as a unit and their meaning is non-compositional.  


\enumsentence{They looked  up the number.} 
\enumsentence{They looked the number up.}
\enumsentence{They looked up the cliff.}
\enumsentence{*They looked the cliff up.}

Sentences (\ex{-3}) and (\ex{-2}) show the possibility of both base position and particle movement fro verb-particle constructions. Lexical items that serve
as particles can often also be used as prepositions with the same verbs. The verb-prepositional phrase  and verb-particle uses
have distinct interpretations and the prepositions cannot be moved as
shown in (\ex{-1}) and (\ex{0}). The english LTAG analysis has both the verb and the
particle as anchors. Tree families for transitive (and
di-transitive) particle verbs, include a tree with the particle
adjacent to the verb and one for each possible moved position for the
basic tree and each relevant extraction. So for example, the tree family for {\it call
up}, Tnx0Vplnx1, includes the following trees:
\begin{description}
\item nx0Vnx1pl
\item nx0Vplnx1
\item W0nx0Vplnx1
\item W0nx0Vnx1pl
\item Inx0Vnx1pl
\item Inx0Vplnx1
\item R0nx0Vnx1pl
\item R0nx0Vplnx1
\item W1nx0Vplnx1
\item R1nx0Vplnx1
\item N0nx0Vplnx1
\item N0nx0Vnx1pl
\item N1nx0Vplnx1
\item Gnx0Vplnx1
\item Gnx0Vnx1pl
\item Dnx0Vplnx1
\item nx1Vplbynx0
\item W0nx1Vplbynx0
\item pW0nx1Vplbynx0
\item W1nx1Vplbynx0
\item R0nx1Vplbynx0
\item pR0nx1Vplbynx0
\item N0nx1Vplbynx0
\item R1nx1Vplbynx0
\item N1nx1Vplbynx0
\end{description}

While particle movement provides an easy diagnostic for transitive
particle verbs, detecting intransive particle verbs is largely
dependent on semantic criteria. In intransitive constructions the
ambiguity between prepostions and particles is eliminated because of
the absence of an NP to complete the PP but there can still be ambiguity
with locative or directional adverbs as shown in (\ex{1}) and (\ex{2}).   

\enumsentence{The fire burned out before much damage was done.}
\enumsentence{The fire burned out toward the town.}



\subsection{Idioms}
The current English LTAG grammar does not have a full implementation
of idioms but does include the tree and syntactic entries neccessary
for the idiom, {\it kick the bucket}.  The analysis of idiom is
similar to that for light verb constructions and particle verbs in
having as multicomponent anchors items that must co-occur and are
interpreted non-compositionally. The simple decalrative tree anchored
by {\it kick the bucket} is shown in figure \ref{kick-bucket}.

%put kick-the-bucket tree here.

\begin{figure}
{\psfig{figure=ps/kick_the_bucket.ps}}
\caption{Tree for {\it kick the bucket}}
\end{figure}
It closely resembles its transitive counterpart $\alpha$nx0Vnx1.  The
difference between the trees is that the idiom tree must have nodes
expanded until the lexical category of each component of the anchor
appears in the tree as a leaf node. In this example, the NP$_{1}$ node
of $\alpha$nx0Vnx1 corresponds to the Det$_{1}$ and N$_{1}$ nodes of
the $\alpha$nx0VDN1 tree. Discussion of plans for expansion of the
idiom coverage using automatic tree generation is discussed in
section? under future research.
