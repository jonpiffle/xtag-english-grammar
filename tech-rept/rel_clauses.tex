\chapter{Relative Clauses}
\label{rel_clauses}

Relative clauses are NP modifiers involving extraction of an argument or an
adjunct. The NP head (the portion of the NP being modified by the relative
clause) is not directly related to the extracted element.  For example in
\ex{1}, {\it the person} is the head NP and is modified by the relative
clause {\it whose mother $\epsilon$ likes Chris}. However, {\em the person}
is not interpreted as the subject of the relative clause which is missing
an overt subject. In other cases, such as \ex{2}, the relationship between
the head NP {\em export exhibitions} may seem to be more direct but even
here we assume that there are two independent relationships: one between
the entire relative clause and the (head) NP it modifies, and another
between the extracted element and its trace inside the relative clause. The
extracted element can further be overt, as in \ex{1}, or covert, as in \ex{2}. 

\enumsentence{[[ the person$_i$ ] [ whose$_i$ mother$_j$ [ $\epsilon_j$ likes Chris ]]]}
\enumsentence{[[ export exhibitions$_i$ ] [ $\epsilon_i$ that [ $\epsilon_i$ included high-tech items ]]]}

The XTAG analysis of relative clauses is essentially identical to the GB
analysis. We represent relative clauses as auxiliary trees that adjoin to
NP's. These trees are anchored by the verb of the relative clause and
appear in the appropriate tree families representing various verb
subcategorizations. Each family has a group of relative clause trees based
on the declarative tree and each passive tree in that family. Furthermore,
within each of these groups, there is a separate relative clause tree
corresponding to each argument that can be extracted from the clause. The
relationship between the relative clause and the head NP is not represented
directly. Rather, it is treated as a semantic relationship which could be
provided by any reasonable compositional theory. The relationship between
the extracted element, NP$_w$ (which can be covert) and the position from
which it was extracted is captured by co-indexing the {\bf $<$trace$>$}
features of the two positions/nodes. If for example, it is {\bf NP$_{1}$}
that is extracted (object extraction), we have the following feature
equations (see Figure~\ref{trans-rel-clause-trees}(a)):

\begin{itemize}

\item {\bf NP$_{w}$.t:$\langle$ trace $\rangle =$NP$_{1}$.t:$\langle$ trace $\rangle$}
\item {\bf NP$_{w}$.t:$\langle$ case $\rangle =$NP$_{1}$.t:$\langle$ case $\rangle$}
\item {\bf NP$_{w}$.t:$\langle$ agr $\rangle =$NP$_{1}$.t:$\langle$ agr $\rangle$}

\end{itemize}

One aspect of the implementation of relative clauses is to allow a covert
{\bf $+<$wh$>$} NP and/or a covert COMP. For example, \ex{1} has a covert
{\bf $+<$wh$>$} NP and overt COMP, \ex{2} has a covert COMP and overt {\bf
$+<$wh$>$}, and \ex{3} has both a covert {\bf $+<$wh$>$} NP and a covert
COMP.

\enumsentence{export exhibitions [[ {\footnotesize $_{NP_{w}}$} $\epsilon$ ]$_{i}$ that [ $\epsilon$$_{i}$ included high-tech items ]]}
\enumsentence{the export exhibition [ which$_{i}$ [ Muriel planned  $\epsilon$$_{i}$ ]]}
\enumsentence{the export exhibition [[ {\footnotesize $_{NP_{w}}$} $\epsilon$ ]$_{i}$ [ Muriel planned  $\epsilon$$_{i}$ ]]}

Consequently, our treatment of relative clauses has different trees to
handle relative clauses with an overt extracted {\em wh}-NP
(Section~\ref{sec:overt-extraction}) and relative clauses with a covert
extracted {\em wh}-NP (Section~\ref{sec:covert-extraction}). Covert and
overt COMP's are handled by adjunction with the already existing auxiliary
trees for complementizers, that are used for sentential complementation
(see Chapter~\ref{scomps-section}).

\section{Relative Clauses with overt extracted {\em wh}-phrases}
\label{sec:overt-extraction}

Relative clauses with an overt extracted {\em wh}-NP
(Figure~\ref{trans-rel-clause-trees}(a)) involve substitution of a $+${\bf
$<$wh$>$} NP into the (extracted) NP$_{w}$ node. The feature equation {\bf
NP$_{w}$.t:$<$wh$>=+$} allows only {\it wh}-phrases to substitute into
this node, such as {\em whose mother}, {\em who}, {\em whom}, {\em which}
(but not {\em when} and {\em where}, which are treated as exhaustive
$+${\em wh} PPs (Figure~\ref{trans-rel-clause-trees}(b))).

\begin{figure}[ htb ]
\begin{tabular}{cc}
\psfig{figure=ps/rel_clauses-files/betaN1nx0Vnx1.ps,height=12.0cm}&
\psfig{figure=ps/rel_clauses-files/betaNpxnx0Vnx1.ps,height=12.0cm}\\
(a)&(b)
\end{tabular}
\caption{Relative clause trees with overt {\em wh}-phrases in the transitive
tree family: (a) object extraction tree $\beta$N1nx0Vnx1, ({\it the man who
Muriel loved}), and (b) adjunct relative clause tree with PP pied-piping
$\beta$Npxnx0Vnx1, {\it the bowl in which Miranda ate her cereal}.}
\label{trans-rel-clause-trees}
\label{2;16,1}
\label{2;15,1}
\end{figure}

Complementizers can never cooccur with the overt extracted {\bf $+<$wh$>$}
NP (cf. *{\it I saw the man who$_i$ that Muriel saw $\epsilon$$_{i}$}).
Consequently, the auxiliary $\beta$COMPs trees are prevented from adjoining
at the {\bf S$_r$} node in these trees by the equation

\begin{itemize}
\item {\bf S$_{r}$.t:$\langle$comp$\rangle =$ nil}
\end{itemize}

in the relative clause tree, which will always fail to unify with the
(non-{\bf nil}) values of the {\bf $<$comp$>$} feature in the $\beta$COMPs
trees (see Figure~\ref{that-comp-tree}). Examples \ex{1} and \ex{2} are
examples for which the tree in Figure~\ref{trans-rel-clause-trees}(a) is
used. Cases of PP pied-piping, as in \ex{3}, are handled in a similar
fashion by building in a PP$_{w}$ substitution node
(Figure~\ref{trans-rel-clause-trees}(b)).\footnote{%
%
Adjunct traces are not represented in the XTAG analysis of adjunct
extraction. Since relative clauses on adjuncts also do not have traces,
feature equations showing the trace coindexation are not present in such
trees. See Section~\ref{sec:adju-RC} for more discussion of adjunct
relative clauses.%
%
} 

\enumsentence{the man who Muriel likes}
\enumsentence{the man whose mother Muriel likes}
\enumsentence{the bowl in which Miriam ate her cereal}

\subsection{Contraints on the mode of the relative clause}
\label{sec:mode-restriction}

Relative clause trees that have {\bf NP$_{w}$} as a substitution node have
the feature equation given below. The examples in \ex{2}--\ex{6} provide the
rationale for this feature setting.

\begin{itemize}
\item {\bf S$_{r}$.t:$\langle$mode$\rangle =$ind}
\end{itemize}

\enumsentence{
the man [[whose wife ]$_{i}$ [ $\epsilon$$_{i}$ cooked this meal ]] ({\bf S$_{r}$.t:$\langle$mode$\rangle =$ind})}
\enumsentence{
*the girl [[who ]$_{i}$ [ $\epsilon$$_{i}$ to win the prize ]] ({\bf S$_{r}$.t:$\langle$mode$\rangle =
$inf})}
\enumsentence{
*the candidate [[who ]$_{i}$ [ $\epsilon$$_{i}$ defeated in the elections ]] ({\bf S$_{r}$.t:$\langle$mode$\rangle 
=$ppart})}
\enumsentence{
*the boy [[whose dog ]$_{i}$ [ $\epsilon$$_{i}$ chasing the cat ]] ({\bf S$_{r}$.t:$\langle$mode$\rangle =$
ger})}
\enumsentence{
the boy [[whose mother ]$_{i}$ [ Bill believes [ $\epsilon$$_{i}$ to be beautiful]]] ({\bf S$_{r}$.t:$\langle$mode$\rangle =$ind})}

Trees with a {\bf PP$_{w}$} substitution node have the feature equation
given below with the rationale provided by examples in \ex{2}--\ex{5}.%
\footnote{%
%
As is the case for {\bf NP$_{w}$} substitution, any $+${\bf $<$wh$>$}PP can
substitute under PP$_{w}$.  This is implemented by the following
equation: {\bf PP$_{w}$.t:$\langle$wh$\rangle=+$}.
%Not all cases of pied-piping involve substitution of {\bf PP$_{w}$}.  In
%some cases, the P may be built in. In cases where part of the pied-piped PP
%is part of the anchor, it continues to function as an anchor even after
%pied-piping i.e. the P node and the {\bf NP$_{w}$} nodes are represented
%separately.%
%
}

\begin{itemize}
\item {\bf S$_{r}$.t:$\langle$mode$\rangle =$ ind/inf}
\end{itemize}

\enumsentence{
the person [[by whom ]$_{i}$ [ this machine was invented $\epsilon$$_{i}$ ]] ({\bf S$_{r}$.t:$\langle$mode$\rangle =$ind})}
\enumsentence{
a company [[in which ]$_{i}$ [ to invest $\epsilon$$_{i}$ ]] ({\bf S$_{r}$.t:$\langle$mode$\rangle =$inf})}
\enumsentence{
*the fork [[with which ]$_{i}$ (Geoffrey) eaten the pudding $\epsilon$$_{i}$
]] ({\bf S$_{r}$.t:$\langle$
mode$\rangle =$ppart})}
\enumsentence{
*the person [[by whom ]$_{i}$ [ (this machine) inventing $\epsilon$$_{i}$
]] ({\bf S$_{r}$.t:$\langle$mode$\rangle =$ger})}

\section{Relative Clauses with Covert Extracted {\it wh}-NP}
\label{sec:covert-extraction}

Relative clauses with a covert extracted {\em wh}-NP
(Figure~\ref{trans-rel-clause-trees2} ) have a NP$_{w}$ node headed by
$\epsilon$$_{w}$, which is built into the trees. Complementizers can adjoin
in at the {\bf S$_r$} node in a manner parallel to sentential
complementation (see Chapter~\ref{scomps-section}). The examples in
\ex{1}-\ex{4} are handled by this tree.

\enumsentence{the cake that Muriel said Steven ate}
\enumsentence{the book for Miranda to read}
\enumsentence{the man looking at Muriel}
\enumsentence{the librarian to check out the book}

\begin{figure}[ htb ]
\begin{tabular}{c}
\centerline{\psfig{figure=ps/rel_clauses-files/betaNc0nx0Vnx1.ps,height=13.0cm}}
\end{tabular}
\caption{Subject extraction tree in the transitive tree family with
non-overt {\em wh}-phrase, $\beta$Nc0nx0Vnx1}
\label{trans-rel-clause-trees2}
\end{figure}

There are two aspects to the implementation of these trees. Firstly, in a
manner parallel to the relative clause trees with overt NP$_w$, there are
constraints on the mode of the relative clause for these trees, which is
realized with the {\bf $<$mode$>$} feature
(Section~\ref{sec:clause-mode}). Secondly, there are cooccurrence
constraints between the mode of the relative clause and the complementizers
that can adjoin in -- these are realized with the {\bf $<$assign-comp$>$}
and {\bf $<$comp$>$} features (Section~\ref{sec:comp-selection}). The
implementation of the cooccurrence constraints is entirely parallel to
sentential complementation, except in one respect: the occurrence of the
null COMP (which in the case of the relative clauses, is represented by
disallowing the adjunction of any COMP -- namely, the $\beta$COMPs
auxiliary tree -- altogether) is subject to further constraints, which we
realize with the {\bf $<$nocomp-mode$>$} feature
(Section~\ref{sec:nocomp-mode}).

\subsection{Constraints on the mode of the relative clause}
\label{sec:clause-mode}

The mode of relative clause varies depending on which argument has been
extracted. For example, subject extraction can occur only in the
indicative, infinitive, and gerundive modes, as can be seen from examples
like \ex{1}--\ex{4}. Object extraction can only occur in the indicative and
infinitive modes, as shown in examples \ex{5}--\ex{8}. This restriction is
implemented by setting the {\bf S$_r$.t:$<$mode$>$} feature to the
appropriate values, such as {\bf ind}, {\bf inf}, {\bf ger},
etc.. Figure~\ref{trans-rel-clause-trees2} shows this restriction
implemented for the relative clause tree with subject extraction in the
transitive tree family.

\enumsentence{the dog [ $\epsilon_w$ that [ $\epsilon$ ate the cake ] ] ({\bf S$_{r}$.t:$\langle$mode$\rangle =$ind})}
\enumsentence{the team [ $\epsilon_w$ [ $\epsilon$ to win the gold medal ] ] ({\bf S$_{r}$.t:$\langle$mode$\rangle =$inf})}
\enumsentence{the girl [ $\epsilon_w$ [ $\epsilon$ reading the book ] ] ({\bf S$_{r}$.t:$\langle$mode$\rangle =$ger})}
\enumsentence{*the woman [ $\epsilon_w$ [ $\epsilon$ seen the sight ] ] ({\bf S$_{r}$.t:$\langle$mode$\rangle =$ppart})}

\noindent
\enumsentence{the toy [ $\epsilon_w$ that [ Miranda likes $\epsilon$ ] ] ({\bf S$_{r}$.t:$\langle$mode$\rangle =$ind})}
\enumsentence{the guava [ $\epsilon_w$ for [ Susan to eat $\epsilon$ ] ] ({\bf S$_{r}$.t:$\langle$mode$\rangle =$inf})}
\enumsentence{*the toy [ $\epsilon_w$ [ Laura liking $\epsilon$ ] ] ({\bf S$_{r}$.t:$\langle$mode$\rangle =$ger})}
\enumsentence{*the book [ $\epsilon_w$ [ Danny torn $\epsilon$ ] ] ({\bf S$_{r}$.t:$\langle$mode$\rangle =$ppart})}

The full set of mode restrictions on the different relative clause trees is
as follows:

\begin{itemize}

\item For all non-passive cases of subject extraction, {\bf S$_{r}$.t:$\langle$mode$\rangle =$ ind/ger/inf} ( see \ex{1}--\ex{4}):

\enumsentence{the girl [ $\epsilon_w$ [ $\epsilon$ reading the magazine ] ] ({\bf S$_{r}$.t:$\langle$mode$\rangle =$ger})}
\enumsentence{the cowboy [ $\epsilon_w$ [ $\epsilon$ to win the fight ] ] ({\bf S$_{r}$.t:$\langle$mode$\rangle =$inf})}
\enumsentence{the man [ $\epsilon_w$ that [ $\epsilon$ loaded the gun ] ] ({\bf S$_{r}$.t:$\langle$mode$\rangle =$ind})}
\enumsentence{*the child [ $\epsilon_w$ (that/for) [ $\epsilon$ eaten the cake ] ] ({\bf S$_{r}$.t:$\langle$mode$\rangle =$ppart})}

\item For all passive cases of subject extraction, {\bf S$_{r}$.t:$\langle$mode$\rangle =$ ind/ger/ppart/inf} (see \ex{1}--\ex{4}):

\enumsentence{the toy [ $\epsilon_w$ that [ $\epsilon$ was broken by the child ] ] ({\bf S$_{r}$.t:$\langle$mode$\rangle =$ind})}
\enumsentence{the man [ $\epsilon_w$ [ $\epsilon$ being arrested by the officer ] ] ({\bf S$_{r}$.t:$\langle$mode$\rangle =$ger})}
\enumsentence{the food [ $\epsilon_w$ [ $\epsilon$ to be served during dinner ] ] ({\bf S$_{r}$.t:$\langle$mode$\rangle =$inf})}
\enumsentence{the candidates [ $\epsilon_w$ [ $\epsilon$ elected by the people ] ] ({\bf S$_{r}$.t:$\langle$mode$\rangle =$ppart})}

\item Finally, for all cases of non-subject extraction, {\bf S$_{r}$.t:$\langle$mode$\rangle =$ ind/inf} (see \ex{1}--\ex{4}): 

\enumsentence{the book [ $\epsilon_w$ [ John will read $\epsilon$ ] ] ({\bf S$_{r}$.t:$\langle$mode$\rangle =$ind})}
\enumsentence{the candidate [ $\epsilon_w$ for [ people to tear $\epsilon$ ] ] ({\bf S$_{r}$.t:$\langle$mode$\rangle =$inf})}
\enumsentence{*the ring [ $\epsilon_w$ (that/for) [ Miranda tearing $\epsilon$ ] ] ({\bf S$_{r}$.t:$\langle$mode$\rangle =$ger})}
\enumsentence{*the table [ $\epsilon_w$ (that/for) [ Danny broken $\epsilon$ ] ] ({\bf S$_{r}$.t:$\langle$mode$\rangle =$ppart})}

\end{itemize}

Relative clause formation with {\bf $<$mode$>$=nom/prep} (for adjectival,
nominal and prepositional predicates) are also allowed, but only with a
covert $_{NP_{w}}$ and an covert COMP. Furthermore, they can be formed only
on the subject of the clause. Some families that have these additional
modes are Tnx0APnx1 \ex{1}, Tnx0ARBPnx1 \ex{2}, Tnx0nx1ARB \ex{3}.

\enumsentence{the accused [ $\epsilon_w$ [ $\epsilon$ void of all hope ] ] ({\bf S$_{r}$.t:$\langle$mode$\rangle =$prep})} 
\enumsentence{the dog [ $\epsilon_w$ [ $\epsilon$ next to the tree ] ] ({\bf S$_{r}$.t:$\langle$mode$\rangle =$prep})}
\enumsentence{the road [ $\epsilon_w$ [ $\epsilon$ seven miles away ] ] ({\bf S$_{r}$.t:$\langle$mode$\rangle =$nom})}

\subsection{Complementizer Selection}
\label{sec:comp-selection}
The {\bf VP.t:$<$assign-comp$>$} feature in the relative clause is assigned
values which represent constraints on COMP selection by the highest verb in
the clause. The feature values are passed up to the {\bf S$_r$} node of the
relative clause by the equation,

\begin{itemize}
\item {\bf
S$_{r}$.b:$\langle$assign-comp$\rangle=$VP.t:$\langle$assign-comp$\rangle$}
\end{itemize}

This ensures proper selection of the appropriate COMP since the auxiliary
tree anchored by each complementizer also has the {\bf $<$assign-comp$>$}
feature with a value appropriate to the particular complementizer in
question (see for example the $\beta$COMPs anchored by {\em that} in
Figure~\ref{that-comp-tree}). Adjunction of any complementizer can
therefore succeed only if the {\bf $<$assign-comp$>$} features in the COMP
tree and the relative clause tree have the same value.

\begin{figure}[ htb ]
\begin{tabular}{c}
\centerline{\psfig{figure=ps/rel_clauses-files/betaCOMPthat.ps,height=7.0cm}}
\end{tabular}
\caption{Tree $\beta$COMPs, anchored by {\it that}}
\label{that-comp-tree}
\end{figure}

So, while the subject extraction tree in
Figure~\ref{trans-rel-clause-trees2} allows {\it that} to adjoin, it
prevents {\it for} from adjoining because the {\bf
S$_r$.b:$<$assign-comp$>$=for} equation in the $\beta$COMPs tree anchored
by {\em for} will fail to unify with the {\bf VP.t:$<$assign-comp$>$=
that/ind\_nil/inf\_nil/ecm} equation, which is coindexed with the {\bf
S$_r$.b:$<$assign-comp$>$} feature in the relative clause tree.


\subsection{Further Constraints on the Null COMP}
\label{sec:nocomp-mode}

In our analysis, the {\it null} complementizer is not represented in the
structure of the relative clause at all -- realization of the null COMP
implies preventing any COMP from adjoining. However, this requires an
additional set of constraints, both for distributional and implementational
reasons. For example, the null COMP is not permitted in cases of subject
extraction with {\bf $<$mode$>$=ind} unless there is an intervening
clause. The evidence can be seen in \ex{1}-\ex{4}, especially in the
contrast between \ex{1} and \ex{2}.

\enumsentence{*the girl [ $\epsilon_w$ [ $\epsilon$ likes Danny ]]] ({\bf $<$mode$>$=ind})}
\enumsentence{the man [ $\epsilon_w$ [ Mary said [ $\epsilon$ likes Dafna ]]] ({\bf $<$mode$>$=ind})}
\enumsentence{the boy [ $\epsilon_w$ [ $\epsilon$ eating the guava ]] ({\bf $<$mode$>$=ger})}
\enumsentence{the man [ $\epsilon_w$ [ $\epsilon_i$ defeated by the cowboy ]]] ({\bf $<$mode$>$=ppart})}
\enumsentence{the boy [ $\epsilon_w$ [ $\epsilon_i$ to win the gold medal ]]] ({\bf $<$mode$>$=inf})}
\enumsentence{the snake [ $\epsilon_w$ [ $\epsilon_i$ next to the tree ]]] ({\bf $<$mode$>$=prep})}
\enumsentence{the town [ $\epsilon_w$ [ $\epsilon_i$ seven miles away ]]] ({\bf $<$mode$>$=nom})}

To model this paradigm, the feature {\bf $\langle$nocomp-mode$\rangle$} is
used in conjunction with the following equations.\footnote{%
%
The {\bf S$_{r}$.t:$\langle$nocomp-mode$\rangle$} value given here appears
in the relative clause trees with subject extraction. Trees with other
constituents extracted will have different values for this feature. For
example, in object extraction trees, this feature has the value {\bf ind}.%
%
}

\begin{itemize}

\item {\bf S$_{r}$.t:$\langle$nocomp-mode$\rangle =$ inf/ger/ppart} (in
relative clause trees with subject extraction)
\item {\bf S$_{r}$.b:$\langle$nocomp-mode$\rangle =$
S$_{r}$.b:$\langle$mode$\rangle$}

\end{itemize}

Given the two equations above, successful unification of the {\bf
S$_r$.t:$<$nocomp-mode$>$} and {\bf S$_r$.b:$<$nocomp-mode$>$} features
implies realization of the null COMP, which, in the subject extracted
relative clauses (see Figure~\ref{trans-rel-clause-trees2}), is possible
only if the relative clause is in the {\bf inf}, {\bf ger}, or {\bf ppart}
mode (see examples above). Since the {\it that} $\beta$COMPs tree selects a
clause in the indicative mode (See Figure~\ref{that-comp-tree}), {\it that}
will never be able to adjoin to a relative clause with the subject
extracted. However, if a clause adjoins first to the relative clause, as
would be the case in \ex{-5}, this adjunction puts the {\bf
S$_r$.t:$<$nocomp-mode$>$} and {\bf S$_r$.b:$<$nocomp-mode$>$} in different
nodes, thus preventing a feature clash between the mode of the relative
clause and the values of the $<$nocomp-mode$>$ feature that are specified
in the subject extracted relative clause tree.

The above feature equations also permit the mode of the relative clause to
be {\bf ind} just in case there is an intervening clause, as in
\ex{-5}. Adjunction of the clause puts the {\bf S$_r$.t:$<$nocomp-mode$>$}
and {\bf S$_r$.b:$<$nocomp-mode$>$} in different nodes, thus preventing a
unification failure. Note, however, that the feature mismatch induced by
the above equations is not remedied by adjunction of just any S-adjunct
since all other S-adjuncts are transparent to the {\bf
$\langle$nocomp-mode$\rangle$} feature because of the following equation,

\begin{itemize}
\item {\bf S$_{m}$.b:$\langle$nocomp-mode$\rangle =$
S$_{f}$.t:$\langle$nocomp-mode$\rangle$}
\end{itemize}

where {\bf S$_{f}$.t} is in the foot node of the adjoining adjunct.


The obligatory adjunction of complementizers implemented above for subject
extracted relative clauses (in the indicative {\bf $<$mode$>$}) contrasts
with what we do with COMP adjunction in subject extracted questions, where
we disallow COMP from adjoining to the embedded S ({\it *Who did Miranda
say that likes Zed?}). We are thus able to capture the facts related to
{\it that-trace} constraints in English.\footnote{%
%
See Chapter~\ref{scomps-section} for a more detailed discussion related to
{\it that-trace} constraints.%
%
}

\section{External syntax}
A relative clause can combine with the NP it modifies in at least 
the following two ways:

\enumsentence{[ the [ toy [ $\epsilon_w$ [ Dafna likes $\epsilon_i$ ]]]]}
\enumsentence{[[ the toy ] [ $\epsilon_w$ [ Dafna likes $\epsilon_i$ ]]]}

Based on cases like \ex{1} and \ex{2}, which are problematic for the
structure in \ex{-1}, the structure in \ex{0} is
adopted.

\enumsentence{ [[ the man and the woman ] [ who met on the bus ]]}
\enumsentence{ [[ the man and the woman ] [ who like each other ]]} 

\begin{figure}[ htb ]
\begin{tabular}{cc}
\centerline{\psfig{figure=ps/rel_clauses-files/NbetaDnx.ps,height=10.0cm}}
\end{tabular}
\caption{\label{trans_rel_clause_trees3} Determiner tree with {\bf $<$rel-clause$>$} feature: $\beta$Dnx}
\end{figure}

As it stands, the relative clause analysis sketched so far will combine in
two ways with the Determiner tree shown in
Figure~(\ref{trans_rel_clause_trees3}),%
%
\footnote{The determiner tree shown has the {\bf $<$rel-clause$>$} feature
built in. The relative clause analysis would give two parses in the absence
of this feature.%
%
} giving us both the possiblities shown in \ex{-3} and \ex{-2}. In order to
block the structure exemplified in \ex{-3}, the feature {\bf
$\langle$rel-clause$\rangle$} is used in combination with the following
equations.

\begin{itemize}
\item {\bf NP$_{r}$.b:$\langle$rel-clause$\rangle=+$} (on the Relative Clause)
\item {\bf NP$_{f}$.t:$\langle$rel-clause$\rangle=-$} (on the Determiner tree)
\end{itemize}

Together, these equations block introduction of the determiner above the
relative clause.

\section{Other Issues}

\subsection{Reduced Relatives}
The analysis presented above accounts for reduced relatives (which are
commonly treated as derived from relative clauses through deletion of the
relative pronoun and if there is a {\em be}, then deletion of that
also). Reduced relatives are permitted only in cases of subject-extraction.
Past participial reduced relatives are only permitted on passive clauses.
See \ex{1}-\ex{8}.


\enumsentence{
the gypsy [ $\epsilon_w$ [ $\epsilon$ playing the banjo ]]]
}
\enumsentence{
*the instrument [ $\epsilon_w$ [ Amis playing $\epsilon$ ]]]
}
\enumsentence{
*the day [ $\epsilon_w$ [ Amis playing the banjo ]]]
}
\enumsentence{
the apple [ $\epsilon_w$ [ $\epsilon$ eaten by Dafna ]]]
}
\enumsentence{
*the child [ $\epsilon_w$ [ the apple eaten by $\epsilon$ ]]]
}
\enumsentence{
*the day [ $\epsilon_w$ [ Amis eaten the apple ]]]
}
\enumsentence{
*the apple [ $\epsilon_w$ [ Dafna eaten $\epsilon$ ]]]
}
\enumsentence{
*the child [ $\epsilon_w$ [ $\epsilon$ eaten the apple ]]]
}

These restrictions are built into the {\bf $<$mode$>$} specifications
of {\bf S$_r$.t}, as explained in Section~\ref{sec:mode-restriction}.

\subsubsection{Restrictive vs. Non-restrictive relatives}

The English XTAG grammar does not contain any  syntactic distinction between
restrictive and non-restrictive relatives because we believe this to
be a semantic and/or pragmatic difference.



\subsection{Stacking of Complementizers}

Complementizers are prevented from stacking, as in example \ex{1}, just as
in sentential complementation.

\enumsentence{*the book [ $\epsilon_w$ [ that [ that [ Muriel wrote
$\epsilon$ ]]]]}

\subsection{Adjunction on PRO}
Adjunction on PRO, which would yield the ungrammatical \ex{1} is blocked.

\enumsentence{*I want [[PRO [ who Muriel likes ] to read a book ]].}

This is done by setting the {\bf $<$case$>$} feature of {\bf NP$_{f}$}
(foot node of the relative clause treee) to be {\bf nom/acc}. The {\bf
$<$case$>$} feature of PRO is {\bf none}. This leads to a feature clash and
blocks adjunction of relative clauses onto PRO.

\subsection{Adjunct relative clauses}
\label{sec:adju-RC}
Two types of trees to handle adjunct relative clauses exist in the XTAG
grammar: one in which there is {\bf PP$_{w}$} substitution and one in which
there is a null {\bf NP$_{w}$} built in and a {\bf COMP} adjoins in. There
is no {\bf NP$_{w}$} substitution tree. This is because of the contrast
between \ex{1} and \ex{2}.  

\enumsentence{the horse [[on whose back ] [ Muriel rode away ]]]}
\enumsentence{*the horse [[whose back ] [ Muriel rode away ]]]}

In general, adjunct relatives are not possible with an overt {\bf
NP$_{w}$}.  We do not consider \ex{1} and \ex{2} to be counterexamples
to the above statements because we consider {\em where} and {\em when} to
be exhaustive {\bf PP}s that head a {\bf PP} initial tree.

\enumsentence{the place [ where [ Muriel wrote her first book ]]]}
\enumsentence{the time [ when [ Muriel lived in Bryn Mawr ]]]}

\subsection{ECM}
Cases where {\em for} assigns exceptional case (cf. \ex{1}, \ex{2}) are
handled, again parallel to the way ECM is done in sentential complementation.

\enumsentence{a book [ $\epsilon_w$ [ for [ him to read $\epsilon$ ]]]}
\enumsentence{the time [ $\epsilon_w$ [ for [ her to leave Haverford ]]]}

The assignment of case by {\em for} is implemented by a combination of the
following equations:

\begin{itemize}
\item {\bf S$_{r}$.b:$\langle$assign-case$\rangle$=acc} (in the {\it for} $\beta$COMPs tree)
\item {\bf S$_{r}$.b:$\langle$assign-case$\rangle =$
NP$_{0}$.t:$\langle$case$\rangle$} (in the Relative clause tree)
\end{itemize}

\section{Cases not handled}
\subsection{Partial treatment of free-relatives}
Free relatives are only partially handled. All free relatives in
non-subject positions and some free relatives on subject positions are
handled. The structure assigned to free relatives treats the extracted {\em
wh}-NP as the head NP of the relative clause. The remaining relative clause
modifies this extracted {\em wh}-NP (cf. \ex{1}-\ex{3}).

\enumsentence{what(ever) [ $\epsilon$$_{w_{i}}$ [ Mary likes $\epsilon$$_{i}$ ]]]}
\enumsentence{where(ever) [ $\epsilon$$_{w}$ [ Mary lives ]]]}
\enumsentence{who(ever) [ $\epsilon$$_{w_{i}}$ [ Muriel thinks [ $\epsilon$$_{i}$ likes Mary ]]]]}

However, simple subject extractions without further embedding are not
handled (cf. \ex{1}).

\enumsentence{who(ever) [ $\epsilon$$_{w_{i}}$ [ $\epsilon$$_{i}$ likes Bill ]]]}
This is because \ex{0} is treated exactly like the ungrammatical \ex{1}.
\enumsentence{*the person [ $\epsilon$$_{w_{i}}$ [ $\epsilon$$_{i}$ likes Bill ]]]}


\subsection{Adjunct P-stranding}
The following cases of adjunct preposition stranding are not handled 
(cf. \ex{1}, \ex{2}).

\enumsentence{the pen Muriel wrote this letter with}
\enumsentence{the street Muriel lives on}

Adjuncts are not built into elementary trees in XTAG. So there is no
clean way to represent adjunct preposition stranding. A better
solution might, probably, be available if we make use of multi-component
adjunction. 

\subsection{Overgeneration}
The following types of ungrammatical examples are currently accepted by
the XTAG grammar. This is because no clean and conceptually attractive way
of ruling them out is obvious to us yet.

\subsubsection{{\em how} as {\em wh}-NP}
In standard American English, {\em how} is not acceptable as a 
relative pronoun (cf. \ex{1}).

\enumsentence{*the way [ how [ PRO to solve this problem ]]]}

However, \ex{0} is accepted by the current grammar.
The only way to rule \ex{0} out would be to introduce a special feature
devoted to this purpose. This is unappealing. Further, there exist
speech registers/dialects of English, where \ex{0} is acceptable. 

\subsubsection{Internal head constraint}
Relative clauses in English (and in an overwhelming number of languages)
obey a `no internal head' constraint. This constraint is exemplified in
the contrast between \ex{1} and \ex{2}.

\enumsentence{the person [ who$_{i}$ Muriel likes $\epsilon_i$ ]]}
\enumsentence{*the person [[which person ]$_{i}$ Muriel likes $\epsilon_i$ ]]}

We know of no good way to rule \ex{0} out, while still ruling \ex{1} in.
\enumsentence{the person [[whose mother ]$_{i}$ Muriel likes $\epsilon_i$ ]]}

Dayal (1996) suggests that `full' NPs such as {\em which person} and
{\em whose mother} are R-expressions while {\em who} and {\em whose}
are pronouns. R-expressions, unlike pronouns, are subject to Condition C.
\ex{-2} is, then, ruled out as a violation of Condition C since {\em 
the person} and {\em which person} are co-indexed and {\em the person}
c-commands {\em which person}. If we accept Dayal's argument, we 
have a principled reason for allowing overgeneration of relative clauses
that violate the internal head constraint, the reason being that 
the XTAG grammar does generate binding theory violations.

\subsubsection{Overt COMP constraint on stacked relatives}
Stacked relatives of the kind in \ex{1} are handled.

\enumsentence{ [[the book [ that Bill likes ]] [ which Mary wrote ]]}

However, there is a constraint on stacked relatives: all but the relative
clause closest to the head-NP must have either an overt {\bf COMP} or an
overt {\bf NP$_{w}$}. Thus \ex{1} is ungrammatical.

\enumsentence{*[[the book [ that Bill likes ]] [ Mary wrote ]]}

We currently know of no good way of handling this constraint, and \ex{0} is
incorrectly accepted by XTAG.
