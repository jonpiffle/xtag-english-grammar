\chapter{Resultatives}
\label{result_clauses}

Resultatives constructions in English XTAG consist of verb-preposition and
verb-adjective combinations.\footnote{%
%
Currently, we do not handle possible verb-noun combinations, as in examples
\ex{1} and \ex{2},
\enumsentence{She painted the barn a weird shade of red.}
\enumsentence{They ran their sneakers a dingy shade of grey.}%
%
} The verbs that can form the resultatives are transitive, intransitive and
ergative verbs that allow for either a prepositional or adjectival
secondary predicate which expresses the result of an event denoted by the
verb. In the examples in \ex{1} -- \ex{3}, the (a) sentences show the
non-resultative declarative construction, whereas the (b) and (c) sentences
show the resultative counterparts with adjectival and prepositional
predicates respectively.

\begin{itemize}
\item {\bf Transitive:}
\enumsentence{John pounded the dough.}
\enumsentence{John pounded the dough flat.}
\enumsentence{John pounded the dough into a flat shape.}

\item {\bf Intransitive:}

\enumsentence{The joggers ran (*their sneakers).}
\enumsentence{The joggers ran their sneakers threadbare.}
\enumsentence{The joggers ran their sneakers into pieces.}

\item {\bf Ergative:}
\enumsentence{The river froze.}
\enumsentence{The river froze solid.}
\enumsentence{The river froze into one big piece of ice.}

\end{itemize}

The resultative construction shows the following properties:  

\begin{itemize}

\item The selection by the verb of the result AP/PP is semantically
restricted, so that in a sentence like \ex{-7}, the verb and the result
adjective predicate seem to form a complex predicate. As can be seen from
the oddness of \ex{2}, the not any adjective can appear as the result
predicate.

\enumsentence{\#John {\it pounded} the dough {\it smooth}}

\item The post-verbal NP behaves like an argument, at least in the case of
transitive resultatives \cite{carrier92}.%
\footnote{The data is less decisive for the case of intransitive
resultatives.%
%
} It receives the same theta-role and obeys the same selectional
restrictions as the direct object in the non resultative counterpart. For
example, middle formation and adjectival passive formation which can only
apply to verbs that have a direct internal argument also apply to
resultatives.

\begin{itemize}

\item {\bf Middle Formation:} 

\enumsentence{New seedlings water flat easily.}

\item {\bf Adjectival Passive Formation:}
 
\enumsentence{The smashed-open safe.}

\end{itemize}

\end{itemize}
	
Two different analyses follow from the research on resultatives. One
analysis provides a treatment in which the entire predicate forms a small
clause argument of the verb (e.g. \cite{hoekstra88}). The second analysis
gives a ternary branching structure to the resultatives in which both the
post verbal NP and the result XP are sisters to the verb
(\cite{carrier92}).

The XTAG treatment of resultatives is essentially identical to the ternary
analysis. The properties of the resultative construction seem to be best
captured by the ternary analysis, which explains the argument-like behavior
of the post-verbal NP and the semantic relation between the verb and result
predicate. Furthermore, a small clause analysis with the post verbal NP as
the small clause subject seems problematic in view of the fact that the
thematic relationship of the verb with the post verbal NP in resultatives
is the same as with the direct object in the non-resultatives.

There are four tree families in the XTAG grammar for representing the
resultatives. The tree families (with their corresponding derived
constructions) are selected by verb/adjective and verb/preposition pairs to
capture the lexical restrictions that hold between the verb and the
adjectival/prepositional predicate. The trees, therefore, are
multi-anchored.%
\footnote{Even though the meaning of the resultative construction may be
constructed compositionally in many cases, the set of allowed
verb/adjective and verb/preposition combinations is highly restricted.%
%
} The four tree families are listed below.   

\begin{itemize}

\item {\bf TRnx0Vnx1Pnx2:} selected by transitive/intransitive verbs with
prepositional result predicates. 

\item {\bf TRnx0Vnx1A2:} selected by transitive/intransitive verbs with
adjectival result predicates.

\item {\bf TREnx1VA2:} selected by ergative verbs with adjectival result
predicates.

\item {\bf TREnx1VPnx2:} selected by ergative verbs with prepositional
result predicates.

\end{itemize}

Figures \ref{result-tree}(a) and \ref{result-tree}(b) show the tree
selected by the transitive/intransitive verbs with adjectival result
predicates and the tree selected by ergative verbs with prepositional
result predicates, respectively.

\begin{figure}[htb]
\centering
\begin{tabular}{cc}
{\psfig{figure=ps/resultative-files/alphanx0RVnx1A.ps,height= 4in}} &
{\psfig{figure=ps/resultative-files/alphanx1RVPnx2.ps,height= 4in}}
\end{tabular}
\caption{Resultative trees with adjectival result predicates, $\alpha$Rnx0Vnx1A2
(a) and prepositional result predicates, $\alpha$REnx1VPnx2 (b)}
\label{result-tree} 
\end{figure}
