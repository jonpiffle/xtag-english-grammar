\section{Time Noun Phrases}
\label{timenps}

%% \subsection{Introduction}


Although in general NPs cannot modify clauses, there is a class of
NPs, with meanings that relate to time, that can do so.\footnote{
  There may be other classes of NPs, such as directional phrases, such
  as {\em north, south} etc., which behave similarly. We have not yet
  analysed these phrases.} We call this class of NPs ``Time~NPs''.
Time~NPs behave essentially like PPs. Like PPs, time~NPs can adjoin at
four places: to the right of an NP, to the right and left of a VP, and
to the left of an~S.

Time~NPs may include determiners, as in {\em this month} in example
(\ex{1}), or may be single lexical items as in {\em today} in example
(\ex{2}).  Like other NPs, time~NPs can also include adjectives, as in
example (\ex{6}).

%% \enumsentence{I go there {\ul every month}}
%% \enumsentence{I am free {\ul today}}
\enumsentence{Elvis left the building \underline{this week}}
\enumsentence{Elvis left the building \underline{today}}
\enumsentence{It has no bearing on our work force \underline{today} (WSJ)}
\enumsentence{The fire \underline{yesterday} claimed two lives}
%% In early trading in Tokyo Tuesday, the Nikkei index rose 35.28
%% points to 35452.72.
%% \enumsentence{It has no bearing \underline{today} on our work force}
\enumsentence{\underline{Today} it has no bearing on our work force}
\enumsentence{Michael \underline{late yesterday} announced a buy-back program}
%% Michael late yesterday announced a $ 3.8 million stock buy-back program .

The XTAG analysis for time~NPs is fairly simple, requiring only the
creation of proper NP auxiliary trees.  Only nouns that can be part of
time~NPs will select the relevant auxiliary trees, and so only these
type of NPs will behave like PPs under the XTAG analysis.
Currently, about 60 words select Time~NP trees, but since these
words can form NPs that include determiners and adjectives, a large
number of phrases are covered by this class of modifiers.

Corresponding to the four positions listed above, time~NPs
can select one of the four trees shown in Figure~\ref{timenp-trees}.

\begin{figure}[htb]
\centering
\begin{tabular}{ccccccc}
{\psfig{figure=ps/timenp-files/betaNs.ps,height=1.5in}}
& \hspace{.5in} &
{\psfig{figure=ps/timenp-files/betaNvx.ps,height=1.5in}}
&  \hspace{.5in} &
{\psfig{figure=ps/timenp-files/betavxN.ps,height=1.5in}}
&  \hspace{.5in} &
{\psfig{figure=ps/timenp-files//betanxN.ps,height=1.5in}}\\
$\beta$Ns&&$\beta$Nvx&&$\beta$vxN&&$\beta$nxN\\
\end{tabular}
\caption{Time Phrase Modifier trees: $\beta$Ns, $\beta$Nvx, $\beta$vxN, $\beta$nxN}
\label{timenp-trees}
\end{figure}

Determiners can be added to time~NPs by adjunction in
the same way that they are added to NPs in other
positions. The trees in Figure~\ref{everymonth} show that the
structures of examples (\ex{-5}) and (\ex{-4}) differ only in the
adjunction of  {\em this} to the time~NP in example (\ex{-5}).

\begin{figure}[htb] 
\centering 
\begin{tabular}{ccc}
\psfig{figure=ps/timenp-files/elvis-thisweek.ps,height=3in}
& \hspace{.5in} &
\psfig{figure=ps/timenp-files/elvis-today.ps,height=3in} \\
\end{tabular}\\
\caption{Time~NPs with and without a determiner} 
\label{everymonth}
\end{figure}

\newpage

The sentence 
\enumsentence{Esso said the Whiting field started production Tuesday (WSJ)} 
has (at least) two different interpretations, depending on whether
{\em Tuesday} attaches to {\em said} or to {\em started}. 
Valid time~NP analyses are available for both these interpretations and 
are shown in Figure~\ref{esso}.

\begin{figure}[htb] \centering \begin{tabular}{ccc}
{\psfig{figure=ps/timenp-files/EssoSaidTuesday.ps,height=3.0in}} & \hspace{.5in} &
{\psfig{figure=ps/timenp-files/EssoStartedTuesday.ps,height=3.0in}} \\ \end{tabular}\\
\caption {Time~NP trees: Two different attachments} \label{esso}
\end{figure}

Example (\ex{0}) shows that there are cases of genuine ambiguity that will
be properly represented with multiple valid parses under our
analysis. 

Derived tree structures for examples (\ex{-4}) -- (\ex{-1}), which
show the four possible time~NP positions are shown in
Figures~\ref{bearingtrees} and \ref{lateyesterday}.  The derivation
tree for example (\ex{-1}) is also shown in
Figure~\ref{lateyesterday}.

\begin{figure}[htb] \centering \begin{tabular}{ccccc}
{\psfig{figure=ps/timenp-files/bearingENDtoday.ps,height=3.0in}} & \hspace{.1in} &
{\psfig{figure=ps/timenp-files/thefireyesterday.ps,height=2.2in}} &  \hspace{.1in} &
%% {\psfig{figure=ps/timenp-files/bearingtoday.ps,height=3.0in}} &  \hspace{.1in} &
{\psfig{figure=ps/timenp-files/todaybearing.ps,height=3.0in}} \\ \end{tabular}\\
\caption {Time~NPs in different positions
($\beta$vxN, $\beta$nxN and $\beta$Ns)} \label {bearingtrees}
\end{figure}

\begin{figure}[htb] 
\centering
\begin{tabular}{ccc}
\psfig{figure=ps/timenp-files/lateyesterday.ps,height=3in} &
\hspace{.02in} &
\psfig{figure=ps/timenp-files/DERIVlateyesterday.ps,height=1.25in} \\
\end{tabular}
\caption {Time~NPs: Derived tree and Derivation ($\beta$Nvx position)} 
\label{lateyesterday} 
\end{figure}

