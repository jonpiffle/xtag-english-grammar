\section{Verb Transformations}

The verbs classes described in Section \ref{verb-classes} each have a number of
trees associated with them.  These trees are the various related forms
(transformations) of sentences that verbs in that class can undergo.  This
section describes the basic structure and functionality of each of these
transformations.  Example trees will be given, although they will, of course,
need to be extrapolated for any other tree family.

\subsection{Yes/No questions}

Yes/No questions are formed by the adjunction of an auxiliary verb to the left
side of the S node of the declarative or passive trees.  This mechanism is
described in more detail in the section on auxiliary verbs
(Section~\ref{yesno-questions}).

\subsection{Wh-moved subject}

The wh-moved subject tree provides for sentences such as {\it Who left?}, {\it
Who wrote the paper?}, and {\it Who was happy?}, depending on the tree family
with which it is associated.  Because wh+ subject do not require an auxiliary
verb, it is difficult to tell whether the subject has moved out of position or
not (see \ref{ask-Beth} for arguments for and against moved subject).  However,
we find that the wh+ subjects pattern like other wh+ elements that explicitly
move to the beginning of the sentence, and unlike wh+ elements that stay at
their place of origin (as in echo-questions, such as {\it John likes
*who*?}\footnote{The word surrounded by the *'s is intended to be read as
highly stressed.}).  This becomes apparent when looking at embedded sentences,
which are generally not allowed to have wh-moved elements ($\ast${\it John
thought that who$_i$ Mary loved t$_i$}), with the exception of specific verbs
which require it (John wondered who Mary loved).  This pattern switches with
regard to echo questions ({\it John thought that Mary loved *who*?} vs
$\ast${\it John wondered *who* Mary loved}).  Wh+ subjects pattern like other
wh-moved elements ($\ast$ {\it John thought that who$_i$ T$_i$ loved
Mary}\footnote{No emphasis on {\it who}, which would change it into an
echo-question reading} {\it John wondered who loved Mary}.  Accordingly, the
tree for the wh+ subject has the NP node extracted to a higher level, and is
contrained to be {\bf <wh> = +}.  In addition, the highest S node is marked
{\bf <extracted> = +}.  It is this extracted value that is used to regaulate
the occurance of these trees in embedded sentences.




\subsection{Wh-moved NP complement}
\subsection{Wh-moved S complement}
\subsection{Wh-moved Adjective complement}
\subsection{Wh-moved object of a PP}
\subsection{Wh-moved PP}
\subsection{Topicalized NP complement}
\subsection{Imperative}
\subsection{Determiner gerund}
\subsection{NP gerund}
\subsection{Relative clause on subject}
\subsection{Relative clause on NP complement}
\subsection{Passive with wh-moved subject (with and without {\it by} phrase)}
\subsection{Passive with wh-moved indirect object (with and without {\it by} phrase)}
\subsection{Passive with wh-moved object of the {\it by} phrase}
\subsection{Passive with wh-moved {\it by} phrase}

