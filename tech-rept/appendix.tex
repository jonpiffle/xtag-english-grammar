\appendix
\section{Tree Naming conventions}

The names of tree families and the names of the trees within tree families havetwo sections, a prefix and a body. The prefix must contain one of the following:

\begin{tabular}{ll}
T&Tree family\\
alpha&initial tree\\
beta&auxiliary tree\\
\end{tabular}

Following an $\alpha$ or a $\beta$ can additionally contain one of:

\begin{tabular}{ll}
I&imperative\\
E&ergative\\
R{0,1,2}&wh-relative clause\{position\}\\
N{0,1,2}&non-wh relative clause\{position\}\\
G&genitive\\
pW{0,1,2}&wh-PP extraction\{position\}\\
W{0,1,2}&wh-NP extraction\{position\}\\
\end{tabular}

Where the numbering of positions is:
\begin{tabular}{ll}
0&subject position\\
1&first argument (e.g. direct object)\\
2&second argument (e.g. indirect object)\\
\end{tabular}

The body of the name consists of a string of the following components
used to designate the leaves of the tree:
\begin{tabular}{ll}
s&sentence\\
a&adjective\\
arb&adverb\\
BE&"be"\\
x&phrasal category\\
d&determiner\\
v&verb\\
lv&light verb\\
conj&conjunction\\
comp&complementizer\\
g&genitive (e.g. 's)\\
It&"it"\\
n&noun\\
p&preposition\\
pl&particle\\
CAPS&anchor\\
by&"by"\\
neg&negation\\
\end{tabular}




\section{Features}

\begin{tabular}{|l|l|}
\hline
Feature&Value\\
\hline
\hline
agr 3rdsing&$+,-$\\
agr num&plur,sing\\
agr pers&1,2,3\\
assign-case&nom,acc,none\\
assign-comp&that,whether,if,for,rel,inf\_nil,ind\_nil\\
card&$+,-$\\
case&nom,acc,none\\
comp&that,whether,if,for,rel,inf\_nil,ind\_nil\\
conditional&$+,-$\\
decreas&$+,-$\\
definite&$+,-$\\
displ-const&$+,-$\\
extracted&$+,-$\\
gen&$+,-$\\
inv&$+,-$\\
invlink&\\
mainv&$+,-$\\
mode&base,ger,ind,inf,imp,nom,ppart,prep,sbjunt\\
neg&$+,-$\\
passive&$+,-$\\
perfect&$+,-$\\
pred&$+,-$\\
predet&$+,-$\\
progressive&$+,-$\\
pron&$+,-$\\
quan&$+,-$\\
tense&pres,past\\
trace&no value, indexing only\\
trans&$+,-$\\
wh&$+,-$\\
\hline
\end{tabular}
