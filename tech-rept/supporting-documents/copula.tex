\documentstyle [12pt]{article}
\pagestyle{myheadings}
\newcommand{\marginfix}{
\setlength{\oddsidemargin}{0.1in}
\setlength{\evensidemargin}{0.1in}
\setlength{\topmargin}{-37pt}
\setlength{\textwidth}{6.5in}
\setlength{\textheight}{9in}
}
\marginfix

\begin{document}

\markright{Progress report: the copula in LTAG}

\begin{center}
{\LARGE Progress Report: The English Copula in a LTAG}
\end{center}
\vspace{2mm}
\begin{center}
{\large \today} \\
{\large Caroline Heycock} \\
\end{center}
\vspace{4mm}

\section{Facts}

\label{facts}

\subsection{The Syntactic Behavior of the Copula}

\label{cop}
The verb {\em be} in English that occurs in the following sentences is
often referred to as the {\em copula:}
\begin{tabbing}
aaa\=(@1)\= a. \= *\= \kill
   \>(+1)\> a. \>  \> Kate is furious. \\
   \>   \> b. \>  \> Kate is a Socialist. \\
   \>   \> c. \>  \> Kate is in a foul mood.
\end{tabbing}
As shown in (+1), the copula may be followed by an AP, a NP, or a PP.
The set of adjectives that can head the AP following copular {\em be} and
the set of adjectives that can occur as modifiers of nouns is not
identical, although there is massive overlap:
\begin{tabbing}
aaa\=(@2)\= a. \= *\= \kill
   \>(+2)\> a. \>  \> The victim was alive. \\
   \>    \> b. \> *\> An alive victim lay amid the rubble. \\
   \>    \> c. \>  \> The former candidate still took an interest in the
                      race. \\
   \>    \> d. \> *\> The candidate was now former.
\end{tabbing}
Restrictions on the complement of the copula may be independent of the
copula itself, however: the same pattern is found in the following
construction despite the absence of {\em be:}
\begin{tabbing}
aaa\=(@3)\= a. \= *\= \kill
   \>(+3)\> a. \>  \> The doctors considered him alive, although he
                       was in a deep coma. \\
   \>    \>  b. \> *\> The analysts all considered the candidate former.
\end{tabbing}
In addition, any semantic restrictions on the subject of the copula
appear to be imposed by the complement rather than the copula itself:
\begin{tabbing}
aaa\=(@4)\= a. \= ?\= \kill
   \>(+4)\> a. \> ?\> The garrulous cliff loomed over the cottage. \\
   \>    \>  b. \> ?\> They considered the cliff garrulous \\
   \>    \>  c. \> ?\> The cliff was garrulous.
\end{tabbing}
\begin{tabbing}
aaa\=(@5)\= a. \= ?\= \kill
   \>(+5)\> a. \>  \> The garrulous bartender leaned over the counter. \\
   \>    \>  b. \>  \> They considered the bartender garrulous \\
   \>    \>  c. \>  \> The bartender was garrulous.
\end{tabbing}
The ``inheritance'' of restrictions on the subject is also found in some 
auxiliaries, and raising verbs like {\em seem\/}.

The syntactic behavior of copular {\em be} is exactly the same as that of
auxiliary {\em be\/}, and contrasts with that of ordinary main verbs:
\begin{enumerate}
 
\item Copular {\em be} inverts with the subject:
\begin{tabbing}
aaa\=(@6)\= a. \= *\= \kill
   \>(+6)\> a. \>  \> {\em Was Rome} built in a day? \\ 
   \>    \>  b. \>  \> {\em Is Cuomo} running? \\
   \>    \>  c. \>  \> {\em Is Dianna} a journalist? \\
   \>    \>  d. \>  \> {\em What was} the Berlin wall, Mummy? \\
   \>    \>  e. \> *\> {\em Wrote Dianna} this story? \\
   \>    \>  f. \> *\> What {\em saw Terry\/}?
\end{tabbing}

\item Tensed copular {\em be} occurs to the left of the negative marker 
      {\em not\/}:
\begin{tabbing}
aaa\=(@7)\= a. \= *\= \kill
   \>(+7)\> a. \>  \> Rome {\em was not} built in a day. \\ 
   \>    \>  b. \>  \> Jackson {\em is not} running. \\
   \>    \>  c. \>  \> David {\em is not} a journalist. \\
   \>    \>  d. \>  \> The revolution {\em was not} violent. \\
   \>    \>  e. \> *\> Dianna {\em wrote not} this story.
\end{tabbing}

\item Tensed copular {\em be} can host the contracted negative marker
{\em n't\/}:
\begin{tabbing}
aaa\=(@8)\= a. \= *\= \kill
   \>(+8)\> a. \>  \> Rome {\em wasn't} built in a day. \\ 
   \>    \>  b. \>  \> Jackson {\em isn't} running. \\
   \>    \>  c. \>  \> David {\em isn't} a journalist. \\
   \>    \>  d. \>  \> The revolution {\em wasn't} violent. \\
   \>    \>  e. \> *\> Dianna {\em wroten't} this story.
\end{tabbing}

\item Tensed copular {\em be} occurs to the left of adverbs in the
unmarked order:
\begin{tabbing}
aaa\=(@9)\= a. \= *\= \kill
   \>(+9)\> a. \>  \> Rome {\em was often} attacked. \\
   \>    \>  b. \>  \> Jackson {\em is usually} running for something. \\
   \>    \>  c. \>  \> David {\em was never} a cricket fan. \\
   \>    \>  d. \>  \> The revolution {\em was occasionally} violent. \\
   \>    \>  e. \> *\> Dianna {\em seems usually} enthusiastic.
\end{tabbing}

\item Morphological irregularities conditioned by syntactic context are
the same for copular and auxiliary {\em be:}
\begin{tabbing}
aaa\=(@10)\= a. \= *\= \kill
   \>(+10)\> a. \> *\> I amn't impressed by his reputation. \\
   \>    \>  b. \> *\> Amn't I helping? \\
   \>    \>  c. \>  \> Aren't I helping? \\
   \>    \>  d. \> *\> I amn't a great Delius fan. \\
   \>    \>  e. \> *\> Amn't I a help? \\
   \>    \>  f. \>  \> Aren't I helpful?
\end{tabbing}

\end{enumerate}

Unlike all other auxiliaries, however, copular {\em be} (by definition)
is not followed by a verbal category, and therefore may be the only verb
in a clause.  In this respect it is like a regular main verb.

\subsection{Other Possibly Related Constructions}

\label{other}

\subsubsection{Raising Constructions}

\label{raise}
As pointed out in Section~\ref{cop} the copula appears not to impose any
restrictions on its subject, but rather passes on restrictions imposed
by the predicative phrase that it takes as a complement. This property is
shared by a number of other verbs. The syntactic restrictions that these
verbs impose on their complements, however, vary. Each may be followed
by one out of various subsets of the following%
%
\footnote{I have included passives such as {\em to be considered}
although I do not know how these will be eventually be handled in the
grammar being developed, and no analysis is proposed in this report.}: 
%
\begin{tabbing}
aaa\=(@11)\= a. \= *\= \kill
   \>(+11)\> a. \>  \> {\em to\/}-infinitive      \\
   \>  \>    b. \>  \> NP   \\
   \>  \>    c. \>  \> AP   \\
   \>  \>    d. \>  \> PP
\end{tabbing}
Note that none of these raising verbs may be followed by a bare
infinitive (various proposals have been advanced to account for this
restriction: see \cite{ksh87} and references cited there).
Examples of these verbs and the types of complement they may take are as
follows: 
\begin{tabbing}
aaa\=(@12)\= a. \= *\= Jenny was considered aa\= *\=        \kill
   \>(+12)\>    \>  \> Jenny {\em \bf is}                \\
   \>     \> a. \>  \>                  \>  \> to leave. \\
   \>     \> b. \>  \>                  \>  \> a genius.    \\    
   \>     \> c. \>  \>                  \>  \> angry.    \\    
   \>     \> d. \>  \>                  \>  \> in a foul mood.
\end{tabbing}
\begin{tabbing}
aaa\=(@13)\= a. \= *\= Jenny was considered aa\= *\=        \kill
   \>(+13)\>    \>  \> Jenny {\em \bf seems}                \\
   \>     \> a. \>  \>                  \>  \> to love you. \\
   \>     \> b. \>  \>                  \>  \> a genius.    \\    
   \>     \> c. \>  \>                  \>  \> angry.    \\    
   \>     \> d. \>  \>                  \> ?\> in a foul mood.
\end{tabbing}
Although {\em seem\/} $+$ {\em to\/}-infinitive is the archetypical raising verb, 
it is less clear whether this is the correct analysis for its other uses.   
Crucially, it does not allow expletive {\em there\/} as a
subject:
\begin{tabbing}
aaa\=(@15)\= a. \= *\= \kill
   \>(+15)\>    \> *\> There became a riot 
\end{tabbing}
\begin{tabbing}
aaa\=(@14)\= a. \= *\= Jenny was considered aa\= *\=        \kill
   \>(+14)\>    \>  \> Jenny {\em \bf became\/}       \\
   \>     \> a. \>  \>                  \> *\> to love you. \\
   \>     \> b. \>  \>                  \>  \> a genius.    \\    
   \>     \> c. \>  \>                  \>  \> angry.    \\    
   \>     \> d. \>  \>                  \> *\> in a foul mood.
\end{tabbing}
In fact, {\em become} should quite probably not be considered a
raising verb. Like {\em seem\/} $+$ NP, it does not allow expletive {\em there\/} 
as a subject.
\begin{tabbing}
aaa\=(@16)\= a. \= *\= Jenny was considered aa\= *\=        \kill
   \>(+16)\>    \>  \> Jenny {\em \bf turned out}                \\
   \>     \> a. \>  \>                  \>  \> to love you. \\
   \>     \> b. \>  \>                  \>  \> a genius.    \\    
   \>     \> c. \>  \>                  \>  \> very clever.    \\    
   \>     \> d. \>  \>                  \> *\> in a foul mood.
\end{tabbing}
It is not clear to me that (+16b,c) are in fact grammatical under
the same reading as (+16a). For example, neither
seem to be possible continuations of the following: {\em I was
expecting Jenny to be stupid, but she turned out \ldots\ .} Instead, the
meaning in these cases seems to be closer to {\em become\/}. Hence, if
we decide not to consider {\em become} a raising verb, then 
{\em turn out\/} should be considered not to be a raising verb when it
takes a following NP or AP.
\begin{tabbing}
aaa\=(@17)\= a. \= *\= Jenny was considered aa\= *\=        \kill
   \>(+17)\>    \>  \> Jenny {\em \bf was considered}                \\
   \>     \> a. \>  \>                  \>  \> to love him. \\
   \>     \> b. \>  \>                  \>  \> a genius.    \\    
   \>     \> c. \>  \>                  \>  \> angry.    \\    
   \>     \> d. \>  \>                  \> ?\> in a foul mood.
\end{tabbing}
\begin{tabbing}
aaa\=(@18)\= a. \= *\= Jenny was considered aa\= *\=        \kill
   \>(+18)\>    \>  \> Jenny {\em \bf was believed}                \\
   \>     \> a. \>  \>                  \>  \> to love him. \\
   \>     \> b. \>  \>                  \> *\> a genius.    \\    
   \>     \> c. \>  \>                  \> *\> angry.    \\    
   \>     \> d. \>  \>                  \> *\> in a foul mood.
\end{tabbing}
\begin{tabbing}
aaa\=(@19)\= a. \= *\= Jenny was considered aa\= *\=        \kill
   \>(+19)\>    \>  \> Jenny {\em \bf was made}                \\
   \>     \> a. \>  \>                  \>  \> to love him. \\
   \>     \> b. \>  \>                  \>  \> a genius.    \\    
   \>     \> c. \>  \>                  \>  \> angry.    \\    
   \>     \> d. \>  \>                  \> *\> in a foul mood.
\end{tabbing}
\begin{tabbing}
aaa\=(@20)\= a. \= *\= Jenny was considered aa\= *\=        \kill
   \>(+20)\>    \>  \> Jenny {\em \bf was heard}                \\
   \>     \> a. \>  \>                  \>  \> to abuse him. \\
   \>     \> b. \>  \>                  \> *\> an orator.    \\    
   \>     \> c. \>  \>                  \> *\> loud.    \\    
   \>     \> d. \>  \>                  \> *\> in a foul mood.
\end{tabbing}

The only correlation that I can establish is that a raising predicate
may be followed by a NP if and only if it may be followed by an AP.


\subsubsection{Small Clause Arguments}

\label{sc}
One way of describing the constructions illustrated in the last section
is to say that---among other things---they involve predication without
the use of the copula. As suggested by (+3) above, predicative NPs, APs,
and PPs may also be predicated of what appear to be non-subjects, again
without the appearance of the copula.  One
motivation for analyzing the constructions in the previous section as
involving raising of the subject out of some kind (or kinds) of
clausal constituent that the raising verb takes as a complement is the
idea that the same kind(s) of clausal constituents appear as the
complements to verbs like {\em consider\/}. 

Again we find that verbs differ in the types of complements they allow:
different verbs may allow an NP followed by one out of various subsets
of the following:
\begin{tabbing}
aaa\=(@21)\= a. \= *\= \kill
   \>(+21)\> a. \>  \> {\em to\/}-infinitive      \\
   \>  \>    b. \>  \> bare infinitive   \\
   \>  \>    c. \>  \> NP   \\
   \>  \>    d. \>  \> AP   \\
   \>  \>    e. \>  \> PP  
\end{tabbing}
Note that here bare infinitives are possible.

Examples of these verbs and the types of complements they may take are
as follows:
\begin{tabbing}
aaa\=(@22)\= a. \= *\= I considered Jenny \= *\=        \kill
   \>(+22)\>    \>  \> I {\em \bf consider} Jenny               \\
   \>     \> a. \>  \>                  \>  \> to abuse him. \\
   \>     \> b. \>  \>                  \> *\> abuse him. \\
   \>     \> c. \>  \>                  \>  \> an orator.    \\    
   \>     \> d. \>  \>                  \>  \> loud.    \\    
   \>     \> e. \>  \>                  \> ?\> in a foul mood.
\end{tabbing}
\begin{tabbing}
aaa\=(@23)\= a. \= *\= I considered Jenny \= *\=        \kill
   \>(+23)\>    \>  \> I {\em \bf believe} Jenny               \\
   \>     \> a. \>  \>                  \>  \> to abuse him. \\
   \>     \> b. \>  \>                  \> *\> abuse him. \\
   \>     \> c. \>  \>                  \> *\> an orator.    \\    
   \>     \> d. \>  \>                  \> *\> loud.    \\    
   \>     \> e. \>  \>                  \> *\> in a foul mood.
\end{tabbing}
\begin{tabbing}
aaa\=(@24)\= a. \= *\= I considered Jenny \= *\=        \kill
   \>(+24)\>    \>  \> I {\em \bf made} Jenny               \\
   \>     \> a. \>  \>                  \> *\> to abuse him. \\
   \>     \> b. \>  \>                  \>  \> abuse him. \\
   \>     \> c. \>  \>                  \>  \> an orator.    \\    
   \>     \> d. \>  \>                  \>  \> angry.    \\    
   \>     \> e. \>  \>                  \> *\> in a foul mood.
\end{tabbing}
\begin{tabbing}
aaa\=(@25)\= a. \= *\= I considered Jenny \= *\=        \kill
   \>(+25)\>    \>  \> I {\em \bf let} Jenny               \\
   \>     \> a. \>  \>                  \> *\> to abuse him. \\
   \>     \> b. \>  \>                  \>  \> abuse him. \\
   \>     \> c. \>  \>                  \> *\> an orator.    \\    
   \>     \> d. \>  \>                  \> *\> angry.    \\    
   \>     \> e. \>  \>                  \> *\> in a foul mood.
\end{tabbing}
\begin{tabbing}
aaa\=(@26)\= a. \= *\= I considered Jenny \= *\=        \kill
   \>(+26)\>    \>  \> I {\em \bf heard} Jenny               \\
   \>     \> a. \>  \>                  \> *\> to abuse him. \\
   \>     \> b. \>  \>                  \>  \> abuse him. \\
   \>     \> c. \>  \>                  \> *\> an orator.    \\    
   \>     \> d. \>  \>                  \> *\> angry.    \\    
   \>     \> e. \>  \>                  \> *\> in a foul mood.
\end{tabbing}
Some of the sentences in (+26) that are starred are grammatical on a
different reading, where the predicative phrase is read as a ``secondary
predicate,'' either on the subject or the object NP, as also found in
the following:
\begin{tabbing}
aaa\=(@27)\= a. \= *\= \kill
   \>(+27)\> a. \>  \> I met Jenny drunk. \\
   \>     \> b. \>  \> I met Jenny a teetotaller. \\
   \>     \> c. \>  \> I met Jenny in a bad mood. 
\end{tabbing}
No analysis of secondary predication will be put forward in this report.

Again the only correlation I can establish is that a verb allows a
following AP if and only if it allows a following NP.



\subsection{Non-predicational Uses of the Copula}

\label{nonpred}
The examples of copular sentences in Section~\ref{cop} appear to
indicate that {\em be} is always followed by a predicative phrase of
some sort
%
\footnote{I am using the adjective {\em predicational\/} in an
intuitive, {\em i.e.\/} undefined sense.}.  
% 
It has been observed, however, that {\em be} may be followed by phrases
that do not appear to be predicational.  Examples include the following:
\begin{tabbing}
aaa\=(@28)\=a. \= *\= \kill
   \>(+28)\>a. \>  \> My aunt is Mrs Cripps.      \\
   \>     \>b. \>  \> James is the man with glasses. \\
   \>     \>c. \>  \> The culprit is Ridley.
\end{tabbing}

Notice that corresponding sentences involving the verbs illustrated in
Sections~\ref{raise} and \ref{sc} are ungrammatical%
%
\footnote{Notice, however, that even if {\em consider} is followed by an
infinitive clause containing {\em be\/}, sentences parallel to (@29b)
and (@30a,b) are about equally bad:
\begin{tabbing}
aaaaaa\=(iii) \= *\= \kill
      \>(i)   \> *\> James was considered to be the man with glasses. \\
      \>(ii)  \> *\> They consider my aunt to be Mrs Cripps. \\
      \>(iii) \> *\> Everyone considered James to be the man with glasses. 
\end{tabbing}
This is not true of the sentences with {\em seem\/}, nor of (+30c) (as
discussed in \cite{mor90}---see Section~\ref{mo} below):
\begin{tabbing}
aaaaaa\=(iii) \= *\= \kill
      \>(iv)   \>  \> I consider the culprit to be Ridley.
\end{tabbing}
This may lend indirect support to Moro's idea that the infinitival clause
in (iv) involves predication (the predicate being the apparent subject of
the infinitival clause)---the generalization about {\em consider\/}
would then be that it has to have a predicative clause complement,
unlike, {\em e.g.\/}, {\em believe\/}. I think, however, that the
extreme awkwardness of (i--iii) above should probably not be considered
in terms of syntactic predication.}:
%
\begin{tabbing}
aaa\=(+29)\=a. \= *\= \kill
   \>(+29)\>a. \> *\> My aunt seems Mrs Cripps.      \\
   \>     \>b. \> *\> James was considered the man with glasses. \\
   \>     \>c. \> *\> The culprit seemed Ridley. 
\end{tabbing}
\begin{tabbing}
aaa\=(+30)\=a. \= *\= \kill
   \>(+30)\>a. \> *\> They consider my aunt Mrs Cripps.      \\
   \>     \>b. \> *\> Everyone considered James the man with glasses. \\
   \>     \>c. \> *\> I consider the culprit Ridley. 
\end{tabbing}

The {\em be} that occurs in the sentences in (+28) exhibits all the
syntactic traits listed in (+6)--(+10) in Section~\ref{cop}.


\section{Suggested Analyses of the Copula}

\subsection{Main Verb Raising to INFL + Small Clause}

In \cite{po89} the copula is generated as the head of a VP, like any
main verb such as {\em sing} or {\em picket.\/} Unlike all other main
verbs (with the exception of {\em have} in some British dialects of
English), however, {\em be} moves out of the VP and into Infl in a
tensed sentence.  This analysis aims to account for the behaviour of
{\em be} as an auxiliary in terms of inversion, negative placement, and
adverb placement, while retaining a sentential structure in which {\em
be} heads the main VP at D-Structure and can thus be the only verb in
the clause.

Pollock claims that the predicative phrase is not an argument of {\em
be}, which he assumes to take a Small Clause complement, consisting of a
node dominating an NP and an predicative AP, NP, or PP. The
initial---subject---NP of the Small Clause then raises to become the
subject of the sentence.  This accounts for the failure of the copula to
impose any selectional restrictions on the subject, and, on the
assumption that verbs like {\em consider} take the same type of Small
Clause complement, this also explains the data in (+2) and (+3) above.

\subsection{Auxiliary + Null Copula}

\label{la}
In \cite{la80} the copula is treated as an auxiliary verb that takes as
its complement a VP headed by a passive verb, a present participle, or a
null verb---the true copula. This verb may then take AP, NP, and PP
complements.  The author points out (p.247) that there are many
languages that have been analysed as having a null copula, but that of
course English has the peculiarity that its null copula requires the
copresence of a particular auxiliary---{\em be.\/}

\subsection{Auxiliary + Predicative Phrase}

\label{gpsg}
In GPSG (\cite{gkps85}, \cite{sgww85}) the copula is treated as an
auxiliary verb (thus allowing for its participation in Subject-Aux
Inversion, etc) that takes an X$^{2}$ category with a + value for the HEAD
feature [PRD] (predicative). AP, NP, PP, and VP can all be [+PRD], but a
F(eature) C(o-occurrence) R(estriction) guarantees that a VP[+PRD] will
be headed by a verb that is either passive or a present participle.  One
of the advantages of being able to give an underspecified category as
the complement is that it is useful in an account of co-ordination, the
main topic of \cite{sgww85}. Notice that {\em be} is the only auxiliary
verb that can take a complement headed by anything other than a verb,
but since in this system an auxiliary is simply a type of verb, this
does not cause any inconsistency.

The type of data in Section~\ref{nonpred}, which appear to show that in
some cases the copula takes a non-predicative complement, are not
discussed. 

GPSG follows \cite{ch70} in adopting the binary valued features [V] and
[N] for decomposing the categories V, N, A, and P: V is [+V,$-$N], N is
[$-$V,+N], A is [+V,+N], and P is [$-$V,$-$N]. Consequently, in
\cite[p.141]{sgww85}, the verb {\em become} can be specified as taking a
predicative complement that is [+N]: that is, it can take either a
predicative NP or AP complement. As illustrated in Sections~\ref{raise}
and \ref{sc} above, NP and AP generally pattern together: this fact can
be stated economically using this category decomposition.
\begin{tabbing}
aaa\=(@31)\= a. \= *\= \kill
   \>(+31)\> a. \>  \> Pat has become a Republican. \\
   \>   \> b. \>  \> Gerry became quite conservative. \\
   \>   \> c. \> *\> Connie has become of the opinion that we should
                     get out. \\
   \>   \> d. \> *\> Tracy became awarded a prize. \\
   \>   \> c. \> *\> Chris will become talking to colleagues. 
\end{tabbing}
In neither paper is there any discussion of how to handle the complete
range of complements to a verb like {\em seem}, which takes AP, NP, and
possibly PP complements, as well as infinitives; the only possibility
appears to be to associate the verb with two I(mmediate) D(ominance)
rules (leaving aside the use of the verb with an expletive subject and
sentential complement).

\subsection{Auxiliary + Small Clause}

\label{mo}
In \cite{mor90} the copula is treated as a special functional
category---the spell-out of tense, which is considered to head its own
projection. It takes as a complement the projection of another
functional category, Agr(eement). This projection corresponds roughly to
a Small Clause, and is considered to be the domain within which
predication takes place.  An NP must then raise out of this projection
to become the subject of the sentence: it may be the subject of the
AgrP, or, if the predicate of the AgrP is an NP, this may raise instead.
As well as occurring as the complement of {\em be\/}, AgrP is
selected by certain verbs such as {\em consider\/}. It follows from this
analysis that when the complement to {\em consider} is a simple AgrP, it will
always consist of a subject followed by a predicate, whereas if the
complement contains the verb {\em be}, the predicate of the AgrP may
raise to the left of {\em be} leaving the subject of the AgrP to the
right:
\begin{tabbing}
aaa\=(@32)\= a. \= *\= \kill
   \>(+32)\> a. \>  \> John$_{i}$ is [$_{AgrP}$ $t_{i}$ the culprit ]. \\
   \>    \> b. \>  \> The culprit$_{i}$ is [$_{AgrP}$ John $t_{i}$ ]. \\
   \>    \> c. \>  \> I consider [$_{AgrP}$ John the culprit]. \\
   \>    \> d. \> *\> I consider the culprit John. \\
   \>    \> e. \>  \> I consider [John$_{i}$ to be [$_{AgrP}$ $t_{i}$ 
                      the culprit ]]. \\
   \>    \> f. \>  \> I consider [the culprit$_{i}$ to be [$_{AgrP}$ 
                      John $t_{i}$ ]].
\end{tabbing}

Moro does not discuss a number of aspects of his analysis, including the
nature of Agr and the implied existence of sentences without VPs. He
also does not discuss the derivation of sentences which appear to
consist of the copula flanked by two non-predicative NPs, as illustrated
in Section~\ref{nonpred} and again here:
\begin{tabbing}
aaa\=(@33)\= a. \= *\= \kill
   \>(+33)\> a. \>  \> John is the man with glasses. \\
   \>    \> b. \>  \> The man with glasses is John. \\
   \>    \> c. \> *\> I consider John the man with glasses.\\
   \>    \> d. \> *\> I consider the man with glasses John.
\end{tabbing}

This use of the copula is not discussed in any of the other analyses
outlined here either.  Moro explicitly claims to be giving a unified
account of the copula, but it is possible that in the other analyses
this ``non-predicational'' {\em be} would be treated simply as a
separate lexical item. Notice for example that it is very awkward to
conjoin predicative and non-predicative NPs after {\em be}:
\begin{tabbing}
aaa\=(@34)\= a. \= *\= \kill
   \>(+34)\> a. \>  \> John is the man with glasses. \\
   \>    \> b. \>  \> John is a lawyer. \\
   \>    \> c. \> *\> John is a lawyer and the man with glasses. \\
   \>    \> d. \> *\> John is the man with glasses and a lawyer.
\end{tabbing}

\section{Possible Treatments in an LTAG}
 
\subsection{Main Verb}

\label{main}
At present, to the extent that the English copula is handled in the
system at all, it is handled like a main verb. If adopted
systematically, this type of treatment would associate the copula with
tree families that are selected by other main verbs: for example, we
already have tree families for verbs that take NP and PP complements.
Treating the copula as a transitive verb when it takes a NP complement
clearly does not capture the ``inheritance'' of selectional requirements
on the subject from the NP complement that was pointed out in
Section~\ref{facts}. In addition, because the copula behaves in many
respects like an auxiliary verb, this apparently straightforward
analysis would complicate the grammar unnecessarily.  Some of the
problems are outlined below.
\begin{enumerate}

\item As pointed out in \cite{za89}, this analysis of the copula poses a
problem for the analysis of negation. In the current system, the
distribution of the negative marker {\em not} is captured by making {\em
not} the anchor for an auxiliary tree with a VP root and foot node, that
is restricted to adjoin only into a VP whose bottom set of features does
not include $<$mode$>$=ind.  That is to say, {\em not} can occur only as the
left sister of a VP headed by a non-finite verb.  This analysis
correctly guarantees the ungrammaticality of (@36), but does not allow
for the grammatical construction in (@37):
\begin{tabbing}
aaa\=(+36)\= a. \= *\= \kill
   \>(+36)\> a. \> *\> Terry not is satisfied with his job. \\
   \>    \> b. \>  *\> Beatrice not is a phonologist.
\end{tabbing}
\begin{tabbing}
aaa\=(+37)\= a. \= *\= \kill
   \>(+37)\> a. \>  \> Terry is not satisfied with his job. \\
   \>    \> b. \>   \> Beatrice is not a phonologist.
\end{tabbing}

\item Again like an auxiliary, the copula does not allow 
{\em do\/}-support%
%
\footnote{Except in emphatic imperatives:
\begin{tabbing}
aaaaa\=(ii) \= \kill
     \>(i)  \> {\em Do be a friend and give me a hand with this!} \\
     \>(ii) \> {\em Do be quiet!}
\end{tabbing}
}. 
%
The currently implemented system, however, achieves the correct
distribution of {\em do} by restricting it to take a VP as its
complement that is $<$conditional$>=-$, $<$perfect$>=-$, $<$progressive$>=-$.
Unless a new feature is added, this will result in the generation of
ungrammatical sentences such as:
\begin{tabbing}
aaa\=(@35)\= a. \= *\= \kill
   \>(+35)\> a. \> *\> Isabelle doesn't be a barrister. \\
   \>    \> b. \>  *\> Does Jane be annoyed? 
\end{tabbing}
This objection is not as serious as it may seem, however, since we shall
require a new feature to avoid generating this type of sentences in
whatever analysis we adopt (see Section~\ref{bep}). 
%
%
%\footnote{This suggests to me that our current analysis of {\em do} 
%captures only indirectly the generalization that {\em do} must take a VP
%headed by a main verb as its complement.}.
%

\item As mentioned in Section~\ref{facts}, the English copula undergoes
inversion, a characteristic of auxiliary verbs, not of main verbs.
Reflecting this fact, none of the tree families for main verbs allow for
inversion%
%
\footnote{But see Section~\ref{bep} below for a proposal to introduce a
feature distinguishing main verbs (here meaning those that invert and
that can appear after {\em do\/}) from auxiliaries. This could be used to
restrict inversion to auxiliaries, thus allowing them in principle to
select tree families also selected by main verbs, without overgeneration
of inversion structures.}.
\end{enumerate}

Although this section has only given a brief summary of some of the
problems that would arise if the copula were treated as a type of
transitive verb, it is I think clear that we should try to improve on
this solution.

\subsection{The Copula as a Raising Verb}

\subsubsection{General Considerations}

In Section~\ref{main} it was pointed out that treating the copula like a
transitive verb does not capture its ``inheritance'' properties.
Other verbs that inherit selectional restrictions on their subject from
within their complement are the ``raising verbs'' {\em seem\/}, {\em
tend\/}, etc., and the auxiliaries.  In our LTAG for English both of
these classes are analysed as verbs that take VP complements:  they 
anchor auxiliary trees with VP root and foot nodes, so that when they
adjoin into a clause they ``push'' the subject away from its original
predicate. In G(overnment) B(inding) Theory a ``raising verb'' is one
that allows an element to move out from within its complement, which can
be of any category. Consequently, in a GB analysis which has the copula
as a raising verb taking a Small Clause complement, the structure of the
Small Clause can be assumed to be either along the lines of (@38a--b) or
the more orthodox---in terms of the \={X}-Schema---(@39a--b):
\begin{tabbing}
aaa\=(+38)\= a. \= *\= \kill
   \>(+38)\> a. \>  \> [$_{S}$ [$_{NP}$ Mary] [$_{NP}$ a doctor]]   \\
   \>     \> b. \>  \> [$_{S}$ [$_{NP}$ Mary] [$_{AP}$ decadent]]   \\
   \>  \>    c. \>  \> Mary$_{i}$ is [$_{S}$ $t_{i}$ a doctor]. 
\end{tabbing}
\begin{tabbing}
aaa\=(+39)\= a. \= *\= \kill
   \>(+39)\> a. \>  \> [$_{NP}$ [$_{NP}$ Mary] [$_{NP}$ a doctor]]   \\
   \>  \>    b. \>  \> [$_{AP}$ [$_{NP}$ Mary] [$_{AP}$ decadent]]   \\
   \>  \>    c. \>  \> Mary$_{i}$ is [$_{NP}$ $t_{i}$ a doctor]. 
\end{tabbing}

Under our current assumptions, however, we cannot produce a plausible
LTAG analysis which involves adjoining into a Small Clause with either
of these structures.  In either case we would require the copula to
anchor all of the following trees:
\begin{tabbing}
aaa\=(@40)\= a. \= *\= \kill
   \>(+40)\> a. \>  \> [$_{NP}$ Cop NP]      \\
   \>  \>    b. \>  \> [$_{AP}$ Cop AP]      \\
   \>  \>    c. \>  \> [$_{PP}$ Cop PP]      
\end{tabbing}
If we assumed the type of structure in (+39a--b), the result of
adjoining in the appropriate one of these trees would be a matrix clause
that was an NP, AP, or PP, rather than an S.  If we assumed the type of
structure in (+38a-b), the result of adjoining in a tree from (+40) would
be a matrix clause that was rooted in S, but which had an NP, AP, or PP
rather than a VP.  While this is not {\em a priori} disastrous, it would
then be impossible to adjoin any kind of auxiliaries or negation, since
these are all represented by auxiliary trees that adjoin into a VP node.
As a result, the grammar would not be able to generate any of the
following type of sentences:
\begin{tabbing}
aaa\=(@41)\= a. \= *\= \kill
   \>(+41)\> a. \>  \> Jeff may have been a baseball player.    \\
   \>     \> b. \>  \> The check will be in the mail. \\
   \>     \> c. \>  \> I am not a crook.
\end{tabbing}




\subsubsection{{\em Be\/} + Predicative Phrase}            

\label{bep}
One possible solution is to treat the maximal projections of lexical
items that can be predicates as Pred(icative) Phrases, rather than VPs,
APs, NPs, and PPs. Thus the structure of a Sentence would be roughly as
in (@42) and that of a Small Clause as in (@43)
\begin{tabbing}
aaa\=(+42)\= a. \= *\= \kill
   \>(+42)\>    \>  \> [$_{S}$ NP [$_{PredP}$ V \ldots\ ]]       
\end{tabbing}
\begin{tabbing}
aaa\=(+43)\= a. \= *\= \kill
   \>(+43)\> a. \>  \> [$_{S}$ NP [$_{PredP}$ A \ldots\ ]]       \\
   \>     \> b. \>  \> [$_{S}$ NP [$_{PredP}$ N \ldots\ ]]       \\
   \>     \> c. \>  \> [$_{S}$ NP [$_{PredP}$ P \ldots\ ]]       
\end{tabbing}
The copula would anchor an auxiliary tree with a PredP root and foot node:
\begin{tabbing}
aaa\=(@44)\= a. \= *\= \kill
   \>(+44)\>    \>  \> [$_{PredP}$ [$_{V}$ be] [$_{PredP}$ ]]    
\end{tabbing}

As for the related constructions: the putative raising verbs illustrated
in Section~\ref{raise} could also anchor an auxiliary tree with a PredP
root and foot node, while the verbs illustrated in Section~\ref{sc}
could anchor auxiliary trees with an S root and foot node.  The
possibility that this second group of verbs should anchor initial trees,
with their complements supplied by substitution, can be rejected on the
basis of the grammaticality of extraction from within the complement:
\begin{tabbing}
aaa\=(@45)\= a. \= *\= \kill
   \>(+45)\> a. \>  \>How intelligent$_{i}$ do you consider Mary $t_{i}$?  \\
   \>  \>    b. \>  \>What$_{i}$ did they make John $t_{i}$? \\
   \>  \>    c. \>  \>Who$_{i}$ did you hear $t_{i}$ insult the flag?
\end{tabbing}
The additional mechanisms that will be necessary to replicate the
correct co-occurrence restrictions will be discussed in what follows.
Notice, however, that the analysis of all these constructions is
independent of the analysis of the copula.  We could, for example, treat
the copula as suggested above while rejecting the idea that the verbs in
Section~\ref{sc} take clausal complements.

If we handle the copula by having it anchor an auxiliary tree that
adjoins at a PredP node in what is essentially a Small Clause, we will
have to ensure that we don't end up generating matrix Small Clauses, or
predicative sentences with auxiliaries but no copula:
\begin{tabbing}
aaa\=(@46)\= a. \= *\= \kill
   \>(+46)\> a. \> *\> Laura a doctor. \\
   \>     \> b. \> *\> David over-educated. \\
   \>     \> c. \> *\> The check will in the mail. \\
   \>     \> d. \> *\> Jeff may have a baseball player.
\end{tabbing}
Both of these ends could be achieved by adding an additional value for
the $<$mode$>$ feature, which we can call ``non-v'' for the moment. This
will be the value for the bottom $<$mode$>$ feature on the PredP
dominating any non-verbal head ({\em i.e.\/} A, N, or P), and it will be
selected for by {\em be\/}, just as it now selects either $<$mode$>=$ppart
or $<$mode$>=$ger. This will prevent Small Clauses occurring as matrix
sentences because $<$mode$>=$non-v will fail to unify with
$<$mode$>=$ind.  It will also prevent predicative sentences with
auxiliaries but no copula because all the auxiliaries specify some value
for the $<$mode$>$ feature that will fail to unify with ``non-v.''

In addition, we will have to add a feature to prevent {\em do} from
adjoining to the left of the copula.  At present we prevent {\em do}
from co-occurring with any auxiliaries by having it set all aspect
features---$<$conditional$>$, $<$perfect$>$, and $<$progressive$>$---to
be ``$-$.'' If we wanted to pursue this strategy, we would have to add a
feature, {\em e.g.\/} $<$cop$>$, that would be given a ``$+$'' value in
the bottom set of features on the VP node dominating the copula. This
strategy is theoretically rather unattractive, however, since the new
feature $<$cop$>$ does not form a natural class with the aspect
features. In fact even without the addition of the copula we need an
additional constraint on the occurrence of {\em do\/}---if we rely on
aspect features alone we will not be able to prevent the occurrence of
{\em do} with passive {\em be\/}.  Rather than rely on this
heterogeneous collection of features to ensure that {\em do} must take a
VP headed by a main verb as its complement I propose that we should
introduce a binary valued feature $<$mainv$>$, a positive value for
which will be selected for by {\em do\/}.

Preventing {\em do} from adjoining to the left of the NP, AP, or PP
complement does not require any additional features, given the new value
for $<$mode$>$ described above: the grammar already constrains {\em do}
to take a VP complement specified as $<$mode$>=$base.

This treatment of the copula should resolve the problem for the analysis
of negation pointed out in \cite[p.7]{za89} (see Section~\ref{main}
above): provided that the auxiliary tree rooted in {\em not\/} is
redrawn to have a PredP root and foot node, while still retaining the
restriction that the $<$mode$>$ feature must not have the value ``ind,''
the grammar will correctly generate sentences with {\em not} between the
copula and the predicative phrase. If we adopt Small Clause analyses for
the various verbs illustrated in Sections~\ref{raise} and \ref{sc} the
grammar will also generate the following type of sentences:
\begin{tabbing}
aaa\=(@47)\= a. \= *\= \kill
   \>(+47)\> a. \>  \> Jenny seemed not angry.      \\
   \>     \> b. \>  \> Jenny was considered not a genius. \\
   \>     \> c. \>  \> I considered Jenny not happy. \\
   \>     \> d. \>  \> I let Jenny not attend.
\end{tabbing}
This is a desirable result, as all these sentences are acceptable.

This treatment also will allow our grammar to replicate the
auxiliary-like behavior of the copula. For example, its participation in
Subject-Aux inversion can be handled in exactly the same way as that of
``passive {\em be\/}'' and ``progressive {\em be\/},'' which are already
included in the grammar.  One aspect of this analysis that seems
theoretically unattractive to me, however, is that the grammar as it
stands now makes the distinction between verbs that invert and those
that don't appear essentially arbitrary. For example, the structures
anchored by a raising verb like {\em seem} appear virtually identical to
those anchored by the various auxiliaries, except that {\em seem} does
not anchor a tree where it appears as a left sister of S (the inversion
structure).  This problem---if it is one---is not affected by the
possible analyses of the copula proposed in this report, however. In
fact, if we decided to create a tree family that would be selected by
all the raising verbs including the auxiliaries as well as main verbs
like {\em seem\/} we could use the feature $<$mainv$>$ proposed above to
restrict inversion to the auxiliaries.


I would now like to discuss some issues that arise if we adopt analyses
of the constructions illustrated in Sections~\ref{raise} and \ref{sc}
along the lines suggested at the beginning of this section.

The proposed ``non-v'' value for the feature $<$mode$>$ is sufficient as
long as we are only considering the copula, since, as illustrated in
Sections~\ref{cop} and \ref{raise}, it can co-occur with an NP, AP, or
PP. If, however, we also want to analyze the passive of {\em make} (or
the verb {\em become\/}---but see the caveat in Section~\ref{raise})
as a raising verb that adjoins into a PredP node, or the active as a
verb that takes a Small Clause complement, we will need to distinguish
between PredPs that dominate adjectival and nominal heads on the one
hand, and those that dominate prepositional heads on the other, in the
light of the following contrasts, illustrated above in
Section~\ref{raise}:
\begin{tabbing}
aaa\=(@48)\= a. \= *\= \kill
   \>(+48)\> a. \>  \> Alice was made irritated by his questions.    \\
   \>     \> b. \>  \> Alice was made a radical by her experiences. \\
   \>     \> c. \> *\> Alice was made in a bad mood by his questions.
\end{tabbing}
\begin{tabbing}
aaa\=(@49)\= a. \= *\= \kill
   \>(+49)\> a. \>  \> Alice became tense.      \\
   \>     \> b. \>  \> Alice became a feminist in her teens. \\
   \>     \> c. \> *\> Alice became in a temper. 
\end{tabbing}
\begin{tabbing}
aaa\=(@50)\= a. \= *\= \kill
   \>(+50)\> a. \>  \> She made him embarassed.      \\
   \>     \> b. \>  \> That made him a believer. \\
   \>     \> c. \> *\> This will make him in a relaxed frame of mind. 
\end{tabbing}

There are two ways of making this distinction that will allow us to make
the correct restrictions.
\begin{enumerate}

\item We could replace the ``non-v'' value proposed for the 
$<$mode$>$ feature by two values, say ``nom'' ({\em i.e.\/} [+N]---nominal and
adjectival head) and ``prep'' (prepositional head).  Of course this
would mean giving verbs that co-occur with all non-verbal PredPs dual
lexical entries.  

\item Alternatively, we could retain ``non-v'' as a value
for $<$mode$>$, using it to distinguish between verbal and non-verbal
predicates, as before, and set up an additional feature defined for
non-verbal complements, say $<$nominal$>$ with two values, ``+'' and
``$-$.'' Then verbs that co-occur with all non-verbal PredPs will take a
complement that is $<$mode$>$=non-v, and $<$nominal$>$ unspecified.

\end{enumerate} 

It would be nice if we could adapt the decomposition of V, N, A, and P
into two binary-valued features as outlined in Section~\ref{gpsg}.  But
while NPs and APs (the [+N] categories) pattern together in the
constructions we are considering, none of the other pairs---NPs and PPs
([$-$V]), APs and VPs ([+V]), VPs and PPs ([$-$N])---appear to have this
property% 
%
\footnote{Except to the meager extent that the possibility of co-occurrence
with a PP [$-$V] implies the possibility of co-occurrence with an NP
[$-$V].}. 
%

In addition, if we consider verbs taking the various types of clausal
complements listed in Section~\ref{sc}, there do not appear to be any
correlations between co-occurrence with a bare infinitive and with a
{\em to\/}-infinitive, or between either of these and co-occurrence with
any of the non-verbal predicates.  As a result, many of the verbs listed
in Sections~\ref{raise} and \ref{sc} will have to have multiple lexical
entries: {\em e.g.\/} {\em seem\/}, one of the most promiscuous verbs, will have
to anchor 2 or 3 trees: on one the foot PredP node will have a top
feature $<$mode$>$=inf; on another $<$mode$>$=non-v; or, if we adopt the
first suggestion for distinguishing between nominal/adjectival and
prepositional complements, on a second tree the foot PredP node will
have the top feature $<$mode$>$=nom, and on a third $<$mode$>$=prep.

There are various objections to the analysis of the copula outlined in
this section, including the following:
\begin{enumerate}

\item PredP is an odd category, which clearly does not fit into the
\={X}-Schema, since it can be headed by V, A, N, or P. (On the other
hand, under the current system the category S does not fit into the
\={X}-Schema either).  Suppose instead we did not have PredP as a
category but just had a binary-valued feature $<$pred$>$ along the lines
of GPSG. The copula would anchor an auxiliary tree whose root and foot
node would be constrained to be $<$pred$>$=+.  All other auxiliary verbs
would be constrained in the same way.  The obvious problem with such a
system would be that not only the trees for the copula but also those
for all auxiliaries, raising verbs, and negation, etc., would have to
exist in quadruplicate so that they could adjoin to VPs, APs, NPs, and
VPs---since even though an auxiliary would not occur next to an N, for
example, it should be able to occur next to a copula with an NP
complement, and since the root and foot node of the tree anchored by the
copula have to be of the same category the root node of a copula with an
NP complement would have the category NP.  This quadruplication could be
avoided if it were possible to have auxiliary trees whose root and foot
nodes were of a variable category---constrained only to be the same, and
set upon adjunction. Of course, this would still result in an analysis where
sentences did not necessarily have VPs---evidently the subject could be
a sister to a maximal projection of any lexical category. This does not
seem to me to be particuarly unappealing for sentences like those in
(@51), but it seems pretty far-fetched for sentences like those in
(@52): 
\begin{tabbing}
aaa\=(+51)\=a. \= *\= \kill
   \>(+51)\>a. \>  \> Jonathan is weary.       \\
   \>     \>b. \>  \> Alyosha is a dreamer
\end{tabbing}
\begin{tabbing}
aaa\=(+52)\=a. \= *\= \kill
   \>(+52)\>a. \>  \> Jonathan could have been being sarcastic.      \\
   \>     \>b. \>  \> Alyosha may have been a dreamer
\end{tabbing}
Also the copula and all the auxiliaries could not be considered heads,
but would rather be modifiers.

\item If the innovation outlined in the last paragraph is not introduced
into the system, the current grammar will have to be reworked completely
so that all VPs are replaced by PredPs.

\end{enumerate}

\subsubsection{{\em Be\/} + a Null Copula}

\label{tagnull}
Another possibility that could be implemented in our LTAG for English is
a version of the analysis proposed in \cite{la80} and outlined briefly
in Section~\ref{la}: that is, we could assume the existence of a null
copular verb.  A predicative A, N, or P would then anchor a sentential
structure as in (@53):
\begin{tabbing}
aaa\=(+53)\=a. \= *\= \kill
   \>(+53)\>a. \>  \> [$_{S}$ NP [$_{VP}$ [$_{V}$ 0] [$_{AP}$ A \ldots ]]]  \\
   \>     \>b. \>  \> [$_{S}$ NP [$_{VP}$ [$_{V}$ 0] [$_{NP}$ N \ldots ]]]  \\
   \>     \>c. \>  \> [$_{S}$ NP [$_{VP}$ [$_{V}$ 0] [$_{PP}$ P \ldots ]]]  
\end{tabbing}

Under this analysis {\em be} is simply an auxiliary, which would anchor
an auxiliary tree with a VP root and foot node, as ``passive {\em be\/}''
and ``progressive {\em be\/}'' do in the current system.

This type of sentential structure with a VP headed by the null copula
could then be used in the analysis of the raising verbs
illustrated in Section~\ref{raise} and the verbs taking Small Clause
complements illustrated in Section~\ref{sc}.

Virtually all of the discussion in the previous section holds for this
analysis as well.  In order to avoid matrix Small Clauses and sentences
with the null copula and other auxiliaries but no {\em be} we would need
to add values for the feature $<$mode$>$ and possibly another feature
$<$nominal$>$, for exactly the reasons outlined in the previous section.

\subsubsection{The Substance of the Two Analyses}

As should be fairly clear, there is little difference between the two
analyses outlined above. The first proposal dispenses with null
categories, but at the expense of introducing a phrase type that can
have four different heads. The second requires us to make less changes
in the current system, and is consistent with \={X}-Theory, but only by
virtue of positing an empty verb.  If making as few changes to the
current system as possible is an important criterion, and adherence to
\={X}-Theory is not, we could simply allow VP to dominate non-verbal
heads (as suggested already by Sharon Cote, reported in \cite{za89}, and
supplement the inventory of features and values as before.  Small
clauses would then have the following structures:
\begin{tabbing}
aaa\=(@80)\=a. \= *\= \kill
   \>(+80)\>a. \>  \> [$_{S}$ NP [$_{VP}$  A \ldots ]]  \\
   \>     \>b. \>  \> [$_{S}$ NP [$_{VP}$  N \ldots ]]  \\
   \>     \>c. \>  \> [$_{S}$ NP [$_{VP}$  P \ldots ]]
\end{tabbing}

\subsection{The Copula in its Non-Predicational Uses}

\label{nonpsol}
As discussed briefly in Section~\ref{nonpred}, the copula is not always
followed by a predicative phrase, and in this respect differs from other
raising verbs%
%
\footnote{In this section I am calling {\em be\/} ``non-predicational''
whenever it is followed by a non-predicative phrase.  This conflates two
possibily different uses: on the one hand the subject also may be
clearly non-predicative, as in (i):
\begin{tabbing}
aaaaa\=(ii) \= \kill
     \>(i)  \>  My grandmother was his aunt.
\end{tabbing}
Or on the other the subject may be predicative (Moro's ``inverse sentences''):
\begin{tabbing}
aaaaa\=(ii) \= \kill
     \>(ii)  \> The culprit was Ridley.
\end{tabbing}
Whilee it may turn out that different analyses are required for
these two cases, at this point I can find no empirical
basis for separate analyses.}.
%
If we adopt the analysis that postulates the adjunction
of a tree anchored by the copula into a PredP node, we are virtually
forced to come up with a separate analysis for the non-predicational
uses of {\em be}, unless we consider ``PredP'' just an arbitrary name
for a category which can equally have a non-predicative head. If we
adopt either of the other two variants it would be possible to have
other raising verbs, {\em e.g.\/} {\em seem}, and all verbs taking Small
Clause complements, {\em e.g.\/} {\em consider\/}, to require that their
VP sisters are specified as $<$pred$> = +$, while {\em be} makes no such
restriction.  However, as pointed out to me by Beth Ann Hockey (p.c.)
this still raises a serious problem: since {\em be} is
introduced by adjunction, the Small Clause into which it adjoins must
have a lexical anchor.  As long as the non-subject part of the Small
Clause is headed by an A, N, or P, this can serve as the anchor.  But
the sister to non-predicational {\em be} can in addition be a full
clause:
\begin{tabbing}
aaa\=(@54)\=a. \= *\= \kill
   \>(+54)\>   \>  \> My objection is that he doesn't know what he's doing.  
\end{tabbing}
It would be highly undesirable to have to include in the tree family for
every possible lexical anchor for the complement clause---in this case
{\em know\/}---a tree that includes the dominating Small Clause as well,
as would be necessary under this analysis%
%
\footnote{An alternative, suggested to me by Megan Moser (personal
communication) would be to make the head of the subject the anchor for
the clause.  This suggestion clearly bears on the analysis of \cite{mor90}, and
deserves to be explored, but it is ignored in the following discussion, basically
for the reason given in the previous footnote.}.
%

I therefore conclude that we will need to have a separate entry in the
syntactic lexicon for non-predicational {\em be}. From the above
discussion it seems clear that it must be the anchor in the clauses in
which it appears, and therefore should not be introduced by adjunction.
Instead, the subject and complement will be introduced by substitution
into the initial tree anchored by non-predicational {\em be\/}.

Notice that extraction from a sentential complement to {\em be} is
ungrammatical: 
\begin{tabbing}
aaa\=(@55)\=a. \= *\= \kill
   \>(+55)\>a. \>  \> My objection is that he said we are inferior. \\
   \>     \>b. \> *\> What$_{i}$ is your objection that he said $t_{i}$?   
\end{tabbing}
\begin{tabbing}
aaa\=(@56)\=a. \= *\= \kill
   \>(+56)\>a. \>  \> My position is that we should aim to beat McDonald. \\
   \>     \>b. \> *\> Who$_{i}$ is your position that we should aim to
                      beat $t_{i}$?
\end{tabbing}
This is exactly as expected if the complement to {\em be} is supplied via
substitution.  The failure of extraction has nothing to do with the
finiteness of the complement clause in these examples.  The following
examples show a near minimal pair with predicational {\em be} in (@57)
(that we are dealing with the {\em be\/} that allows raising is shown by
the grammaticality of the expletive subject in the (b) sentence), and
non-predicational {\em be\/} in (@58):
\begin{tabbing}
aaa\=(+57)\=a. \= *\= \kill
   \>(+57)\>a. \>  \> The challenger is to provide a non-stipulative 
                      solution.\\
   \>     \>b. \>  \> There is to be a meeting at 10 tomorrow. \\
   \>     \>c. \>  \> What$_{i}$ is the challenger to provide $t_{i}$?
\end{tabbing}
\begin{tabbing}
aaa\=(+58)\=a. \= *\= \kill
   \>(+58)\>a. \>  \> The challenge is to provide a non-stipulative
                      solution.       \\
   \>     \>b. \> *\> What$_{i}$ is the challenge to provide $t_{i}$?
\end{tabbing}
(I assume that in (+58) the infinitival complement to {\em be} is an
infinitival clause with a PRO subject.)

As pointed out in \cite{mor90}, extraction out of complex NPs in this
positions is also ungrammatical:
\begin{tabbing}
aaa\=(@90) \= *\= \kill
   \>(+90) \> *\> Who do you think that the cause was a picture of?
\end{tabbing}
This contrasts with extraction out of complex NPs that are the objects
of transitive verbs, as in (@91), or predicative {\em be}, as in (@92):
\begin{tabbing}
aaa\=(+91)\=a. \= \kill
   \>(+91)\>a. \> Who do you think that she bought a picture of? \\
   \>     \>b. \> What do you think that she was tearing up a picture of?
\end{tabbing}
\begin{tabbing}
aaa\=(+92)\=a. \= \kill
   \>(+92)\>a. \> Who do you think that she is an admiror of? \\
   \>     \>b. \> What do you think that this is an imitation of?
\end{tabbing}
I do not know how this contrast is to be captured in an LTAG, as I do
not know how extraction out of complex NPs is currently handled.

Although non-predicational {\em be} is not a raising verb, it still
exhibits all the other auxiliary-like behavior set out in (+6)--(+10)
above, and therefore should probably not be associated with any existing
tree family for main verbs, but instead will require a separate tree
family that should include the trees for inversion.  We will also have
to adopt the same strategy we use for predicational {\em be} to prevent
{\em do}-support.

The simplest analysis of the basic structure anchored by
non-predicational {\em be} is something along the lines of (@59):
\begin{tabbing}
aaa\=(+59)\=a. \= *\= \kill
   \>(+59)\>   \>  \> [$_{S}$ NP [$_{VP}$ [$_{V}$ be] NP/PP/S']] 
\end{tabbing}
However, this structure fails to interact correctly with the current
analysis of negation---{\em i.e.\/} the problem pointed out in
\cite{za89} that was solved for predicational {\em be} reasserts itself
here.  If we maintain the current analysis of negation we need {\em be}
to take a VP (or PredP) complement to which the tree for {\em not} can
adjoin. The only way that I can see to do this is to have
non-predicational {\em be} anchor the kind of structure in (@60a) or
(+60b): 
\begin{tabbing}
aaa\=(+60)\=a. \= *\= \kill
   \>(+60)\>a. \>  \> [$_{S}$ NP [$_{VP}$ [$_{V}$ be] [$_{VP}$ [$_{V}$
                      0] NP/PP/S' ]]] \\
   \>     \>b. \>  \> [$_{S}$ NP [$_{VP}$ [$_{V}$ be] [$_{VP}$ 
                         NP/PP/S' ]]]
\end{tabbing}
This is not very attractive since it involves recursive
initial trees. At the moment, however, I have no other solution unless
the current implementation of negation is changed.

%The following is rather a digression. Notice that the following solution
%{\em cannot\/} be adopted in an LTAG (at least as I currently understand
%it---corrections welcomed).  Instead of having non-predicational {\em
%be} anchor an initial tree, we allow the null copula to anchor the
%following type of structure, where the subject NP
%node and the complement NP/PP/\={S} node are filled by substitution:
%\begin{tabbing}
%aaa\=(@61)\=a. \= *\= \kill
%   \>(+61)\>   \>  \> [$_{S}$ NP [$_{VP}$ [$_{V}$ 0] NP/PP/S']]
%\end{tabbing}
%The VP dominating the null copula in this structure is
%$<$pred$> = -$. The auxiliary {\em be\/} would not specify any
%value for this feature on the VP into which it would adjoin, while
%raising predicates such as {\em seem\/} would only adjoin into VPs that
%were $<$pred$> = +$.  Assuming that this value would be passed up
%to the corresponding feature on the dominating S-node, we could also
%restrict Small Clause complements of verbs like {\em consider} to be 
%$<$pred$> = +$. The reason that this solution is ruled out is that
%it requires a null category to be an anchor.
%
%It appears to be desirable to rule out this analysis, since if the null
%copula is allowed to take non-predicative complements, it is only by
%stipulation that we can prevent the grammar from allowing the
%non-predicative Small Clauses that it generates from occurring without
%{\em be\/} ({\em i.e.\/} with raising verbs like {\em seem\/} and as
%complements to {\em consider\/}), whereas if non-predicative Small
%Clauses without {\em be} are never generated (as must be the case if
%{\em be\/} is the anchor of the initial tree in the non-predicative
%case) this result follows automatically.  Just as a point of interest,
%though, in the only language I have come across myself that could be
%analyzed as having a null copula---Classical Arabic---the ``predicative
%phrase'' can be referential. {\em E.g.\/} one says not only ``She
%beautiful,'' but also ``She Miss Salwa Ayoub''%
%%
%\footnote{This is also true of modern Hebrew (Robert Rubinoff, personal communication)}.
%%
%The
%``predicative phrase'' can also be a clause. But what
%would the anchor for these sentences be in an LTAG?

\section{Implementation Work}

The decisions made for actual implementation are as follows:
\begin{enumerate}

\item {\em Two basic different syntactic types for the copula.} 
What has been referred to in this report as ``predicational {\em be\/}'' anchors
basically the same auxiliary VP trees as all the other auxiliaries;
``Non-predicational {\em be\/}'' anchors a separate tree family.  The possibility
of having non-predicational {\em be} select the regular transitive tree family was
rejected. The principal reason for this was the different behavior of negation
(see Section~\ref{nonpsol} for a brief discussion). 

\item {\em Small Clauses.\/}
``Predicational {\em be}'' adjoins into small clauses that consist of sentences
with a VP headed by a null copula.  This is the analysis outlined in
Section~\ref{tagnull} above. These small clauses are also used in the generation
of the types of sentences described in Section~\ref{other}.

\item {\em New features and feature-values.}
The binary-valued feature $<$mainv$>$ is used to prevent {\em do} taking a VP
headed by {\em be} as its complement (see Section~\ref{bep}). 
Possible values for $<$mode$>$  now  include ``nom'' and ``prep'' in order to
constrain the occurrence of small clauses headed by nouns, adjectivess, and
prepositions (see Section~\ref{bep}).

\end{enumerate}

The implementation of this analysis is to be found in a working demo that was
developed jointly with Beth Ann Hockey. The demo is in ******************, and
documentation is given in ****************

\begin{thebibliography}{Gazdar et al al 85}

\bibitem[Chomsky 70]{ch70} Chomsky, N. (1970). Remarks on
Nominalization.  In R. Jacobs and P. Rosenbaum (eds), {\em Readings in
English Transformational Grammar.\/} Ginn and Company, Waltham, MA:
184--221. 

\bibitem[Gazdar et al 85]{gkps85} Gazdar, G., E. Klein, G. Pullum, and 
I. Sag.  (1985) {\em Generalized Phrase Structure Grammar.} Harvard
University Press, Cambridge MA.

\bibitem[Kroch et al 87]{ksh87}
Kroch, A., B. Santorini, and C. Heycock.  Bare
Infinitives and External Arguments.  In: {\em Proceedings of NELS
1987.\/}

\bibitem[Lapointe 80]{la80} Lapointe, S. (1980). A lexical analysis of
the English auxiliary verb system. In T. Hoekstra et al (eds), {\em
Lexical Grammar.\/} Foris, Dordrecht: 215--254. 

\bibitem[Moro 90]{mor90} Moro, A. (1990). {\em There} as a Raised
Predicate. Paper presented at GLOW.

\bibitem[Moser 90]{mos90} Moser, M. (1990). Auxiliary verbs in English
TAGs. Ms, University of Pennsylvania.

\bibitem[Pollock 89]{po89} Pollock, J-Y. (1989). Verb Movement, UG, and
the Structure of IP. {\em Linguistic Inquiry\/} 20.3: 365--424.

\bibitem[Sag et al 85]{sgww85} Sag, I., G. Gazdar, T. Wasow, S. Weisler.
(1985) Coordination and how to distinguish categories. {\em Natural
Language and Linguistic Theory\/} 3: 117--171.

\bibitem[Zanuttini 89]{za89} Zanuttini, R. (1989). Some Aspects of
Negation in English LTAGs. Ms, University of Pennsylvania.

\end{thebibliography}

\end{document}
