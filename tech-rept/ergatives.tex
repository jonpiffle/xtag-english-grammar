\chapter{Ergatives}
\label{ergatives}

Verbs in English that we will call {\it ergative} \footnote {The terminology is
from \cite{Burzio86}. See also \cite{Perlmutter78} and \cite{Rosen81} for
discussion within the Relational Grammar framework.} display the kind of
alternation shown in the sentences in \ex{1} and \ex{2} below.

\enumsentence{The sun melted the ice .}
\enumsentence{The ice melted .}

The object of the transitive sentence in \ex{-1} corresponds to the subject
of the intransitive sentence in \ex{0}. 

\section{Various Approaches}

The literature discussing such pairs as \ex{-1} and \ex{0} is based largely
on syntactic models that involve movement, particularly GB.  Within that
framework two basic approaches are discussed:

\begin{itemize}
\item {\bf Derived Intransitive}\\ The intransitive member of the
ergative pair is derived through processes of movement and deletion from an
underlying transitive structure \cite{Burzio86,HaleKeyser86,HaleKeyser87}.

\item {\bf Pure Intransitive}\\ The intransitive member is intransitive at all levels of the
syntax and the lexicon and is not related to the transitive member
syntactically or lexically \cite{Napoli88}.
\end{itemize}

\section{Xtag Analysis}

Although XTAG does not have derivational movement, the relationships
between the two arguments can be translated into the FB-LTAG framework.  In
the XTAG grammar the difference between these two approaches is not a
matter of movement but rather a question of tree family selection.  The
relation between sentences represented in terms of movement in other
frameworks is represented in XTAG by membership in the same tree family or
selection of multiple families which preserve the argument relations across
families. Wh-questions and their indicative counterparts are one example of
the former, whereas ergatives and ditransitives with PP shift exemplify the
latter.  Adopting the Pure Intransitive approach suggested by
\cite{Napoli88} would mean placing the intransitive ergatives in a tree
family with other intransitive verbs and separate from the transitive
variants of the same verbs.  This would result in a grammar that
represented intransitive ergatives as more closely related to other
intransitives than to their transitive counterparts.  The only hint of the
relation between the intransitive ergatives and the transitive ergatives
would be that ergative verbs would select both tree families. While this is
a workable solution, it is an unattractive one for the English XTAG grammar
because semantic coherence is lost.  In particular, constancy in thematic
role is represented by constancy in node names across sentence types within
a tree family. For example, if the object of a declarative tree is NP$_{1}$
the subject of the passive tree(s) in that family will also be NP$_{1}$.

The analysis that has been implemented in the English XTAG grammar is
an adaptation of the Derived Intransitive approach.  The ergatives
select the two families Tnx0Vnx1 and TEnx1V.  TEnx1V is quite similar
to the regular intransitive family Tnx0V, except that it encodes a
different type of semantic function---the syntactic subject of TEnx1V
is given the index of `1' to represent that it is the logical object
of the verb.  The two families Tnx0Vnx1 and TEnx1V create the two
possibilities needed to account for the data.

\begin{itemize}
\item {\bf intransitive ergative/transitive alternation.}  These verbs have
transitive and intransitive variants as shown in sentences~\ex{1} and
\ex{2}.

\enumsentence{The sun melted the ice .}
\enumsentence{The ice melted .}

In the English XTAG grammar, verbs with this behavior select family
\{Tnx0Vnx1 $\cup$ TEnx1V\}.  The first family in the union handles
sentences such as \ex{-1} and the second covers \ex{0}.

\item {\bf transitive only.}  Verbs of this type do not allow the
intransitive ergative variants, as can be seen in the pattern shown in
sentences~\ex{1} and \ex{2}.

\enumsentence{Elmo borrowed a book .}
\enumsentence{$\ast$A book borrowed .}

Such verbs select only the family Tnx0Vnx1, and thus do not have the
ergative trees in TEnx1V available to them.

\end{itemize}

\begin{figure}[htb]
\centering
\mbox{}
\psfig{figure=ps/erg-files/alphaEnx1V.ps,height=4.0cm}
\caption{Ergative Tree: $\alpha$Enx1V}
\label{decl-erg-tree}
\label{2;14,1}
\end{figure}

The declarative ergative tree is shown in Figure~\ref{decl-erg-tree}.
Note that the index of the subject NP indicates that it originated as
the object of the verb.























