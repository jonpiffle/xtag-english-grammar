\section{Embedded clauses}
In the LTAG formalism,  arguments of a lexical items including
subjects, appear in the initial tree anchored by that lexical item.  A
sentential arguement appears  as an S node in the appropriate
position within an elementary tree anchored by the lexical item that
selects it. This is the case for sentential
complements of verbs, prepositions and nouns and for sentential
subjects. Following standard GB approach, the English LTAG grammar
does not allow VP complements but treats verb-anchored structures
without overt subjects as having PRO subjects.  

In the English LTAG grammar, complementizers select the type of clause
to which they adjoin through constraints on the {\bf mode} feature
of the S footnode in the $\beta$COMPs tree shown in (\ex{1}).

The grammar handles the complementizers: {\it that\/}, {\it
whether\/}, {\it if\/}, {\it for\/}, and no complementizer, and the
clause types: indicative, infinitival, gerundive, and subjunctive.
The {\bf comp} feature reflects the value of the complementizer if one
has adjoined to the clause and has the value {\bf nil} otherwise.

 This grammar does not have a category distinction between S and
S'. Clauses with and without complementizers are all of the category
S. The differences attributed to bar level in GB are represented by
the {\bf comp} feature in the English LTAG grammar.  The {\bf
comp} feature also insures that there are no multiple
complementizers by requiring the footnode of $\beta$COMPs to have {\bf
$<$comp = nil$>$}.  Note that there are no empty
complementizers in this system. The absence of an overt complementizer
is represented by having no complementizer adjoined to S. That is, in
all cases, absence of a complementizer is taken to actually be absence
of a complementizer.

Another component of the complementizer system, the {\bf
assign-comp} feature, is used to represent the requirements of
particular types of clauses for particular complementizers.  So while
the {\bf comp} feature represents  constraints originating from the VP
dominating the clause, the {\bf assign-comp} feature represents constraints
originating from the highest VP in the clause. The {\bf assign-comp}
feature is particularly useful in accounting for the variation in
subjects of infinitival clauses.  

\enumsentence{John wants her to win.}
\enumsentence{John wants to win.}
\enumsentence{John wants for her to win.}
\enumsentence{*John wants for she to win.}
\enumsentence{*John wants for to win.}


Examples (\ex{-2}), (\ex{-1}) and (\ex{0}) show that an accusative
case subject is obligatory in an infinitive clause if the
complementizer {\it for\/} is present. The infinitive clauses in both
(\ex{-4}) and (\ex{-3}) are analyzed in the English LTAG grammar as
having PRO subjects.  The apparent subject of {\it to win\/} in
(\ex{-4}) is taken to be an object of the verb rather than the subject
of the infinitive clause.  To capture these facts, two infinitive
{\it to}'s are posited. One infinitive {\it to\/} has {\bf
$<$assign-case$>$ =none$>$} which forces a PRO subject. The other
infinitive {\it to\/} has no value at all for {\bf assign-case} and
will only occur with the complementizer {\it for\/} because in those
instances {\it for} supplies the {\bf assign-case} value. The {\bf
assign-comp} feature is used with the non-case-assigning {\it to} to
impose obligatory adjunction of the complementizer {\it for}.

 
\subsection{Sentential Complements}
\subsubsection{Sentential Complements of Verbs}
Verbs that select for sententential complements select the {\bf mode}
and {\bf comp} values for those complements. Since with very few
exceptions\footnote{For example, long distance extraction is not
possible from the S complement in it-clefts} long distance extraction
is possible from sentential complements, the S complement nodes are
adjunction nodes. Figure \ref{think} shows the declarative tree
$\beta$nx0Vs1, anchored by {\it think}.  

%\begin{figure}
%\caption{ \label{think}  $\beta$nx0Vs1}
%\end{figure}

The need for an adjunction node rather than a substitution node  at
S$_{1}$ may not be obvious untill one considers the derivation of
sentences with long distance extractions.  For example, the
declarative in (\ex{1}) is derived by adjoining the tree in figure
\ref{clara-wrote-a-book} to the S$_{1}$ node of the tree in
figure \ref{ernest-thinks}.  Since there are no bottom features on
S$_{1}$, the same final result could have been achieved with a
substitution node at S$_{1}$. 
\enumsentence{Ernest thinks Clara wrote a book}

%\begin{figure}
%\caption{ \label{clara-wrote-a-book} Tree for the sentence {\it Clara
%wrote a book}}
%\end{figure}

%\begin{figure}
%\caption{ \label{ernest-thinks} Tree for [$_{S}${\it Ernest thinks\/}
%S $_{S}$] 
%\end{figure}

However, the adjunction node is crucial in deriving (\ex{1}).  

\enumsentence{Who does Ernest think wrote a book?}

This example is derived from the trees for {\it who wrote a book?} shown
in figure \ref{who-wrote-a-book} and for {\it Ernest think} S shown in
figure \ref{ernest-think}.

\subsection{Sentential Subjects}

Verbs that select sentential subjects anchor trees that have an S node
in the subject position rather that an NP node.  Since extraction is
not possible from sentential subjects, they are implemented as
substitution nodes in the English LTAG grammar.  Restrictions on
sentential subjects such as the required "that" complementizer for
indicatives, are enforced by feature values specified on the S
substitution node in the elementary tree.  

The distinction between {\bf inf\_nil} and {\bf ind\_nil} captures a
difference between infinitive and indicative clauses in subject
position which will be discussed in detail in the section on
sentential subjects.

\subsection{Relative Clauses}
Relative clauses are represented in the grammar by auxiliary trees
that adjoin to NP. These trees are anchored by the verb in the clause
and appear in the appropriate tree families for the various verb
subcategorizations. Our analysis of relative clauses allows a single
tree to provide the structure for various relative clause types. For
example, the Rnx0Vnx1 tree shown in () is used for the relative
clauses shown in (\ex{0})-(\ex{5}).

%insert tree

\enumsentence{the man that Muriel likes}
\enumsentence{the man who Muriel likes}
\enumsentence{the man Muriel likes}
\enumsentence{what Muriel likes}
\enumsentence{the book for Muriel to read}
\enumsentence{the man reading the book}

The relative pronouns {\it who\/}, and {\it which\/} are treated
uniformly with {\it that}
as complementizers but are restricted to relative clauses. These relative
complementizers anchor the tree $\beta$COMPs  and adjoin onto the S node
of the relative clauses trees.  The complementizer
analysis also extends to infinitives with for complementizers as in
example (\ex{-1}).  
The relative clauses in (\ex{-5})-(\ex{0}) vary by clause type, type of complementizer
and the wh-status of the head NP.  Examples (\ex{-5})-(\ex{-3}) have indicative
clauses with {\bf $<$wh = -$>$} head NP's and show variation between
{\it that\/}, {\it who\/} and
no complementizers.   
By not putting a restriction on the {\bf wh} feature on the NP footnode of
relative clauses, free relatives such as (\ex{-2}) are accounted for
as {\bf $<$wh =
+$>$}  NP head with no complementizer.

