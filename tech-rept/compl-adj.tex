\chapter{Tree Families and Subcategorization Frames}
\label{underview}

This chapter is meant to be an introductory look at grammar
organization in XTAG. It serves as basic information the reader would
need to read the more detailed information about the grammar in
subsequent parts of the technical report.

\section{Tree Families and Subcategorization Frames}
\label{subcat-frames}

Elementary trees for predicative words\footnote{ Such as non-auxiliary
verbs or predicative nouns. } are used to represent the linguistic
notion of subcategorization.  The anchor of the elementary tree
subcategorizes for the other elements that appear in the tree, forming
a clausal or sentential structure.  Tree families group together trees
that belong to the same subcategorization frame. Consider the
following uses of the verb {\it buy}:

\enumsentence{Srini bought a book.}
\enumsentence{Srini bought Beth a book.}

In sentence (\ex{-1}), the verb {\it buy} subcategorizes for a direct
object NP.  The elementary tree anchored by {\it buy} is shown in
Figure~\ref{subcat-trees}(a) and includes nodes for the NP complement
of {\it buy} and for the NP subject.  In addition to this declarative
tree structure, the tree family also contains the trees that would be
related to each other transformationally in a movement based approach,
i.e passivization, imperatives, wh-questions, relative clauses, and so
forth.  Each tree family selected by its anchor represents all the
various syntactic environments that it can appear in.

Sentence (\ex{0}) shows that {\it buy} also subcategorizes for
a double NP object.  This means that {\it buy} also selects the double
NP object subcategorization frame, or tree family, with its own set of
transformationally related sentence structures.
Figure~\ref{subcat-trees}(b) shows the declarative structure for this
set of sentence structures.

\begin{figure}[ht]
\centering
\begin{tabular}{ccc}
{\psfig{figure=ps/compl-adj-files/alphanx0Vnx1_bought.ps,height=1.8in}} & 
\hspace*{0.5in} & 
{\psfig{figure=ps/compl-adj-files/alphanx0Vnx2nx1_bought.ps,height=1.8in}}\\
(a) & \hspace*{0.5in} & (b) \\ 
\end{tabular}
\caption{Different subcategorization frames for the verb {\it buy}}
\label{subcat-trees}
\end{figure}

Entire classes of anchors select a tree family. All the transitive
verbs form a class which selects the transitive tree family. Recall
that a tree family is a group of trees related by some syntactic
transformations. Since an entire class of anchors are selecting this
tree family, the assumption is that these syntactic transformations
are valid for each member of this class. For instance, wh- extraction
is a syntactic transformation that will apply regardless of any
idiosyncratic properties of any particular anchor of the tree family.
Chapter~\ref{verb-classes} contains more information about
the different tree families in the grammar.

There are some syntactic transformations, however, that are sensitive
to the properties of a particular anchor within the same
subcategorization frame. The ergative (or transitive-inchoative)
alternation for transitive verbs is one such transformation. Only a
subset of the transitive verbs can undergo this transformation. A more
rigorous definition of tree family that accounts for such lexical
idiosyncrasies within an otherwise homogeneous family is discussed in
detail in Appendix~\ref{families}.

\section{Tree Families and Lexical Rules}
\label{families}

In the XTAG English grammar, tree families provide us with a compact
means to specify the set of trees which a particular lexical anchor
takes. This set represents all syntactic environments in which this
anchor appears. One way to represent tree families is to have a
universal set of trees, with each individual anchor taking some subset
of that larger set.  However, since syntactic transformations are not
usually sensitive to the anchor of the tree family, in practice we
assign a tree family to entire classes of words. 

Chapter~\ref{underview} defines the notion of tree family used in the
XTAG grammar. In this chapter we discuss in more detail what it means
for a particular anchor to select several tree families keeping in
mind that entire classes of words which share the same
subcategorization are assigned tree families in the syntactic
database. Also, there are some syntactic transformations that are
sensitive to the properties of a particular word within the same
subcategorization frame.

In general, the trees within a particular tree family can be thought
of as being those which are syntactically or transformationally
related. The tree in Figure~\ref{trans-base-tree} allows us to parse
sentences such as (\ex{1}). This tree represents the subcategorization
of the verb and forms the base for the generation of an entire tree
family. Each family has one such {\em base tree}. One method for
generating all the trees in the family that are syntactically related to
the base tree is to use metarules (see Appendix~\ref{metarules}). One
such member of the transitive tree family is the tree in
Figure~\ref{trans-extracted-tree} which is the tree where the object
of the $\alpha$nx0Vnx1 has been extracted. This new tree
$\alpha$W1nx0Vnx1 now allows us to parse sentences such as (\ex{2}).

\enumsentence{Susie likes Hobbes.}
\enumsentence{Who does Susie like?}


\begin{figure}[htb]
\centering
\begin{tabular}{c}
\psfig{figure=ps/verb-class-files/alphanx0Vnx1.ps,height=3.4cm}
\end{tabular}
\caption{Declarative Transitive Tree:  $\alpha$nx0Vnx1}
\label{trans-base-tree}
\end{figure}

\begin{figure}[htb]
\centering
\mbox{}
\psfig{figure=ps/extraction-files/alphaW1nx0Vnx1.ps,height=10.0cm}
\caption{Transitive tree with object extraction: $\alpha$W1nx0Vnx1}
\label{trans-extracted-tree}
\end{figure} 

The semantic interpretation of arguments across different trees in a
family remains constant.  That is, if $\mbox{\em NP}_1$ is the
recipient argument of the predicate in the base indicative tree, then
it is the recipient when it is fronted as well.

There are some syntactic transformations, however, that are sensitive to the
properties of a particular word within the same subcategorization frame. The
ergative (or transitive-inchoative alternation) for transitive verbs is one
such transformation. Only a subset of the transitive verbs can undergo this
transformation. Let us call such transformations {\em lexical rules} to
distinguish them from more general {\em syntactic transformations} (like
wh-extraction) which are not lexically idiosyncratic.

To explain how these two different kinds of transformations interact
let us take the example of the ergative subset of the transitive
verbs:

Ergative verbs such as {\it melt} undergo the ergative transformation (see
Chapter~\ref{ergatives}). Such verbs are a subset of the set of transitive
verbs, which do not all take this transformation; for example verbs like {\it
borrow} cannot take this transformation. To handle the transitive/ergative
distinction, we could create a tree family which exclusively handles the
regular transitives and another one which handles the ergatives.  In doing so,
however, we would be duplicating structure since the ergative family would
contain all of the trees found in the transitive family in addition to the
purely ergative trees.  An alternative is to consider the ergative distinction
as arising from a {\em lexical rule} which takes the base tree of the
transitive family ($\alpha$nx0Vnx1) and produces the base tree for a strictly
ergative family ($\alpha$Enx1V).  Then all of the syntactic transformations
which are available in the grammar can be applied to that new base tree to form
the entire ergative tree family.  In this particular instance, the indexation
on the syntactic object NP of $\alpha$nx0Vnx1 ($\mbox{\em NP}_1$) is retained
on the syntactic subject of the ergative tree, which has the same semantic role
in both constructions.

The same problem appears in the ditransitive tree family, in which a subset of
anchors also undergo the dative-shift transformation. We give an identical
solution to this case as we do for the ergatives.

Hence, tree families can be related to each other by lexical rules.
Construing tree families in this way allows us to take advantage of
shared structure when creating a grammar. Introducing this shallow
hierarchy in the definition of family allows us to have a more compact
organization of the grammar. This departs from earlier conceptions of
tree families which maintained that argument positions in different
tree families were in no way related; thus, before, if a lexical item
anchored two different tree families, each anchoring instance was
considered a different predicate.  Under both the previous and the
present conceptions, tree families are the exhausitive domain of a
base tree and the trees generated by syntactic transformations on that
base.  However, the present conception of tree families provides for a
new level of distinction -- that of families related by a lexical
rule. These related families are the exhaustive domain of a particular
predicate that is affected by a lexical rule.

The XTAG grammar encodes the notion of lexical rules implicitly rather
than explicitly. It is implicit in the anchorings of certain lexical
items.  For example, ergative verbs such as {\it melt} anchor both
Tnx0Vnx1 and TEnx1V, and their lexical family is the set of trees in
the union of these two tree families.  See the chapters on ergatives
(see Chapter~\ref{ergatives}) and ditransitives (see
Chapter~\ref{double-objs}) for more in-depth discussions of particular
cases of lexical families.

To briefly summarize, we can now give the following definitions:

\begin{description}
\item[tree family:] the set of trees generated from a base tree by the
syntactic transformations which are available in the grammar.

\item[lexical family:] the set of tree families generated for a given
predicate from an initial base tree and the base trees which are
generated by the lexical rules which are specified for that predicate.

\end{description}

It should be noted, however, that we will continue to refer to tree families
and lexical families both by the generic name {\it tree family}, unless the
distinction is crucial to the point we are trying to make.

\section{Complements and Adjuncts}
\label{compl-adj}

Complements and adjuncts have very different structures in the XTAG grammar.
Complements are included in the elementary tree anchored by the verb that
selects them, while adjuncts do not originate in the same elementary tree as
the verb anchoring the sentence, but are instead added to a structure by
adjunction.  The contrasts between complements and adjuncts have been
extensively discussed in the linguistics literature and the classification of a
given element as one or the other remains a matter of debate (see
\cite{rizzi90},
\cite{larson88}, \cite{jackendoff90}, \cite{larson90}, \cite{cinque90}, 
\cite{obernauer84}, \cite{lasnik-saito84}, and \cite{chomsky86}).  The guiding
rule used in developing the XTAG grammar is whether or not the sentence is
ungrammatical without the questioned structure.\footnote{Iteration of a
structure can also be used as a diagnostic: {\it Srini bought a book at the
bookstore on Walnut Street for a friend}.} Consider the following
sentences:

\enumsentence{Srini bought a book.}
\enumsentence{Srini bought a book at the bookstore.}
\enumsentence{Fei ventured into the cave.}
\enumsentence{$\ast$Fei ventured.}

Prepositional phrases are common adjuncts, and when they are used as
adjuncts they have a tree structure such as that shown in
Figure~\ref{compl-adjunct}(a).  This adjunction tree would adjoin into
the tree shown in Figure~\ref{subcat-trees}(a) to generate sentence
(\ex{-2}).  There are verbs, however, such as {\it venture}, {\it
hunger} and {\it differentiate}, that take prepositional phrases as
complements.  Sentences (\ex{-1}) and (\ex{0}) clearly show that the
prepositional phrase are not optional for {\it venture}.
% For these
%sentences, the prepositional phrase will be an initial tree (as shown
%in Figure~\ref{compl-adjunct}(b)) that substitutes into an elementary
%tree, such as the one anchored by the verb {\it arrange} in
%Figure~\ref{compl-adjunct}(c).
For verbs such as this, the prepositional phrase is articulated within the
verb's elementary tree, as shown in Figure~\ref{compl-adjunct}(b). The
preposition and its complement noun phrase are substituted into this elementary
tree. 


\begin{figure}[ht]
\centering
\begin{tabular}{ccc}
{\psfig{figure=ps/compl-adj-files/betavxPnx_at.ps,height=1.8in}} &
\hspace{0.5in} &
%{\psfig{figure=ps/compl-adj-files/alphaPXPnx_for.ps,height=1.3in}} &
%\hspace{0.5in} & 
{\psfig{figure=ps/compl-adj-files/alphanx0Vpnx1_ventured_.ps,height=1.8in}}\\
(a) & \hspace{0.5in} & (b) \\ 
\end{tabular}
\caption{Trees illustrating the difference between Complements and Adjuncts}
\label{compl-adjunct}
\label{2;1,9}
\end{figure}


Virtually all parts of speech, except for main verbs, function as both
complements and adjuncts in the grammar.  More information is available in this
report on various parts of speech as complements: adjectives (e.g. section
\ref{nx0Vax1-family}), nouns (e.g.  section~\ref{nx0Vnx1-family}), and
prepositions (e.g. section~\ref{nx0Vpnx1-family}); and as adjuncts: adjectives
(section~\ref{adj-modifier}), adverbs (section~\ref{adv-modifier}), nouns
(section~\ref{noun-modifier}), and prepositions (section~\ref{prep-modifier}).

\section{Other assumptions made in grammar development}

The morphological, syntactic, and tree databases together comprise the
English grammar\footnote{ See Chapter~\ref{overview} for details on
these levels of representation. }.  A lexical item that is not in the
databases receives a default tree selection and features for its part
of speech and morphology.  

In designing the grammar, a decision was made early on to err on the
side of acceptance whenever there are conflicting opinions as to
whether or not a construction is grammatical.  In this sense, the XTAG
English grammar is intended to function primarily as an acceptor
rather than a generator of English sentences.  The range of syntactic
phenomena that can be handled is large and includes auxiliaries
(including inversion), copula, raising and small clause constructions,
topicalization, relative clauses, infinitives, gerunds, passives,
adjuncts, it-clefts, wh-clefts, PRO constructions, noun-noun
modifications, determiner sequences, genitives, negation, noun-verb
contractions, clausal adjuncts and imperatives.

\subsection{Non-S constituents}

Although sentential trees are generally considered to be special cases
in any grammar, insofar as they make up a `starting category', it is
the case that any initial tree constitutes a phrasal constituent.
These initial trees may have substitution nodes that need to be filled
(by other initial trees), and may be modified by adjunct trees,
exactly as the trees rooted in S.  Although grouping is possible
according to the heads or anchors of these trees, we have not found
any classification similar to the subcategorization frames for verbs
that can be used by a lexical entry to create a tree family of the set
of trees selected by such entries.  These trees are selected one by
one by each lexical item, according to each lexical item's
idiosyncrasies.  The grammar described by this technical report places
them into several files for ease of use, but these files do not
constitute tree families in the way that the subcategorization frames
do.








