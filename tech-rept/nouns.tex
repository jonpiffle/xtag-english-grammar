\section{Nouns}

There are two basic kinds of noun phrases in XTAG, those with and without a determiner.

\Nouns with sentential complements
\Predicative nouns
\Noun gerunds
\NP and N conjunction
\Relative clauses

\Noun-Noun Compounds

In the English LTAG grammar, all nouns select the trees shown in
figure \ref{n-comp} which allows them to do noun-noun modification.

\begin{figure}[h]
%put in n-n coupmpound tree(s)
\end{figure}

This allows for the derivation of NP's such as {\it the oatmeal cookies}
and {\it a basic initialization routine}

\begin{figure}[h]
%derived tree for the oatmeal cookies and a basic initialization routine
\end{figure}

Notice that the noun-noun modifier trees differ from the adjective
tree shown in figure \ref{adj-tree}.

\begin{figure}
%adj-tree
\end{figure}

This similarity accurately represents the shared noun modifying
function of nouns in compounds and adjectives.  Constraints on
noun-noun compounding that have been discussed in the linguistics
literature have been based on semantic considerations. Since we are
currently not attempting to encode sematics in the English LTAG
grammar, our implementation of noun-noun compounding is quite
unconstrained and produces multiple derivations. This greatly
increases ambiguity in the grammar, but this increase in ambiguity
seems a necessary one.  Different derivations are needed to account
for the difference between (\ex{1}) and (\ex{2}).

\enumsentence{[Chinese [history teacher]] (a chinese person who
teaches history)}
\enumsentence{[[Chinese history] teacher] ( person who teaches chinese
history)}







