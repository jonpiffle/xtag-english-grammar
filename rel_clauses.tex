\chapter{Relative Clauses}
\label{rel_clauses}

Relative clauses are NP modifiers. For relative clauses on arguments,
an argument in the clause is extracted, and the NP head (the
portion of the NP being modified by the relative clause) is
interpreted as having the same role in the clause as the extracted
item.  For example in (\ex{1}) {\it export exhibitions} is the head NP
and is modified by the relative clause {\it $\epsilon$ included high-tech
items}. {\it Export exhibitions} is interpreted as the subject of the
relative clause which is missing an overt subject.

\enumsentence{export exhibitions that included high-tech items}

Relative clauses are represented in the English XTAG grammar by auxiliary trees
that adjoin to NP's. These trees are anchored by the verb in the clause and
appear in the appropriate tree families for the various verb
subcategorizations. Within a tree family there will be groups of relative
clause trees based on the declarative tree and each passive tree. Within each
of these groups, there is a separate relative clause tree corresponding to each
possible argument that can be extracted from the clause. The relationship
between the extracted position and the head NP is captured by co-indexing the
{\bf $<$trace$>$} features of the extracted NP and the NP foot node in the
relative clause tree.  Representative examples from the transitive tree family
are shown with a relevant subset of their features in
Figures~\ref{trans-rel-clause-trees}(a) and \ref{trans-rel-clause-trees}(b).

\begin{figure}[htb]
\begin{tabular}{cc}
\psfig{figure=ps/rel_clauses-files/betaN1nx0Vnx1.ps,height=10.0cm}&
\psfig{figure=ps/rel_clauses-files/betaN0nx0Vnx1.ps,height=10.5cm}\\
(a)&(b)
\end{tabular}
\caption{Relative clause trees in the transitive tree family: $\beta$N1nx0Vnx1
(a) and $\beta$N0nx0Vnx1 (b)}
\label{trans-rel-clause-trees}
\label{2;16,1}
\label{2;15,1}
\end{figure}


Our treatment of relative clauses allows a single tree to provide the structure
for various relative clause types. For example, the tree shown in
Figure~\ref{trans-rel-clause-trees}(a) is used for all of the relative
clauses shown in sentences (\ex{1})-(\ex{6}).

\enumsentence{the man that Muriel likes}
\enumsentence{the man who Muriel likes}
\enumsentence{the man Muriel likes}
\enumsentence{what Muriel likes}
\enumsentence{the book for Muriel to read}
\enumsentence{the book Muriel is reading}

This variety of clause types is achieved through combinations of different
clause types using the {\bf $<$mode$>$} feature, different complementizers
using the {\bf $<$comp$>$} and {\bf $<$assign-comp$>$} features and {\bf
$<$wh$>$=+} or {\bf $<$wh$>$=--} NP heads. A detailed discussion of how the
{\bf $<$mode$>$}, {\bf $<$comp$>$} and {\bf $<$assign-comp$>$} features are
used to account for embedded clauses in general can be found in
Chapter~\ref{scomps-section}.

The relative pronouns {\it who} and {\it which} are treated as complementizers
restricted to relative clauses.  Their treatment as complementizers is
consistent with our treatment of the complementizers {\it that} and {\it for}
in other embedded clause environments as well as in relative clauses. Like
other complementizers, the relative complementizers use the tree in
Figure~\ref{betaCOMPs}.


\begin{figure}[htb]
\centering
\mbox{}
\psfig{figure=ps/rel_clauses-files/betaCOMPs.ps,height=6.5cm}
\caption{Complementizer tree: $\beta$COMPs}
\label{betaCOMPs}
\end{figure}

The relative complementizers, {\it who} and {\it which}, have {\bf rel} as
their value for the feature {\bf $<$comp$>$}. This feature value insures that
{\it who} and {\it which} do not adjoin onto sentential complements, subjects
or adjunct modifiers because only relative clause trees allow complementizers
with the value {\bf rel}.  Relative clause trees such as the one in
Figure~\ref{trans-rel-clause-trees}(a) also allow other complementizers with
the appropriate clause type. For example, in sentence (\ex{-1}) the infinitive
relative clause with an overt subject requires the complementizer {\it for}
just as an infinitive with an overt subject would in other embedded
clauses. Similarly, the adjunction of the complementizer {\it that} is optional
in indicative relative clauses with non-subject extractions, such as in
sentences (\ex{-5}) and (\ex{-3}), just as it is in sentential complements. The
same system of features, {\bf $<$comp$>$}, {\bf $<$mode$>$} and {\bf
$<$assign-comp$>$}, is used in all cases of embedded clauses, including
relative clauses, to insure the proper cooccurrence of complementizers and
clause types.

Under this account, free relatives as in sentence~(\ex{-2}) require no
additional mechanisms. They are simply {\bf $<$wh$>$=+} NP heads with
complementizerless relative clauses. For example, the clause {\it Mary likes
$\epsilon$}, using the tree in Figure~\ref{trans-rel-clause-trees}(a), adjoins
onto the NP {\it what} to derive {\it what Mary likes $\epsilon$}.

The English XTAG grammar does not contain any  syntactic distinction between
restrictive and non-restrictive relatives because we believe this to
be a semantic and/or pragmatic difference.
