\section{Relative Clauses}
Relative clauses are represented in the grammar by auxiliary trees
that adjoin to NP. These trees are anchored by the verb in the clause
and appear in the appropriate tree families for the various verb
subcategorizations. Our analysis of relative clauses allows a single
tree to provide the structure for various relative clause types. For
example, the Rnx0Vnx1 tree shown in () is used for the relative
clauses shown in ()-()

%insert tree

	() the man that Muriel likes
	() the man who Muriel likes
	() the man Muriel likes
	() what Muriel likes
	() the book for Muriel to read
	() the man reading the book

The relative pronouns who, and which are treated uniformly with that
as complementizers restricted to relative clauses. These relative
complementizers anchor the tree $\beta$???? and adjoin onto the S node
of the relative clauses trees.  The complementizer
analysis also extends to infinitives with for complementizers as in
example ().  
The relative clauses in ()-() vary by clause type, type of complementizer
and the wh-status of the head NP.  Examples ()-() have indicative
clauses with wh = - head NP's and show variation between that, who and
no complementizers.   
By not putting a restriction on the wh feature on the NP footnode of
relative clauses, free relatives such as () are accounted for as wh =
+ NP head with no complementizer.

