\section{Relative Clauses}
\label{rel_clauses}

Relative clauses are NP modifiers. For relative clauses on arguements,
an arguement in the clause is extracted, and the NP head (the
portion of the NP being modified by the relative clause) is
interpreted as having the same role in the clause as the extracted
item.  For example in (\ex{1}) {\it export exhibitions} is the head NP
and is modified by the relative clause {\it $\epsilon$ included high-tech
items}. {\it Export exhibitions} is interpreted as the subject of the
the relative clause which is missing an overt subject.

\enumsentence{export exhibitions that included high-tech items}

Relative clauses are represented in the English LTAG grammar by
auxiliary trees that adjoin to NPs. These trees are anchored by the
verb in the clause and appear in the appropriate tree families for the
various verb subcategorizations. Within a tree family there will be
group of relative clause trees based on the declarative tree and each
passive tree. Within each of these groups is a separate relative
clause tree corresponding to each possible arguement that can be
extracted from the clause. The relationship between the extracted
position and the head NP is captured by coindexing the {\bf
$<$trace$>$} features of the extracted NP and the NP footnode in the
relative clause tree.  Representative examples from
Tnx0Vnx1 are shown with a relevant subset
of their features in figures \ref{betaN1nx0Vnx1} and
\ref{betaN1nx0Vnx1}.

\begin{figure}[htbp]
\psfig{figure=/mnt/linc/extra/xtag/work/doc/tech-rept/ps/rel_clauses-files/betaN1nx0Vnx1.ps,height=15.0cm}
\caption{ \label{betaN1nx0Vnx1} $\beta$N1nx0Vnx1}
\end{figure}

\begin{figure}[htbp]
\psfig{figure=/mnt/linc/extra/xtag/work/doc/tech-rept/ps/rel_clauses-files/betaN0nx0Vnx1.ps,height=15.0cm}
\caption{ \label{betaN0nx0Vnx1} $\beta$N0nx0Vnx1}
\end{figure}


Our treatment of relative clauses allows a single tree to provide the
structure for various relative clause types. For example, the
$\beta$N1xn0Vnx1 tree shown in figure (\ref{betaN1nx0Vnx1}) is used
for all of the the relative clauses shown in (\ex{1})-(\ex{6})
\vspace{0.5cm}

\enumsentence{the man that Muriel likes}
\enumsentence{the man who Muriel likes}
\enumsentence{the man Muriel likes}
\enumsentence{what Muriel likes}
\enumsentence{the book for Muriel to read}
\enumsentence{the man reading the book}

This variety of clause types is achieved through combinations of
different clause types using the {\bf $<$mode$>$} feature, different
complementizers using the {\bf $<$comp$>$} and {\bf $<$assign-comp$>$}
and {\bf $+$} or {\bf $-$} {\bf $<$wh$>$} NP heads. A detailed
discussion of how the {\bf $<$mode$>$}, {\bf $<$comp$>$} and {\bf
$<$assign-comp$>$} features are used to account for embedded clauses
in general can be found in Section
\ref{scomps-section}.  

The relative pronouns {\it who}, and {\it which} are treated as
complementizers restricted to relative clauses.  Their treatment as
complementizers is consistent with our treatment of the
complementizers {\it that} and {\it for} in other embedded clause
environments as well as in relative clauses. Like other
complementizers, the relative complementizers use the tree
$\beta$COMPs in figure (\ref{betaCOMPs}).


\begin{figure}[htbp]
\center{
\psfig{figure=/mnt/linc/extra/xtag/work/doc/tech-rept/ps/rel_clauses-files/betaCOMPs.ps,height=10.0cm}
\caption{ \label{betaCOMPs} Tree:  $\beta$COMPs}
}
\end{figure}

The relative complementizers, {\it who} and {\it which}, have {\bf
rel} as their value for the feature {\bf $<$comp$>$}. This feature
value insures that {\it who} and {\it which} do not adjoin onto
sentential complements, subjects or adjunct modifiers because only
relative clause trees allow complementizers with the value {\bf rel}.
Relative clause trees such as $\beta$N1xn0Vnx1 in figure
(\ref{betaN1nx0Vnx1}) also allow other complementizers with the
approriate clause type. For example, in (\ex{-1}) the infinitive
relative clause with an overt subject requires the complementizer {\it
for} just as an infinitive with an overt subject would in other
embedded clauses. Similarly, the adjunction of the complementizer {\it
that} is optional in indicative relative clauses with non-subject
extractions, such as (\ex{-3}) and (\ex{-5}), just as it is in
sentential complements. The same system of features, $<$comp$>$,
$<$mode$>$, and $<$assign-comp$>$, is used in all cases of embedded
clauses including relative clauses to insure the proper cooccurance of
complementizers and clause types.

Under this type of account, free relatives such as (\ex{-3}) require
no additional mechanisms. They are simply $+$wh NP heads with
complementizerless relative clauses. For example, the clause {\it Mary likes
$\epsilon$} using the $\beta$N1nx0Vnx1 tree in figure
\ref{betaN1nx0Vnx1} adjoins onto the NP {\it what} to derive {\it
what Mary likes $\epsilon$}.

The English Xtag grammar does not contain any  syntactic distinction between
restrictive and non-restrictive relatives because we believe this to
be a semantic and/or pragmatic difference.
