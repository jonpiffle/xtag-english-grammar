\chapter{The English Copula, Raising Verbs, and Small Clauses}
\label{small-clauses}

The English copula, raising verbs, and small clauses are all handled in the
English XTAG grammar by a common analysis based on sentential clauses
headed by non-verbal elements.  Since there are a number of different
analyses in the literature of how these phenomena are related (or not), we
will present first the data for all three phenomena, then various analyses
from the literature, finishing with the analysis used in the English XTAG
grammar.\footnote{%
%
This chapter is strongly based on \cite{heycock91}.
Sections \ref{sm-clause-data} and \ref{sm-clause-other-analyses} are
greatly condensed from her paper, while the description of the XTAG
analysis in section \ref{sm-clause-xtag-analysis} is an updated and
expanded version.%
%
}

\section{Usages of the copula, raising verbs, and small clauses}
\label{sm-clause-data}

\subsection{Copula}
\label{copula-data}

The verb {\it be} as used in sentences {\ex{1}}-{\ex{3} is often
referred to as the \xtagdef{copula}.  It can be followed by a noun,
an adjective, or a prepositional phrase.

\enumsentence{Carl is a jerk .}
\enumsentence{Carl is upset .}
\enumsentence{Carl is in a foul mood .}

Although the copula may look like a main verb at first glance, its syntactic
behavior follows the auxiliary verbs rather than main verbs.  In particular,

\begin{itemize}
\item Copula {\it be} inverts with the subject.
\enumsentence{is Beth writing her dissertation ?\\
		is Beth upset ?\\
		$\ast$wrote Beth her dissertation ?}

\item Copula {\it be} occurs to the left of the negative marker {\it
not}.
\enumsentence{Beth is not writing her dissertation .\\
		Beth is not upset .\\
		$\ast$Beth wrote not her dissertation .}

\item Copula {\it be} can contract with the negative marker {\it not}.
\enumsentence{Beth isn't writing her dissertation .\\
		Beth isn't upset .\\
		$\ast$Beth wroten't her dissertation .}

\item Copula {\it be} can contract with pronominal subjects.
\enumsentence{She's writing her dissertation .\\
		She's upset .\\
		$\ast$She'ote her dissertation .}

\item Copula {\it be} occurs to the left of adverbs in the unmarked order.
\enumsentence{Beth is often writing her dissertation .\\
		Beth is often upset .\\
		$\ast$Beth wrote often her dissertation .}
\end{itemize}

Unlike all the other auxiliaries, however, copula {\it be} is not followed by
a verbal category (by definition) and therefore must be the rightmost verb.  In
this respect, it is like a main verb.

The semantic behavior of the copula is also unlike main verbs.  In particular,
any semantic restrictions or roles placed on the subject come from the
complement phrase (NP, AP, PP) rather than from the verb, as illustrated in
sentences ({\ex{1}}) and ({\ex{2}}).  Because the complement phrases predicate
over the subject, these types of sentences are often called
\xtagdef{predicative} sentences.

\enumsentence{The bartender was garrulous .}
\enumsentence{?The cliff was garrulous .}


\subsection{Raising Verbs}
\label{raising-verbs}

Raising verbs are the class of verbs that share with the copula the property
that the complement, rather than the verb, places semantic constraints on
the subject.  

\enumsentence{Carl seems a jerk .\\
		Carl seems upset .\\
		Carl seems in a foul mood .}

\enumsentence{Carl appears a jerk .\\
		Carl appears upset .\\
		Carl appears in a foul mood .}

The raising verbs are similar to auxiliaries in that they order with other
verbs, but they are unique in that they can appear to the left of the
infinitive, as seen in the sentences in ({\ex{1}}).  They cannot, however,
invert or contract like other auxiliaries ({\ex{2}}), and they appear to the
right of adverbs ({\ex{3}}).

\enumsentence{Carl seems to be a jerk .\\
		Carl seems to be upset .\\
		Carl seems to be in a foul mood .}

\enumsentence{$\ast$seems Carl to be a jerk ?\\
		$\ast$Carl seemn't to be upset .\\
		$\ast$Carl`ems to be in a foul mood .}

\enumsentence{Carl often seems to be upset .\\
		$\ast$Carl seems often to be upset .}


\subsection{Small Clauses}

One way of describing small clauses is as predicative sentences without the
copula.  Since matrix clauses require tense, these clausal structures appear
only as embedded sentences.  They occur as complements of certain verbs, each
of which may allow certain types of small clauses but not others, depending on its
lexical idiosyncrasies.

\enumsentence{I consider [Carl a jerk] .\\
		I consider [Carl upset] .\\
		?I consider [Carl in a foul mood] .}

\enumsentence{I prefer [Carl in a foul mood] .\\
		??I prefer [Carl upset] .}


\subsection{Raising Adjectives}
\label{raising-adjs}

Raising adjectives are the class of adjectives that 
share with the copula and raising verbs the property
that the complement, rather than the verb, places semantic constraints on
the subject.  

They appear with the copula in a matrix clause, as in ({\ex{1}}).  However,
in other cases, such as that of small clauses ({\ex{2}}), they do not
have to appear with the copula.

\enumsentence{Carl is likely to be a jerk .\\
		Carl is likely to be upset .\\
		Carl is likely to be in a foul mood .\\
                Carl is likely to perjure himself .}

\enumsentence{I consider Carl likely to perjure himself .}



\section{Various Analyses}
\label{sm-clause-other-analyses}

\subsection{Main Verb Raising to INFL + Small Clause}

In \cite{pollack89} the copula is generated as the head of a VP, like any main
verb such as {\it sing} or {\it buy}. Unlike all other main verbs\footnote{with
the exception of {\it have} in British English. See
footnote~\ref{have-footnote} in Chapter~\ref{auxiliaries}.}, however, {\it be}
moves out of the VP and into Infl in a tensed sentence.  This analysis aims to
account for the behavior of {\it be} as an auxiliary in terms of inversion,
negative placement and adverb placement, while retaining a sentential structure
in which {\it be} heads the main VP at D-Structure and can thus be the only
verb in the clause.

Pollock claims that the predicative phrase is not an argument of {\it be},
which instead he assumes to take a small clause complement, consisting of a
node dominating an NP and a predicative AP, NP or PP. The subject NP of the
small clause then raises to become the subject of the sentence.  This accounts
for the failure of the copula to impose any selectional restrictions on the
subject.  Raising verbs such as {\it seem} and {\it appear}, presumably, take the
same type of small clause complement.

\subsection{Auxiliary + Null Copula}
\label{la}

In \cite{lapointe80} the copula is treated as an auxiliary verb that takes as its
complement a VP headed by a passive verb, a present participle, or a null verb
(the true copula). This verb may then take AP, NP or PP complements.  The
author points out that there are many languages that have been analyzed as
having a null copula, but that English has the peculiarity that its
null copula requires the co-presence of the auxiliary {\it be}.

\subsection{Auxiliary + Predicative Phrase}
\label{gpsg}

In GPSG (\cite{gazdar85}, \cite{sag85}) the copula is treated as an auxiliary
verb that takes an X$^{2}$ category with a + value for the head feature
[PRD] (predicative). AP, NP, PP and VP can all be [+PRD], but a
Feature Co-occurrence Restriction guarantees that a [+PRD] VP will be
headed by a verb that is either passive or a present participle.

GPSG follows \cite{chomsky70} in adopting the binary valued features [V] and
[N] for decomposing the verb, noun, adjective and preposition categories.  In
that analysis, verbs are [+V,$-$N], nouns are [$-$V,+N], adjectives [+V,+N] and
prepositions [$-$V,$-$N].  NP and AP predicative complements generally pattern
together; a fact that can be stated economically using this category
decomposition.  In neither \cite{sag85} nor \cite{chomsky70} is there any
discussion of how to handle the complete range of complements to a verb like
{\it seem}, which takes AP, NP and PP complements, as well as infinitives.  The
solution would appear to be to associate the verb with two sets of rules for
small clauses, leaving aside the use of the verb with an expletive subject and
sentential complement.

\subsection{Auxiliary + Small Clause}

\label{mo}
In \cite{moro90} the copula is treated as a special functional category - a
lexicalization of tense, which is considered to head its own projection. It
takes as a complement the projection of another functional category, Agr
(agreement). This projection corresponds roughly to a small clause, and is
considered to be the domain within which predication takes place.  An NP must
then raise out of this projection to become the subject of the sentence: it may
be the subject of the AgrP, or, if the predicate of the AgrP is an NP, this may
raise instead.  In addition to occurring as the complement of {\it be}, AgrP is
selected by certain verbs such as {\it consider}. It follows from this analysis
that when the complement to {\it consider} is a simple AgrP, it will always
consist of a subject followed by a predicate, whereas if the complement
contains the verb {\it be}, the predicate of the AgrP may raise to the left of
{\it be}, leaving the subject of the AgrP to the right.

\enumsentence{John$_{i}$ is [$_{AgrP}$ $t_{i}$ the culprit ] .}
\enumsentence{The culprit$_{i}$ is [$_{AgrP}$ John $t_{i}$ ] .}
\enumsentence{I consider [$_{AgrP}$ John the culprit] .}
\enumsentence{I consider [John$_{i}$ to be [$_{AgrP}$ $t_{i}$ the culprit ]] .}
\enumsentence{I consider [the culprit$_{i}$ to be [$_{AgrP}$ John $t_{i}$ ]] .}

Moro does not discuss a number of aspects of his analysis, including the
nature of Agr and the implied existence of sentences without VP's. 

\section{XTAG analysis}
\label{sm-clause-xtag-analysis}

\begin{figure}[htbp]
\centering
\begin{tabular}{ccccc}
{\psfig{figure=ps/sm-clause-files/alphanx0N1.ps,height=2.3in}} &
\hspace{0.5in} &
{\psfig{figure=ps/sm-clause-files/alphanx0Ax1.ps,height=2.4in}} &
\hspace{0.5in} &
{\psfig{figure=ps/sm-clause-files/alphanx0Pnx1.ps,height=2.4in}} \\
(a)&&(b)&&(c)\\
\end{tabular}
\caption{Predicative trees: $\alpha$nx0N1 (a), $\alpha$nx0Ax1 (b) and $\alpha$nx0Pnx1 (c)}
\label{predicative-trees}
\label{1;1,7}
\label{1;1,9}
\end{figure}

The XTAG grammar provides a uniform analysis for the copula, raising verbs and
small clauses by treating the maximal projections of lexical items that can be
predicated as predicative clauses, rather than simply noun, adjective and
prepositional phrases.  The copula adjoins in for matrix clauses, as do the
raising verbs.  Certain other verbs (such as {\it consider}) can take the
predicative clause as a complement, without the adjunction of the copula, to
form the embedded small clause.

The structure of a predicative clause, then, is roughly as seen in
({\ex{1}})-({\ex{3}}) for NP's, AP's and PP's.  The XTAG trees corresponding
to these structures\footnote{There are actually two other predicative trees in
the XTAG grammar.  Another predicative noun phrase tree is needed for noun
phrases without determiners, as in the sentence {\it They are firemen}, and
another prepositional phrase tree is needed for exhaustive prepositional
phrases, such as {\it The workers are below}.} are shown in
Figures~\ref{predicative-trees}(a),
\ref{predicative-trees}(b), and \ref{predicative-trees}(c), 
respectively.

\enumsentence{[$_{S}$ NP [$_{VP}$  N \ldots ]]}
\enumsentence{[$_{S}$ NP [$_{VP}$  A \ldots ]]}
\enumsentence{[$_{S}$ NP [$_{VP}$  P \ldots ]]}



The copula {\it be} and raising verbs all get the basic auxiliary tree as
explained in the section on auxiliary verbs (section \ref{aux-non-inverted}).
Unlike the raising verbs, the copula also selects the inverted auxiliary tree
set.  Figure~\ref{Vvx-with-nomprep} shows the basic auxiliary tree anchored by
the copula {\it be}.  The {\bf $<$mode$>$} feature is used to distinguish the
predicative constructions so that only the copula and raising verbs adjoin onto
the predicative trees.  

\begin{figure}[htb]
\centering
\begin{tabular}{c}
{\psfig{figure=ps/sm-clause-files/betaVvx_is-with-features.ps,height=5.7in}} \\
\end{tabular}
\caption{Copula auxiliary tree: $\beta$Vvx}
\label{Vvx-with-nomprep}
\end{figure}

There are two possible values of {\bf $<$mode$>$} that correspond to the
predicative trees, {\bf nom} and {\bf prep}.  They correspond to a modified
version of the four-valued [N,V] feature described in section \ref{gpsg}.  The
{\bf nom} value corresponds to [N+], selecting the NP and AP predicative
clauses.  As mentioned earlier, they often pattern together with respect to
constructions using predicative clauses.  The remaining prepositional phrase
predicative clauses, then, correspond to the {\bf prep} mode.

Figure~\ref{upset-with-features} shows the predicative adjective tree from
Figure~\ref{predicative-trees}(b) now anchored by {\it upset} and with the
features visible.  As mentioned, {\bf $<$mode$>$=nom} on the VP node prevents
auxiliaries other than the copula or raising verbs from adjoining into this
tree.  In addition, it prevents the predicative tree from occurring as a matrix
clause.  Since all matrix clauses in XTAG must be mode indicative ({\bf ind})
or imperative ({\bf imp}), a tree with {\bf $<$mode$>$=nom} or {\bf
$<$mode$>$=prep} must have an auxiliary verb (the copula or a raising verb)
adjoin in to make it {\bf $<$mode$>$=ind}.


\begin{figure}[htb]
\centering
\begin{tabular}{c}
{\psfig{figure=ps/sm-clause-files/alphanx0Ax1_upset-with-features.ps,height=6.3in}} \\
\end{tabular}
\caption{Predicative AP tree with features: $\alpha$nx0Ax1}
\label{upset-with-features}
\label{1;1,4}
\end{figure}

The distribution of small clauses as embedded complements to some verbs is also
managed through the mode feature.  Verbs such as {\it consider} and {\it
prefer} select trees that take a sentential complement, and then restrict that
complement to be {\bf $<$mode$>$=nom} and/or {\bf $<$mode$>$=prep},
depending on the lexical idiosyncrasies of that particular verb.  Many verbs
that don't take small clause complements do take sentential complements that
are {\bf $<$mode$>$=ind}, which includes small clauses with the copula
already adjoined.  Hence, as seen in sentence sets ({\ex{1}})-({\ex{3}}),
{\it consider} takes only small clause complements, {\it prefer} takes both
{\bf prep} (but not {\bf nom}) small clauses and indicative clauses, while {\it
feel} takes only indicative clauses.

\enumsentence{She considers Carl a jerk .\\
		?She considers Carl in a foul mood .\\
		$\ast$She considers that Carl is a jerk .}

\enumsentence{$\ast$She prefers Carl a jerk .\\
		She prefers Carl in a foul mood .\\
		She prefers that Carl is a jerk .}

\enumsentence{$\ast$She feels Carl a jerk .\\
		$\ast$She feels Carl in a foul mood .\\
		She feels that Carl is a jerk .}

\noindent
Figure \ref{consider-with-features} shows the tree anchored by {\it consider}
that takes the predicative small clauses.

\begin{figure}[htb]
\centering
\begin{tabular}{c}
{\psfig{figure=ps/sm-clause-files/betanx0Vs1_consider-with-features.ps,height=2.3in}} \\
\end{tabular}
\caption{{\it Consider} tree for embedded small clauses}
\label{consider-with-features}
\end{figure}

Raising verbs such as {\it seems} work essentially the same as the
auxiliaries, in that they also select the basic auxiliary tree, as in
Figure~\ref{Vvx-with-nomprep}.  The only difference is that 
the value of {\bf $<$mode$>$} 
on the VP foot node might be different, depending on what types of
complements the raising verb takes.  Also, two of the raising verbs take
an additional tree, $\beta$Vpxvx, shown in Figure~\ref{Vpxvx}, which
allows for an experiencer argument, as in {\it John seems to me
to be happy}.  

\begin{figure}[htb]
\centering
\begin{tabular}{c}
{\psfig{figure=ps/sm-clause-files/betaVpxvx.ps,height=2.0in}} \\
\end{tabular}
\caption{Raising verb with experiencer tree: $\beta$Vpxvx}
\label{Vpxvx}
\end{figure}


Raising adjectives, such as {\it likely}, take the tree shown in
Figure~\ref{Vvx-adj}.  This tree combines aspects of the auxiliary
tree $\beta$Vvx and the adjectival predicative tree shown in
Figure~\ref{predicative-trees}(b).  As with $\beta$Vvx, it adjoins
in as a VP auxiliary tree.  However, since it is anchored by an
adjective, not a verb, it is similar to the adjectival predicative
tree in that it has an $\epsilon$ at the V node, and a feature value
of {\bf $<$mode$>$=nom} which is passed up to the VP root indicates
that it is an adjectival predication.  This serves the same purpose
as in the 
case of the tree in Figure~\ref{upset-with-features}, and forces another
auxiliary verb, such as the copula, to adjoin in to make it
{\bf $<$mode$>$=ind}.

\begin{figure}[htb]
\centering
\begin{tabular}{c}
{\psfig{figure=ps/sm-clause-files/betaVvx-adj.ps,height=2.0in}} \\
\end{tabular}
\caption{Raising adjective tree: $\beta$Vvx-adj}
\label{Vvx-adj}
\end{figure}

\subsection{Raising Passives}
\label{sm-clause-xtag-ECM}
As discussed in Section~\ref{ecm-verbs}, the passives of Exceptional
Case Marking verbs are also treated as raising verbs.  The passive
therefore also selects the Vvx tree, and like other raising verbs such
as {\it seems} will select for an infinitival complement.  However, 
unlike {\it seems}, the {\bf $<$mode$>$} feature is set to {\bf ppart},
which therefore forces an auxiliary to also adjoin, as with other passives,
as described in Chapter~\ref{passives}.  For example, to derive (\ex{1}), 
the Vvx tree for {\it was} adjoins at the root of the 
tree for {\it expected} in in Figure~\ref{expects-passive}(a), which
adjoins into a 
derivation for {\it Bob to talk} at the VP node.

\enumsentence{Bob was expected to talk .}

\begin{figure}[hbt]
\centering
\begin{tabular}{ccccc}
{\psfig{figure=ps/sm-clause-files/expects-Vvx.ps,height=1.5in}} & 
\hspace{0.4in}&
{\psfig{figure=ps/sm-clause-files/betaVvxbynx_expected_.ps,height=1.5in}} & 
\hspace{0.4in}&
{\psfig{figure=ps/sm-clause-files/betaVbynxvx_expected_.ps,height=1.5in}} \\
(a) & & (b) & & (c) \\
\end{tabular}
\caption{ECM raising passive trees: $\beta$Vvx (a), $\beta$Vvxbynx (b), 
 $\beta$Vbynxvx (c) }
\label{expects-passive}
\end{figure}


Also, the {\it by} phrase associated with the raising passives can
appear to the left or right of the infinitival complement, as in
(\ex{1}) and (\ex{2}).

\enumsentence{Bob was expected by Bill to talk .}
\enumsentence{Bob was expected to talk by Bill .}

To handle these cases, the raising passives such as {\it expect} can
also select the trees shown in figures~\ref{expects-passive}(b)
and \ref{expects-passive}(c).  ECM verbs such as {\it expect} therefore
select the ECM tree family and also, separately, the three trees in 
figure~\ref{expects-passive}.

Also, it has long been noted that passives of both full and bare infinitive 
ECM constructions are full infinitives, as in (\ex{1}) and (\ex{2}).

\enumsentence{Bob sees the harmonica fall .}
\enumsentence{The harmonica was seen to fall .}
\enumsentence{$\ast$The harmonica was seen fall .}

Under the TAG ECM analysis, this fact is easy to implement.  The foot
node of the ECM passive tree is simply set to have {\bf $<$mode$>$=inf},
which prevents the derivation of (\ex{0}).  Therefore, while verbs
selecting full and bare infinitives will differ in the {\bf $<$mode$>$} 
specified for the active form when selecting the ECM tree family, they
all use {\bf $<$mode$>$=inf} for the passive tree.

Selecting the passive tree separately from the ECM tree family for the
active trees has the advantage of allowing verbs which only have the passive to
not select the active trees.  There are a number of such cases,
as in (\ex{1}), where
the raising passive exists but not its active counterpart.  For more 
discussion of this issue, see \cite{kj85}.

\enumsentence{John is said to be a crook .}

There are also some cases for which the raising passive complement does
not take {\bf $<$mode$>$=inf}, but rather a {\bf nom/prep} complement,
as in (\ex{1}) and (\ex{2}):

\enumsentence{John is considered a crook .}
\enumsentence{The Americans are believed involved in the coup .}
\enumsentence{$\ast$They believe the Americans involved in the coup .}

Such cases are handled by selecting the Vvx, Vbynxvx, and Vvxbynx trees
with the appropriate {\bf $<$mode$>$} on the foot node, instead of just
{\bf inf}.  In some cases, as in (\ex{0}), the corresponding active
sentence sounds quite bad, and this is prohibited by not selecting the
ECM family for the active cases with that mode.  For example, {\it believe}
selects the ECM family with {\bf $<$mode$>$=inf} and the raising passive
trees with {\bf $<$mode$>$=inf/nom/prep}.

\section{Non-predicative {\it BE}}
\label{equative-be-xtag-analysis}

The examples with the copula that we have given seem to indicate that {\it be}
is always followed by a predicative phrase of some sort.  This is not the case,
however, as seen in sentences such as ({\ex{1}})-({\ex{6}}).  The noun phrases in
these sentences are not predicative.  They do not take raising verbs, and they
do not occur in embedded small clause constructions.

\enumsentence{my teacher is Mrs. Wayman .}
\enumsentence{Doug is the man with the glasses .}

\enumsentence{$\ast$My teacher seems Mrs. Wayman .}
\enumsentence{$\ast$Doug appears the man with the glasses .}

\enumsentence{$\ast$I consider [my teacher Mrs. Wayman] .}
\enumsentence{$\ast$I prefer [Doug the man with the glasses] .}

In addition, the subject and complement can exchange positions in these type of
examples but not in sentences with predicative {\it be}.  Sentence ({\ex{1}})
has the same interpretation as sentence ({\ex{-4}}) and differs only in the
positions of the subject and complement NP's. Similar sentences, with a
predicative {\it be}, are shown in ({\ex{2}}) and ({\ex{3}}).  In this case,
the sentence with the exchanged NP's ({\ex{3}}) is ungrammatical.

\enumsentence{The man with the glasses is Doug .}
\enumsentence{Doug is a programmer .}
\enumsentence{$\ast$A programmer is Doug .}

The non-predicative {\it be} in ({\ex{-8}}) and ({\ex{-7}}), also called
\xtagdef{equative be}, patterns differently, both syntactically and
semantically, from the predicative usage of {\it be}.  Since these sentences
are clearly not predicative, it is not desirable to have a tree structure that
is anchored by the NP, AP, or PP, as we have in the predicative sentences.  In
addition to the conceptual problem, we would also need a mechanism to block
raising verbs from adjoining into these sentences (while allowing them for true
predicative phrases), and prevent these types of sentence from being embedded
(again, while allowing them for true predicative phrases).  

%%An additional
%%indication that distinct trees are necessary is the difference in the
%%grammaticality of extraction for the material to the right of the predicative
%%and non-predicative {\it be}.  The sentences in ({\ex{1}}) contain predicative
%%{\it be} as evidenced by the ungrammaticality of switching the subject {\it the
%%challenger} with the clause {\it to provide a non-stipulative solution}. In
%%contrast the grammaticality of the sentence produced by the equivalent switch
%%in ({\ex{2}}) shows that the {\it be} in those sentences is
%%non-predicative. Notice that extraction is possible from the clause with
%%predicative {\it be} but not with non-predicative {\it be}.
%%
%%\enumsentence{The challenger is to provide a non-stipulative solution .\\
%%$\ast$To provide a non-stipulative solution is the challenger .\\
%%What$_{i}$ is the challenger to provide $t_{i}$ ?}
%%
%%\enumsentence{The challenge is to provide a non-stipulative solution .\\
%%To provide a non-stipulative solution is the challenge .\\
%%		$\ast$What$_{i}$ is the challenge to provide $t_{i}$ ?}
%%
%%The ungrammaticality of the extracted NP in ({\ex{0}}) is explained if the
%%clause is a complement added by substitution, since extraction is not possible
%%from within substituted elements.  This is true in XTAG of the complements of
%%equative {\it be}.  In contrast, the predicative {\it be} in ({\ex{-1}})
%%adjoins to {\it What$_{i}$ the challenger to provide
%%$\epsilon_{i}$}\footnote{The elementary tree underlying this sentence would not
%%include the infinitive {\it to}.  Since the details of how {\it to} is adjoined
%%do not affect the thrust of the current argument, we assume a stage of
%%derivation where {\it to} has already been adjoined.}. Since the extraction is
%%already part of the elementary tree anchored by {\it provide} in the
%%predicative analysis, the extraction is unproblematic.

\begin{figure}[htb]
\centering
\begin{tabular}{ccc}
{\psfig{figure=ps/sm-clause-files/alphanx0BEnx1_is.ps,height=1.9in}} &
\hspace{1.0in}&
{\psfig{figure=ps/sm-clause-files/alphaInvnx0BEnx1_is.ps,height=2.5in}} \\
(a)&&(b)\\
\end{tabular}
\caption{Equative {\it BE} trees: $\alpha$nx0BEnx1 (a) and $\alpha$Invnx0BEnx1 (b)}
\label{equative-be}
\label{1;1,6}
\end{figure}

Although non-predicative {\it be} is not a raising verb, it does exhibit the
auxiliary verb behavior set out in section \ref{copula-data}.  It inverts,
contracts, and so forth, as seen in sentences ({\ex{1}}) and ({\ex{2}}), and
therefore can not be associated with any existing tree family for main verbs.
It requires a separate tree family that includes the tree for inversion.
Figures~\ref{equative-be}(a) and \ref{equative-be}(b) show the declarative and
inverted trees, respectively, for equative {\it be}.

\enumsentence{is my teacher Mrs. Wayman ?}
\enumsentence{Doug isn't the man with the glasses .} 


