\chapter{Determiners and Noun Phrases}
\label{det-comparitives}

In our English XTAG grammar,\footnote{A more detailed discussion of
this analysis can be found in \cite{ircs:det98}.} all nouns select the
noun phrase (NP) tree structure shown in Figure~\ref{np-tree}.  Common
nouns do not require determiners in order to form grammatical
NPs. Rather than being ungrammatical, singular countable nouns without
determiners are restricted in interpretation and can only be
interpreted as mass nouns.  Allowing all nouns to head determinerless
NPs correctly treats the individuation in countable NPs as a property
of determiners. Common nouns have negative(``-'') values for
determiner features in the lexicon in our analysis and can only
acquire a positive(``+'') value for those features if determiners
adjoin to them.  Other types of NPs such as pronouns and proper nouns
have been argued by Abney \cite{Abney87} to either be determiners or
to move to the determiner position because they exhibit
determiner-like behavior. We can capture this insight in our system by
giving pronouns and proper nouns positive values for determiner
features. For example pronouns and proper nouns would be marked as
definite, a value that NPs containing common nouns can only obtain by
having a definite determiner adjoin. In addition to the determiner
features, nouns also have values for features such as reflexive
({\bf refl}), case, pronoun ({\bf pron}) and conjunction ({\bf conj}).


\begin{figure}[ht]
\centering
\begin{tabular}{c}
%\rule[.1in]{8.35cm}{0.01in}
{\psfig{figure=/mnt/linc/xtag/work/doc/tech-rept/ps/det-files/alphaNXN.ps,height=16.0cm}}\\
\end{tabular}

\caption{NP Tree}
\label{np-tree}
%\rule[.1in]{8.35cm}{0.01in}
\end{figure}

  A single tree structure is selected by simple determiners, an
auxiliary tree which adjoins to NP. An example of this determiner tree
anchored by the determiner {\it these\/} is shown in
Figure~\ref{det-trees}. In addition to the determiner features the
tree in Figure~\ref{det-trees} has noun features such as {\bf case}
(see section 4.4.2), the {\bf conj} feature to control conjunction
(see Chapter \ref{conjunction}), {\bf rel-clause$-$} (see Chapter
\ref{rel_clauses}) and {\bf gerund$-$} (see Chapter
\ref{gerunds-chapter}) which prevent determiners from adjoining on top
of relative clauses and gerund NPs respectively, and the {\bf
displ-const} feature which is used to simulate multi-component
adjunction.

Complex determiners such as genitives and partitives also anchor tree
structures that adjoin to NP. They differ from the simple determiners
in their internal complexity. Details of our treatment of these more
complex constructions appear in Sections \ref{genitives} and
\ref{partitives}.  Sequences of determiners, as in the NPs {\it all
her dogs\/} or {\it those five dogs\/} are derived by multiple
adjunctions of the determiner tree, with each tree anchored by one of
the determiners in the sequence. The order in which the determiner
trees can adjoin is controlled by features.

\begin{figure}[ht]
\centering
\begin{tabular}{c}
%\rule[.1in]{8.35cm}{0.01in}
{\psfig{figure=/mnt/linc/xtag/work/doc/tech-rept/ps/det-files/betaDnx-these.ps,height=14cm}}
\end{tabular}
\caption{Determiner Trees with Features}
\label{det-trees}
%\rule[.1in]{8.35cm}{0.01in}
\end{figure}


This treatment of determiners as adjoining onto NPs is similar to that
of \cite{Abeille90:TAG}, and allows us to capture one of the insights of the DP
hypothesis, namely that determiners select NPs as complements. In
Figure~\ref{det-trees} the determiner and its NP complement appear in
the configuration that is typically used in LTAG to represent
selectional relationships. That is, the head serves as the anchor of
the tree and it's complement is a sister node in the same elementary tree.


The XTAG treatment of determiners uses nine features for representing
their properties: definiteness ({\bf definite}), quantity
({\bf quan}), cardinality ({\bf card}), genitive ({\bf gen}), 
decreasing ({\bf decreas}), constancy ({\bf const}), {\bf wh}, agreement ({\bf
agr}), and complement ({\bf compl}). Seven of these
features were developed by semanticists for their accounts of semantic
phenomena (\cite{KeenanStavi86:LP}, \cite{BarwiseCooper81:LP},
\cite{Partee90:BK}), another was developed for a semantic
account of determiner negation by one of the authors of this
determiner analysis (\cite{Mateyak97}), and the last is the familiar
agreement feature. When used together these features also account for
a substantial portion of the complex patterns of English determiner
sequencing. Although we do not claim to have exhaustively covered the
sequencing of determiners in English, we do cover a large subset, both
in terms of the phenomena handled and in terms of corpus coverage. The
XTAG grammar has also been extended to include complex determiner
constructions such as genitives and partitives using these determiner
features.

Each determiner carries with it a set of values for these features
that represents its own properties, and a set of values for the
properties of NPs to which can adjoin. The features are crucial to
ordering determiners correctly. The semantic definitions
underlying the features are given below.

\begin{description}

\item[Definiteness:] Possible Values [+/--]. \\
A function f is definite iff f is non-trivial and whenever
f(s)~$\neq~\emptyset$ then it is always the intersection of one or
more individuals.  \cite{KeenanStavi86:LP}

\item[Quantity:]  Possible Values [+/--]. \\
If A and B are sets denoting an NP and associated predicate, respectively; E is
a domain in a model M, and F is a bijection from M$_{1}$ to M$_{2}$, then we
say that a determiner satisfies the constraint of quantity if
Det$_{E_{1}}$AB~$\leftrightarrow$~Det$_{E_{2}}$F(A)F(B). \cite{Partee90:BK}

\item[Cardinality:]  Possible Values [+/--]. \\
A determiner D is cardinal iff D $\in$ cardinal numbers~$\geq$~1.

\item[Genitive:]  Possible Values [+/--]. \\
Possessive pronouns and the possessive morpheme ({\it 's}) are marked {\bf
gen$+$}; all other nouns are {\bf gen$-$}.

\item[Decreasing:]  Possible Values [+/--]. \\
A set of Q properties is decreasing iff whenever s$\leq$t and t$\in$Q then
s$\in$Q. A function f is decreasing iff for all properties f(s) is a decreasing
set.

A non-trivial NP (one with a Det) is decreasing iff its denotation in any model
is decreasing. \cite{KeenanStavi86:LP}

\item[Constancy:] Possible Values [+/--]. \\
If A and B are sets denoting an NP and associated predicate, respectively, and
E is a domain, then we say that a determiner displays constancy if
(A$\cup$B)~$\subseteq$~E~$\subseteq$~E$^{\prime}$ then
Det$_{E}$AB~$\leftrightarrow$~Det$_{E^{\prime}}$AB. Modified from
\cite{Partee90:BK}

\item[Complement:] Possible Values [+/--]. \\
A determiner Q is positive complement if and only if for every set X, there
exists a continuous set of possible values for the size of the negated
determined set, NOT(QX), and the cardinality of QX is the only aspect of QX
that can be negated. (adapted from \cite{Mateyak97})

\end{description}

The {\bf wh}-feature has been discussed in the linguistics literature mainly in relation to wh-movement and with respect to NPs and nouns as well as determiners. We give a shallow but useful working definition of the {\bf wh}-feature below:

\begin{description}

\item[Wh:]  Possible Values [+/--]. \\
Interrogative determiners are {\bf wh$+$}; all other determiners are
{\bf wh$-$}. 
\end{description}

The {\bf agr} feature is inherently a noun feature.  While determiners
are not morphologically marked for agreement in English many of them
are sensitive to number.  Many determiners are semantically either
singular or plural and must adjoin to nouns which are the same. For
example, {\it a\/} can only adjoin to singular nouns ({\it a dog\/} vs
{\it $\ast$a dogs\/} while {\it many\/} must have plurals ({\it many
dogs\/} vs {\it $\ast$many dog\/}). Other determiners such as {\it some} are
unspecified for agreement in our analysis because they are compatible
with either singulars or plurals ({\it some dog}, {\it some
dogs}). The possible values of agreement for determiners are: [3sg, 3pl, 3].




The determiner tree in Figure~\ref{det-trees} shows the appropriate
feature values for the determiner {\it these}, while Table \ref{det-values}
shows the corresponding feature values of several other common determiners.


\begin{table}
\centering
\begin{tabular}{|l||c|c|c|c|c|c|c|c|c|}
\hline
Det&definite&quan&card&gen&wh&decreas&const&agr&compl\\
\hline
\hline
all&$-$&$+$&$-$&$-$&$-$&$-$&$+$&3pl&$+$\\
both&$+$&$-$&$-$&$-$&$-$&$-$&$+$&3pl&$+$\\
this&$+$&$-$&$-$&$-$&$-$&$-$&$+$&3sg&$-$\\
these&$+$&$-$&$-$&$-$&$-$&$-$&$+$&3pl&$-$\\
that&$+$&$-$&$-$&$-$&$-$&$-$&$+$&3sg&$-$\\
those&$+$&$-$&$-$&$-$&$-$&$-$&$+$&3pl&$-$\\
what&$-$&$-$&$-$&$-$&$+$&$-$&$+$&3&$-$\\
whatever&$-$&$-$&$-$&$-$&$-$&$-$&$+$&3&$-$\\
which&$-$&$-$&$-$&$-$&$+$&$-$&$+$&3&$-$\\
whichever&$-$&$-$&$-$&$-$&$-$&$-$&$+$&3&$-$\\
the&$+$&$-$&$-$&$-$&$-$&$-$&$+$&3&$-$\\
each&$-$&$+$&$-$&$-$&$-$&$-$&$+$&3sg&$-$\\
every&$-$&$+$&$-$&$-$&$-$&$-$&$+$&3sg&$+$\\
a/an&$-$&$+$&$-$&$-$&$-$&$-$&$+$&3sg&$+$\\
some$_{1}$&$-$&$+$&$-$&$-$&$-$&$-$&$+$&3&$-$\\
some$_{2}$&$-$&$+$&$-$&$-$&$-$&$-$&$-$&3pl&$-$\\
any&$-$&$+$&$-$&$-$&$-$&$-$&$+$&3sg&$+$\\
another&$-$&$+$&$-$&$-$&$-$&$-$&$+$&3sg&$+$\\
few&$-$&$+$&$-$&$-$&$-$&$+$&$-$&3pl&$-$\\
a few&$-$&$+$&$-$&$-$&$-$&$-$&$+$&3pl&$-$\\
many&$-$&$+$&$-$&$-$&$-$&$-$&$-$&3pl&$+$\\
many a/an&$-$&$+$&$-$&$-$&$-$&$-$&$-$&3sg&$+$\\
several&$-$&$+$&$-$&$-$&$-$&$-$&$+$&3pl&$-$\\
various&$-$&$-$&$-$&$-$&$-$&$-$&$+$&3pl&$-$\\
sundry&$-$&$-$&$-$&$-$&$-$&$-$&$+$&3pl&$-$\\
no&$-$&$+$&$-$&$-$&$-$&$+$&$+$&3&$-$\\
neither&$-$&$-$&$-$&$-$&$-$&$+$&$+$&3&$-$\\
either&$-$&$-$&$-$&$-$&$-$&$-$&$+$&3&$-$\\
\hline
\hline
GENITIVE&$+$&$-$&$-$&$+$&$-$&$-$&$+$&UN\footnotemark&$-$\\
CARDINAL&$-$&$+$&$+$&$-$&$-$&$-$&$+$&3pl\footnotemark\ &$-$\footnotemark\ \\
PARTITIVE&$-$&+/-\footnotemark\ &$-$&$-$&$-$&$-$&$+$&UN&+/-\\
\hline
\end{tabular}
 \caption{Determiner Features associated with D anchors}
\label{det-values}
\end{table}
\addtocounter{footnote}{-3}
\footnotetext{We use the symbol UN to represent the fact that the selectional
restrictions for a given feature are unspecified, meaning the noun phrase that
the determiner selects can be either positive or negative for this
feature.}
\stepcounter{footnote}
\footnotetext{Except {\it one} which is 3sg.}
\stepcounter{footnote}
\footnotetext{Except {\it one} which is {\bf compl+}.}
\stepcounter{footnote}
\footnotetext{A partitive can be either {\bf quan+} or {\bf quan-}, depending
upon the nature of the noun that anchors the partitive.  If the anchor noun is
modified, then the quantity feature is determined by the modifier's quantity
value.}


In addition to the features that represent their own properties, determiners
also have features to represent the selectional
restrictions they impose on the NPs they take as complements.  The
selectional restriction features of a determiner appear on the NP footnode of
the auxiliary tree that the determiner anchors.  The NP$_{f}$ node in Figure~\ref{det-trees} shows the selectional feature
restriction imposed by {\it these}\footnote{In addition to this tree, {\it
these} would also anchor another auxiliary tree that adjoins onto {\bf card+}
determiners.}, while Table~\ref{det-ordering} shows the corresponding
selectional feature restrictions of several other determiners.
\small
\begin{table}
\centering
\begin{tabular}{|l||c|c|c|c|c|c|c|c|c||l|}
\hline
Det&defin&quan&card&gen&wh&decreas&const&agr&compl&{\it e.g.}\\
\hline
\hline
&$-$&$-$&$-$&$-$&$-$&$-$&$-$&3pl&$-$&{\it dogs}\\
all&$+$&$-$&$-$&UN&$-$&UN&UN&3pl&$-$&{\it these dogs}\\
&UN&UN&$+$&UN&UN&UN&UN&3pl&UN&{\it five dogs}\\
\hline
&$-$&$-$&$-$&$-$&$-$&$-$&$-$&3pl&$-$&{\it dogs}\\
{both}&$+$&$-$&$-$&UN&$-$&UN&UN&3pl&$-$&{\it these dogs}\\
\hline
&$-$&$-$&$-$&$-$&$-$&$-$&$-$&3sg&$-$&{\it dog}\\
&$-$&$+$&UN&UN&$-$&$+$&$-$&3&UN&{\it few dogs}\\
{this/that}&$-$&$+$&UN&UN&$-$&$-$&$-$&3pl&$+$&{\it many dogs}\\
&UN&UN&$+$&UN&UN&UN&UN&3sg&UN&{\it five dogs}\\
\hline
&$-$&$-$&$-$&$-$&$-$&$-$&$-$&3pl&$-$&{\it dogs}\\
these/those&$-$&$+$&UN&UN&$-$&$+$&$-$&3pl&UN&{\it few dogs}\\
&UN&UN&$+$&UN&UN&UN&UN&3pl&UN&{\it five dogs}\\
\hline
what/which&$-$&$-$&$-$&$-$&$-$&$-$&$-$&3&$-$&{\it dog(s)}\\
whatever&$-$&$+$&UN&UN&$-$&$+$&$-$&3&UN&{\it few dogs}\\
whichever&UN&UN&$+$&UN&UN&UN&UN&3&UN&{\it many dogs}\\
\hline
&$-$&$-$&$-$&$-$&$-$&$-$&$-$&3&$-$&{\it dog(s)}\\
the&$-$&$+$&UN&UN&$-$&$+$&$-$&3&UN&{\it few dogs}\\
&$+$&$-$&$-$&$-$&$-$&$-$&$-$&UN&$-$&{\it the me}\\
&$-$&$+$&UN&UN&$-$&$-$&$-$&3pl&$+$&{\it many dogs}\\
&UN&UN&$+$&UN&UN&UN&UN&3&UN&{\it five dogs}\\
\hline
&$-$&$-$&$-$&$-$&$-$&$-$&$-$&3sg&$-$&{\it dog}\\
every/each&$-$&$+$&UN&UN&$-$&$+$&$-$&3&UN&{\it few dogs}\\
&UN&UN&$+$&UN&UN&UN&UN&3&UN&{\it five dogs}\\
\hline
a/an&$-$&$-$&$-$&$-$&$-$&$-$&$-$&3sg&$-$&{\it dog}\\
\hline
some$_{1,2}$&$-$&$-$&$-$&$-$&$-$&$-$&$-$&3&$-$&{\it dog(s)}\\
some$_{1}$&UN&UN&$+$&UN&UN&UN&UN&3pl&UN&{\it dogs}\\
\hline
&$-$&$-$&$-$&$-$&$-$&$-$&$-$&3sg&$-$&{\it dog}\\
any&$-$&$+$&UN&UN&$-$&$+$&$-$&3&UN&{\it few dogs}\\
&UN&UN&$+$&UN&UN&UN&UN&3&UN&{\it five dogs}\\
\hline
&$-$&$-$&$-$&$-$&$-$&$-$&$-$&3sg&$-$&{\it dog}\\
another&$-$&$+$&UN&UN&$-$&$+$&$-$&3&UN&{\it few dogs}\\
&UN&UN&$+$&UN&UN&UN&UN&3&UN&{\it five dogs}\\
\hline
few&$-$&$-$&$-$&$-$&$-$&$-$&$-$&3pl&$-$&{\it dogs}\\
\hline
a few&$-$&$-$&$-$&$-$&$-$&$-$&$-$&3pl&$-$&{\it dogs}\\
\hline
many&$-$&$-$&$-$&$-$&$-$&$-$&$-$&3pl&$-$&{\it dogs}\\
\hline
many a/an&$-$&$-$&$-$&$-$&$-$&$-$&$-$&3sg&$-$&{\it dog}\\
\hline
several&$-$&$-$&$-$&$-$&$-$&$-$&$-$&3pl&$-$&{\it dogs}\\
\hline
various&$-$&$-$&$-$&$-$&$-$&$-$&$-$&3pl&$-$&{\it dogs}\\
\hline
sundry&$-$&$-$&$-$&$-$&$-$&$-$&$-$&3pl&$-$&{\it dogs}\\
\hline
no&$-$&$-$&$-$&$-$&$-$&$-$&$-$&3&$-$&{\it dog(s)}\\
\hline
neither&$-$&$-$&$-$&$-$&$-$&$-$&$-$&3sg&$-$&{\it dog}\\
\hline
either&$-$&$-$&$-$&$-$&$-$&$-$&$-$&3sg&$-$&{\it dog}\\
\hline
\end{tabular}
\caption{Selectional Restrictions Imposed by Determiners on the NP
foot node}
\label{det-ordering}
\end{table}

\begin{table}[htb]
\centering
\begin{tabular}{|l||c|c|c|c|c|c|c|c|c|}
\hline\hline
Det&definite&quan&card&gen&wh&decreas&const&agr&compl\\
\hline
\hline
&$-$&$-$&$-$&$-$&$-$&$-$&$-$&3&$-$\\
&$-$&$+$&UN&UN&$-$&$+$&$-$&3&UN\\
GENITIVE&$-$&$+$&UN&UN&$-$&$-$&$-$&3pl&$+$\\
&UN&UN&$+$&UN&UN&UN&UN&3&UN\\
&$-$&$+$&$-$&$-$&$-$&$-$&$+$&3pl&$-$\\
&$-$&$-$&$-$&$-$&$-$&$-$&$+$&3pl&$-$\\
\hline
CARDINAL&$-$&$-$&$-$&$-$&$-$&$-$&$-$&3pl\footnotemark&$-$\\
\hline
PARTITIVE&UN&UN&UN&UN&$-$&UN&UN&UN&UN\\
\hline
\end{tabular}
\caption{Selectional Restrictions Imposed by Groups of
Determiners/Determiner Constructions}
\label{det-ordering2}
\end{table}
\footnotetext{{\it one} differs from the rest of CARD in selecting
singular nouns}


\normalsize


\section{The Wh-Feature}
\label{agr-section}
A determiner with a {\bf wh+} feature is always the left-most
determiner in linear order since no determiners have selectional
restrictions that allow them to adjoin onto an NP with a +wh feature
value.  The presence of a wh+ determiner makes the entire NP wh+, and
this is correctly represented by the coindexation of the determiner
and root NP nodes' values for the wh-feature. Wh+ determiners'
selectional restrictions on the NP foot node of their tree only allows them
adjoin onto NPs that are {\bf wh-} or unspecified for the
wh-feature. Therefore ungrammatical sequences such as {\it $\ast$which what
dog} are impossible.  The adjunction of {\bf wh +} determiners onto
{\bf wh+} pronouns is also prevented by the same mechanism.

\section{Multi-word Determiners}
The system recognizes the multi-word determiners {\it a few} and {\it many a}.
The features for a multi-word determiner are located on the parent node of its 
two components (see Figure~\ref{multi-det-tree}).  We chose to represent these 
determiners as multi-word constituents because neither determiner retains the 
same set of features as either of its parts.  For example, the determiner 
{\it a} is 3sg and {\it few} is decreasing, while {\it a few} is 3pl and 
increasing.  Additionally, {\it many} is 3pl and {\it a} displays constancy, 
but {\it many a} is 3sg and does not display constancy.  Example sentences 
appear in (\ex{1})-(\ex{2}).

\begin{itemize}
\item{Multi-word Determiners}
\enumsentence{{\bf a few} teaspoons of sugar should be adequate .}
\enumsentence{{\bf many a} man has attempted that stunt, but none have
succeeded .}

\end{itemize}

\begin{figure}[htb]
\centering
\begin{tabular}{cc}
{\psfig{figure=/mnt/linc/xtag/work/doc/tech-rept/ps/det-files/betaDDnx.ps,height=5.0in}}
\end{tabular}\\
\caption{Multi-word Determiner tree:  $\beta$DDnx}
\label{multi-det-tree}
\end{figure} 



\section{Genitive Constructions}
\label{genitives}        

There are two kinds of genitive constructions: genitive pronouns, and genitive
NP's (which have an explicit genitive marker, {\it 's}, associated with them).
It is clear from examples such as {\it her dog returned home\/} and
{\it her five dogs returned home} vs {\it
$\ast$dog returned home\/} that genitive pronouns function as determiners and as
such, they sequence with the rest of the determiners.  The features for the
genitives are the same as for other determiners.  Genitives are not required to agree with
either the determiners or the nouns in the NPs that they modify. The
value of the {\bf agr} feature for an NP with a genitive determiner
depends on the NP to which the genitive determiner adjoins. While it
might seem to make sense to take {\it their} as 3pl, {\it my} as 1sg,
and {\it Alfonso's} as 3sg, this number and person information only
effects the genitive NP itself and bears no relationship to the number
and person of the NPs with these items as determiners. Consequently,
we have represented {\bf agr} as unspecified for genitives in Table
\ref{det-values}.

Genitive NP's are particularly interesting because they are potentially
recursive structures.  Complex NP's can easily be embedded within a determiner.

\enumsentence{[[[John]'s friend from high school]'s uncle]'s mother came to town.}

There are two things to note in the above example.  One is that in embedded
NPs, the genitive morpheme comes at the end of the NP phrase, even if the head
of the NP is at the beginning of the phrase.  The other is that the determiner
of an embedded NP can also be a genitive NP, hence the possibility of recursive
structures.

In the XTAG grammar, the genitive marker, {\it 's}, is separated from the
lexical item that it is attached to and given its own category (G).  In
this way, we can allow the full complexity of NP's to come from the
existing NP system, including any recursive structures.  As with the simple
determiners, there is one auxiliary tree structure for genitives which
adjoins to NPs. As can be seen in \ref{gen-trees}, this tree is anchored by
the genitive marker {\it 's} and has a branching D node which accomodates
the additional internal structure of genitive determiners. Also, like simple
determiners, there is one initial tree structure
(Figure~\ref{subst-genNP-tree}) available for substitution where needed, as
in, for example, the Determiner Gerund NP tree (see
Chapter~\ref{gerunds-chapter} for discussion on determiners for gerund
NP's).

\begin{figure}[ht]
\centering
\begin{tabular}{c}
%\rule[.1in]{8.35cm}{0.01in}
{\psfig{figure=/mnt/linc/xtag/work/doc/tech-rept/ps/det-files/betaGnx-features.ps,height=13.0cm}}\\
\end{tabular}
\caption{Genitive Determiner Tree}
\label{gen-trees}
%\rule[.1in]{8.35cm}{0.01in}
\end{figure}

\begin{figure}[htb]
\centering
\begin{tabular}{c}
{\psfig{figure=/mnt/linc/xtag/work/doc/tech-rept/ps/det-files/alphaDnxG.ps,height=1.8in}}\\
\end{tabular}
\caption{Genitive NP tree for substitution: $\alpha$DnxG}
\label{subst-genNP-tree}
\end{figure}

Since the NP node which is sister
to the G node could also have a genitive determiner in it, the type of
genitive recursion shown in (\ex{0}) is quite naturally accounted for
by the genitive tree structure used in our analysis.

\section{Partitive Constructions}        
\label{partitives}                        

The deciding factor for including an analysis of partitive constructions(e.g.\ {\it some kind
of}, {\it all of\/}) as a complex determiner constructions was the
behavior of the agreement features.  If partitive constructions are analyzed as
an NP with an adjoined PP, then we would expect to get agreement with the head
of the NP (as in ({\ex{1}})).  If, on the other hand, we analyze them
as a determiner construction, then we would expect to get agreement with the
noun that the determiner sequence modifies (as we do in ({\ex{2}})).

\enumsentence{a {\it kind} [of these machines] {\it is} prone to failure.}
\enumsentence{[a kind of] these {\it machines are} prone to failure.}

In other words, for partitive constructions, the semantic head of the NP is the second rather than the first noun in linear order. That the agreement shown in ({\ex{0}}) is possible suggests that the second noun in linear order in these constructions should also be treated as the syntactic head. Note that both the partitive and PP readings are usually possible for a particular NP. In the cases where either the partitive or the PP reading is preferred, we take it to be just that, a preference, most appropriately modeled not in the grammar but in a component such as the heuristics used with the XTAG parser for reducing the analyses produced to the most likely. 

In our analysis the partitive tree in Figure~\ref{part-tree} is anchored
by one of a limited group of nouns that can appear in the determiner
portion of a partitive construction. A rough semantic characterization
of these nouns is that they either represent quantity (e.g. {\it part, half,
most, pot, cup, pound} etc.) or classification (e.g. {\it type, variety,
kind, version} etc.).  In the absence of a more implementable
characterization we use a list of such nouns compiled from a
descriptive grammar \cite{quirk85}, a thesaurus, and from online
corpora. In our grammar the nouns on the list are the only ones that
select the partitive determiner tree. 


\begin{figure}[ht]
\centering
\begin{tabular}{c}
%\rule[.1in]{8.35cm}{0.01in}
{\psfig{figure=/mnt/linc/xtag/work/doc/tech-rept/ps/det-files/betaNofnx.ps,height=17.0cm}}\\
\end{tabular}
\caption{Partitive Determiner Tree}
\label{part-tree}
%\rule[.1in]{8.35cm}{0.01in}
\end{figure}

Like other determiners, partitives can adjoin to an NP consisting of just
a noun ({\it `[a certain kind of] machine'}), or adjoin to NPs
that already have determiners ({\it `[some parts of] these
machines'}. Notice that just as for the genitives, the complexity and
the recursion are contained below the D node and rest of the structure
is the same as for simple determiners.

\section{Adverbs, Noun Phrases, and Determiners}
\label{adverbial-section}

%% Quirk, Sections 7.62, 7.53

Many adverbs interact with the noun phrase and determiner system in English.
For example, consider sentences (\ref{approx})-(\ref{double}) below.

\enumsentence{\label{approx}{\bf Approximately} thirty people came to the lecture.}
\enumsentence{\label{practically}{\bf Practically} every person in the theater was laughing
hysterically during that scene.}
\enumsentence{\label{only}{\bf Only} John's crazy mother can make stuffing that tastes so good.}
\enumsentence{\label{relatively}{\bf Relatively} few programmers remember how to program in
COBOL.}
\enumsentence{\label{not}{\bf Not} every martian would postulate that all humans speak a
universal language.}
\enumsentence{\label{enough}{\bf Enough} money was gathered to pay off the group gift.}
\enumsentence{\label{quite}{\bf Quite} a few burglaries occurred in that neighborhood last
year.}
\enumsentence{\label{double}I wanted to be paid {\bf double} the amount they offered.}

Although there is some debate in the literature as to whether these should be
classified as determiners or adverbs, we believe that these items that
interact with the NP and determiner system are in fact adverbs.   These items
exhibit a broader distribution than either determiners or adjectives in that
they can modify many other phrasal categories, including adjectives, verb
phrases, prepositional phrases, and other adverbs.

Using the determiner feature system, we can obtain a close approximation to an
accurate characterization of the behavior of the adverbs that interact with
noun phrases and determiners.  Adverbs can adjoin to either a determiner or a
noun phrase (see figure~\ref{det-adv-trees}), with the adverbs restricting what
types of NPs or determiners they can modify by imposing feature requirements on
the foot D or NP node.  For
example, the adverb {\it approximately}, seen in (\ref{approx})
above, selects for determiners that are {\bf card+}.  The adverb {\it enough}
in (\ref{enough}) is an example of an adverb that selects for a noun phrase,
specifically a noun phrase that is not modified by a determiner.

\begin{figure}[ht]
\centering
\begin{tabular}{ccc}
{\psfig{figure=/mnt/linc/xtag/work/doc/tech-rept/ps/det-files/advdet.ps,height=5.0cm}}&&
{\psfig{figure=/mnt/linc/xtag/work/doc/tech-rept/ps/det-files/advnoun.ps,height=5.0cm}}\\
(a)&&(b)
\end{tabular}
\caption{(a) Adverb modifying a determiner; (b) Adverb modifying a noun phrase}
\label{det-adv-trees}
%\rule[.1in]{8.35cm}{0.01in}
\end{figure}

Most of the adverbs that modify determiners and NPs divide into six classes,
with some minor variation within classes, based on the pattern of these
restrictions.  Three of the classes are adverbs that modify determiners, while
the other three modify NPs.

The largest of the five classes is the class of adverbs that modify cardinal
determiners.  This class includes, among others, the adverbs {\it about}, {\it
at most}, {\it exactly}, {\it nearly}, and {\it only}.  These adverbs have the
single restriction that they must adjoin to determiners that are {\bf card+}.
Another class of adverbs consists of those that can modify the determiners {\it
every}, {\it all}, {\it any}, and {\it no}.  The adverbs in this class are {\it
almost}, {\it nearly}, and {\it practically}.  Closely related to this class
are the adverbs {\it mostly} and {\it roughly}, which are restricted to
modifying {\it every} and {\it all}, and {\it hardly}, which can only modify
{\it any}.  To select for {\it every}, {\it all}, and {\it any}, these adverbs
select for determiners that are [{\bf quan+}, {\bf card-}, {\bf const+}, {\bf
compl+}], and to select for {\it no}, the adverbs choose a determiner that is
[{\bf quan+}, {\bf decreas+}, {\bf const+}].  The third class of adverbs that
modify determiners are those that modify the determiners {\it few} and {\it
many}, representable by the feature sequences [{\bf quan+}, {\bf decreas+},
{\bf const-}] and [{\bf quan+}, {\bf decreas-}, {\bf const-}, {\bf 3pl}, {\bf
compl+}], respectively.  Examples of these adverbs are {\it awfully}, {\it
fairly}, {\it relatively}, and {\it very}.

Of the three classes of adverbs that modify noun phrases, one actually consists
of a single adverb {\it not}, that only modifies determiners that are {\bf
compl+}.  Another class consists of the focus adverbs, {\it at least}, {\it
even}, {\it only}, and {\it just}.  These adverbs select NPs that are {\bf wh-}
and {\bf card-}.  For the NPs that are {\bf card+}, the focus adverbs actually
modify the cardinal determiner, and so these adverbs are also included in the
first class of adverbs mentioned in the previous paragraph.  The last major
class that modify NPs consist of the adverbs {\it double} and {\it twice},
which select NPs that are [{\bf definite+}] (i.e., {\it the}, {\it
this/that/those/these}, and the genitives).

Although these restrictions succeed in recognizing the correct
determiner/adverb sequences, a few unacceptable sequences slip through.  For
example, in handling the second class of adverbs mentioned above, {\it every},
{\it all}, and {\it any} share the features [{\bf quan+}, {\bf card-}, {\bf
const+}, {\bf compl+}] with {\it a} and {\it another}, and so {\it
$\ast$nearly a man} is acceptable in this system.  In addition to this
over-generation within a major class, the adverb {\it quite} selects for
determiners and NPs in what seems to be a purely idiosyncratic fashion.
Consider the following examples.

\eenumsentence{\label{quite2}\item[a.] {\bf Quite} a few members of the audience had to
leave.
	\item[b.] There were {\bf quite} many new participants at this
year's conference.
	\item[c.] {\bf Quite} few triple jumpers have jumped that far.
	\item[d.] Taking the day off was {\bf quite} the right thing to do.
	\item[e.] The recent negotiation fiasco is {\bf quite} another issue.
	\item[f.] Pandora is {\bf quite} a cat!}

In examples (\ref{quite2}a)-(\ref{quite2}c), {\it quite} modifies the determiner, while in (\ref{quite2}d)-(\ref{quite2}f),
{\it quite} modifies the entire noun phrase.  Clearly, it functions in a
different manner in the two sets of sentences; in (\ref{quite2}a)-(\ref{quite2}c), {\it quite}
intensifies the amount implied by the determiner, whereas in (\ref{quite2}d)-(\ref{quite2}f), it
singles out an individual from the larger set to which it belongs.  To capture
the selectional restrictions needed for (\ref{quite2}a)-(\ref{quite2}c), we utilize the two sets of
features mentioned previously for selecting {\it few} and {\it many}.  However,
{\it a few} cannot be singled out so easily; using the sequence [{\bf quan+},
{\bf card-}, {\bf decreas-}, {\bf const+}, {\bf 3pl}, {\bf compl-}], we also
accept the ungrammatical NPs {\it $\ast$quite several members} and {\it
$\ast$quite some members} (where {\it quite} modifies {\it some}).  In
selecting {\it the} as in (d) with the features [{\bf definite+}, {\bf gen-}, {\bf
3sg}], {\it quite} also selects {\it this} and {\it that}, which are
ungrammatical in this position.  Examples (\ref{quite2}e) and (\ref{quite2}f) present yet another
obstacle in that in selecting {\it another} and {\it a}, {\it quite}
erroneously selects {\it every} and {\it any}.

It may be that there is an undiscovered semantic feature that would
alleviate these difficulties.  However, on the whole, the determiner feature system we have proposed can be
used as a surprisingly efficient method of characterizing the interaction of
adverbs with determiners and noun phrases.









