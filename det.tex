\chapter{Determiners and Noun Phrases}
\label{det-comparitives}

In our English XTAG grammar,\footnote{A more detailed discussion of the analysis in this chapter is currently being revised to appear in a forthcoming CSLI volume on Tree Adjoining Grammar. Drafts of the CSLI chapter should be available for anyone particularly interested around October 1997.} all nouns select the noun phrase (NP)
tree structure shown in Figure~\ref{np-tree}.  Common nouns do not
require determiners in order to form grammatical NPs. Rather than
being ungrammatical, singular countable nouns without determiners are
restricted in interpretation and can only be interpreted as mass
nouns.  Allowing all nouns to head determinerless NPs correctly treats
the individuation in countable NPs as a property of
determiners. Common nouns have negative(``-'') values for determiner
features in the lexicon in our analysis and can only acquire a
positive(``+'') value for those features if determiners ajoin to them.
Other types of NPs such as pronouns and proper nouns have been argued
by Abney \cite{Abney87} to either be determiners or to move to the
determiner position because they exhibit determiner-like behavior. We
can capture this insight in our system by giving pronouns and proper
nouns positive values for determiner features. For example pronouns
and proper nouns would be marked as definite, a value that common
nouns can only obtain by having a definite determiner adjoin.


\begin{figure}[ht]
\centering
\begin{tabular}{c}
%\rule[.1in]{8.35cm}{0.01in}
{\psfig{figure=/mnt/linc/xtag/work/doc/tech-rept/ps/det-files/alphaNXN.ps,height=16.0cm}}\\
\end{tabular}
\caption{NP Tree}
\label{np-tree}
%\rule[.1in]{8.35cm}{0.01in}
\end{figure}

  A single tree structure is selected by simple determiners, an
auxiliary tree which adjoins to NP. An example of this determiner tree
anchored by the determiner {\it these\/} is shown in
Figure~\ref{det-trees}. Complex determiners such as partitives and
genitives also anchor tree structures that adjoin to NP. They differ
from the simple determiners in their internal complexity. Details of
our treatment of these more complex constructions appear in Sections 3
and 4.  Sequences of determiners, as in the NPs {\it all her dogs\/} or {\it
those five dogs\/} are derived by multiple adjunctions
of the determiner tree, with each tree anchored by one of the determiners in
the sequence. The order in which the determiner trees can adjoin is
controlled by features.

\begin{figure}[ht]
\centering
\begin{tabular}{c}
%\rule[.1in]{8.35cm}{0.01in}
{\psfig{figure=/mnt/linc/xtag/work/doc/tech-rept/ps/det-files/betaDnx-these.ps,height=14cm}}
\end{tabular}
\caption{Determiner Trees with Features}
\label{det-trees}
%\rule[.1in]{8.35cm}{0.01in}
\end{figure}


This treatment of determiners as adjoining onto NPs is similar to that
of \cite{Abeille90:TAG}, and allows us to capture one of the insights of the DP
hypothesis, namely that Determiners select NPs as complements. In
Figure~\ref{det-trees} the Determiner and its NP complement appear in
the configuration that is typically used in LTAG to represent
selectional relationships. That is, the head serves as the anchor of
the tree and it's complement is a sister node in the same elementary tree.


The XTAG treatment of determiners uses a set of eight
features for representing their properties. Most of these determiner features
were developed by semanticists in their accounts of semantic phenomena
related to quantifiers
\cite{KeenanStavi86:LP,Partee90:BK} but when used together also
account for a substantial portion of the complex patterns of
determiner sequencing. Although we do not claim to have exhaustively
covered the complex determiner system of English, we do cover a large
subset, both in terms of the phenomena handled and in terms of corpus
coverage. The XTAG grammar has also been extended to include complex
determiner constructions such as genitives and partitives using these
determiner features.

Each determiner carries with it a set of values for these features
that represents its own properties, and a set of values for the
properties of NPs to which can adjoin. Features are crucial to
ordering determiners correctly.  We have identified eight features
which are sufficient to order the determiners.  These features are:
{\bf definiteness}, {\bf quantity}, {\bf cardinality}, {\bf genitive},
{\bf decreasing}, {\bf constancy}, {\bf wh}, and {\bf agr}. {\bf
Definiteness}, {\bf quantity}, {\bf cardinality}, {\bf genitive}, {\bf
decreasing} and {\bf constancy} have been previously proposed as
semantic properties of determiners.  The semantic definitions
underlying the features are given below.

\begin{description}

\item[Definiteness:] Possible Values [+/--]. \\
A function f is definite iff f is non-trivial and whenever
f(s)~$\neq~\emptyset$ then it is always the intersection of one or
more individuals.  \cite{KeenanStavi86:LP}

\item[Quantity:]  Possible Values [+/--]. \\
If A and B are sets denoting an NP and associated predicate, respectively; E is
a domain in a model M, and F is a bijection from M$_{1}$ to M$_{2}$, then we
say that a determiner satisfies the constraint of quantity if
Det$_{E_{1}}$AB~$\leftrightarrow$~Det$_{E_{2}}$F(A)F(B). \cite{Partee90:BK}

\item[Cardinality:]  Possible Values [+/--]. \\
A determiner D is cardinal iff D $\in$ cardinal numbers~$\geq$~1.

\item[Genitive:]  Possible Values [+/--]. \\
Possessive pronouns and the possessive morpheme ({\it 's}) are marked {\bf
gen$+$}; all other nouns are {\bf gen$-$}.

\item[Decreasing:]  Possible Values [+/--]. \\
A set of Q properties is decreasing iff whenever s$\leq$t and t$\in$Q then
s$\in$Q. A function f is decreasing iff for all properties f(s) is a decreasing
set.

A non-trivial NP (one with a Det) is decreasing iff its denotation in any model
is decreasing. \cite{KeenanStavi86:LP}

\item[Constancy:] Possible Values [+/--]. \\
If A and B are sets denoting an NP and associated predicate, respectively, and
E is a domain, then we say that a determiner displays constancy if
(A$\cup$B)~$\subseteq$~E~$\subseteq$~E$^{\prime}$ then
Det$_{E}$AB~$\leftrightarrow$~Det$_{E^{\prime}}$AB. Modified from
\cite{Partee90:BK}

\end{description}

The {\bf wh}-feature has been discussed in the linguistics literature mainly in relation to wh-movement and with respect to NPs and nouns as well as determiners. We give a shallow but useful working definition of the {\bf wh}-feature below:

\begin{description}

\item[Wh:]  Possible Values [+/--]. \\
Interrogative determiners are {\bf wh$+$}; all other determiners are
{\bf wh$-$}. 
\end{description}

The {\bf agr} feature is inherently a noun feature.  While determiners
are not morphologically marked for agreement in English many of them
are sensitive to number.  Many determiners are semantically either
singular or plural and must adjoin to nouns which are the same. For
example, {\it a\/} can only adjoin to singular nouns ({\it a dog\/} vs
{\it $\ast$a dogs\/} while {\it many\/} must have plurals ({\it many
dogs\/} vs {\it $\ast$many dog\/}). Other determiners such as {\it some} are
unspecified for agreement in our analysis because they are compatible
with either singulars or plurals ({\it some dog}, {\it some
dogs}). The possible values of agreement for determiners are: [3sg, 3pl, 3sgpl].




The determiner tree in Figure~\ref{det-trees} shows the appropriate
feature values for the determiner {\it these}, while Table \ref{det-values}
shows the corresponding feature values of several other common determiners.

\small
\begin{table}
\centering
\begin{tabular}{|l||c|c|c|c|c|c|c|c|}
\hline
Det&defin&quan&card&gen&wh&decreas&const&agr\\
\hline
\hline
all&$+$&$+$&$-$&$-$&$-$&$-$&$+$&3pl\\
this&$+$&$-$&$-$&$-$&$-$&$-$&$+$&3sg\\
that&$+$&$-$&$-$&$-$&$-$&$-$&$+$&3sg\\
what&$+$&$-$&$-$&$-$&$+$&$-$&$+$&3sgpl\\
the&$+$&$-$&$-$&$-$&$-$&$-$&$+$&3sgpl\\
every&$+$&$+$&$-$&$-$&$-$&$-$&$+$&3sg\\
each&$+$&$+$&$-$&$-$&$-$&$-$&$+$&3sg\\
any&$-$&$+$&$-$&$-$&$-$&$-$&$+$&3sg\\
a&$-$&$+$&$-$&$-$&$-$&$-$&$+$&3sg\\
no&$+$&$+$&$-$&$-$&$-$&$-$&$+$&3sgpl\\
few&$-$&$+$&$-$&$-$&$-$&$+$&$-$&3pl\\
many&$-$&$+$&$-$&$-$&$-$&$-$&$-$&3pl\\
GEN&$+$&$-$&$-$&$+$&$-$&$-$&$+$&\\
CARD&$+$&$+$&$+$&$-$&$-$&$-$&$+$&3pl\footnotemark\ \\
PART&$-$&&$-$&$-$&$-$&$-$&$+$&\\
\hline
\end{tabular}
 \caption{Determiner Features}
\label{det-values}
\end{table}\addtocounter{footnote}{0}\footnotetext{{except {\it one} which is 3sg}} 

\normalsize

In addition to the features that represent their own properties, determiners
also have features to represent the selectional
restrictions they impose on the NPs they take as complements.  The
selectional restriction features of a determiner appear on the NP footnode of
the auxiliary tree that the determiner anchors.  The NP$_{f}$ node in Figure~\ref{det-trees} shows the selectional feature
restriction imposed by {\it these}\footnote{In addition to this tree, {\it
these} would also anchor another auxiliary tree that adjoins onto {\bf card+}
determiners.}, while Table~\ref{det-ordering} shows the corresponding
selectional feature restrictions of several other determiners.

\small
\begin{table}
\centering
\begin{tabular}{|l||c|c|c|c|c|c|c|c|}
\hline
Det&defin&quan&card&gen&wh&decreas&const&agr\\
\hline
\hline
all&$+$&$-$&$-$&&$-$&&&\\
&&&$+$&&&&&\\
this&$-$&&&&$-$&$+$&&\\
&&&$+$&&&&&\\
that&$-$&&&&$-$&$+$&&\\
&&&$+$&&&&&\\
what&$-$&&&&$-$&$+$&&\\
&&&$+$&&&&&\\
the&$-$&&&&$-$&&$-$&\\
&&&$+$&&&&&\\
every&$-$&&&&$-$&$+$&&\\
&&&$+$&&&&&\\
each&$-$&&&&$-$&$+$&&\\
&&&$+$&&&&&\\
any&$-$&&&&$-$&$+$&&\\
&&&$+$&&&&&\\
a&$-$&&&&$-$&$+$&&\\
many&\multicolumn{8}{c|}{only nouns}\\
no&\multicolumn{8}{c|}{only nouns}\\
GEN&\multicolumn{8}{c|}{only nouns}\\
CARD&\multicolumn{8}{c|}{only nouns}\\
PART&&&&&$-$&&&\\
\hline
\end{tabular}
\caption{Selectional Restrictions Imposed by Determiners}
\label{det-ordering}
\end{table}

\normalsize


\section{The Wh-Feature}
\label{agr-section}
A determiner with a {\bf wh+} feature is always the left-most
determiner in linear order since no determiners have selectional
restrictions that allow them to adjoin onto an NP with a +wh feature
value.  The presence of a wh+ determiner makes the entire NP wh+, and
this is correctly represented by the coindexation of the determiner
and root NP nodes' values for the wh-feature. Wh+ determiners'
selectional restrictions on the NP foot node of their tree only allows them
adjoin onto NPs that are {\bf wh-} or unspecified for the
wh-feature. Therefore ungrammatical sequences such as {\it which what
dog} are impossible.  The adjunction of {\bf wh +} determiners onto
{\bf wh+} pronouns is also prevented by the same mechanism.

\section{Multi-word Determiners}
The system recognizes the multi-word determiners {\it a few} and {\it many a}.
The features for a multi-word determiner are located on the parent node of its 
two components (see Figure~\ref{multi-det-tree}).  We chose to represent these 
determiners as multi-word constituents because neither determiner retains the 
same set of features as either of its parts.  For example, the determiner 
{\it a} is 3sg and {\it few} is decreasing, while {\it a few} is 3pl and 
increasing.  Additionally, {\it many} is 3pl and {\it a} displays constancy, 
but {\it many a} is 3sg and does not display constancy.  Example sentences 
appear in (\ex{1})-(\ex{2}).

\begin{itemize}
\item{Multi-word Determiners}
\enumsentence{{\bf a few} teaspoons of sugar should be adequate .}
\enumsentence{{\bf many a} man has attempted that stunt, but none have
succeeded .}

\end{itemize}

\begin{figure}[htb]
\centering
\begin{tabular}{cc}
{\psfig{figure=/mnt/linc/xtag/work/doc/tech-rept/ps/det-files/betaDDnx.ps,height=5.0in}}
\end{tabular}\\
\caption{Multi-word Determiner tree:  $\beta$DDnx}
\label{multi-det-tree}
\end{figure} 



\section{Genitive Constructions}

There are two kinds of genitive constructions: genitive pronouns, and genitive
NP's (which have an explicit genitive marker, {\it 's}, associated with them).
It is clear from examples such as {\it her dog was run over\/} vs {\it
$\ast$dog was run over\/} that genitive pronouns function as determiners and as
such, they sequence with the rest of the determiners.  The features for the
genitives are the same as for other determiners.  No {\bf agr} is specified in
Table \ref{det-values}, since the {\bf number} and {\bf person} of the genitive
will depend on its particular form (i.e.\ {\it my} vs {\it their}).  Genitives
are not required to agree with either the determiners or the nouns that they
modify.

Genitive NP's are particularly interesting because they are potentially
recursive structures.  Complex NP's can easily be embedded within a determiner.

\enumsentence{[[[John]'s friend from high school]'s uncle]'s mother came to town.}

There are two things to note in the above example.  One is that in embedded
NP's, the genitive morpheme comes at the end of the NP phrase, even if the head
of the NP is at the beginning of the phrase.  The other is that the determiner
of an embedded NP can also be a genitive NP, hence the possibility of recursive
structures.

In the XTAG grammar, the genitive marker, {\it 's}, is separated from the
lexical item that it is attached to and given its own category (G).  In this
way, we can allow the full complexity of NP's to come from the existing NP
system, including any recursive structures.  As with the simple determiners, there is one
auxiliary tree structure for genitives which adjoins to NPs. As can be
seen in \ref{gen-trees},  this tree is anchored by the genitive marker
{\it 's} and has a branching D node which accomodates the additional
complexity of genitive determiners. 

%Figure of alphaDXnxG-features and betanxGdx-features here 

\begin{figure}[ht]
\centering
\begin{tabular}{c}
%\rule[.1in]{8.35cm}{0.01in}
{\psfig{figure=/mnt/linc/xtag/work/doc/tech-rept/ps/det-files/betaGnx-features.ps,height=13.0cm}}\\
\end{tabular}
\caption{Genitive Determiner Tree}
\label{gen-trees}
%\rule[.1in]{8.35cm}{0.01in}
\end{figure}

Since the NP node which is sister
to the G node could also have a genitive determiner in it, the type of
genitive recursion shown in \ex{-1} is quite naturally accounted for
by the genitive tree structure used in our analysis.

\section{Partitive Constructions}

The deciding factor for including an analysis of partitive constructions(e.g.\ {\it some kind
of}, {\it all of\/}) as a complex determiner constructions was the
behavior of the agreement features.  If partitive constructions are analyzed as
an NP with an adjoined PP, then we would expect to get agreement with the head
of the NP (as in Sentence~({\ex{1}})).  If, on the other hand, we analyze them
as a determiner construction, then we would expect to get agreement with the
noun that the determiner sequence modifies (as we do in Sentence~({\ex{2}})).

\enumsentence{a {\it kind} [of these machines] {\it is} prone to failure.}
\enumsentence{[a kind of] these {\it machines are} prone to failure.}

In otherwords, for partitive constructions, the semantic head of the NP is the second rather than the first noun in linear order. That the agreement shown in ({\ex{0}}) is possible suggests that the second noun in linear order in these constructions should also be treated as the syntactic head. Note that both the partitive and PP readings are usually possible for a particular NP. In the cases where either the partitive or the PP reading is preferred, we take it to be just that, a preference, most appropriately modeled not in the grammar but in the a component such as the heuristics used with the XTAG parser for reducing the analyses produced to the most likely. 

In our analysis the partitive tree in Figure~ref{part-tree} is anchored
by one of a limited group of nouns that can appear in the determiner
portion of a partitive construction. A rough semantic characterization
of these nouns is that they either represent quantity (e.g. part, half,
most, pot, cup, pound etc.) or classification (e.g. type, variety,
kind, version etc.).  In the abscence of a more implementable
characterization we use a list of such nouns compiled from a
descriptive grammar \cite{quirk85}, a thesaurus, and from online
corpora. In our grammar the nouns on the list are the only ones that
select the partitive determiner tree. 


\begin{figure}[ht]
\centering
\begin{tabular}{c}
%\rule[.1in]{8.35cm}{0.01in}
{\psfig{figure=/mnt/linc/xtag/work/doc/tech-rept/ps/det-files/betaNofnx.ps,height=17.0cm}}\\
\end{tabular}
\caption{Partitive Determiner Tree}
\label{part-tree}
%\rule[.1in]{8.35cm}{0.01in}
\end{figure}

Like other determiners, partitives can adjoin to an NP consisting of just
a noun ({\it `[a certain kind of] machine'}), or adjoin to NPs
that already have determiners ({\it `[some parts of] these
machines'}. Notice that just as for the genitives, the complexity and
the recursion are contained below the D node and rest of the structure
is the same as for simple determiners.



\section{Determiner Adverbs}
\label{adverbial-section}

%% Quirk, Sections 7.62, 7.53

There are some adverbs that interact with the NP and determiner system
\cite{quirk85}, although there is some debate in the literature as to
whether these should be classified as determiners or adverbs.
Sentences~(\ex{1})~-~(\ex{3}) contain examples of this phenomena.

\enumsentence{Hardly any attempt was made at restitution.}
\enumsentence{Only Albert would say such a thing.}
\enumsentence{Almost all the people had left by 5pm.}


Adverbs that modify NP's or determiners have restrictions on what types of NP's
or determiners they can modify. They divide into three classes based on the
pattern of these restrictions.  The adverbs {\it especially}, {\it even}, {\it
just}, and {\it only} form a class that can modify any NP that is {\bf wh--},
including Proper Nouns.  A second class, consisting of adverbs such as {\it
hardly}, {\it merely}, and {\it simply}, modifies NP's with determiners that
are {\bf definite--} and {\bf const+}, or that are {\bf gen+}.  This second
class of adverbs can also modify NP's with {\it the} as a determiner.  They do
not modify NP's without determiners.  The third class, exemplified by {\it
almost}, {\it approximately}, and {\it relatively}, modifies the determiner
itself.  These adverbs are restricted to modifying {\bf card+} determiners, as
well as {\it all}, {\it double}, and {\it half}.  The distinction between
adverbs that modify NP's and ones that modify determiners can be seen in the
NP's in ({\ex{1}})~and~({\ex{2}}).

\enumsentence{[Just][half the people]}
\enumsentence{[Approximately half][the people]}

This analysis of determiner adverbs is still somewhat preliminary and
only the adverbs that modify {\bf card+} determiners have been fully
implemented in the XTAG English grammar.



