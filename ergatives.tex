\chapter{Ergatives}
\label{ergatives}

Verbs in English that are termed ergative display the kind of
alternation shown in the sentences in (\ex{1}) below.

\enumsentence{The sun melted the ice .\\
The ice melted .}

The pattern of ergative pairs as seen in (\ex{0}) is for the object of the
transitive sentence to be the subject of the intransitive sentence.
The literature discussing such pairs is based largely on syntactic
models that involve movement, particularly GB.  Within that framework
two basic approaches are discussed:

\begin{itemize}
\item {\bf Derived Intransitive}\\ The intransitive member of the
ergative pair is derived through processes of movement and deletion from:
\begin{itemize}
\item a transitive D-structure \cite{Burzio86}; or	
\item transitive lexical structure \cite{HaleKeyser86,HaleKeyser87}
\end{itemize}

\item {\bf Pure Intransitive}\\ The intransitive member is intransitive at all levels of the
syntax and the lexicon and is not related to the transitive member
syntactically or lexically \cite{Napoli88}.
\end{itemize}


Obviously, the Derived Intransitive approach's notions of movement in the
lexicon or in the grammar cannot be represented as such in lexicalized tag.
However, distinctions drawn in these arguments can be translated to the FB-LTAG
framework.  In the XTAG grammar the difference between these two approaches is
not a matter of movement but rather a question of tree family membership.  The
relation between sentences represented in terms of movement in other frameworks
is represented in XTAG by membership in the same tree family. Wh-questions and
their indicative counterparts are one example of this.  Adopting the Pure
Intransitive approach suggested by \cite{Napoli88} would mean placing the
intransitive ergatives in a tree family with other intransitive verbs and
separate from the transitive variants of the same verbs.  This would result in
a grammar that represented intransitive ergatives as more closely related to
other intransitives than to their transitive counterparts.  The only hint of
the relation between the intransitive ergatives and the transitive ergatives
would be that ergative verbs would select both tree families. While
this is a workable solution, it is an unattractive one for the English XTAG
grammar because semantic coherence is implicitly associated with tree families
in our analysis of other constructions.  In particular, constancy in thematic
role is represented by constancy in node names across sentence types within a
tree family. For example, if the object of a declarative tree is NP$_{1}$ the
subject of the passive tree(s) in that family will also be NP$_{1}$.

The analysis that has been implemented in the English XTAG grammar is an
adaptation of the Derived Intransitive approach. The ergative verbs select one
family, Tnx0Vnx1, that contains both transitive and intransitive trees.  The
{\bf$<$trans$>$} feature appears on the intransitive ergative trees with the
value {\bf --} and on the transitive trees with the value {\bf +}.  This
creates the two possibilities needed to account for the data.

\begin{itemize}
%%\item {\bf intransitive ergative only.} These verbs have the feature
%%value  {\bf$<$trans$>$=--}, so they can unify only with the
%%intransitive trees within Tnx0Vnx1. This correctly captures the
%%pattern shown in (\ex{1}) and (\ex{2}).
%%
%%\enumsentence{The leaves fell .}
%%\enumsentence{$\ast$The wind fell the leaves .}
%%
\item {\bf intransitive ergative/transitive alternation.}  These verbs
have transitive and intransitive variants as shown in sentences~(\ex{1}) and
(\ex{2}).

\enumsentence{The sun melted the ice cream .}
\enumsentence{The ice cream melted .}


In the English XTAG grammar, verbs with this behavior are left unspecified as
to value for the {\bf$<$trans$>$} feature.  This lack of specification allows
these verbs to anchor either type of tree in the Tnx0Vnx1 tree family because
the unspecified {\bf$<$trans$>$} value of the verb can unify with either {\bf
+} or {\bf --} values in the trees.

\item {\bf transitive only.}  Verbs of this type select only the
transitive trees and do not allow intransitive ergative variants as in
the pattern show in sentences~(\ex{1}) and (\ex{2}).

\enumsentence{Elmo borrowed a book .}
\enumsentence{$\ast$A book borrowed .}

The restriction to selecting only transitive trees is accomplished by
setting the {\bf$<$trans$>$} feature value to {\bf +} for these verbs.
\end{itemize}

\begin{figure}[htb]
\centering
\mbox{}
\psfig{figure=ps/erg-files/alphaEnx1V.ps,height=4.0cm}
\caption{Ergative Tree: $\alpha$Enx1V}
\label{decl-erg-tree}
\label{2;14,1}
\end{figure}

The declarative ergative tree is shown in Figure~\ref{decl-erg-tree} with the
{\bf $<$trans$>$} feature displayed.  Note that the index of the subject NP
indicates that it originated as the object of the verb.























