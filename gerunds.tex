\chapter{Gerund NP's}
\label{gerunds-chapter}

There are two main types of gerundives identified in the linguistics
literature. One is the class of {\it derived nominalizations} exemplified
in (\ex{1}), also called {\it nominal gerundives} or {\it action
nominalizations}. The other is the class of so-called {\it sentential} or
{\it VP gerundives} exemplified in (\ex{2}).

\enumsentence{Some think that {\bf the selling of bonds} is beneficial.}

\enumsentence{Are private markets approving of {\bf Washington bashing Wall Street}?}

In the English XTAG grammar, the two classes of gerunds are analyzed
differently - owing to the differences in their internal structure - and
assigned different trees within each tree family. The derived
nominalizations are termed {\bf Determiner gerunds}, and the sentential or
VP gerunds are termed {\bf NP gerunds}.

The external distribution of both classes of gerunds is parallel to that of
noun phrases. In particular, it has been observed that most places where
NP's are allowed, a gerundive clause is also allowed.\footnote{Restricting
the selection of gerundives in NP positions is handled by the feature {\bf
gerund = -}. For example, the subject and object NP's in the ``equative
BE'' tree (Tnx0BEnx1) have the feature {\bf gerund = -}, which prevents
gerunds (which have the feature {\bf gerund = +}) from substituting into
these NP positions.} The bold face portions of sentences (\ex{3})--(\ex{5})
show examples of gerunds as a subject and as the object of a preposition.

\enumsentence{\ldots and {\bf avoiding such losses} will take a monumental
effort.}

\enumsentence{{\bf Mr. Nolen's wandering} doesn't make him a weirdo.}

\enumsentence{Are private markets approving of {\bf Washington bashing Wall
Street}?}


In order to capture their distributional behavior, both types of trees are
assigned NP as the category label of their top node.

\section{Determiner Gerunds}
The Determiner Gerund tree in Figure~\ref{detgerund-tree} has an initial D
and instantiates the direct object as an {\it of} PP.\footnote{Note that
the determiner can adjoin to the gerund only from {\it within} the gerund
tree. Adjunction of determiners to the gerund root node is prevented by
constraining determiners to select NP's with the feature {\bf gerund = -}
This rules out sentences like {\it Private markets approved of (*the) [the
selling of bonds]}.} It is used for gerunds such as the ones in bold face
in sentences (\ex{6}), (\ex{7}) and (\ex{8}).

\begin{figure}[htb]
\centering
\begin{tabular}{c}
{\psfig{figure=ps/gerund-files/alphaDnx0Vnx1.ps,height=3.2in}}\\
$\alpha$Dnx0Vnx1\\
\end{tabular}
\caption{Determiner Gerund tree from the transitive tree family: $\alpha$Dnx0Vnx1}
\label{detgerund-tree}
\end{figure}

\enumsentence{Some think that {\bf the selling of bonds} is beneficial.}

\enumsentence{{\bf His painting of Mona Lisa} is highly acclaimed.}

\enumsentence{Are private markets approving of {\bf Washington's bashing of Wall Street}?}

There are two trees that can substitute into the D node: a simple
determiner tree, which can be anchored by a simple determiner (cf. example
(\ex{6})) or a possessive pronoun (cf. example (\ex{7})), and a
possessive NP tree (cf. example (\ex{8})) with the root node labeled as
D (see Figure~\ref{DnxG}).\footnote{The trees for genitive pronouns and
genitive NP's have the root node labelled as D because they seem to
function as determiners and as such, sequence with the rest of the
determiners. See Chapter~\ref{det-comparitives} for discussion on
Determiner trees.}

\begin{figure}[htb]
\centering
\begin{tabular}{c}
{\psfig{figure=ps/gerund-files/alphaDnxG.ps,height=1.5in}}\\
\end{tabular}
\caption{Possessive NP Determiner tree: $\alpha$DnxG}
\label{DnxG}
\end{figure}

The anchoring verb is required to be {\bf $<$mode$>$ = ger}. Auxiliaries
cannot left-adjoin to the anchoring verb in the Determiner gerund tree, as
can be seen in the ungrammatical sentences (\ex{9}) and (\ex{10}).

\enumsentence{* {\bf The having sold of bonds} is beneficial.}
\enumsentence{* {\bf The having been selling of bonds} is beneficial.}

Since the verb projects an N instead of a VP, this prevents auxiliary
adjunction. It should be noted, however, that the crucial motivation for
making the verb project an N instead of a VP arises from the ability of the
verb in these trees to be adjectivally modified (cf. example
(\ex{11})). Adjectival modification would not have been possible if
the projected node had been labeled VP.

\enumsentence{{\bf Her rapid writing of the book} was astonishing.}

\section{Sentential Gerunds}
Sentential gerundives show a number of structural peculiarities, but the
main puzzle about them in the linguistics literature has been that they
have the internal properties of sentences. Thus, as the example in
(\ex{12}) demonstrates, they may contain adverbial expressions and
object NP's, but adjectival modifiers and determiners are excluded.

\enumsentence{{\bf Her writing the book so rapidly} was astonishing.}

Such properties of internal structure are characteristic of clauses rather
than NP's, and thereby, in the English XTAG grammar, we adopt a position
similar to that of \cite{Rosenbaum67} and \cite{Emonds70} -- that these
gerunds are NP's exhaustively dominating a clause. Consequently, the tree
assigned to the NP gerund (cf. Figure~\ref{NPgerund-tree}) looks exactly
like the declarative transitive tree for clauses except for the root node
label and the feature values. The anchoring verb projects a VP. Auxiliary
adjunction is allowed (cf. example (\ex{13})), subject to one
constraint -- that the highest verb in the verbal sequence be in gerundive
form, regardless of whether it is a main or auxiliary verb.

\begin{figure}[htb]
\centering
\begin{tabular}{cc}
{\psfig{figure=ps/gerund-files/alphaGnx0Vnx1.ps,height=3.2in}}\\
$\alpha$Gnx0Vnx1\\
\end{tabular}
\caption{NP Gerund tree from the transitive tree family: $\alpha$Gnx0Vnx1}
\label{NPgerund-tree}
\end{figure}

\enumsentence{Private markets approved of {\bf Washington having bashed
Wall Street}}

This constraint is implemented by requiring the topmost VP node to be {\bf
$<$mode$>$ = ger}. In the absence of any adjunction, the anchoring verb
itself is forced to be gerundive, as in examples (\ex{3}), (\ex{4}), and
(\ex{5}). But if the verbal sequence has more than one verb, then the
sequence and form of the verbs is limited by the restrictions that each
verb in the sequence imposes on the succeeding verb. The nature of these
restrictions for sentential clauses, and the manner in which they are
implemented in XTAG, are both discussed in Chapter~\ref{auxiliaries}. The
analysis and implementation discussed there differs from that required for
gerunds only in one respect -- that the highest verb in the verbal sequence
is required to be {\bf $<$mode$>$ = ger}.\\

Additionally, the subject in the NP gerund tree is required to have {\bf
$<$case$>$=acc/none/gen}, i.e., it can be either a PRO (cf. example
\ex{3}), a genitive NP (cf. example \ex{4}), or an accusative NP
(cf. example \ex{5}). The whole NP formed by the gerund can occur in either
nominative or accusative positions.\\

One question that arises with respect to gerunds is whether there is anything
special about their distribution as compared to other types of NP's.  In fact,
it appears that gerund NP's can occur in any NP position.  Some verbs might not
seem to be very accepting of gerund NP arguments, as in (\ex{14}), but we
believe this to be a semantic incompatibility rather than a syntactic problem
since the same structures are possible with other lexical items.

\enumsentence{? [$_{NP}$John's tinkering$_{NP}$] ran.}
\enumsentence{[$_{NP}$John's tinkering$_{NP}$] worked.}

By having the root node of gerund trees be NP, the gerunds have the same
distribution as any other NP in the English XTAG grammar without doing
anything exceptional.\footnote{with the exception of NP's in ``equative
BE'' structures.} The clause structure is captured by the form of the trees
and by inclusion in the tree families.\\

\section{Gerund Passives}

It was mentioned above that gerunds display certain clausal properties. It
is therefore not surprising that gerunds also have their own set of
transformationally related structures. For example, gerunds allow
passivization just like their sentential counterparts (cf. examples
(\ex{16}) and (\ex{17})). 

\enumsentence{The lawyers objected to {\bf the slanderous book being
written by John}.}
\enumsentence{Susan could not forget {\bf having
been embarassed by the vicar}.}

In the English XTAG grammar, gerund passives are treated in an almost
exactly similar fashion to sentential passives, and are assigned separate
trees within the appropriate tree families. The passives occur in pairs,
one with the {\it by} phrase, and another without it. There are two feature
restrictions imposed on the passive trees: (a) only verbs with {\bf
$<$mode$>$ = ppart} (i.e., verbs with passive morphology) can be the
anchors, and (b) the highest verb in the verb sequence is required to be
{\bf $<$mode$>$ = ger}. The two restrictions, together, ensure the
selection of only those sequences of auxiliary verb(s) that select {\bf
$<$mode$>$ = ppart} and {\bf $<$passive$>$ = +} (such as {\it being} or
{\it having been} but NOT {\it having}). The passive trees are assumed to
be related to only the NP gerund trees (and not the Determiner gerund
trees), since passive structures involve movement of some object to the
subject position (in a movement analysis). Also, like the sentential
passives, gerund passives are found in most tree families that have a
direct object in the declarative tree. Figure~\ref{passive-trees} shows
the gerund passive trees for the transitive tree family.


\begin{figure}[htb]
\centering
\begin{tabular}{cc}
{\psfig{figure=ps/gerund-files/alphaGnx1Vbynx0.ps,height=3.8in}}&
{\psfig{figure=ps/gerund-files/alphaGnx1V.ps,height=3.8in}}
\\
(a) $\alpha$Gnx1Vbynx0&(b) $\alpha$Gnx1V\\
\end{tabular}
\caption{Passive Gerund trees from the transitive tree family: $\alpha$Gnx1Vbynx0 (a) and
$\alpha$Gnx1V (b)}
\label{passive-trees}
\end{figure}




























