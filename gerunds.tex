\chapter{Gerund NP's}
\label{gerunds-chapter}

There are two types of gerunds identified in the linguistics
literature. One is the class of {\it derived nominalizations} (also 
called {\it nominal gerundives} or {\it action nominalizations}) 
exemplified in (\ex{1}), which instantiates the direct object within an
{\it of} PP.
The other is the class of so-called {\it sentential} or
{\it VP gerundives} exemplified in (\ex{2}). In the English XTAG grammar,
the derived nominalizations are termed {\bf determiner gerunds}, and the
sentential or VP gerunds are termed {\bf NP gerunds}.

\enumsentence{Some think that {\bf the selling of bonds} is beneficial.}

\enumsentence{Are private markets approving of {\bf Washington bashing Wall
Street}?}

Both types of gerunds exhibit a similar distribution, appearing in most
places where NP's are allowed.\footnote{an exception being the NP positions
in ``equative BE'' sentences, such as, {\it John is my father}.}  The bold
face portions of sentences (\ex{1})--(\ex{3}) show examples of gerunds as a
subject and as the object of a preposition.

\enumsentence{{\bf Avoiding such losses} will take a monumental effort.}
\enumsentence{{\bf Mr. Nolen's wandering} doesn't make him a weirdo.}
\enumsentence{Are private markets approving of {\bf Washington bashing Wall
Street}?}

The motivation for splitting the gerunds into two classes is semantic as
well as structural in nature. Semantically, the two gerunds are in sharp
contrast with each other. NP gerunds refer to an action, i.e., a way of
doing something, whereas determiner gerunds refer to a fact. Structurally,
there are a number of properties (extensively discussed in \cite{Lees60})
that show that NP gerunds have the syntax of verbs, whereas determiner
gerunds have the syntax of basic nouns.  Firstly, the fact that the direct
object of the determiner gerund can only appear within an {\it of} PP
suggests that the determiner gerund, like nouns, is not a case assigner and
needs insertion of the preposition {\it of} for assignment of case to the
direct object. NP gerunds, like verbs, need no such insertion and can
assign case to their direct object.  Secondly, like nouns, only determiner
gerunds can appear with articles (cf. example (\ex{1}) and
(\ex{2})). Thirdly, determiner gerunds, like nouns, can be modified by
adjectives (cf. example (\ex{3})), whereas NP gerunds, like verbs, resist
such modification (cf. example (\ex{4})). Fourthly, nouns, unlike verbs,
cannot co-occur with aspect (cf. example (\ex{5}) and (\ex{6})). Finally,
only NP gerunds, like verbs, can take adverbial modification (cf. example
(\ex{7}) and (\ex{8})).

\enumsentence{\ldots the proving of the theorem\ldots. \hspace{1.0in} (det
ger with article)}
\enumsentence{* \ldots the proving the theorem\ldots. \hspace{1.0in} (NP ger
with article)}
\enumsentence{John's rapid writing of the book\ldots. \hspace{1.0in} (det
ger with Adj)}
\enumsentence{* John's rapid writing the book\ldots. \hspace{1.0in} (NP ger
with Adj)}
\enumsentence{* John's having written of the book\ldots. \hspace{1.0in}
(det ger with aspect)}
\enumsentence{John having written the book\ldots. \hspace{1.0in} (NP ger
with aspect)}
\enumsentence{* His writing of the book rapidly\ldots. \hspace{1.0in} (det
ger with Adverb)}
\enumsentence{His writing the book rapidly\ldots. \hspace{1.0in} (NP ger
with Adverb)}

In English XTAG, the two types of gerunds are assigned separate
trees within each tree family, but in order to capture their similar
distributional behavior, both are assigned NP as the category label of
their top node. The feature {\bf gerund = +/--} distinguishes gerund NP's from
regular NP's where needed.\footnote{This feature is also needed to restrict
the selection of gerunds in NP positions. For example, the subject and
object NP's in the ``equative BE'' tree (Tnx0BEnx1) cannot be filled by
gerunds, and are therefore assigned the feature {\bf
gerund = --}, which prevents gerunds (which have the feature {\bf gerund =
+}) from substituting into these NP positions.} The determiner gerund 
and the NP gerund trees are discussed in section~(\ref{detger-sec}) and
~(\ref{NPger-sec}) respectively.

\section{Determiner Gerunds}
\label{detger-sec}
The determiner gerund tree in Figure~\ref{detgerund-tree} is anchored by a
V, capturing the fact that the gerund is derived from a verb. The verb
projects an N and instantiates the direct object as an {\it of} PP. The
nominal category projected by the verb can now display all the syntactic
properties of basic nouns, as discussed above. For example, it can be
straightforwardly modified by adjectives; it cannot co-occur with aspect;
and it can appear with articles. The only difference of the determiner
gerund nominal with the basic nominals is that the former cannot occur
without the determiner, whereas the latter can. The determiner gerund 
tree therefore has an initial D modifying the N.\footnote{Note that
the determiner can adjoin to the gerund only from {\it within} the gerund
tree. Adjunction of determiners to the gerund root node is prevented by
constraining determiners to only select NP's with the feature {\bf gerund = --}.
This rules out sentences like {\it Private markets approved of (*the) [the
selling of bonds]}.} It is used for gerunds such as the ones in bold face
in sentences (\ex{1}), (\ex{2}) and (\ex{3}).

\begin{figure}[htb]
\centering
\begin{tabular}{c}
{\psfig{figure=ps/gerund-files/alphaDnx0Vnx1.ps,height=4.0in}}\\
$\alpha$Dnx0Vnx1\\
\end{tabular}
\caption{Determiner Gerund tree from the transitive tree family: $\alpha$Dnx0Vnx1}
\label{detgerund-tree}
\label{2;12,1}

\end{figure}

The D node can take a simple determiner (cf. example (\ex{1})), a
genitive pronoun (cf. example (\ex{2})), or a genitive NP (cf. example
(\ex{3})).\footnote{The trees for genitive pronouns and genitive NP's
have the root node labelled as D because they seem to function as
determiners and as such, sequence with the rest of the
determiners. See Chapter~\ref{det-comparitives} for discussion on
determiner trees.}

\enumsentence{Some think that {\bf the selling of bonds} is beneficial.}
\enumsentence{{\bf His painting of Mona Lisa} is highly acclaimed.}
\enumsentence{Are private markets approving of {\bf Washington's bashing of Wall Street}?}

%\begin{figure}[htb]
%\centering
%\begin{tabular}{c}
%{\psfig{figure=ps/gerund-files/alphaDnxG.ps,height=1.5in}}\\
%\end{tabular}
%\caption{Possessive NP Determiner tree: $\alpha$DnxG}
%\label{DnxG}
%\end{figure}

%The anchoring verb is required to be {\bf $<$mode$>$ = ger}. Auxiliaries
%cannot left-adjoin to the anchoring verb in the determiner gerund tree, as
%can be seen in the ungrammatical sentences (\ex{1}) and (\ex{2}).

%\enumsentence{* {\bf The having sold of bonds} is beneficial.}
%\enumsentence{* {\bf The having been selling of bonds} is beneficial.}

%Since the verb projects an N instead of a VP, this prevents auxiliary
%adjunction. It should be noted, however, that the crucial motivation for
%making the verb project an N instead of a VP arises from the ability of the
%verb in these trees to be adjectivally modified (cf. example
%(\ex{1})). Adjectival modification would not have been possible if
%the projected node had been labeled VP.

%\enumsentence{{\bf Her rapid writing of the book} was astonishing.}

\section{NP Gerunds}
\label{NPger-sec}
NP gerunds show a number of structural peculiarities, the crucial one being
that they have the internal properties of sentences. In the English XTAG
grammar, we adopt a position similar to that of \cite{Rosenbaum67} and
\cite{Emonds70} -- that these gerunds are NP's exhaustively dominating a
clause. Consequently, the tree assigned to the transitive NP gerund tree
(cf. Figure~\ref{NPgerund-tree}) looks exactly like the declarative
transitive tree for clauses except for the root node label and the feature
values. The anchoring verb projects a VP. Auxiliary adjunction is allowed,
subject to one constraint -- that the highest verb in the verbal sequence
be in gerundive form, regardless of whether it is a main or auxiliary verb.
This constraint is implemented by requiring the topmost VP node to be {\bf
$<$mode$>$ = ger}. In the absence of any adjunction, the anchoring verb
itself is forced to be gerundive. But if the verbal sequence has more than
one verb, then the sequence and form of the verbs is limited by the
restrictions that each verb in the sequence imposes on the succeeding
verb. The nature of these restrictions for sentential clauses, and the
manner in which they are implemented in XTAG, are both discussed in
Chapter~\ref{auxiliaries}. The analysis and implementation discussed there
differs from that required for gerunds only in one respect -- that the
highest verb in the verbal sequence is required to be {\bf $<$mode$>$ =
ger}.


\begin{figure}[htb]
\centering
\begin{tabular}{cc}
{\psfig{figure=ps/gerund-files/alphaGnx0Vnx1.ps,height=4.0in}}\\
$\alpha$Gnx0Vnx1\\
\end{tabular}
\caption{NP Gerund tree from the transitive tree family: $\alpha$Gnx0Vnx1}
\label{NPgerund-tree}
\label{2;13,1}
\end{figure}


Additionally, the subject in the NP gerund tree is required to have {\bf
$<$case$>$=acc/none/gen}, i.e., it can be either a PRO (cf. example
\ex{1}), a genitive NP (cf. example \ex{2}), or an accusative NP
(cf. example \ex{3}). The whole NP formed by the gerund can occur in either
nominative or accusative positions.


\enumsentence{\ldots John does not like {\bf wearing a hat}.}
\enumsentence{Are private markets approving of {\bf Washington's bashing Wall
Street}?}
\enumsentence{Mother disapproved of {\bf me wearing such casual clothes}.}

One question that arises with respect to gerunds is whether there is anything
special about their distribution as compared to other types of NP's.  In fact,
it appears that gerund NP's can occur in any NP position.  Some verbs might not
seem to be very accepting of gerund NP arguments, as in (\ex{1}) below, but we
believe this to be a semantic incompatibility rather than a syntactic problem
since the same structures are possible with other lexical items.

\enumsentence{? [$_{NP}$John's tinkering$_{NP}$] ran.}
\enumsentence{[$_{NP}$John's tinkering$_{NP}$] worked.}

By having the root node of gerund trees be NP, the gerunds have the same
distribution as any other NP in the English XTAG grammar without doing
anything exceptional. The clause structure is captured by the form of the trees
and by inclusion in the tree families.

\section{Gerund Passives}

It was mentioned above that the NP gerunds display certain clausal
properties. It is therefore not surprising that they too have their own set
of transformationally related structures. For example, NP gerunds allow
passivization just like their sentential counterparts (cf. examples
(\ex{1}) and (\ex{2})).

\enumsentence{The lawyers objected to {\bf the slanderous book being
written by John}.}
\enumsentence{Susan could not forget {\bf having
been embarrassed by the vicar}.}

In the English XTAG grammar, gerund passives are treated in an almost
exactly similar fashion to sentential passives, and are assigned separate
trees within the appropriate tree families. The passives occur in pairs,
one with the {\it by} phrase, and another without it. There are two feature
restrictions imposed on the passive trees: (a) only verbs with {\bf
$<$mode$>$ = ppart} (i.e., verbs with passive morphology) can be the
anchors, and (b) the highest verb in the verb sequence is required to be
{\bf $<$mode$>$ = ger}. The two restrictions, together, ensure the
selection of only those sequences of auxiliary verb(s) that select {\bf
$<$mode$>$ = ppart} and {\bf $<$passive$>$ = +} (such as {\it being} or
{\it having been} but NOT {\it having}). The passive trees are assumed to
be related to only the NP gerund trees (and not the determiner gerund
trees), since passive structures involve movement of some object to the
subject position (in a movement analysis). Also, like the sentential
passives, gerund passives are found in most tree families that have a
direct object in the declarative tree. Figure~\ref{pass-trees} shows
the gerund passive trees for the transitive tree family.


\begin{figure}[htb]
\centering
\begin{tabular}{cc}
{\psfig{figure=ps/gerund-files/alphaGnx1Vbynx0.ps,height=4.0in}}&
{\psfig{figure=ps/gerund-files/alphaGnx1V.ps,height=4.0in}}
\\
(a) $\alpha$Gnx1Vbynx0&(b) $\alpha$Gnx1V\\
\end{tabular}
\caption{Passive Gerund trees from the transitive tree family: $\alpha$Gnx1Vbynx0 (a) and
$\alpha$Gnx1V (b)}
\label{pass-trees}
\end{figure}




























