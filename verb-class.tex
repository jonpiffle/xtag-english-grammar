%When adding new families here, also add them to table.tex and add the
%examples to the appropriate enumsentences.X file.

\chapter{Verb Classes}
\label{verb-classes}

Each main\footnote{Auxiliary verbs are handled under a different
mechanism.  See Chapter~\ref{auxiliaries} for details.} verb in the
syntactic lexicon selects at least one tree family\footnote{See
section \ref{tree-db} for explanation of tree families.}
(subcategorization frame).  Since the tree database and syntactic
lexicon are already separated for space efficiency (see
Chapter~\ref{overview}), each verb can efficiently select a large
number of trees by specifying a tree family, as opposed to each of the
individual trees.  This approach allows for a considerable reduction
in the number of trees that must be specified for any given verb or
form of a verb.

There are currently 57 tree families in the system.\footnote{An
  explanation of the naming convention used in naming the trees and
  tree families is available in Appendix~\ref{tree-naming}.}  This
chapter gives a brief description of each tree family and shows the
corresponding declarative tree\footnote{Before lexicalization, the
  $\diamond$ indicates the anchor of the tree.}, along with any
peculiar characteristics or trees.  It also indicates which
transformations are in each tree family, and gives the number of verbs
that select that family.\footnote{Numbers given are as of December 2000
  and are subject to some change with further development of the
  grammar.} A few sample verbs are given, along with example
sentences.


\section{Intransitive: Tnx0V}\index{verbs, intransitive}
\label{nx0V-family}

\begin{description}
  
\item[Description:] This tree family is selected by verbs that do not
  require an object complement of any type.  Adverbs, prepositional
  phrases and other adjuncts may adjoin on, but are not required for
  the sentences to be grammatical.  1,941 verbs select this family.

\item[Examples:]  {\it eat}, {\it sleep}, {\it dance} \\
{\it Al ate .} \\ 
{\it Seth slept .} \\ 
{\it Hyun danced .}

\item[Declarative tree:]  See Figure~\ref{nx0V-tree}.

\begin{figure}[htb]
\centering
\begin{tabular}{c}
\psfig{figure=ps/verb-class-files/alphanx0V.ps,height=3.4cm}
\end{tabular}
\caption{Declarative Intransitive Tree:  $\alpha$nx0V}
\label{nx0V-tree}
\end{figure}

\item[Other available trees:] wh-moved subject, 
subject relative clause with overt and covert extracted {\it wh}-NP, 
adjunct (gap-less) relative clause with covert extracted {\it wh}-NP,
adjunct (gap-less) relative clause with PP pied-piping,
imperative, 
determiner gerund, 
NP gerund, 
pre-nominal participial modifier, 
PRO subject, 
NP gerund with PRO subject.

\end{description}


\section{Ergative: TEnx1V}\index{verbs,ergative}
\label{Enx1V-family}

\begin{description}
  
\item[Description:] This tree family is selected by verbs for which
  the syntactic subject corresponds to the logical object, which is
  indicated by the index of `1' on the subject NP.  See
  Chapter~\ref{ergatives} for more information about the ergative
  alternation.  292 verbs select the ergative tree family.

\item[Examples:] {\it sink}, {\it melt}\\
{\it The ship sank .} \\ 
{\it The ice melted .} \\ 

\item[Declarative tree:] See Figure~\ref{Enx1V-tree}.

\begin{figure}[htb]
\centering
\begin{tabular}{c}
\psfig{figure=ps/verb-class-files/alphaEnx1V.ps,height=3.4cm}
\end{tabular}
\caption{Declarative Ergative Tree:  $\alpha$Enx1V}
\label{Enx1V-tree}
\end{figure}

\item[Other available trees:] wh-moved subject, subject relative clause
with overt and covert extracted {\it wh}-NP's, adjunct (gap-less) relative
clause with covert extracted {\it wh}-NP, adjunct (gap-less) relative clause with PP pied-piping,
NP gerund, determiner gerund, imperative, PRO subject, NP gerund with PRO
subject, prenominal participial modifier.

\end{description}


\section{Transitive: Tnx0Vnx1}\index{verbs,transitive}
\label{nx0Vnx1-family}

\begin{description}
  
\item[Description:] This tree family is selected by verbs that require
  only an NP object complement.  The NP's may be complex structures,
  including gerund NP's and NP's that take sentential complements.
  This does not include light verb constructions (see
  sections~\ref{nx0lVN1-family} and \ref{nx0lVN1Pnx2-family}).  4,506
  verbs select the transitive tree family.

\item[Examples:] {\it eat}, {\it dance}, {\it take}, {\it like}\\
{\it Al ate an apple .} \\ 
{\it Seth danced the tango .} \\ 
{\it Hyun is taking an algorithms course .} \\
{\it Anoop likes the fact that the semester is finished .}

\item[Declarative tree:] See Figure~\ref{nx0Vnx1-tree}.

\begin{figure}[htb]
\centering
\begin{tabular}{c}
\psfig{figure=ps/verb-class-files/alphanx0Vnx1.ps,height=3.4cm}
\end{tabular}
\caption{Declarative Transitive Tree:  $\alpha$nx0Vnx1}
\label{nx0Vnx1-tree}
\end{figure}

\item[Other available trees:] 
wh-moved subject, 
wh-moved object, 
subject relative clause with overt and covert extracted {\it wh} NP's, 
adjunct (gap-less) relative clause with covert extracted {\it wh} NP/with PP pied-piping, 
object relative clause with overt and covert extracted {\it wh}-NP's, 
imperative, 
determiner gerund, 
NP gerund, passive with {\it by} phrase, 
passive without {\it by} phrase, 
passive with wh-moved subject and {\it by} phrase, 
passive with wh-moved subject and no {\it by} phrase,
passive with wh-moved object out of the {\it by} phrase, 
passive with wh-moved {\it by} phrase, 
passive with relative clause on subject and {\it by} phrase overt and covert extracted {\it wh} NP's, 
passive with relative clause on subject and no {\it by} phrase with overt and covert extracted {\it wh} NP's, 
passive with relative clause on object on the {\it by} phrase with overt and covert extracted {\it wh} NP's/with PP pied-piping,
gerund passive with {\it by} phrase, 
gerund passive without {\it by} phrase, 
PRO subject, 
NP gerund with PRO subject, 
passive with {\it by}-phrase with PRO subject, 
passive without {\it by} phrase with PRO subject, 
gerund passive with {\it by} phrase with PRO subject, 
gerund passive without {\it by} phrase with PRO subject, 
prenominal participial modifiers,
derived adjectives.

\end{description}



\section{Ditransitive: Tnx0Vnx2nx1}\index{verbs,ditransitive}
\label{nx0Vnx2nx1-family}

\begin{description}

\item[Description:]  This tree family is selected by verbs that take exactly 
two NP complements.  The apparent ditransitive alternates involving
verbs in this class and benefactive PP's (e.g. {\it John baked a cake
for Mary}) are analyzed as transitives (see
section~\ref{nx0Vnx1-family}) with a PP adjunct. Benefactives are
taken to be adjunct PP's because they are optional (e.g. {\it John
baked a cake} vs. {\it John baked a cake for Mary}).  167 verbs select
the ditransitive tree family.

\item[Examples:] {\it ask}, {\it cook}, {\it win} \\
{\it Christy asked Mike a question .} \\ 
{\it Doug cooked his father dinner .} \\
{\it Dania won her sister a stuffed animal .}

\item[Declarative tree:]  See Figure~\ref{nx0Vnx2nx1-tree}.

\begin{figure}[htb]
\centering
\begin{tabular}{c}
\psfig{figure=ps/verb-class-files/alphanx0Vnx2nx1.ps,height=3.4cm}
\end{tabular}
\caption{Declarative Ditransitive Tree:  $\alpha$nx0Vnx2nx1}
\label{nx0Vnx2nx1-tree}
\end{figure}

\item[Other available trees:] wh-moved subject, wh-moved direct object,
wh-moved indirect object, subject relative clause with covert and overt
extracted {\it wh}-NP's, adjunct (gap-less) relative clause with covert
extracted {\it wh} NP/with PP pied-piping, direct object relative clause
with covert and overt extracted {\it wh}-NP's, indirect object relative
clause with covert and overt extracted {\it wh}-NP's, imperative,
determiner gerund, NP gerund, passive with {\it by} phrase, passive without
{\it by} phrase, passive with wh-moved subject and {\it by} phrase, passive
with wh-moved subject and no {\it by} phrase, passive with wh-moved object
out of the {\it by} phrase, passive with wh-moved {\it by} phrase, passive
with wh-moved indirect object and {\it by} phrase, passive with wh-moved
indirect object and no {\it by} phrase, passive with relative clause on
subject and {\it by} phrase with covert and overt extracted {\it wh}-NP's,
passive with relative clause on subject and no {\it by} phrase with covert
and overt extracted {\it wh}-NP's, passive with relative clause on object
of the {\it by} phrase with covert and overt extracted {\it wh}-NP's/with
PP pied-piping, passive with relative clause on the indirect object and
{\it by} phrase with covert and overt extracted {\it wh}-NP's, passive with
relative clause on the indirect object and no {\it by} phrase with covert
and overt extracted {\it wh}-NP's, passive with/without {\it by}-phrase
with adjunct (gap-less) relative clause with covert extracted {\it
wh}-NP/with PP pied-piping, gerund passive with {\it by} phrase, gerund
passive without {\it by} phrase, PRO subject, passive with and without {\it
by}-phrase with PRO subject, NP gerund with PRO subject, NP gerund passive
with and without the {\it by}-phrase and PRO subject.


\end{description}


\section{Ditransitive with PP: Tnx0Vnx1pnx2}\index{verbs, NP with VP verbs}
\label{nx0Vnx1pnx2-family}

\begin{description}

\item[Description:]  This tree family is selected by ditransitive verbs that
take a noun phrase followed by a prepositional phrase.  The
preposition is not constrained in the syntactic lexicon.  The
preposition must be required and not optional - that is, the sentence
must be ungrammatical with just the noun phrase (e.g. {\it $\ast$John
put the table}).  No verbs, therefore, should select both this tree
family and the transitive tree family (see
section~\ref{nx0Vnx1-family}).  There are 62 verbs that select this
tree family.

\item[Examples:] {\it ensconce}, {\it put}, {\it usher} \\
{\it Mary ensconced herself on the sofa .}   \\
{\it He put the book on the table .}  \\
{\it He ushered the patrons into the theater .}

\item[Declarative tree:]  See Figure~\ref{nx0Vnx1pnx2-tree}.

\begin{figure}[htb]
\centering
\begin{tabular}{c}
\psfig{figure=ps/verb-class-files/alphanx0Vnx1pnx2.ps,height=4.0cm}
\end{tabular}
\caption{Declarative Ditransitive with PP Tree:  $\alpha$nx0Vnx1pnx2}
\label{nx0Vnx1pnx2-tree}
\end{figure}

\item[Other available trees:] wh-moved subject, wh-moved direct object,
wh-moved object of PP, wh-moved PP, subject relative clause with overt and
covert extracted {\it wh}-NP's, adjunct (gap-less) relative clause with
covert extracted {\it wh}-NP/with PP pied-piping, direct object relative
clause with overt and covert extracted {\it wh}-NP's, object of PP relative
clause with overt and covert extracted {\it wh}-NP's/with PP pied-piping,
imperative, determiner gerund, NP gerund, passive with {\it by} phrase,
passive without {\it by} phrase, passive with wh-moved subject and {\it by}
phrase, passive with wh-moved subject and no {\it by} phrase, passive with
wh-moved object out of the {\it by} phrase, passive with wh-moved {\it by}
phrase, passive with wh-moved object out of the PP and {\it by} phrase,
passive with wh-moved object out of the PP and no {\it by} phrase, passive
with wh-moved PP and {\it by} phrase, passive with wh-moved PP and no {\it
by} phrase, passive with relative clause on subject and {\it by} phrase
with overt and covert extracted {\it wh}-NP's, passive with relative clause
on subject and no {\it by} phrase with overt and covert extracted {\it
wh}-NP's, passive with relative clause on object of the {\it by} phrase
with overt and covert extracted {\it wh}-NP's/with PP pied-piping, passive
with relative clause on the object of the PP and {\it by} phrase with overt
and covert extracted {\it wh}-NP's/with PP pied-piping, passive with
relative clause on the object of the PP and no {\it by} phrase with overt
and covert extracted {\it wh}-NP's/with PP pied-piping, passive with and
without {\it by} phrase with adjunct (gap-less) relative clause with covert
extracted {\it wh}-NP/with PP pied-piping, gerund passive with {\it by}
phrase, gerund passive without {\it by} phrase, PRO subject, passive with
and without {\it by}-phrase with PRO subject, NP gerund with PRO subject,
NP gerund passive with and without the {\it by}-phrase and PRO subject.

\end{description}



\section{Multiple anchor ditransitive with PP:
Tnx0Vnx1Pnx2}\index{verbs, multi-anchor ditransitive verbs}
\label{nx0Vnx1Pnx2-family}

\begin{description}

\item[Description:]  This tree family is selected by ditransitive verbs that
take a noun phrase followed by a prepositional phrase headed by a
particular preposition.  The preposition is constrained by making it
one of the anchors.  There are 84 verbs that select this tree family.

\item[Examples:] {\it gear for}, {\it give to}, {\it remind of} \\
{\it The attorney geared his client for the trial .}
{\it He gave the book to his teacher .}  \\
{\it The city reminded John of his home town .}

\item[Declarative tree:]  See Figure~\ref{nx0Vnx1Pnx2-tree}.

\begin{figure}[htb]
\centering
\begin{tabular}{c}
\psfig{figure=ps/verb-class-files/alphanx0Vnx1Pnx2.ps,height=4.0cm}
\end{tabular}
\caption{Declarative Multiple anchor Ditransitive with PP Tree:  $\alpha$nx0Vnx1Pnx2}
\label{nx0Vnx1Pnx2-tree}
\end{figure}

\item[Other available trees:] wh-moved subject, wh-moved direct object,
wh-moved object of PP, wh-moved PP, subject relative clause with overt and
covert extracted {\it wh}-NP's, adjunct (gap-less) relative clause with
covert extracted {\it wh}-NP/with PP pied-piping, direct object relative
clause with overt and covert extracted {\it wh}-NP's, object of PP relative
clause with overt and covert extracted {\it wh}-NP's/with PP pied-piping,
imperative, determiner gerund, NP gerund, passive with {\it by} phrase,
passive without {\it by} phrase, passive with wh-moved subject and {\it by}
phrase, passive with wh-moved subject and no {\it by} phrase, passive with
wh-moved object out of the {\it by} phrase, passive with wh-moved {\it by}
phrase, passive with wh-moved object out of the PP and {\it by} phrase,
passive with wh-moved object out of the PP and no {\it by} phrase, passive
with wh-moved PP and {\it by} phrase, passive with wh-moved PP and no {\it
by} phrase, passive with relative clause on subject and {\it by} phrase
with overt and covert extracted {\it wh}-NP's, passive with relative clause
on subject and no {\it by} phrase with overt and covert extracted {\it
wh}-NP's, passive with relative clause on object of the {\it by} phrase
with overt and covert extracted {\it wh}-NP's/with PP pied-piping, passive
with relative clause on the object of the PP and {\it by} phrase with overt
and covert extracted {\it wh}-NP's/with PP pied-piping, passive with
relative clause on the object of the PP and no {\it by} phrase with overt
and covert extracted {\it wh}-NP's/with PP pied-piping, passive with and
without {\it by} phrase with adjunct (gap-less) relative clause with covert
extracted {\it wh}-NP/with PP pied-piping, gerund passive with {\it by}
phrase, gerund passive without {\it by} phrase, PRO subject, passive with
and without {\it by}-phrase with PRO subject, NP gerund with PRO subject,
NP gerund passive with and without the {\it by}-phrase and PRO subject.

\end{description}


% This section has been commented out since the Tnx0Vnx1tonx2 family
% is no longer part of the grammar.  (jason: 12/9/98)
% 
% \section{Ditransitive with PP shift: Tnx0Vnx1tonx2}\index{verbs,ditransitive
% with PP shift}
% \label{nx0Vnx1Pnx2-family}
% 
% \begin{description}
% 
% \item[Description:]  This tree family is selected by ditransitive verbs that
% undergo a shift to a {\it to} prepositional phrase.  These ditransitive verbs
% are clearly constrained so that when they shift, the prepositional phrase must
% start with {\it to}.  This is in contrast to the Ditransitives with PP in
% section~\ref{nx0Vnx1pnx2-family}, in which verbs may appear in [NP V NP PP]
% constructions with a variety of prepositions.  Both the dative shifted and
% non-shifted PP complement trees are included.  56 verbs select this family.
% 
% \item[Examples:] {\it give}, {\it promise}, {\it tell} \\
% {\it Bill gave Hillary flowers .} \\ 
% {\it Bill gave flowers to Hillary .} \\
% {\it Whitman promised the voters a tax cut .} \\
% {\it Whitman promised a tax cut to the voters .} \\
% {\it Pinnochino told Gepetto a lie .} \\
% {\it Pinnochino told a lie to Gepetto .}
% 
% \item[Declarative tree:]  See Figure~\ref{nx0Vnx1Pnx2-tree}.
% 
% \begin{figure}[htb]
% \centering
% \begin{tabular}{ccc}
% \psfig{figure=ps/verb-class-files/alphanx0Vnx1Pnx2.ps,height=5.2cm} &
% \hspace{1.0in}&
% \psfig{figure=ps/verb-class-files/alphanx0Vnx2nx1.ps,height=3.3cm} \\
% (a) && (b)
% \end{tabular}
% \caption{Declarative Ditransitive with PP shift Trees: $\alpha$nx0Vnx1Pnx2~(a)
% and $\alpha$nx0Vnx2nx1~(b)}
% \label{nx0Vnx1Pnx2-tree}
% \end{figure}
% 
% \item[Other available trees:] {\bf Non-shifted:} wh-moved subject, wh-moved
% direct object, wh-moved indirect object, subject relative clause with overt and covert extracted {\it wh}-NP's, 
% adjunct (gap-less) relative clause with covert extracted {\it wh}-NP/with PP pied-piping, direct
% object relative clause with covert extracted {\it wh}-NP/with PP pied-piping, indirect object relative clause
% with overt and covert extracted {\it wh}-NP's/with PP pied-piping, imperative, NP
% gerund, passive with {\it by} phrase, passive without {\it by} phrase,
% passive with wh-moved subject and {\it by} phrase, passive with wh-moved
% subject and no {\it by} phrase, passive with wh-moved object out of the
% {\it by} phrase, passive with wh-moved {\it by} phrase, passive with
% wh-moved indirect object and {\it by} phrase, passive with wh-moved
% indirect object and no {\it by} phrase, passive with relative clause on
% subject and {\it by} phrase with overt and covert extracted {\it wh}-NP's, 
% passive with relative clause on subject and no
% {\it by} phrase with overt and covert extracted {\it wh}-NP's, passive with relative clause on object of the {\it by}
% phrase with overt and covert extracted {\it wh}-NP's/with PP pied-piping, 
% passive with relative clause on the indirect object and {\it by}
% phrase with overt and covert extracted {\it wh}-NP's/with PP pied-piping, 
% passive with relative clause on the indirect object and no {\it by}
% phrase with overt and covert extracted {\it wh}-NP's/with PP pied-piping, 
% passive with/without {\it by}-phrase with adjunct (gap-less) relative clause
% with covert extracted {\it wh}-NP/with PP pied-piping,
% gerund passive with {\it by} phrase, gerund passive without {\it
% by} phrase;\\ 
% {\bf Shifted:} wh-moved subject, wh-moved direct object,
% wh-moved object of PP, wh-moved PP, subject relative clause with overt and covert extracted {\it wh}-NP's, 
% adjunct (gap-less) relative clause with covert extracted {\it wh}-NP/with PP pied-piping, direct object
% relative clause with covert extracted {\it wh}-NP/with PP pied-piping, object of PP relative clause with and without 
% comp/with PP pied-piping, imperative, determiner
% gerund, NP gerund, passive with {\it by} phrase, passive without {\it by}
% phrase, passive with wh-moved subject and {\it by} phrase, passive with
% wh-moved subject and no {\it by} phrase, passive with wh-moved object out
% of the {\it by} phrase, passive with wh-moved {\it by} phrase, passive with
% wh-moved object out of the PP and {\it by} phrase, passive with wh-moved
% object out of the PP and no {\it by} phrase, passive with wh-moved PP and
% {\it by} phrase, passive with wh-moved PP and no {\it by} phrase, passive
% with relative clause on subject and {\it by} phrase with overt and covert extracted {\it wh}-NP's, passive with relative
% clause on subject and no {\it by} phrase with overt and covert extracted {\it wh}-NP's, passive with relative clause on
% object of the {\it by} phrase with overt and covert extracted {\it wh}-NP's/with PP pied-piping, 
% passive with relative clause on the object
% of the PP and {\it by} phrase with overt and covert extracted {\it wh}-NP's/with PP pied-piping, 
% passive with relative clause on the object
% of the PP and no {\it by} phrase with overt and covert extracted {\it wh}-NP's/with PP pied-piping, 
% passive with/without {\it by}-phrase with adjunct (gap-less) relative clause
% with covert extracted {\it wh}-NP/with PP pied-piping, gerund passive with {\it by} phrase,
% gerund passive without {\it by} phrase.
% 
% 
% \end{description}
% 

\section{Sentential Complement with NP: Tnx0Vnx1s2}\index{verbs,Sentential
Complement with NP} 
\label{nx0Vnx1s2-family}

\begin{description}
  
\item[Description:] This tree family is selected by verbs that take both an
NP and a sentential complement.  The sentential complement may be
infinitive or indicative.  The type of clause is specified by each
individual verb in its syntactic lexicon entry.  A given verb may select
more than one type of sentential complement.  The declarative tree, and
many other trees in this family, are auxiliary trees, as opposed to the
more common initial trees.  These auxiliary trees adjoin onto an S node in
an existing tree of the type specified by the sentential complement.  This
is the mechanism by which TAGs are able to maintain long-distance
dependencies (see Chapter~\ref{extraction}), even over multiple embeddings
(e.g. {\it What did Bill tell Mary that John said?}).  83 verbs select this
tree family.

\item[Examples:] {\it beg}, {\it expect}, {\it tell} \\
{\it Srini begged Mark to increase his disk quota .} \\
{\it Beth told Jim that it was his turn .}

\item[Declarative tree:]  See Figure~\ref{nx0Vnx1s2-tree}.

\begin{figure}[htb]
\centering
\begin{tabular}{c}
\psfig{figure=ps/verb-class-files/betanx0Vnx1s2.ps,height=3.4cm}
\end{tabular}
\caption{Declarative Sentential Complement with NP Tree:  $\beta$nx0Vnx1s2}
\label{nx0Vnx1s2-tree}
\end{figure}

\item[Other available trees:] wh-moved subject, wh-moved object, wh-moved
sentential complement, subject relative clause with overt and covert
extracted {\it wh}-NP's, adjunct (gap-less) relative clause with covert
extracted {\it wh}-NP/with PP pied-piping, object relative clause with
overt and covert extracted {\it wh}-NP's, imperative, determiner gerund, NP
gerund, passive with {\it by} phrase before sentential complement, passive
with {\it by} phrase after sentential complement, passive without {\it by}
phrase, passive with wh-moved subject and {\it by} phrase before sentential
complement, passive with wh-moved subject and {\it by} phrase after
sentential complement, passive with wh-moved subject and no {\it by}
phrase, passive with wh-moved object out of the {\it by} phrase, passive
with wh-moved {\it by} phrase, passive with relative clause on subject and
{\it by} phrase before sentential complement with overt and covert
extracted {\it wh}-NP's, passive with relative clause on subject and {\it
by} phrase after sentential complement with overt and covert extracted {\it
wh}-NP's, passive with relative clause on subject and no {\it by} phrase
with overt and covert extracted {\it wh}-NP's, passive with/without {\it
by}-phrase with adjunct (gap-less) relative clause with covert extracted
{\it wh}-NP/with PP pied-piping, gerund passive with {\it by} phrase before
sentential complement, gerund passive with {\it by} phrase after the
sentential complement, gerund passive without {\it by} phrase,
parenthetical reporting clause, PRO subject, passive with PRO subject with
{\it by}-phrase before and after sentential complement, passive with PRO
subject without {\it by}-phrase, NP gerund with PRO subject, NP gerund
passive with PRO subject with {\it by}-phrase before and after sentential
complement, NP gerund passive with PRO subject without {\it by}-phrase.

\end{description}

\section{Intransitive Verb Particle: Tnx0Vpl}\index{verbs,verb-particle,intransitive}
\label{nx0Vpl}

\begin{description}

\item[Description:] The trees in this tree family are anchored by both the
verb and the verb particle.  Both appear in the syntactic lexicon and
together select this tree family.  Intransitive verb particles can be
difficult to distinguish from intransitive verbs with adverbs adjoined
on. The main diagnostics for including verbs in this class are whether the
meaning is compositional or not, and whether there is a transitive version
of the verb/verb particle combination with the same or similar meaning.
The existence of an alternate compositional meaning is a strong indication
for a separate verb particle construction.  There are 159 verb/verb
particle combinations.

\item[Examples:] {\it add up}, {\it come out}, {\it sign off} \\
{\it The numbers never quite added up .} \\
{\it John finally came out (of the closet) .} \\
{\it I think that I will sign off now .}

\item[Declarative tree:]  See Figure~\ref{nx0Vpl-tree}.

\begin{figure}[htb]
\centering
\begin{tabular}{c}
\psfig{figure=ps/verb-class-files/alphanx0Vpl.ps,height=3.4cm}
\end{tabular}
\caption{Declarative Intransitive Verb Particle Tree:  $\alpha$nx0Vpl}
\label{nx0Vpl-tree}
\end{figure}

\item[Other available trees:] wh-moved subject, subject relative clause
with overt and covert extracted {\it wh}-NP's, adjunct (gap-less) relative
clause with covert extracted {\it wh}-NP/with PP pied-piping, imperative,
determiner gerund, NP gerund, PRO subject, NP gerund with PRO subject.

\end{description}

\section{Transitive Verb Particle: Tnx0Vplnx1}\index{verbs,particle,transitive}
\label{nx0Vplnx1-family}

\begin{description}
  
\item[Description:] Verb/verb particle combinations that take an NP
complement select this tree family.  Both the verb and the verb particle
are anchors of the trees. Particle movement has been taken as the
diagnostic to distinguish verb particle constructions from intransitives
with adjoined PP's.  If the alleged particle is able to undergo particle
movement, in other words appear both before and after the direct object,
then it is judged to be a particle.  Items that do not undergo particle
movement are taken to be prepositions.  In many, but not all, of the verb
particle cases, there is also an alternate prepositional meaning in which
the lexical item did not move.  (e.g. {\it He looked up the number (in the
phonebook).  He looked the number up. Srini looked up the road (for
Purnima's car).  $\ast$He looked the road up.})  There are 548 verb/verb
particle combinations.

\item[Examples:] {\it blow off}, {\it make up}, {\it pick out} \\
{\it He blew off his linguistics class for the third time .} \\
{\it He blew his linguistics class off for the third time .} \\
{\it The dyslexic leprechaun made up the syntactic lexicon .} \\
{\it The dyslexic leprechaun made the syntactic lexicon up .} \\
{\it I would like to pick out a new computer .} \\
{\it I would like to pick a new computer out .} 

\item[Declarative tree:]  See Figure~\ref{nx0Vplnx1-tree}.

\begin{figure}[htb]
\centering
\begin{tabular}{ccc}
\psfig{figure=ps/verb-class-files/alphanx0Vplnx1.ps,height=3.4cm} &
\hspace{1.0in}&
\psfig{figure=ps/verb-class-files/alphanx0Vnx1pl.ps,height=3.4cm} \\
(a)&&(b)
\end{tabular}
\caption{Declarative Transitive Verb Particle Tree: $\alpha$nx0Vplnx1~(a) and
$\alpha$nx0Vnx1pl~(b)}
\label{nx0Vplnx1-tree}
\end{figure}

\item[Other available trees:] wh-moved subject with particle before the NP,
wh-moved subject with particle after the NP, wh-moved object, subject
relative clause with particle before the NP with overt and covert extracted
{\it wh}-NP's, subject relative clause with particle after the NP with
overt and covert extracted {\it wh}-NP's, object relative clause with overt
and covert extracted {\it wh}-NP's, adjunct (gap-less) relative clause with
particle before the NP with covert extracted {\it wh}-NP/with PP
pied-piping, adjunct (gap-less) relative clause with particle after the NP
with covert extracted {\it wh}-NP/with PP pied-piping, imperative with
particle before the NP, imperative with particle after the NP, determiner
gerund with particle before the NP, NP gerund with particle before the NP,
NP gerund with particle after the NP, passive with {\it by} phrase, passive
without {\it by} phrase, passive with wh-moved subject and {\it by} phrase,
passive with wh-moved subject and no {\it by} phrase, passive with wh-moved
object out of the {\it by} phrase, passive with wh-moved {\it by} phrase,
passive with relative clause on subject and {\it by} phrase with overt and
covert extracted {\it wh}-NP's, passive with relative clause on subject and
no {\it by} phrase with overt and covert extracted {\it wh}-NP's, passive
with relative clause on object of the {\it by} phrase with overt and covert
extracted {\it wh}-NP's/with PP pied-piping, passive with/without {\it
by}-phrase with adjunct (gap-less) relative clause with covert extracted
{\it wh}-NP/with PP pied-piping, gerund passive with {\it by} phrase,
gerund passive without {\it by} phrase, PRO subject with verb particle
before and after the NP, passive with PRO subject with and without the {\it
by} phrase, NP gerund with PRO subject with verb particle before and after
the NP, NP gerund passive with PRO subject with and without the {\it by}
phrase.

\end{description}

\section{Ditransitive Verb Particle: Tnx0Vplnx2nx1}\index{verbs,particle,ditransitive}
\label{nx0Vplnx1nx2}

\begin{description}

\item[Description:] Verb/verb particle combinations that select this tree
family take 2 NP complements.  Both the verb and the verb particle anchor
the trees, and the verb particle can occur before, between, or after the
noun phrases.  Perhaps because of the complexity of the sentence, these
verbs do not seem to have passive alternations ({\it $\ast$A new bank
account was opened up Michelle by me}).  There are 4 verb/verb particle
combinations that select this tree family.  The exhaustive list is given in
the examples.

\item[Examples:] {\it dish out}, {\it open up}, {\it pay off}, {\it rustle up}
\\
{\it I opened up Michelle a new bank account .} \\
{\it I opened Michelle up a new bank account .} \\
{\it I opened Michelle a new bank account up .}


\item[Declarative tree:]  See Figure~\ref{nx0Vplnx2nx1-tree}.

\begin{figure}[htb]
\centering
\begin{tabular}{ccc}
\psfig{figure=ps/verb-class-files/alphanx0Vplnx2nx1.ps,height=3.0cm} &
\psfig{figure=ps/verb-class-files/alphanx0Vnx2plnx1.ps,height=3.0cm} &
\psfig{figure=ps/verb-class-files/alphanx0Vnx2nx1pl.ps,height=3.0cm} \\
(a) & (b)  & (c)
\end{tabular}
\caption{Declarative Ditransitive Verb Particle Tree: $\alpha$nx0Vplnx2nx1~(a),
$\alpha$nx0Vnx2plnx1~(b) and $\alpha$nx0Vnx2nx1pl~(c)}
\label{nx0Vplnx2nx1-tree}
\end{figure}

\item[Other available trees:] wh-moved subject with particle before the
NP's, wh-moved subject with particle between the NP's, wh-moved subject
with particle after the NP's, wh-moved indirect object with particle before
the NP's, wh-moved indirect object with particle after the NP's, wh-moved
direct object with particle before the NP's, wh-moved direct object with
particle between the NP's, subject relative clause with particle before the
NP's with overt and covert extracted {\it wh}-NP's, subject relative clause
with particle between the NP's with overt and covert extracted {\it
wh}-NP's, subject relative clause with particle after the NP's with overt
and covert extracted {\it wh}-NP's, indirect object relative clause with
particle before the NP's with overt and covert extracted {\it wh}-NP's,
indirect object relative clause with particle after the NP's with overt and
covert extracted {\it wh}-NP's, direct object relative clause with particle
before the NP's with overt and covert extracted {\it wh}-NP's, direct
object relative clause with particle between the NP's with and without
comp, adjunct (gap-less) relative clause with covert extracted {\it
wh}-NP/with PP pied-piping, imperative with particle before the NP's,
imperative with particle between the NP's, imperative with particle after
the NP's, determiner gerund with particle before the NP's, NP gerund with
particle before the NP's, NP gerund with particle between the NP's, NP
gerund with particle after the NP's, PRO subject with particles before
NP's, PRO subject with particles between the NP's, PRO subject with
particles after the NP's, NP gerund with PRO subject with particles before
NP's, NP gerund with PRO subject with particles between the NP's, NP gerund
with PRO subject with particles after the NP's,

\end{description}





\section{Intransitive with PP: Tnx0Vpnx1}\index{verbs,intransitive with PP}
\label{nx0Vpnx1-family}
\begin{description}

\item[Description:]  The verbs that select this tree family are not strictly 
intransitive, in that they {\bf must} be followed by a prepositional
phrase.  Verbs that are intransitive and simply {\bf can} be followed
by a prepositional phrase do not select this family, but instead have
the PP adjoin onto the intransitive sentence.  Accordingly, there
should be no verbs in both this class and the intransitive tree family
(see section~\ref{nx0V-family}).  The prepositional phrase is not
restricted to being headed by any particular lexical item.  The PP is
expanded to facilitate extraction of the PP-internal NP.  Note that
these are not transitive verb particles (see
section~\ref{nx0Vplnx1-family}), since the head of the PP does not
move.  22 verbs select this tree family.

\item[Examples:] {\it grab}, {\it venture} {\it slog} \\
{\it Seth grabbed for the brass ring.} \\
{\it Jones ventured into the cave.}\\
{\it The soldiers slogged grudgingly through the mud.}

\item[Declarative tree:]  See Figure~\ref{nx0Vpnx1-tree}.

\begin{figure}[htb]
\centering
\begin{tabular}{c}
\psfig{figure=ps/verb-class-files/alphanx0Vpnx1.ps,height=4.0in}
\end{tabular}
\caption{Declarative Intransitive with PP Tree:  $\alpha$nx0Vpnx1}
\label{nx0Vpnx1-tree}
\end{figure}

\item[Other available trees:] wh-moved subject, wh-moved object of the PP,
wh-moved PP, subject relative clause with overt and covert extracted {\it
wh}-NP's, adjunct (gap-less) relative clause with covert extracted {\it
wh}-NP/with PP pied-piping, object of the PP relative clause with overt and
covert extracted {\it wh}-NP's/with PP pied-piping, imperative, determiner
gerund, NP gerund, passive with {\it by} phrase, passive without {\it by}
phrase, passive with wh-moved subject and {\it by} phrase, passive with
wh-moved subject and no {\it by} phrase, passive with wh-moved {\it by}
phrase, passive with relative clause on subject and {\it by} phrase with
overt and covert extracted {\it wh}-NP's, passive with relative clause on
subject and no {\it by} phrase with overt and covert extracted {\it
wh}-NP's, passive with relative clause on object of the {\it by} phrase
with overt and covert extracted {\it wh}-NP's/with PP pied-piping, passive
with/without {\it by}-phrase with adjunct (gap-less) relative clause with
covert extracted {\it wh}-NP/with PP pied-piping, gerund passive with {\it
by} phrase, gerund passive without {\it by} phrase, PRO subject, passive
with PRO subject with and without the {\it by} phrase, NP gerund with PRO
subject, NP gerund passive with PRO subject with and without the {\it by}
phrase.

\end{description}

\section{Multiple anchor PP complement: Tnx0VPnx1}\label{verbs,prepositional complement} 
\label{nx0VPnx1-family}

\begin{description}

\item[Description:] This tree family is selected by multiple anchor
verb/preposition pairs which together have a non-compositional
interpretation.  For example, {\it think of} has the non-compositional
interpretation involving the inception of a notion or mental entity in
addition to the interpretation in which the agent is thinking about
someone or something.  To allow adverbs to appear between the verb and
the preposition, the trees contain an extra VP level.  206
verb/preposition pairs select this tree family.

\item[Examples:] {\it think of}, {\it believe in}, {\it depend on} \\
{\it Calvin thought of a new idea .}\\
{\it Hobbes believes in sleeping all day .}\\
{\it Bill depends on drinking coffee for stimulation .}\\

\item[Declarative tree:] See Figure~\ref{nx0VPnx1-tree}.

\begin{figure}[htb]
\centering
\begin{tabular}{c}
\psfig{figure=ps/verb-class-files/alphanx0VPnx1.ps,height=4.8cm}
\end{tabular}
\caption{Declarative PP Complement Tree:  $\alpha$nx0VPnx1}
\label{nx0VPnx1-tree}
\end{figure}

\item[Other available trees:] wh-moved subject, wh-moved object, subject
relative clause with overt and covert extracted {\it wh}-NP's, adjunct
(gap-less) relative clause with covert extracted {\it wh}-NP/with PP
pied-piping, object relative clause with overt and covert extracted {\it
wh}-NP's, imperative, determiner gerund, NP gerund, passive with {\it by}
phrase, passive without {\it by} phrase, passive with wh-moved subject and
{\it by} phrase, passive with wh-moved subject and no {\it by} phrase,
passive with wh-moved object out of the {\it by} phrase, passive with
wh-moved {\it by} phrase, passive with relative clause on subject and {\it
by} phrase with overt and covert extracted {\it wh}-NP's, passive with
relative clause on subject and no {\it by} phrase with overt and covert
extracted {\it wh}-NP's, passive with relative clause on object on the {\it
by} phrase with overt and covert extracted {\it wh}-NP's/with PP
pied-piping, passive with/without {\it by}-phrase with adjunct (gap-less)
relative clause with covert extracted {\it wh}-NP/with PP pied-piping,
gerund passive with {\it by} phrase, gerund passive without {\it by}
phrase.  In addition, two other trees that allow transitive verbs to
function as adjectives (e.g. {\it the thought of idea}) are also in the
family, PRO subject, passive with PRO subject with and without the {\it by}
phrase, NP gerund with PRO subject, NP gerund passive with PRO subject with
and without the {\it by} phrase.

\end{description}


\section{Sentential Complement: Tnx0Vs1}\label{verbs,sentential complement}
\label{nx0Vs1-family}

\begin{description}
  
\item[Description:] This tree family is selected by verbs that take just a
sentential complement.  The sentential complement may be of type
infinitive, indicative, or small clause (see Chapter~\ref{small-clauses}).
The type of clause is specified by each individual verb in its syntactic
lexicon entry, and a given verb may select more than one type of sentential
complement.  The declarative tree, and many other trees in this family, are
auxiliary trees, as opposed to the more common initial trees.  These
auxiliary trees adjoin onto an S node in an existing tree of the type
specified by the sentential complement.  This is the mechanism by which
TAGs are able to maintain long-distance dependencies (see
Chapter~\ref{extraction}), even over multiple embeddings (e.g. {\it What
did Bill think that John said?}). 353 verbs select this tree family.

\item[Examples:]  {\it consider}, {\it think} \\
{\it Dania considered the algorithm unworkable .}\\
{\it Srini thought that the program was working .} \\


\item[Declarative tree:]  See Figure~\ref{nx0Vs1-tree}.

\begin{figure}[htb]
\centering
\begin{tabular}{c}
\psfig{figure=ps/verb-class-files/betanx0Vs1.ps,height=3.4cm}
\end{tabular}
\caption{Declarative Sentential Complement Tree:  $\beta$nx0Vs1}
\label{nx0Vs1-tree}
\end{figure}

\item[Other available trees:] wh-moved subject, wh-moved sentential
complement, subject relative clause with overt and covert extracted {\it
wh}-NP's, adjunct (gap-less) relative clause with covert extracted {\it
wh}-NP/with PP pied-piping, imperative, determiner gerund, NP gerund,
parenthetical reporting clause trees, PRO subject, NP gerund with PRO
subject.

\end{description}




\section{Intransitive with Adjective: Tnx0Vax1}\index{verbs,intransitive with adjective}
\label{nx0Vax1-family}

\begin{description}

\item[Description:] The verbs that select this tree family take an
adjective as a complement.  The adjective may be regular, comparative, or
superlative.  It may also be formed from the special class of adjectives
derived from the transitive verbs (e.g. {\it agitated, broken}).  See
section~\ref{nx0Vnx1-family}).  Unlike the Intransitive with PP verbs (see
section~\ref{nx0Vpnx1-family}), some of these verbs may also occur as bare
intransitives as well.  This distinction is drawn because adjectives do not
normally adjoin onto sentences, as prepositional phrases do.  Other
intransitive verbs can only occur with the adjective, and these select only
this family.  The verb class is also distinguished from the adjective small
clauses (see section~\ref{nx0Ax1-family}) because these verbs are not
raising verbs.  34 verbs select this tree family.

\item[Examples:] {\it become}, {\it grow}, {\it smell} \\
{\it The greenhouse became hotter .} \\
{\it The plants grew tall and strong .} \\
{\it The flowers smelled wonderful .}

\item[Declarative tree:]  See Figure~\ref{nx0Vax1-tree}.

\begin{figure}[htb]
\centering
\begin{tabular}{c}
\psfig{figure=ps/verb-class-files/alphanx0Vax1.ps,height=3.4cm}
\end{tabular}
\caption{Declarative Intransitive with Adjective Tree:  $\alpha$nx0Vax1}
\label{nx0Vax1-tree}
\end{figure}

\item[Other available trees:] wh-moved subject, wh-moved adjective ({\it
how}), subject relative clause with overt and covert extracted {\it
wh}-NP's, adjunct (gap-less) relative clause with covert extracted {\it
wh}-NP/with PP pied-piping, imperative, NP gerund, PRO subject, NP gerund
with PRO subject.

\end{description}




\section{Transitive Sentential Subject:  Ts0Vnx1}\index{verbs,sentential subject}
\label{s0Vnx1-family}

\begin{description}

\item[Description:] The verbs that select this tree family all take
sentential subjects, and are often referred to as `psych' verbs, since they
all refer to some psychological state of mind.  The sentential subject can
be indicative (complementizer required) or infinitive (complementizer
optional).  98 verbs that select this tree family.

\item[Examples:] {\it delight}, {\it impress}, {\it surprise} \\
{\it that the tea had rosehips in it delighted Christy .} \\
{\it to even attempt a marathon impressed Dania .} \\
{\it For Jim to have walked the dogs surprised Beth .}

\item[Declarative tree:]  See Figure~\ref{s0Vnx1-tree}.

\begin{figure}[htb]
\centering
\begin{tabular}{c}
\psfig{figure=ps/verb-class-files/alphas0Vnx1.ps,height=3.4cm}
\end{tabular}
\caption{Declarative Sentential Subject Tree:  $\alpha$s0Vnx1}
\label{s0Vnx1-tree}
\end{figure}

\item[Other available trees:] wh-moved subject, adjunct (gap-less) relative
clause with covert extracted {\it wh}-NP/with PP pied-piping.

\end{description}

\section{Light Verbs: Tnx0lVN1}\index{verbs, light}
\label{nx0lVN1-family}

\begin{description}
  
\item[Description:] The verb/noun pairs that select this tree families are
pairs in which the interpretation is non-compositional and the noun
contributes argument structure to the predicate (e.g. {\it The man took a
walk.} vs. {\it The man took a radio}).  The verb and the noun occur
together in the syntactic database, and both anchor the trees.  The verbs
in the light verb constructions are {\it do}, {\it give}, {\it have}, {\it
make} and {\it take}.  The noun following the light verb is (usually) in a
bare infinitive form ({\it have a good cry}) and usually occurs with {\it
a(n)}.  However, we include deverbal nominals ({\it take a bath}, {\it give
a demonstration}) as well.  Constructions with nouns that do not contribute
an argument structure ({\it have a cigarette}, {\it give} NP {\it a black
eye}) are excluded.  In addition to semantic considerations of light verbs,
they differ syntactically from Transitive verbs
(section~\ref{nx0Vnx1-family}) as well in that the noun in the light verb
construction does not extract.  Some of the verb-noun anchors for this
family, like {\it take aim} and {\it take hold} disallow determiners, while
others require particular determiners.  For example, {\it have think} must
be indefinite and singular, as attested by the ungrammaticality of *{\it
John had the think/some thinks}.  Another anchor, {\it take leave} can
occur either bare or with a possesive pronoun (e.g., {\it John took his
leave}, but not *{\it John took the leave}).  This is accomplished through
feature specification on the lexical entries.  There are 242 verb/noun
pairs that select the light verb tree.

\item[Examples:] {\it give groan}, {\it have discussion}, {\it make comment} \\
{\it The audience gave a collective groan .} \\
{\it We had a big discussion about closing the libraries .} \\
{\it The professors made comments on the paper .}

\item[Declarative tree:]  See Figure~\ref{nx0lVN1-tree}.

\begin{figure}[htb]
\centering
\begin{tabular}{c}
\psfig{figure=ps/verb-class-files/alphanx0lVN1.ps,height=4.0cm}\\
\end{tabular}
\caption{Declarative Light Verb Tree: $\alpha$nx0lVN1}
\label{nx0lVN1-tree}
\end{figure}

\item[Other available trees:] wh-moved subject, subject relative clause
with overt and covert extracted {\it wh}-NP's, adjunct (gap-less) relative
clause with covert extracted {\it wh}-NP/with PP pied-piping, imperative,
determiner gerund, NP gerund, PRO subject, NP gerund with PRO subject.

\end{description}

\section{Ditransitive Light Verbs with PP Shift: Tnx0lVN1Pnx2}\index{verbs,ditransitive light verbs with PP shift}
\label{nx0lVN1Pnx2-family}

\begin{description}

\item[Description:] The verb/noun pairs that select this tree family are
pairs in which the interpretation is non-compositional and the noun
contributes argument structure to the predicate (e.g. {\it Dania made Srini
a cake.} vs.  {\it Dania made Srini a loan}).  The verb and the noun occur
together in the syntactic database, and both anchor the trees.  The verbs
in these light verb constructions are {\it give} and {\it make}.  The noun
following the light verb is (usually) a bare infinitive form (e.g. {\it
make a promise to Anoop}).  However, we include deverbal nominals
(e.g. {\it make a payment to Anoop}) as well.  Constructions with nouns
that do not contribute an argument structure are excluded.  In addition to
semantic considerations of light verbs, they differ syntactically from the
Ditransitive with PP Shift verbs (see section~\ref{nx0Vnx1Pnx2-family}) as
well in that the noun in the light verb construction does not extract.
Also, passivization is severely restricted.  Special determiner requirments
and restrictions are handled in the same manner as for the Tnx0lVN1 family.
There are 18 verb/noun pairs that select this family.

\item[Examples:] {\it give look}, {\it give wave}, {\it make promise} \\
{\it Dania gave Carl a murderous look .} \\
{\it Amanda gave us a little wave as she left .} \\
{\it Dania made Doug a promise .} 

\item[Declarative tree:]  See Figure~\ref{nx0lVN1Pnx2-tree}.

\begin{figure}[htb]
\centering
\mbox{}
\begin{tabular}{cc}
\psfig{figure=ps/verb-class-files/alphanx0lVN1Pnx2.ps,height=5.1cm}
\psfig{figure=ps/verb-class-files/alphanx0lVnx2N1.ps,height=5.1cm} \\
(a) & (b) \vspace*{1.2cm}\\
\end{tabular}
\caption{Declarative Light Verbs with PP Tree: $\alpha$nx0lVN1Pnx2~(a),
$\alpha$nx0lVnx2N1~(b)}
\label{nx0lVN1Pnx2-tree}
\end{figure}

\item[Other available trees:] {\bf Non-shifted:} wh-moved subject, wh-moved
indirect object, subject relative clause with overt and covert extracted
{\it wh}-NP's, adjunct (gap-less) relative clause with covert extracted
{\it wh}-NP/with PP pied-piping, indirect object relative clause with overt
and covert extracted {\it wh}-NP's/with PP pied-piping, imperative, NP
gerund, passive with {\it by} phrase, passive with {\it by}-phrase with
adjunct (gap-less) relative clause with covert extracted {\it wh}-NP/with
PP pied-piping, gerund passive with {\it by} phrase, gerund passive without
{\it by} phrase, PRO subject, NP gerund with PRO subject, passive with PRO
subject with {\it by} phrase, NP gerund passive with and without the {\it
by} phrase\\ {\bf Shifted:} wh-moved subject, wh-moved object of PP,
wh-moved PP, subject relative clause with overt and covert extracted {\it
wh}-NP's, object of PP relative clause with overt and covert extracted {\it
wh}-NP's/with PP pied-piping, imperative, determiner gerund, NP gerund,
passive with {\it by} phrase with adjunct (gap-less) relative clause with
covert extracted {\it wh}-NP/with PP pied-piping, gerund passive with {\it
by} phrase, gerund passive without {\it by} phrase, PRO subject, NP gerund
with PRO subject
\end{description}




\section{NP It-Cleft: TItVnx1s2}
\label{ItVnx1s2-family}

\begin{description}

\item[Description:] This tree family is selected by {\it be} as the main
verb and {\it it} as the subject. Together these two items serve as a
multi-component anchor for the tree family.  This tree family is used for
it-clefts in which the clefted element is an NP and there are no gaps in
the clause which follows the NP.  The NP is interpreted as an adjunct of
the following clause. See Chapter~\ref{it-clefts} for additional
discussion.

\item[Examples:] {\it it be} \\
{\it it was yesterday that we had the meeting .}

\item[Declarative tree:]  See Figure~\ref{ItVnx1s2-tree}.

\begin{figure}[htb]
\centering
\begin{tabular}{c}
\psfig{figure=ps/verb-class-files/alphaItVnx1s2.ps,height=4.9cm}
\end{tabular}
\caption{Declarative NP It-Cleft Tree:  $\alpha$ItVnx1s2}
\label{ItVnx1s2-tree}
\end{figure}

\item[Other available trees:] inverted question, wh-moved object with {\it
be} inverted, wh-moved object with {\it be} not inverted, adjunct
(gap-less) relative clause with covert extracted {\it wh}-NP/with PP
pied-piping.

\end{description}

\section{PP It-Cleft: TItVpnx1s2}
\label{ItVpnx1s2-family}

\begin{description}
  
\item[Description:] This tree family is selected by {\it be} as the main
verb and {\it it} as the subject. Together these two items serve as a
multi-component anchor for the tree family.  This tree family is used for
it-clefts in which the clefted element is a PP and there are no gaps in the
clause which follows the PP.  The PP is interpreted as an adjunct of the
following clause. See Chapter~\ref{it-clefts} for additional discussion.

\item[Examples:] {\it it be} \\
{\it it was at Kent State that the police shot all those students .}

\item[Declarative tree:]  See Figure~\ref{ItVpnx1s2-tree}.

\begin{figure}[htb]
\centering
\begin{tabular}{c}
\psfig{figure=ps/verb-class-files/alphaItVpnx1s2.ps,height=5.0cm}
\end{tabular}
\caption{Declarative PP It-Cleft Tree:  $\alpha$ItVpnx1s2}
\label{ItVpnx1s2-tree}
\end{figure}

\item[Other available trees:] inverted question, wh-moved prepositional
phrase with {\it be} inverted, wh-moved prepositional phrase with {\it be}
not inverted, adjunct (gap-less) relative clause with covert extracted {\it
wh}-NP/with PP pied-piping.

\end{description}

\section{Adverb It-Cleft: TItVad1s2}
\label{ItVad1s2-family}

\begin{description}

\item[Description:] This tree family is selected by {\it be} as the main
verb and {\it it} as the subject. Together these two items serve as a
multi-component anchor for the tree family.  This tree family is used for
it-clefts in which the clefted element is an adverb and there are no gaps
in the clause which follows the adverb.  The adverb is interpreted as an
adjunct of the following clause. See Chapter~\ref{it-clefts} for additional
discussion.

\item[Examples:] {\it it be} \\
{\it it was reluctantly that Dania agreed to do the tech report .}

\item[Declarative tree:]  See Figure~\ref{ItVad1s2-tree}.

\begin{figure}[htb]
\centering
\begin{tabular}{c}
\psfig{figure=ps/verb-class-files/alphaItVad1s2.ps,height=4.9cm}
\end{tabular}
\caption{Declarative Adverb It-Cleft Tree:  $\alpha$ItVad1s2}
\label{ItVad1s2-tree}
\end{figure}

\item[Other available trees:] inverted question, wh-moved adverb {\it how}
with {\it be} inverted, wh-moved adverb {\it how} with {\it be} not
inverted, adjunct (gap-less) relative clause with covert extracted {\it
wh}-NP/with PP pied-piping.

\end{description}



\section{Adjective Small Clause Tree: Tnx0Ax1}\index{verbs,small-clause}
\label{nx0Ax1-family}

\begin{description}
  
\item[Description:] These trees are not anchored by verbs, but by
adjectives.  They are explained in much greater detail in the section on
small clauses (see section~\ref{sm-clause-xtag-analysis}).  This section is
presented here for completeness.  4150 adjectives select this tree family.

\item[Examples:] {\it addictive}, {\it dangerous}, {\it wary}\\
{\it cigarettes are addictive .} \\
{\it smoking cigarettes is dangerous .} \\
{\it John seems wary of the Surgeon General's warnings .}

\item[Declarative tree:]  See Figure~\ref{nx0Ax1-tree}.

\begin{figure}[htb]
\centering
\begin{tabular}{c}
\psfig{figure=ps/verb-class-files/alphanx0Ax1.ps,height=4.0cm}
\end{tabular}
\caption{Declarative Adjective Small Clause Tree:  $\alpha$nx0Ax1}
\label{nx0Ax1-tree}
\end{figure}

\item[Other available trees:] wh-moved subject, wh-moved adjective {\it
how}, relative clause on subject with overt and covert extracted {\it
wh}-NP's, imperative, NP gerund, adjunct (gap-less) relative clause with
covert extracted {\it wh}-NP/with PP pied-piping, PRO subject, NP gerund
with PRO subject.

\end{description}

\section{Adjective Small Clause with Sentential Complement: Tnx0A1s1}
\label{nx0A1s1-family}

\begin{description}
  
\item[Description:] This tree family is selected by adjectives that take
sentential complements.  The sentential complements can be indicative or
infinitive.  Note that these trees are anchored by adjectives, not verbs.
Small clauses are explained in much greater detail in
section~\ref{sm-clause-xtag-analysis}.  This section is presented here for
completeness.  673 adjectives select this tree family.

\item[Examples:] {\it able}, {\it curious}, {\it disappointed} \\
{\it Christy was able to find the problem .} \\
{\it Christy was curious whether the new analysis was working .} \\
{\it Christy was sad that the old analysis failed .} 

\item[Declarative tree:]  See Figure~\ref{nx0A1s1-tree}.

\begin{figure}[htb]
\centering
\begin{tabular}{c}
\psfig{figure=ps/verb-class-files/alphanx0A1s1.ps,height=4.7cm}
\end{tabular}
\caption{Declarative  Adjective Small Clause with Sentential Complement Tree:  $\alpha$nx0A1s1}
\label{nx0A1s1-tree}
\end{figure}

\item[Other available trees:] wh-moved subject, wh-moved adjective {\it
how}, relative clause on subject with overt and covert extracted {\it
wh}-NP's, imperative, NP gerund, adjunct (gap-less) relative clause with
covert extracted {\it wh}-NP/with PP pied-piping, PRO subject, NP gerund
with PRO subject.

\end{description}


\section{Adjective Small Clause with Sentential Subject: Ts0Ax1}
\label{s0Ax1-family}

\begin{description}
  
\item[Description:] This tree family is selected by adjectives that take
sentential subjects.  The sentential subjects can be indicative or
infinitive.  Note that these trees are anchored by adjectives, not verbs.
Most adjectives that take the Adjective Small Clause tree family (see
section~\ref{nx0Ax1-family}) take this family as well.\footnote{No great
attempt has been made to go through and decide which adjectives should
actually take this family and which should not.}  Small clauses are
explained in much greater detail in section~\ref{sm-clause-xtag-analysis}.
This section is presented here for completeness. 4,007 adjectives select
this tree family.

\item[Examples:] {\it decadent}, {\it incredible}, {\it uncertain} \\
{\it to eat raspberry chocolate truffle ice cream is decadent .} \\
{\it that Carl could eat a large bowl of it is incredible .} \\
{\it whether he will actually survive the experience is uncertain .}

\item[Declarative tree:]  See Figure~\ref{s0Ax1-tree}.

\begin{figure}[htb]
\centering
\begin{tabular}{c}
\psfig{figure=ps/verb-class-files/alphas0Ax1.ps,height=4.0cm}
\end{tabular}
\caption{Declarative Adjective Small Clause with Sentential Subject Tree:  $\alpha$s0Ax1}
\label{s0Ax1-tree}
\end{figure}

\item[Other available trees:] wh-moved subject, wh-moved adjective, adjunct
(gap-less) relative clause with covert extracted {\it wh}-NP/with PP
pied-piping.

\end{description}



\section{Equative {\it BE}: Tnx0BEnx1}
\label{nx0BEnx1-family}

\begin{description}

\item[Description:] This tree family is selected only by the verb {\it be}.
It is distinguished from the predicative NP's (see
section~\ref{nx0N1-family}) in that two NP's are equated, and hence
interchangeable (see Chapter~\ref{small-clauses} for more discussion on the
English copula and predicative sentences).  The XTAG analysis for equative
{\it be} is explained in greater detail in
section~\ref{equative-be-xtag-analysis}.

\item[Examples:] {\it be} \\
{\it That man is my uncle.}

\item[Declarative tree:]  See Figure~\ref{nx0BEnx1-tree}.

\begin{figure}[htb]
\centering
\begin{tabular}{c}
\psfig{figure=ps/verb-class-files/alphanx0BEnx1.ps,height=5.1cm}
\end{tabular}
\caption{Declarative Equative {\it BE} Tree:  $\alpha$nx0BEnx1}
\label{nx0BEnx1-tree}
\end{figure}

\item[Other available trees:] inverted-question.

\end{description}




\section{NP Small Clause: Tnx0N1}
\label{nx0N1-family}

\begin{description}
  
\item[Description:] The trees in this tree family are not anchored by
verbs, but by nouns.  Small clauses are explained in much greater detail in
section~\ref{sm-clause-xtag-analysis}.  This section is presented here for
completeness.  5,476 nouns select this tree family.

\item[Examples:] {\it author}, {\it chair}, {\it dish} \\
{\it Dania is an author .} \\
{\it that blue, warped-looking thing is a chair .} \\
{\it those broken pieces were dishes .}

\item[Declarative tree:]  See Figure~\ref{nx0N1-tree}.

\begin{figure}[htb]
\centering
\begin{tabular}{c}
\psfig{figure=ps/verb-class-files/alphanx0N1.ps,height=4.8cm}
\end{tabular}
\caption{Declarative NP Small Clause Trees: $\alpha$nx0N1}
\label{nx0N1-tree}
\end{figure}

\item[Other available trees:] wh-moved subject, wh-moved object, relative
clause on subject with overt and covert extracted {\it wh}-NP's,
imperative, NP gerund, adjunct (gap-less) relative clause with covert
extracted {\it wh}-NP/with PP pied-piping, PRO subject, NP gerund with PRO
subject.

\end{description}



%%  No nouns select this tree family.  What is up?
%%\section{NP Small Clauses with Sentential Complement: Tnx0N1s2, Tnx0N1s2}
%%\label{nx0N1s2-family}
%%
%%\begin{description}
%%
%%\item[Description:]
%%
%%\item[Examples:]
%%
%%\item[Declarative tree:]  See Figure~\ref{nx0N1s2-tree}.
%%
%%\begin{figure}[htb]
%%\centering
%%\begin{tabular}{cc}
%%\psfig{figure=ps/verb-class-files/betanx0N1s2.ps,height=4.0cm} &
%%psfig{figure=ps/verb-class-files/betanx0dxN1s2.ps,height=4.0cm} \\
%%$\beta$nx0N1s2 &$\beta$nx0dxN1s2 
%%\end{tabular}
%%\caption{Declarative NP Small Clauses with Sentential Complement Tree}
%%\label{nx0N1s2-tree}
%%\end{figure}
%%
%%\item[Other available trees:]  Wh-moved subject, Wh-moved object, relative clause on object, imperative.
%%
%%\end{description}



\section{NP Small Clause with Sentential Complement: Tnx0N1s1}
\label{nx0N1s1-family}

\begin{description}

\item[Description:] This tree family is selected by the small group of
nouns that take sentential complements by themselves (see
section~\ref{NPA}).  The sentential complements can be indicative or
infinitive, depending on the noun.  Small clauses in general are explained
in much greater detail in the section~\ref{sm-clause-xtag-analysis}.  This
section is presented here for completeness.  144 nouns select this family.

\item[Examples:] {\it admission}, {\it claim}, {\it vow} \\
{\it The affidavits are admissions that they killed the sheep .} \\
{\it there is always the claim that they were insane .} \\
{\it this is his vow to fight the charges .}

\item[Declarative tree:]  See Figure~\ref{nx0N1s1-tree}.

\begin{figure}[htb]
\centering
\begin{tabular}{c}
\psfig{figure=ps/verb-class-files/alphanx0N1s1.ps,height=4.0cm} 
\end{tabular}
\caption{Declarative NP with Sentential Complement Small Clause Tree:
$\alpha$nx0N1s1}
\label{nx0N1s1-tree}
\end{figure}

\item[Other available trees:] wh-moved subject, wh-moved object, relative
clause on subject with overt and covert extracted {\it wh}-NP's,
imperative, NP gerund, adjunct (gap-less) relative clause with covert
extracted {\it wh}-NP/with PP pied-piping, PRO subject, NP gerund with PRO
subject.

\end{description}



\section{NP Small Clause with Sentential Subject:  Ts0N1}
\label{s0N1-family}

\begin{description}

\item[Description:] This tree family is selected by nouns that take
sentential subjects.  The sentential subjects can be indicative or
infinitive.  Note that these trees are anchored by nouns, not verbs.  Most
nouns that take the NP Small Clause tree family (see
section~\ref{nx0N1-family}) take this family as well.\footnote{No great
attempt has been made to go through and decide which nouns should actually
take this family and which should not.}  Small clauses are explained in
much greater detail in section~\ref{sm-clause-xtag-analysis}.  This section
is presented here for completeness.  5,466 nouns select this tree family.

\item[Examples:] {\it dilemma}, {\it insanity}, {\it tragedy} \\
{\it whether to keep the job he hates is a dilemma .} \\
{\it to invest all of your money in worms is insanity .} \\
{\it that the worms died is a tragedy .}

\item[Declarative tree:]  See Figure~\ref{s0N1-tree}.

\begin{figure}[htb]
\centering
\begin{tabular}{c}
\psfig{figure=ps/verb-class-files/alphas0N1.ps,height=4.0cm} 
\end{tabular}
\caption{Declarative NP Small Clause with Sentential Subject Tree: $\alpha$s0N1}
\label{s0N1-tree}
\end{figure}

\item[Other available trees:] wh-moved subject, adjunct (gap-less) relative
clause with covert extracted {\it wh}-NP/with PP pied-piping, PRO subject,
NP gerund with PRO subject.

\end{description}

\section{PP Small Clause: Tnx0Pnx1}
\label{nx0Pnx1-family}

\begin{description}

\item[Description:] This family is selected by prepositions that can occur
in small clause constructions.  For more information on small clause
constructions, see section~\ref{sm-clause-xtag-analysis}.  This section is
presented here for completeness.  39 prepositions select this tree family.

\item[Examples:] {\it around}, {\it in}, {\it underneath} \\
{\it Chris is around the corner .} \\
{\it Trisha is in big trouble .} \\
{\it The dog is underneath the table .}

\item[Declarative tree:]  See Figure~\ref{nx0Pnx1-tree}.

\begin{figure}[htb]
\centering
\begin{tabular}{c}
\psfig{figure=ps/verb-class-files/alphanx0Pnx1.ps,height=4.0cm}
\end{tabular}
\caption{Declarative PP Small Clause  Tree:  $\alpha$nx0Pnx1}
\label{nx0Pnx1-tree}
\end{figure}

\item[Other available trees:] wh-moved subject, wh-moved object of PP,
relative clause on subject with overt and covert extracted {\it wh}-NP's,
relative clause on object of PP with overt and covert extracted {\it
wh}-NP's/with PP pied-piping, imperative, NP gerund, adjunct (gap-less)
relative clause with covert extracted {\it wh}-NP/with PP pied-piping, PRO
subject, NP gerund with PRO subject.

\end{description}





\section{Exhaustive PP Small Clause: Tnx0Px1}
\label{nx0Px1-family}

\begin{description}

\item[Description:] This family is selected by {\bf exhaustive}
prepositions that can occur in small clauses.  Exhaustive prepositions are
prepositions that function as prepositional phrases by themselves.  For
more information on small clause constructions, please see
section~\ref{sm-clause-xtag-analysis}.  The section is included here for
completeness.  33 exhaustive prepositions select this tree family.

\item[Examples:] {\it abroad}, {\it below}, {\it outside} \\
{\it Dr. Joshi is abroad .} \\
{\it The workers are all below .} \\
{\it Clove is outside .}

\item[Declarative tree:]  See Figure~\ref{nx0Px1-tree}.

\begin{figure}[htb]
\centering
\begin{tabular}{c}
\psfig{figure=ps/verb-class-files/alphanx0Px1.ps,height=4.0cm}
\end{tabular}
\caption{Declarative Exhaustive PP Small Clause Tree:  $\alpha$nx0Px1}
\label{nx0Px1-tree}
\end{figure}

\item[Other available trees:] wh-moved subject, wh-moved PP, relative
clause on subject with overt and covert extracted {\it wh}-NP's,
imperative, NP gerund, adjunct (gap-less) relative clause with covert
extracted {\it wh}-NP/with PP pied-piping, PRO subject, NP gerund with PRO
subject.

\end{description}


\section{PP Small Clause with Sentential Subject: Ts0Pnx1}
\label{s0Pnx1-family}

\begin{description}

\item[Description:] This tree family is selected by prepositions that take
sentential subjects.  The sentential subject can be indicative or
infinitive.  Small clauses are explained in much greater detail in
section~\ref{sm-clause-xtag-analysis}.  This section is presented here for
completeness.  39 prepositions select this tree family.

\item[Examples:] {\it beyond}, {\it unlike} \\
{\it that Ken could forget to pay the taxes is beyond belief .} \\
{\it to explain how this happened is outside the scope of this discussion .} \\
{\it for Ken to do something right is unlike him .}


\item[Declarative tree:]  See Figure~\ref{s0Pnx1-tree}.

\begin{figure}[htb]
\centering
\begin{tabular}{c}
\psfig{figure=ps/verb-class-files/alphas0Pnx1.ps,height=4.0cm}
\end{tabular}
\caption{Declarative PP Small Clause with Sentential Subject Tree:  $\alpha$s0Pnx1}
\label{s0Pnx1-tree}
\end{figure}

\item[Other available trees:] wh-moved subject, relative clause on object
of the PP with overt and covert extracted {\it wh}-NP's/with PP
pied-piping, adjunct (gap-less) relative clause with covert extracted {\it
wh}-NP/with PP pied-piping.

\end{description}

\section{Intransitive Sentential Subject:  Ts0V}\index{verbs,sentential subject}
\label{s0V-family}

\begin{description}

\item[Description:] Only the verb {\it matter} selects this tree family.
The sentential subject can be indicative (complementizer required) or
infinitive (complementizer optional).

\item[Examples:] {\it matter} \\
{\it to arrive on time matters considerably .} \\
{\it that Joshi attends the meetings matters to everyone .}

\item[Declarative tree:]  See Figure~\ref{s0V-tree}.

\begin{figure}[htb]
\centering
\begin{tabular}{c}
\psfig{figure=ps/verb-class-files/alphas0V.ps,height=3.0cm}
\end{tabular}
\caption{Declarative Intransitive Sentential Subject Tree:  $\alpha$s0V}
\label{s0V-tree}
\end{figure}

\item[Other available trees:]  wh-moved subject, 
adjunct (gap-less) relative clause with covert extracted {\it wh}-NP/with PP pied-piping.

\end{description}

\section{Sentential Subject with `to' complement:  Ts0Vtonx1}\index{verbs,sentential
subject, PP complement}
\label{s0Vtonx1-family}

\begin{description}

\item[Description:] The verbs that select this tree family are {\it fall},
{\it occur} and {\it leak}.  The sentential subject can be indicative
(complementizer required) or infinitive (complementizer optional).


\item[Examples:]  {\it fall}, {\it occur}, {\it leak}\\
{\it to wash the car fell to the children .} \\
{\it that he should leave occurred to the party crasher .} \\
{\it whether the princess divorced the prince leaked to the press .}

\item[Declarative tree:]  See Figure~\ref{s0Vtonx1-tree}.

\begin{figure}[htb]
\centering
\begin{tabular}{c}
\psfig{figure=ps/verb-class-files/alphas0Vtonx1.ps,height=5.4cm}
\end{tabular}
\caption{Sentential Subject Tree with `to' complement:  $\alpha$s0Vtonx1}
\label{s0Vtonx1-tree}
\end{figure}

\item[Other available trees:] wh-moved subject, adjunct (gap-less) relative
clause with covert extracted {\it wh}-NP/with PP pied-piping.

\end{description}

\section{PP Small Clause, with Adv and Prep anchors: Tnx0ARBPnx1}
\label{nx0ARBPnx1-family}

\begin{description}

\item[Description:] This family is selected by multi-word prepositions that
can occur in small clause constructions.  In particular, this family is
selected by two-word prepositions, where the first word is an adverb, the
second word a preposition.  Both components of the multi-word preposition
are anchors.  For more information on small clause constructions, see
section~\ref{sm-clause-xtag-analysis}.  8 multi-word prepositions select
this tree family.

\item[Examples:] {\it ahead of}, {\it close to} \\
{\it The little girl is ahead of everyone else in the race .} \\
{\it The project is close to completion .} \\

\item[Declarative tree:]  See Figure~\ref{nx0ARBPnx1-tree}.

\begin{figure}[htb]
\centering
\begin{tabular}{c}
\psfig{figure=ps/verb-class-files/alphanx0ARBPnx1.ps,height=4.9cm}
\end{tabular}
\caption{Declarative PP Small Clause tree with two-word preposition, where the 
first word is an adverb, and the second word is a preposition:  $\alpha$nx0ARBPnx1}
\label{nx0ARBPnx1-tree}
\end{figure}

\item[Other available trees:] wh-moved subject, wh-moved object of PP,
relative clause on subject with overt and covert extracted {\it wh}-NP's,
relative clause on object of PP with overt and covert extracted {\it
wh}-NP's, adjunct (gap-less) relative clause with covert extracted {\it
wh}-NP/with PP pied-piping, imperative, NP Gerund, PRO subject, NP gerund
with PRO subject.

\end{description}


\section{PP Small Clause, with Adj and Prep anchors: Tnx0APnx1}
\label{nx0APnx1-family}

\begin{description}

\item[Description:] This family is selected by multi-word prepositions that
can occur in small clause constructions.  In particular, this family is
selected by two-word prepositions, where the first word is an adjective,
the second word a preposition.  Both components of the multi-word
preposition are anchors.  For more information on small clause
constructions, see section~\ref{sm-clause-xtag-analysis}.  7 multi-word
prepositions select this tree family.

\item[Examples:] {\it according to}, {\it void of} \\
{\it The operation we performed was according to standard procedure .} \\
{\it He is void of all feeling .} \\

\item[Declarative tree:]  See Figure~\ref{nx0APnx1-tree}.

\begin{figure}[htb]
\centering
\begin{tabular}{c}
\psfig{figure=ps/verb-class-files/alphanx0APnx1.ps,height=5.3cm}
\end{tabular}
\caption{Declarative PP Small Clause tree with two-word preposition, where the 
first word is an adjective, and the second word is a preposition: $\alpha$nx0APnx1}
\label{nx0APnx1-tree}
\end{figure}

\item[Other available trees:] wh-moved subject, relative clause on subject
with overt and covert extracted {\it wh}-NP's, relative clause on object of
PP with overt and covert extracted {\it wh}-NP's, wh-moved object of PP,
adjunct (gap-less) relative clause with covert extracted {\it wh}-NP/with
PP pied-piping, PRO subject.

\end{description} 


\section{PP Small Clause, with Noun and Prep anchors: Tnx0NPnx1}
\label{nx0NPnx1-family}

\begin{description}

\item[Description:] This family is selected by multi-word prepositions that
can occur in small clause constructions.  In particular, this family is
selected by two-word prepositions, where the first word is a noun, the
second word a preposition.  Both components of the multi-word preposition
are anchors.  For more information on small clause constructions, see
section~\ref{sm-clause-xtag-analysis}. 1 multi-word preposition selects
this tree family.

\item[Examples:] {\it thanks to} \\
{\it The fact that we are here tonight is thanks to the valiant efforts of our 
staff .} \\

\item[Declarative tree:] See Figure~\ref{nx0NPnx1-tree}.

\begin{figure}[htb]
\centering
\begin{tabular}{c}
\psfig{figure=ps/verb-class-files/alphanx0NPnx1.ps,height=5.3cm}
\end{tabular}
\caption{Declarative PP Small Clause tree with two-word preposition, where the
first word is a noun, and the second word is a preposition:  $\alpha$nx0NPnx1}
\label{nx0NPnx1-tree}
\end{figure}

\item[Other available trees:] wh-moved subject, wh-moved object of PP,
relative clause on subject with overt and covert extracted {\it wh}-NP's,
relative clause on object with covert extracted {\it wh}-NP, adjunct
(gap-less) relative clause with covert extracted {\it wh}-NP/with PP
pied-piping, PRO subject.

\end{description}

\section{PP Small Clause, with Prep anchors: Tnx0PPnx1}
\label{nx0PPnx1-family}

\begin{description}

\item[Description:] This family is selected by multi-word prepositions that
can occur in small clause constructions.  In particular, this family is
selected by two-word prepositions, where both words are prepositions. Both
components of the multi-word preposition are anchors.  For more information
on small clause constructions, see section~\ref{sm-clause-xtag-analysis}.
9 multi-word prepositions select this tree family.

\item[Examples:] {\it on to}, {\it inside of} \\
{\it that detective is on to you .} \\
{\it The red box is inside of the blue box .} \\

\item[Declarative tree:] See Figure~\ref{nx0PPnx1-tree}.

\begin{figure}[htb]
\centering
\begin{tabular}{c}
\psfig{figure=ps/verb-class-files/alphanx0PPnx1.ps,height=5.3cm}
\end{tabular}
\caption{Declarative PP Small Clause tree with two-word preposition, where both
words are prepositions:  $\alpha$nx0PPnx1}
\label{nx0PPnx1-tree}
\end{figure}    

\item[Other available trees:] wh-moved subject, wh-moved object of PP,
relative clause on subject with overt and covert extracted {\it wh}-NP's,
relative clause on object of PP with and without comp/with PP pied-piping,
imperative, wh-moved object of PP, adjunct (gap-less) relative clause with
covert extracted {\it wh}-NP/with PP pied-piping, PRO subject, NP gerund
with PRO subject.

\end{description}


\section{PP Small Clause, with Prep and Noun anchors: Tnx0PNaPnx1}
\label{nx0PNaPnx1-family}

\begin{description}

\item[Description:] This family is selected by multi-word prepositions that
can occur in small clause constructions.  In particular, this family is
selected by three-word prepositions.  The first and third words are always
prepositions, and the middle word is a noun.  The noun is marked for null
adjunction since it cannot be modified by noun modifiers.  All three
components of the multi-word preposition are anchors.  For more information
on small clause constructions, see section~\ref{sm-clause-xtag-analysis}.
9 multi-word preposition select this tree family.

\item[Examples:] {\it in back of}, {\it in line with}, {\it on top of} \\
{\it The red plaid box should be in back of the plain black box .} \\
{\it The evidence is in line with my newly concocted theory .} \\
{\it She is on top of the world .} \\
{\it *She is on direct top of the world .} \\

\item[Declarative tree:] See Figure~\ref{nx0PNaPnx1-tree}.

\begin{figure}[htb]
\centering
\begin{tabular}{c}
\psfig{figure=ps/verb-class-files/alphanx0PNaPnx1.ps,height=5.5cm}
\end{tabular}
\caption{Declarative PP Small Clause tree with three-word preposition,
where the middle noun is marked for null adjunction:  $\alpha$nx0PNaPnx1}
\label{nx0PNaPnx1-tree}
\end{figure}

\item[Other available trees:] wh-moved subject, wh-moved object of PP,
relative clause on subject with overt and covert extracted {\it wh}-NP's,
relative clause on object of PP with overt and covert extracted {\it
wh}-NP's/with PP pied-piping, adjunct (gap-less) relative clause with
covert extracted {\it wh}-NP/with PP pied-piping, imperative, NP Gerund,
PRO subject, NP gerund with PRO subject.

\end{description}

\section{PP Small Clause with Sentential Subject, and Adv and Prep anchors: Ts0ARBPnx1}
\label{s0ARBPnx1-family}

\begin{description}

\item[Description:] This tree family is selected by multi-word prepositions
that take sentential subjects. In particular, this family is selected by
two-word prepositions, where the first word is an adverb, the second word a
preposition.  Both components of the multi-word preposition are
anchors. The sentential subject can be indicative or infinitive.  Small
clauses are explained in much greater detail in
section~\ref{sm-clause-xtag-analysis}.  2 prepositions select this tree
family.

\item[Examples:]  {\it due to}, {\it contrary to} \\
{\it that David slept until noon is due to the fact that he never sleeps during
the week .} \\
{\it that Michael's joke was funny is contrary to the usual status of his comic
attempts .} \\

\item[Declarative tree:]  See Figure~\ref{s0ARBPnx1-tree}.
 
\begin{figure}[htb]
\centering
\begin{tabular}{c}
\psfig{figure=ps/verb-class-files/alphas0ARBPnx1.ps,height=5.5cm}
\end{tabular}
\caption{Declarative PP Small Clause with Sentential Subject Tree, with 
two-word preposition, where the first word is an adverb, and the second word is
a preposition:  $\alpha$s0ARBPnx1}
\label{s0ARBPnx1-tree}
\end{figure}

\item[Other available trees:] wh-moved subject, relative clause on object
of the PP with overt and covert extracted {\it wh}-NP's, adjunct (gap-less)
relative clause with covert extracted {\it wh}-NP/with PP pied-piping.

\end{description}

\section{PP Small Clause with Sentential Subject, and Adj and Prep anchors: Ts0APnx1}
\label{s0APnx1-family}

\begin{description}

\item[Description:] This tree family is selected by multi-word prepositions
that take sentential subjects. In particular, this family is selected by
two-word prepositions, where the first word is an adjective, the second
word a preposition.  Both components of the multi-word preposition are
anchors. The sentential subject can be indicative or infinitive.  Small
clauses are explained in much greater detail in
section~\ref{sm-clause-xtag-analysis}.  4 prepositions select this tree
family.

\item[Examples:] {\it devoid of}, {\it according to} \\ 
{\it that he could walk out on her is devoid of all reason .} \\
{\it that the conversation erupted precisely at that moment was according to my
theory .} \\

\item[Declarative tree:]  See Figure~\ref{s0APnx1-tree}.
        
\begin{figure}[htb]
\centering
\begin{tabular}{c}
\psfig{figure=ps/verb-class-files/alphas0APnx1.ps,height=5.5cm}
\end{tabular}
\caption{Declarative PP Small Clause with Sentential Subject Tree, with 
two-word preposition, where the first word is an adjective, and the second word
is a preposition:  $\alpha$s0APnx1}
\label{s0APnx1-tree}
\end{figure}

\item[Other available trees:] wh-moved subject, relative clause on object
of the PP with overt and covert extracted {\it wh}-NP's, adjunct (gap-less)
relative clause with covert extracted {\it wh}-NP/with PP pied-piping.

\end{description}


\section{PP Small Clause with Sentential Subject, and Noun and Prep anchors: Ts0NPnx1}
\label{s0NPnx1-family}

\begin{description}

\item[Description:] This tree family is selected by multi-word prepositions
that take sentential subjects. In particular, this family is selected by
two-word prepositions, where the first word is a noun, the second word a
preposition.  Both components of the multi-word preposition are
anchors. The sentential subject can be indicative or infinitive.  Small
clauses are explained in much greater detail in
section~\ref{sm-clause-xtag-analysis}.  1 preposition selects this tree
family.

\item[Examples:] {\it thanks to} \\
{\it that she is worn out is thanks to a long day in front of the computer
terminal .} \\ 

\item[Declarative tree:]  See Figure~\ref{s0NPnx1-tree}.

\begin{figure}[htb]
\centering
\begin{tabular}{c}
\psfig{figure=ps/verb-class-files/alphas0NPnx1.ps,height=5.5cm}
\end{tabular}
\caption{Declarative PP Small Clause with Sentential Subject Tree, with 
two-word preposition, where the first word is a noun, and the second word is a preposition:  $\alpha$s0NPnx1}
\label{s0NPnx1-tree}
\end{figure}

\item[Other available trees:] wh-moved subject, relative clause on object
of the PP with overt and covert extracted {\it wh}-NP's, adjunct (gap-less)
relative clause with covert extracted {\it wh}-NP/with PP pied-piping.

\end{description}

\section{PP Small Clause with Sentential Subject, and Prep anchors: Ts0PPnx1}
\label{s0PPnx1-family}

\begin{description}


\item[Description:] This tree family is selected by multi-word prepositions
that take sentential subjects. In particular, this family is selected by
two-word prepositions, where both words are prepositions.  Both components
of the multi-word preposition are anchors. The sentential subject can be
indicative or infinitive.  Small clauses are explained in much greater
detail in section~\ref{sm-clause-xtag-analysis}.  3 prepositions select
this tree family.

\item[Examples:] {\it outside of} \\
{\it that Mary did not complete the task on time is outside of the scope of 
this discussion .} \\

\item[Declarative tree:]  See Figure~\ref{s0PPnx1-tree}. 

\begin{figure}[htb]
\centering
\begin{tabular}{c}
\psfig{figure=ps/verb-class-files/alphas0PPnx1.ps,height=5.5cm}
\end{tabular}
\caption{Declarative PP Small Clause with Sentential Subject Tree, with 
two-word preposition, where both words are prepositions:  $\alpha$s0PPnx1}
\label{s0PPnx1-tree}
\end{figure}

\item[Other available trees:] wh-moved subject, relative clause on object
of the PP with overt and covert extracted {\it wh}-NP's, adjunct (gap-less)
relative clause with covert extracted {\it wh}-NP/with PP pied-piping.

\end{description}

\section{PP Small Clause with Sentential Subject, and Prep and Noun anchors: Ts0PNaPnx1}
\label{s0PNaPnx1-family}

\begin{description}
  
\item{Description:} This tree family is selected by multi-word prepositions
that take sentential subjects. In particular, this family is selected by
three-word prepositions.  The first and third words are always
prepositions, and the middle word is a noun.  The noun is marked for null
adjunction since it cannot be modified by noun modifiers.  All three
components of the multi-word preposition are anchors.  Small clauses are
explained in much greater detail in section~\ref{sm-clause-xtag-analysis}.
4 prepositions select this tree family.

\item[Examples:]  {\it on account of}, {\it in support of} \\
{\it that Joe had to leave the beach was on account of the hurricane .} \\
{\it that Maria could not come is in support of my theory about her .} \\
{\it *that Maria could not come is in direct/strict/desparate support of my
theory about her .} \\

\item[Declarative tree:]  See Figure~\ref{s0PNaPnx1-tree}.

\begin{figure}[htb]
\centering
\begin{tabular}{c}
\psfig{figure=ps/verb-class-files/alphas0PNaPnx1.ps,height=5.5cm}
\end{tabular}
\caption{Declarative PP Small Clause with Sentential Subject Tree, with 
three-word preposition, where the middle noun is marked for null adjunction:
$\alpha$s0PNaPnx1} 
\label{s0PNaPnx1-tree}
\end{figure}
        
\item[Other available trees:] wh-moved subject, relative clause on object
of the PP with overt and covert extracted {\it wh}-NP's, adjunct (gap-less)
relative clause with covert extracted {\it wh}-NP/with PP pied-piping.

\end{description}

%\section{Predicative Adjective with Sentential Subject and Complement: Ts0A1s1}
%\label{s0A1s1-family}

%\begin{description}
  
%\item{Description:} This tree family is selected by predicative
%  adjectives that take sentential subjects and a sentential
%  complement. This tree family is selected by {\it likely} and {\it
    certain}.

%\item[Examples:]  {\it likely}, {\it certain} \\
%{\it that Max continues to drive a Jaguar is certain to make Bill jealous .} \\
%{\it for the Jaguar to be towed seems likely to make Max very angry .} \\

%\item[Declarative tree:]  See Figure~\ref{s0A1s1-tree}.

%\begin{figure}[htb]
%\centering
%\begin{tabular}{c}
%\psfig{figure=ps/verb-class-files/alphas0A1s1.ps,height=4.8cm}
%\end{tabular}
%\caption{Predicative Adjective with Sentential Subject and Complement:
%$\alpha$s0A1s1} 
%\label{s0A1s1-tree}
%\end{figure}
        
%\item[Other available trees:] wh-moved subject, 
%adjunct (gap-less) relative clause with covert extracted {\it wh}-NP/with PP pied-piping.

%\end{description}


\section{Sentential Subject with Small Clause Complement: Ts0Vs1}
\label{s0Vs1-family}

\begin{description}
  
\item{Description:} This tree family is selected by verbs that
take a sentential subject and a small clause complement.  
This tree family is selected by {\it make}.
%and {\it let}.

\item[Examples:]  {\it make}
%%%%%{\it let} \\
{\it that Max drives a Jaguar makes Bill jealous .} \\
%{\it that John fell let Bill win the race .} \\

\item[Declarative tree:]  See Figure~\ref{s0Vs1-tree}.

\begin{figure}[htb]
\centering
\begin{tabular}{c}
\psfig{figure=ps/verb-class-files/alphas0Vs1.ps,height=4.8cm}
\end{tabular}
\caption{Sentential Subject with Small Clause Complement: $\alpha$s0Vs1} 
\label{s0Vs1-tree}
\end{figure}
        
\item[Other available trees:] wh-moved subject.
\end{description}


\section{Locative Small Clause with Ad anchor: Tnx0nx1ARB}
\label{nx0nx1ARB-family}

\begin{description}

\item[Description:] These trees are not anchored by verbs, but by adverbs
that are part of locative adverbial phrases. Locatives are explained in
much greater detail in the section on the locative modifier trees (see
section~\ref{locatives}). The only remarkable aspect of this tree family is
the wh-moved locative tree, $\alpha$W1nx0nx1ARB, shown in
Figure~\ref{W1nx0nx1ARB-tree}. This is the only tree family with this type
of transformation, in which the entire adverbial phrase is wh-moved but not
all elements are replaced by wh items (as in {\it how many city blocks away
is the record store?}). Locatives that consist of just the locative adverb
or the locative adverb and a degree adverb (see Section \ref{locatives} for
details) are treated as exhaustive PPs and therefore select that tree
family (Section~\ref{nx0Px1-family}) when used predicatively. For an
extensive description of small clauses, see
Section~\ref{sm-clause-xtag-analysis}. 26 adverbs select this tree family.

\item[Examples:] {\it ahead}, {\it offshore}, {\it behind} \\
{\it the crash is three blocks ahead} \\
{\it the naval battle was many kilometers offshore} \\
{\it how many blocks behind was Max?} \\

\item[Declarative tree:]  See Figure~\ref{nx0nx1ARB-tree}.

\begin{figure}[htb]
\centering
\begin{tabular}{c}
\psfig{figure=ps/verb-class-files/alphanx0nx1ARB.ps,height=5.0cm}
\end{tabular}
\caption{Declarative Locative Adverbial Small Clause Tree:  $\alpha$nx0nx1ARB}
\label{nx0nx1ARB-tree}
\label{3;nx0nx1ARB}
\end{figure}

\begin{figure}[htb]
\centering
\begin{tabular}{c}
\psfig{figure=ps/verb-class-files/alphaW1nx0nx1ARB.ps,height=6.0cm}
\end{tabular}
\caption{Wh-moved Locative Small Clause Tree:  $\alpha$W1nx0nx1ARB}
\label{W1nx0nx1ARB-tree}
\label{3;W1nx0nx1ARB}
\end{figure}

\item[Other available trees:] wh-moved subject, relative clause on subject
with overt and covert extracted {\it wh}-NP's, wh-moved locative,
imperative, NP gerund, PRO subject, NP gerund with PRO subject.

\end{description}

\section{Exceptional Case Marking: TXnx0Vs1}\index{verbs,ecm}
\label{Xnx0Vs1-family}

\begin{description}

\item[Description:] This tree family is selected by verbs that are
classified as exceptional case marking, meaning that the verb asssigns
accusative case to the subject of the sentential complement.  This is in
contrast to verbs in the Tnx0Vnx1s2 family
(section~\ref{nx0Vnx1s2-family}), which assign accusative case to a NP
which is not part of the sentential complement.  ECM verbs take sentential
complements which are either an infinitive or a ``bare'' infinitive.  As
with the Tnx0Vs1 family (section~\ref{nx0Vs1-family}), the declarative and
other trees in the Xnx0Vs1 family are auxiliary trees, as opposed to the
more common initial trees.  These auxiliary trees adjoin onto an S node in
an existing tree of the type specified by the sentential complement.  This
is the mechanism by which TAGs are able to maintain long-distance
dependencies (see Chapter~\ref{extraction}), even over multiple embeddings
(e.g. {\it Who did Bill expect to eat beans?}) or {\it who did Bill expect
Mary to like?}  See section~\ref{ecm-verbs} for details on this family.  21
verbs select this tree family.

\item[Examples:]  {\it expect}, {\it see} \\
{\it Van expects Bob to talk .}
{\it Bob sees the harmonica fall .}

\item[Declarative tree:]  See Figure~\ref{Xnx0Vs1-tree}.

\begin{figure}[htb]
\centering
\begin{tabular}{c}
\psfig{figure=ps/verb-class-files/betaXnx0Vs1.ps,height=3.4cm}
\end{tabular}
\caption{ECM Tree:  $\beta$Xnx0Vs1}
\label{Xnx0Vs1-tree}
\end{figure}

\item[Other available trees:] wh-moved subject, subject relative clause
with overt and covert extracted {\it wh}-NP's, adjunct (gap-less) relative
clause with overt and covert extracted {\it wh}-NP's/with PP pied-piping,
imperative, NP gerund, PRO subject.

\end{description}

\section{Idiom with V, D, and N anchors: Tnx0VDN1}\index{verbs,idiomatic}
\label{nx0VDN1-family}

\begin{description}

\item[Description:] This tree family is selected by idiomatic phrases in
which the verb, determiner, and NP are all frozen (as in {\it He kicked the
bucket.}).  Only a limited number of transformations are allowed, as
compared to the normal transitive tree family (see
section~\ref{nx0Vnx1-family}).  Other idioms that have the same structure
as {\it kick the bucket}, and that are limited to the same transformations
would select this tree, while different tree families are used to handle
other idioms.  Note that {\it John kicked the bucket} is actually
ambiguous, and would result in two parses - an idiomatic one (meaning that
John died), and a compositional transitive one (meaning that there is an
physical bucket that John hit with his foot). 21 idioms select this family.

\item[Examples:] {\it kick the bucket}, {\it bury the hatchet}, {\it
break the ice} \\
{\it Nixon kicked the bucket.}
{\it The opponents finally buried the hatchet.}
{\it The group activity really broke the ice.}

\item[Declarative tree:]  See Figure~\ref{nx0VDN1-tree}.

\begin{figure}[htb]
\centering
\begin{tabular}{c}
\psfig{figure=ps/verb-class-files/alphanx0Vdn1.ps,height=5.2cm}
\end{tabular}
\caption{Declarative Transitive Idiom Tree:  $\alpha$nx0VDN1}
\label{nx0VDN1-tree}
\end{figure}

\item[Other available trees:] subject relative clause with and without
comp, declarative, wh-moved subject, imperative, NP gerund, adjunct gapless
relative with covert extracted {\it wh}-NP/with PP pied-piping, passive,
w/wo by-phrase, wh-moved object of by-phrase, wh-moved by-phrase, relative
(with overt and covert extracted {\it wh}-NP's) on subject of passive, PP
relative, PRO subject, NP gerund with PRO subject.

\end{description}


\section{Idiom with V, D, A, and N anchors: Tnx0VDAN1}\index{verbs,idiomatic}
\label{nx0VDAN1-family}

\begin{description}

\item[Description:] This tree family is selected by transitive idioms that
are anchored by a verb, determiner, adjective, and noun. 2 idioms select
this family.

\item[Examples:] {\it have a green thumb}, {\it sing a different tune} \\
{\it Martha might have a green thumb, but it's uncertain after the death of all the plants.} \\
{\it After his conversion John sang a different tune.} \\

\item[Declarative tree:]  See Figure~\ref{nx0VDAN1-tree}.

\begin{figure}[htb]
\centering
\begin{tabular}{c}
\psfig{figure=ps/verb-class-files/alphanx0VDAN1.ps,height=5.0cm}
\end{tabular}
\caption{Declarative Idiom with V, D, A, and N Anchors Tree: $\alpha$nx0VDAN1}
\label{nx0VDAN1-tree}
\label{3;nx0VDAN1}
\end{figure}

\item[Other available trees:] Subject relative clause with overt and covert
extracted {\it wh}-NP's, adjunct relative clause with covert extracted {\it
wh}-NP/with PP pied-piping, wh-moved subject, imperative, NP gerund,
passive without {\it by} phrase, passive with {\it by} phrase, passive with
wh-moved object of {\it by} phrase, passive with wh-moved {\it by phrase},
passive with relative on object of {\it by} phrase with overt and covert
extracted {\it wh}-NP's, PRO subject, NP gerund with PRO subject.

\end{description}


\section{Idiom with V and N anchors: Tnx0VN1}\index{verbs,idiomatic}
\label{nx0VN1-family}

\begin{description}

\item[Description:]
This tree family is selected by transitive idioms that are anchored by a 
verb and noun. 8 idioms select this family.

\item[Examples:] {\it draw blood}, {\it cry wolf} \\
{\it Graham's retort drew blood.} \\
{\it The neglected boy cried wolf.} \\

\item[Declarative tree:]  See Figure~\ref{nx0VN1-tree}.

\begin{figure}[htb]
\centering
\begin{tabular}{c}
\psfig{figure=ps/verb-class-files/alphanx0VN1.ps,height=5.0cm}
\end{tabular}
\caption{Declarative Idiom with V and N Anchors Tree: $\alpha$nx0VN1}
\label{nx0VN1-tree}
\label{3;nx0VN1}
\end{figure}

\item[Other available trees:] Subject relative clause with overt and covert
extracted {\it wh}-NP's, adjunct relative clause with covert extracted {\it
wh}-NP/with PP pied-piping, wh-moved subject, imperative, NP gerund,
passive without {\it by} phrase, passive with {\it by} phrase, passive with
wh-moved object of {\it by} phrase, passive with wh-moved {\it by phrase},
passive with relative on object of {\it by} phrase with overt and covert
extracted {\it wh}-NP's, PRO subject, NP gerund with PRO subject.

\end{description}


\section{Idiom with V, A, and N anchors: Tnx0VAN1}\index{verbs,idiomatic}
\label{nx0VAN1-family}

\begin{description}

\item[Description:]
This tree family is selected by transitive idioms that are anchored by a 
verb, adjective, and noun. 2 idioms select this family.

\item[Examples:] {\it break new ground}, {\it cry bloody murder} \\
{\it The avant-garde film breaks new ground.} \\
{\it The investors cried bloody murder after the suspicious takeover.} \\

\item[Declarative tree:]  See Figure~\ref{nx0VAN1-tree}.

\begin{figure}[htb]
\centering
\begin{tabular}{c}
\psfig{figure=ps/verb-class-files/alphanx0VAN1.ps,height=5.0cm}
\end{tabular}
\caption{Declarative Idiom with V, A, and N Anchors Tree: $\alpha$nx0VAN1}
\label{nx0VAN1-tree}
\label{3;nx0VAN1}
\end{figure}

\item[Other available trees:] Subject relative clause with overt and covert
extracted {\it wh}-NP's, adjunct relative clause with covert extracted {\it
wh}-NP/with PP pied-piping, wh-moved subject, imperative, NP gerund,
passive without {\it by} phrase, passive with {\it by} phrase, passive with
wh-moved object of {\it by} phrase, passive with wh-moved {\it by phrase},
passive with relative on object of {\it by} phrase with overt and covert
extracted {\it wh}-NP's, PRO subject, NP gerund with PRO subject.

\end{description}



\section{Idiom with V, D, A, N, and Prep anchors: Tnx0VDAN1Pnx2}\index{verbs,idiomatic}
\label{nx0VDAN1Pnx2-family}

\begin{description}

\item[Description:] This tree family is selected by transitive idioms that
are anchored by a verb, determiner, adjective, noun, and preposition. 2
idioms select this family.

\item[Examples:] {\it make a big deal about}, {\it make a great show of} \\
{\it John made a big deal about a miniscule dent in his car.} \\
{\it The company made a big show of paying generous dividends.} \\

\item[Declarative tree:]  See Figure~\ref{nx0VDAN1Pnx2-tree}.

\begin{figure}[htb]
\centering
\begin{tabular}{c}
\psfig{figure=ps/verb-class-files/alphanx0VDAN1Pnx2.ps,height=5.0cm}
\end{tabular}
\caption{Declarative Idiom with V, D, A, N, and Prep Anchors Tree: $\alpha$nx0VDAN1Pnx2}
\label{nx0VDAN1Pnx2-tree}
\label{3;nx0VDAN1Pnx2}
\end{figure}

\item[Other available trees:] Subject relative clause with overt and covert
extracted {\it wh}-NP's, adjunct relative clause with covert extracted {\it
wh}-NP/with PP pied-piping, wh-moved subject, imperative, NP gerund,
passive without {\it by} phrase, passive with {\it by} phrase, passive with
wh-moved object of {\it by} phrase, passive with wh-moved {\it by phrase},
%passive with relative on object of {\it by} phrase with and 
%with overt and covert extracted {\it wh}-NP's, 
outer passive without {\it by} phrase, outer passive with {\it by} phrase, 
outer passive with wh-moved {\it by} phrase, outer passive with wh-moved 
object of {\it by} phrase, 
outer passive without {\it by} phrase with relative on the subject with overt and covert extracted {\it wh}-NP's, 
outer passive with {\it by} phrase with relative on subject with overt and
covert extracted {\it wh}-NP's, PRO subject, passive with PRO subject with
and without the {\it by} phrase, NP gerund with PRO subject.

\end{description}



\section{Idiom with V, A, N, and Prep anchors: Tnx0VAN1Pnx2}\index{verbs,idiomatic}
\label{nx0VAN1Pnx2-family}

\begin{description}

\item[Description:]
This tree family is selected by transitive idioms that are anchored by a 
verb, adjective, noun, and preposition. 1 idiom selects this family.

\item[Examples:] {\it make short work of} \\
{\it John made short work of the glazed ham.} \\


\item[Declarative tree:]  See Figure~\ref{nx0VAN1Pnx2-tree}.

\begin{figure}[htb]
\centering
\begin{tabular}{c}
\psfig{figure=ps/verb-class-files/alphanx0VAN1Pnx2.ps,height=5.0cm}
\end{tabular}
\caption{Declarative Idiom with V, A, N, and Prep Anchors Tree: $\alpha$nx0VAN1Pnx2}
\label{nx0VAN1Pnx2-tree}
\label{3;nx0VAN1Pnx2}
\end{figure}

\item[Other available trees:] Subject relative clause with overt and covert
extracted {\it wh}-NP's, adjunct relative clause with covert extracted {\it
wh}-NP/with PP pied-piping, wh-moved subject, imperative, NP gerund,
passive without {\it by} phrase, passive with {\it by} phrase, passive with
wh-moved object of {\it by} phrase, passive with wh-moved {\it by phrase},
%passive with relative on object of {\it by} phrase with and 
%without comp, 
outer passive without {\it by} phrase, outer passive with {\it by} phrase,
outer passive with wh-moved {\it by} phrase, outer passive with wh-moved
object of {\it by} phrase, outer passive without {\it by} phrase with
relative on the subject with overt and covert extracted {\it wh}-NP's,
outer passive with {\it by} phrase with relative on subject with overt and
covert extracted {\it wh}-NP's, PRO subject, passive with PRO subject with
and without the {\it by} phrase, NP gerund with PRO subject.

\end{description}


\section{Idiom with V, N, and Prep anchors: Tnx0VN1Pnx2}\index{verbs,idiomatic}
\label{nx0VN1Pnx2-family}

\begin{description}

\item[Description:]
This tree family is selected by transitive idioms that are anchored by a 
verb, noun, and preposition. 6 idioms select this family.

\item[Examples:] {\it look daggers at}, {\it keep track of} \\
{\it Maria looked daggers at her ex-husband across the courtroom.} \\
{\it The company kept track of its inventory.} \\

\item[Declarative tree:]  See Figure~\ref{nx0VN1Pnx2-tree}.

\begin{figure}[htb]
\centering
\begin{tabular}{c}
\psfig{figure=ps/verb-class-files/alphanx0VN1Pnx2.ps,height=5.0cm}
\end{tabular}
\caption{Declarative Idiom with V, N, and Prep Anchors Tree: $\alpha$nx0VN1Pnx2}
\label{nx0VN1Pnx2-tree}
\label{3;nx0VN1Pnx2}
\end{figure}

\item[Other available trees:] Subject relative clause with overt and covert
extracted {\it wh}-NP's, adjunct relative clause with covert extracted {\it
wh}-NP/with PP pied-piping, wh-moved subject, imperative, NP gerund,
passive without {\it by} phrase, passive with {\it by} phrase, passive with
wh-moved object of {\it by} phrase, passive with wh-moved {\it by phrase},
%passive with relative on object of {\it by} phrase, 
outer passive without {\it by} phrase, outer passive with {\it by} phrase,
outer passive with wh-moved {\it by} phrase, outer passive with wh-moved
object of {\it by} phrase, outer passive without {\it by} phrase with
relative on the subject with overt and covert extracted {\it wh}-NP's,
outer passive with {\it by} phrase with relative on subject with overt and
covert extracted {\it wh}-NP's, PRO subject, passive with PRO subject with
and without the {\it by} phrase, NP gerund with PRO subject.

\end{description}


\section{Idiom with V, D, N, and Prep anchors: Tnx0VDN1Pnx2}\index{verbs,idiomatic}
\label{nx0VDN1Pnx2-family}

\begin{description}

\item[Description:]
This tree family is selected by transitive idioms that are anchored by a 
verb, determiner, noun, and preposition. 9 idioms select this family.

\item[Examples:] {\it make a mess of}, {\it keep the lid on} \\
{\it John made a mess of his new suit.} \\
{\it The tabloid didn't keep a lid on the imminent celebrity nuptials.} \\

\item[Declarative tree:]  See Figure~\ref{nx0VDN1Pnx2-tree}.

\begin{figure}[htb]
\centering
\begin{tabular}{c}
\psfig{figure=ps/verb-class-files/alphanx0VDN1Pnx2.ps,height=5.0cm}
\end{tabular}
\caption{Declarative Idiom with V, D, N, and Prep Anchors Tree: $\alpha$nx0VDN1Pnx2}
\label{nx0VDN1Pnx2-tree}
\label{3;nx0VDN1Pnx2}
\end{figure}

\item[Other available trees:] Subject relative clause with overt and covert
extracted {\it wh}-NP's, adjunct relative clause with covert extracted {\it
wh}-NP/with PP pied-piping, wh-moved subject, imperative, NP gerund,
passive without {\it by} phrase, passive with {\it by} phrase, passive with
wh-moved object of {\it by} phrase, passive with wh-moved {\it by phrase},
%passive with relative on object of {\it by} phrase, 
outer passive without {\it by} phrase, outer passive with {\it by} phrase,
outer passive with wh-moved {\it by} phrase, outer passive with wh-moved
object of {\it by} phrase, outer passive without {\it by} phrase with
relative on the subject with overt and covert extracted {\it wh}-NP's,
outer passive with {\it by} phrase with relative on subject with overt and
covert extracted {\it wh}-NP's, PRO subject, passive with PRO subject with
and without the {\it by} phrase, NP gerund with PRO subject.

\end{description}


%%\section{Template}
%%
%%\begin{description}
%%
%%\item[Description:]
%%
%%\item[Declarative tree:]  See Figure~\ref{decl-???-tree}.
%%
%%\begin{figure}[htb]
%%\centering
%%\begin{tabular}{c}
%%\psfig{figure=ps/verb-class-files/alpha???.ps,height=4.0cm}
%%\end{tabular}
%%\caption{Declarative Transitive Tree:  $\alpha$nx0Vnx1}
%%\label{decl-????-tree}
%%\end{figure}
%%
%%\item[Other available trees:]
%%
%%\item[Examples:]
%%
%%\end{description}

\section{Transitive/Intransitive Resultatives with Adjectives: TRnx0Vnx1A2}\index{result-verbs-transitive-A}
\label{tr-result_A}

\begin{description}

\item[Description:]

This tree family by transitive and intransitive verbs that form a complex
predicate with adjectives. 55 multi-word anchors select this family.

\item[Example:] {\it hit unconscious}, {\it hammer flat} \\
{\it Bill hit the boy unconscious.}
{\it Miranda hammered the metal flat.} \\

\item[Declarative tree:]  See Figure~\ref{Rnx0Vnx1A2-tree}.

\begin{figure}[htb]
\centering
\begin{tabular}{c}
\psfig{figure=ps/verb-class-files/alphaRnx0Vnx1A2.ps,height=5.0cm}
\end{tabular}
\caption{Resultative multi-anchored by transitive/intransitive verbs and
adjectives, $\alpha$Rnx0Vnx1A2}
\label{Rnx0Vnx1A2-tree}
\label{3;Rnx0Vnx1A2}
\end{figure}

\item[Other available trees:] passive with and without {\it by} phrase,
wh-moved subject, subject relative clause with overt and covert extracted
{\it wh}-NP, wh-moved object, object relative clause with covert extracted
{\it wh}-NP, passive with and without {\it by} phrase on wh-moved object,
passive with and without {\it by} phrase on object relative clause with
covert extracted {\it wh}-NP, imperative, wh-moved subject on passive with
and without {\it by} phrase, wh-question on object of {\it by} phrase in
subject extracted relative clauses with overt and covert extracted {\it
wh}-NP/with PP pied-piping, multi anchored participial modifiers, relative
clause on PP adjunct with overt and covert extracted {\it wh}-NP's,
relative clause on PP adjunct in passives with and without {\it by} phrase
with overt and covert extracted {\it wh}-NP's, NP gerund with and without
{\it by} phrase, wh-moved adjective complement, passive on wh-moved
adjective complement with and without {\it by} phrase.

\end{description}

\section{Transitive/Intransitive Resultatives with Prepositions: TRnx0Vnx1Pnx2}\index{result-verbs-transitive-P}
\label{tr-result_P}

\begin{description}

\item[Description:]

This tree family by transitive and intransitive verbs that form a complex
predicate with prepositions. 11 multi-word anchors select this family.

\item[Example:] {\it grind into}, {\it beat to} \\
{\it Bill ground the wheat into the dough.}
{\it Miranda beat the box to a pulp.} \\

\item[Declarative tree:]  See Figure~\ref{Rnx0Vnx1Pnx2-tree}.

\begin{figure}[htb]
\centering
\begin{tabular}{c}
\psfig{figure=ps/verb-class-files/alphaRnx0Vnx1Pnx2.ps,height=5.0cm}
\end{tabular}
\caption{Resultative multi-anchored by transitive/intransitive verbs and
prepositions, $\alpha$Rnx0Vnx1Pnx2}
\label{Rnx0Vnx1Pnx2-tree}
\label{3;Rnx0Vnx1Pnx2}
\end{figure}

\item[Other available trees:] passive with and without {\it by} phrase,
wh-moved subject, subject relative clause with overt and covert extracted
{\it wh}-NP, wh-moved object, object relative clause with covert extracted
{\it wh}-NP, passive with and without {\it by} phrase on wh-moved object,
passive with and without {\it by} phrase on object relative clause with
covert extracted {\it wh}-NP, imperative, wh-moved subject on passive with
and without {\it by} phrase, wh-question on object of {\it by} phrase in
subject extracted relative clauses with overt and covert extracted {\it
wh}-NP/with PP pied-piping, relative clause on PP adjunct with overt and
covert extracted {\it wh}-NP's, relative clause on PP adjunct in passives
with and without {\it by} phrase with overt and covert extracted {\it
wh}-NP's, NP gerund with and without {\it by} phrase, wh-moved adjective
complement, passive on wh-moved adjective complement with and without {\it
by} phrase.

\end{description}

\section{Ergative Resultatives with Adjectival Predicates: TREnx1VA2}\index{verbs-result-erg-A}
\label{E-result_A}

\begin{description}

\item[Description:]

This tree family by ergative verbs that form a complex predicate with
adjectives. 14 multi-word anchors select this family.

\item[Example:] {\it freeze solid}, {\it break open} \\
{\it The milk froze solid.}
{\it The chest broke open.} \\

\item[Declarative tree:]  See Figure~\ref{REnx1VA2-tree}.

\begin{figure}[htb]
\centering
\begin{tabular}{c}
\psfig{figure=ps/verb-class-files/alphaREnx1VA2.ps,height=5.0cm}
\end{tabular}
\caption{Resultative multi-anchored by transitive/intransitive verbs and
adjectives, $\alpha$REnx1VA2}
\label{REnx1VA2-tree}
\label{3;REnx1VA2}
\end{figure}

\item[Other available trees:] wh-moved subject, subject relative clause
with overt and covert extracted {\it wh}-NP's, wh-moved object, imperative,
subject extracted relative clauses with PP pied-piping, relative clause on
PP adjunct with overt and covert extracted {\it wh}-NP's, NP gerund,
wh-moved adjective complement.

\end{description}

\section{Ergative Resultatives with Adjectival Predicates: TREnx1VPnx2}\index{verbs-result-erg-P}
\label{E-result_P}

\begin{description}

\item[Description:]

This tree family by ergative verbs that form a complex predicate with
prepositions. 6 multi-word anchors select this family.

\item[Example:] {\it bend into}, {\it melt into} \\
{\it The iron rod bent into a curve.}
{\it The icecream melted into milk.} \\

\item[Declarative tree:]  See Figure~\ref{REnx1VPnx2-tree}.

\begin{figure}[htb]
\centering
\begin{tabular}{c}
\psfig{figure=ps/verb-class-files/alphaREnx1VPnx2.ps,height=5.0cm}
\end{tabular}
\caption{Resultative multi-anchored by transitive/intransitive verbs and
adjectives, $\alpha$REnx1VPnx2}
\label{REnx1VPnx2-tree}
\label{3;REnx1VPnx2}
\end{figure}

\item[Other available trees:] wh-moved subject, subject relative clause
with overt and covert extracted {\it wh}-NP's, wh-moved object, object
relative clause with overt and covert extracted {\it wh}-NP's, imperative,
relative clauses with PP pied-piping, relative clause on PP adjunct with
overt and covert extracted {\it wh}-NP's, NP gerund, wh-moved adjective
complement.

\end{description}