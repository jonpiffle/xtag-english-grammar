\section{Adverbs}

Three major issues in accounting for abverbs are:
\begin{enumerate}
\item accounting for the variation in adverb position as shown in
(\ex{1})-(\ex{3}). 

\enumsentence{(Now) he is (now) living in N.Y. (now).}
\enumsentence{(Bravely), Frank (bravely) had (bravely) fought in Spain (bravely).}
\enumsentence{(Here), you can (here) see the symptom (here).}

\item Whether wh-adverbs require a trace; and
\item criteria for subclasses of adverbs.
\end{enumerate}

\subsection{Variation in adverb position}
 Roughly speaking, adverbs can be in sentence-initial, medial and
final positions.  Different adverbs occur in different positions and
some adverbs occur in multiple positions.  To account for this order
variation Emonds(\shortcite{Emonds}) and
Jackendoff(\shortcite{Jackendoff}) use several base generated
positions plus transformations.  Ernst(\shortcite{Ernst}) accounts for
the order variation with an entirely base generated system.  


\subsection{Adverb trace}
The motivation of having gaps in a transformational analysis of
extraction construction like wh-question, topicalization, etc. is
based on the consideration that these constructions behave
as though there is a node of the appropriate type in the position of
the extraction site (cf. Jacobson 82). The following are some
empirical evidence supporting this consideration:

\begin{itemize}
\item Subcategorization of P and V: Transitive verbs and prepositions 
cannot, in general, occur without a following NP, but they can occur in 
these constructions.\\

\enumsentence{Which do you think your father put in the garage?}
\enumsentence{$\ast$My father put in the garage.}
\enumsentence{Who did you talk with?}
\enumsentence{$\ast$I talked with.}

\item In some dialects nominal wh-phrases make a clear case-marking
distinction with regard to the position of the extraction site.\\

\enumsentence{$\ast$Who/Whom did John see?}
\enumsentence{Who/$\ast$Whom did John say was coming.}

\item person/number agreement:\\

\enumsentence{{\em Which boy\/} might he say {\em likes\/} Mary?}
\enumsentence{$\ast${\em Which boy\/} might he say {\em like\/} Mary?}
\enumsentence{{\em Which boys\/} might he say {\em like\/} Mary?}
\enumsentence{$\ast${\em Which boys\/} might he say {\em likes\/} Mary?}

\item {\em wanna\/} contraction:\\

\enumsentence{I wanna go to a movie.}
\enumsentence{Who do you wanna go to the movie with?}
\enumsentence{$\ast$Who do you wanna go to the movie with Chris?}

\end{itemize}

If we follow the line of reasoning above in positing a trace for
wh-phrases, there is no reason to posit a trace for adverbial
wh-phrases: Adverbials are not subcategorized for by verbs or
prepositions. There is no overt morphological evidence to identify the
position of the extraction site of adverbs. 

In addition, Baltin(\cite{Baltin}) has pointed out that PP-topicalized
sentences like (\ref{pp-top}) are grammatical, while the corresponding
{\it wh}-question in (\ref{pp-top-q}) is ungrammatical:

\enumsentence{After the party$_{i}$, I wonder who$_{j}$ e$_{j}$will
stay e$_{i}$.}
\enumsentence{$\ast$When$_{i}$ do you wonder who$_{j}$ e$_{j}$ will stay
e$_{i}$?}

Baltin uses this contrast to argue that preposed adverbs may not leave
traces, since under standard GB theory assumptions the trace of the
adverb in (\ref{pp-top}) would not be properly governed. 

\subsection{Adverb subclasses}
All approaches to adverbs are faced with the problem of dividing them
into classes because adverbs do not behave uniformly with respect to
positions in which they can occur. The classes shown below that have
been proposed by  \cite{Quirk}   are based primarilly on
semantic considerations. 

\begin{itemize}
 \item Time-adverbials ({\em when})
  \begin{enumerate}
   \item Time-when adverbials denote a point or period of time, and/or
imply the point from which that time is measured.
     \begin{verse}
      Do come and see again.\\
      The meeting starts tomorrow \\
      I'm just finishing my homework.({\em just\/} always in mid-position)\\
      (Then) he (then) has (then) lived in N.Y.(then).\\
      (Recently) they (recently) had an accident (recently)
     \end{verse}
   \item Time duration adverbials denote (1) length of time or (2) duration
         from some preceding point of time. 
     \begin{verse}
      They always/temorarily out of work.
     \end{verse}
   \item Time frequency adverbials denote either (1) definite
         frequency, or (2) indefinte frequency. They can occur any of
         the three positions. eg. {\em weekly, generally, normally,
         always, regulary, sometimes, rarely, etc.}
  \end{enumerate}
 \item Place adverbials ({\em where\/})
     \begin{verse}
      (Outside), the boys were jumping and skipping (outside).\\
      (In the nursery) the children were playing happily but noisily.\\
      The children were playing happily but noisily (in the nursery).
     \end{verse}
 \item Manner adverbs ({\em how\/}) are assumed to modify VP's.
     \begin{verse}
      They live frugally.\\
      They examined the specimen microscopically.\\
      He put the point well/*He well put the point \\
      The point was put well/The point was well-put.
     \end{verse}     
 \item Sentence adverbs are periphral to the sentence structure, and have
       a wide range of possible structures:  
     \begin{verse}
      Frankly, he hasn't got a chance.\\
      in all frankness \\
      to be frank/to speak fankly/to put it frankly \\
      frankly speaking/putting it frankly \\
      put frankly \\
      if I may be frank/if I can speak frankly/if I can put it frankly
     \end{verse}
\end{itemize}
       More sentence adverbs are
       {\em admittedly, certainly, definitely, indeed, surely, perhaps,
          possibly; in fact, actually, really; officially,
          superficially, technically, theoretically; fortunately,
          hopefully, luckily, naturally, preferably, strangely,
          surprisingly, etc.}
          Many other sentence adverbials (eg. however, therefore,
          moreover) have a connective role.
Many adverbs can function as both manner and sentence adverbs,
depending on the position of the adverb. 

Emonds(\shortcite{Emonds76}) posits three base positions for adverbs
and has different adverbs associated with each position. He uses
tranformations to derive positions other than his three base positions
or to account for adverbs that can occur in more than one position.
Base positions and possible tranformations for an adverb depend on its
class.

Jackendoff(\shortcite{Jackendoff72}) has six subclasses of adverbs,
according to their pattern of occurrence, and meaning in each of: sentence
initial position, AuxRange and sentence final position.  He uses a
number of different transformations depending (primarily) on the
semantics of the adverbs. He has to resort to semantics to formulate
the right kind of transformation. So again, categorizing the the
adverbs is crucial to the analysis.

Ernst(\shortcite{Ernst84} has a base-generation approach to adverbs.
His base rules generate all possible AdvP positions.  The general
type of semantics of an adverb specifies what positions that adverb
can range over, so again semantic classification is crucial.

In summary, accounting for the syntactic behavior of adverbs requires
classification of adverbs by semantics and/or distributional criteria.
  

\subsection{LTAG treatment of adverbs}
In the English LTAG grammar VP-/S-modifying adverbs anchor the
auxiliary trees $\beta$adS, $\beta$Sad, $\beta$VPad and $\beta$adVP.
Besides the VP-/S-modifying adverbs, the English LTAG grammar includes
adverbs that modify other categories asshown in (\ex{1})-(\ex{4})\footnote{This categorization is adapted  from {\em A Communicative
Grammar of English\/} [Leech and Svartvik 75].}

\enumsentence{modifying an adjective:\\ {\em extremely\/} good\\
{\em rather\/} tall\\rich {\em enough\/} }
\enumsentence{modifying an adverb:\\ oddly {\em enough\/}\\ {\em very\/} well}
\enumsentence{ modifying a PP:\\ {\em right\/} through the wall}
\enumsentence{modifying a determiner: He has {\em hardly\/} any
friends\\ {\em over\/} two hundred deaths}



We have separate trees for each of the modified categories and for
pre- and post-modification where needed.   The trees are auxiliary
trees $\beta$ARBx, $\beta$xARB, where X is a variable over S, VP, NP, A,
Ad, PP, and an initial tree $\alpha$Ad. Examples of these trees are
shown in figures (\ref{sARB})-(\ref{pxARB})

%need all the adverb trees




Order variation of an adverb is represented by different trees. For
instance, if an adverb can occur in sentence initial, preverbal and
sentence final positions as shown in (8) and (9), it can represented by
four different trees $\beta$ARBS, $\beta$ARBvx, $\beta$vxARB, and
$\beta$sARB.

\begin{verse}
(8) ({\em Cleverly\/}) he ({\em cleverly\/}) answered the questions
({\em cleverly\/}).\\
(9) My car does not start on the first trial everyday.
\end{verse}

We assign two trees $\beta$vxARB and $\beta$sARB to sentence final
adverbs since they can be interpreted as modifying the VP or the whole
S: e.g. (8) with the adverb {\em cleverly\/} in sentence final
position has two interpretations (a) It was clever that he answered
the question (S-modifying), (b) He answered question in a clever
manner (VP-modifying).\footnote{This line of reasoning, that
sentence-final adverbs could be represented by either $\beta$vxARB and
$\beta$sARB depending on their interpretation, is inconsistent with the
reasoning given in 3.3 with regard to the syntax/semantics mapping. To
make the story consistent, we can represent all sentence final adverbs
as $\beta$sARB which can modify either the S or VP.} (9) also has two
interpretations: (a) For everyday, it is not the case that my car
starts on the first trial, where the adverb {\em everyday\/} takes the
scope over the whole sentence (S-modifying), (b) It is not the case
that my car starts on the first trial everyday, where {\em everyday\/}
takes the scope over the VP (VP-modifying).

The kind of treatment we give to adverbs here is very much in line
with the base-generation approach proposed by [Ernst 84], which
assumes all positions where an adverb can occur to be base-generated,
and the semantics of the adverb specifies a range of possible
positions occupied by each adverb.
However, classification of adverbs in the English LTAG grammar does not directly take
semantics into account.  Adverbs are simply assigned the appropriate
trees to account for their distributions.  Adverbs that select the
same trees, could be considered a class, and although it would be interesting
if these classes corresponded some semantic distinctions, we have not
investigated this possiblity at this point in the developement of the
grammar.

\subsubsection{Nontransformational Treatment of Wh-adverbs}

In the current system, we posit no traces for wh-adverbs in line with
the base-generation approach for various positions of adverbs,
and assign the sentence initial wh-adverbs the same auxiliary
tree for other sentence initial adverbs ($\beta$adS) with the feature
[+wh] added on {\em ad\/}. 

Under this treatment, the derived tree of the sentence {\em How did
you go to New York?\/} looks like (14), with no trace:

\begin{verbatim}
(14)               S
                 /   \
               Ad      Sr
             [+wh]    /  \
               |     Va   S
              how    |   /  \
                    did NP   VP
                        |   /  \
                       you V    PP
                           |   /  \
                          go  P    NP
                              |    |
                             to  New York
\end{verbatim}
 
\subsubsection{ECP Effects and Wh-island Constraint}

Since we are not assuming a trace for adverbs (including wh-phrases),
it seems to be worth noting how the long-distance dependency of
wh-adverbs including the wh-island effect -- which has a simple
structural explanation (cf. [Kroch 87, 89]) -- is accounted for in the 
current system. 

Consider the examples given below:

\begin{tabbing}
(15) \= a. \= xxxxxxxxxxxxxxxxxxxxxxxxxxxxxxxxxxxxxxxxxxxxxxxxx \kill
(15) \> a. When$_{i}$ do you think that Mary will come {\em e\/}$_{i}$?\\
     \> b. *When$_{i}$ do you wonder who$_{j}$ {\em e\/}$_{j}$ will stay
{\em e\/}$_{i}$? \\
(16) \> a. \> *Who$_{i}$ does he think that [{\em e\/}$_{i}$ left]? \\
     \> b. \> When$_{i}$ does he think that [we left $_{i}$]? \\
(17) \> a. \> ?On that shelf$_{i}$, how many books$_{j}$ can you fit {\em
e\/}$_{j}$ {\em e\/}$_{i}$? \\
     \> b. \> *That many books$_{i}$, on what shelf$_{j}$ {\em e\/}$_{i}$
can fit {\em e\/}$_{j}$? 
\end{tabbing}

\noindent
In [Kroch 89], the (un)grammaticality of the above pairs of examples
are explained in terms of following mechanisms along with the notion
of {\em extended domain of locality\/}:

\begin{itemize}
\item A well-formedness constraint on elementary trees that to the
effect that they can contain no more than one wh-phrases in sentence
initial positions for wh-island constraint (cf.\ (15) above)
\item Three types of proper government for ECP:
 \begin{enumerate}
  \item Lexical government for complement trace
  \item Strictly local antecedent government for a subject trace
  \item Local antecedent government for an adverb trace
 \end{enumerate}
\end{itemize}

The (un)grammaticality of the examples can be easily explained by
using the same mechanisms described above. In some sense, the
explanatory mechanism becomes much simpler since we can get away with
the `local antecedent government' for an adverb trace.

\subsubsection{Complement Extraction out of Wh-islands}

In [Kroch 89], the contrast between (18a) and (18b) is explained by
employing multi-component adjunction:\footnote{We are omitting the
details of the explanation here. For the details, refer to [Kroch
89].}

\begin{tabbing}
(18) \= a. \= ?What$_{i}$ were you wondering [how$_{j}$ to say {\em
e\/}$_{i}$ {\em e\/}$_{j}$]? \kill
(18) \> a. \> ?What$_{i}$ were you wondering [how$_{j}$ to say {\em
e\/}$_{i}$ {\em e\/}$_{j}$]? \\
     \> b. \> *How$_{i}$ were you wondering [what$_{j}$ to say {\em
e\/}$_{j}$ {\em e\/}$_{j}$]?
\end{tabbing}

\noindent
The ungarammaticality of (18b), however, can be accounted for in
another way than the ECP account in [Kroch 89]. Namely, the
restriction on initial trees described above can rule out the
sentence.\footnote{Subject extraction of the same sort can also be
explained in the same way as adverb extraction in (18b).} Ruling out
the ungrammaticality of the same sentence in two different ways seems
to be undesirable according to the occam's razor.\footnote{On the
other hand, this redundency in the TAG mechanism might be unavoidable, 
considering that the ECP in the standard GB theory
subsumes the wh-island effect, while the restriction on
initial trees in TAG discussed in [Kroch 89] is meant to account for
wh-island effect, and the TAG version of ECP is formulated to take
care of the classical ECP effect in GB theory(?).}

Instead, if we don't assume the adverb trace, (18b) is ruled out only
by the restriction on initial trees, although the decision of what is
the preferrable way of ruling out the sentence must be linguistically
well-motivated.

\subsubsection{Derivation of the Unbounded Dependency of Wh-adverbs in
Bridge Verb Construction}

In the current system which does not posit a trace for adverbs, the
sentences such as given below have two separate derivations:

\begin{verse}
(19) Why do you think that Mary left angry?\\
(20) When did you think the TV set was stolen?
\end{verse}

\noindent
In (19) and (20), the sentence initial adverbs {\em why\/} and {\em
when\/} can be originated either from the embedded clause or from the 
matrix clause. Depending on the derivation, the sentences have different 
interpretations with respect to the scope of the adverbs. 

Notice that this treatment of wh-adverbs necessarily presuppose that the
unit of semantics as well as syntax is strictly local, and the
interpretation of the whole sentence crucially hinges on the derivation
history of the derived tree. (cf. [Shieber \& Schabes 90] for 
quantifier scope interpretation)



\subsection{Inconsistency/Problems in the Adverb Treatment in the Current System}

In the previous system (until 1990 Spring), the structures used for adverbs
were limited to right and left attachment to S, and right and left attachment
to VP, which are meant to provide the proper scope for each adverb. 

The decision of the (syntactic) strucuture of adverbs on the basis of scope
consideration, however, has the following problem:

\begin{itemize}
\item It presupposes that any adverbs which take the scope over the sentence
must have have a struture where the adverb is attached right or left to
S depending on the surface position of the adverb.
\item This treatment, however, cannot give a coherent story about sentential
adverbs positioned among auxiliaries as in {\em John could probably have
done the homework this morning\/}. The adverb {\em probably\/} is
S-modifying, but its surface order dictates that it should not be
attached left or right to S but between the first auxiliary {\em could\/}
and the second auxiliary {\em have\/}.
\end{itemize}

The problem is due to the two irreconcilable assumptions that all
adverbs are base-generated, and that the structure of adverbs must
reflect the semantics of the adverbs (i.e. one-to-one mapping between
syntax and semantics).

There seem to be at least two ways of making the inconsistency
linguistically coherent: One is to abandon the idea of having the
semantics be reflected in the syntax, i.e. there is no one-to-one
correpondence between the position of an adverbs and its scope, hence
to admit that a sentence medial adverb can take scope over the whole
sentence, or a sentence final adverb doesn't have to be realized in
two different structures to reflect its proper scope. The other is to
adopt the idea of having a trace for adverbs, so that the base
position of an adverb reflects the scope interpretation of the adverb
-- in this case we need to adopt the transformational approach very
similiar to that discussed in section 2. 

