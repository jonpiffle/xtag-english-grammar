\chapter{Where to Find What}
\label{table-intro}

The two page table that follows gives an overview of what types of trees occur
in various tree families with pointers to discussion in this report.  An
entry in a cell of the table indicates that the tree(s) for the construction
named in the row header are included in the tree family named in the column
header. Entries are of two types.  If the particular tree(s) are displayed
and/or discussed in this report the entry gives a page number reference to the
relevant discussion or figure.\footnote{Since Chapter~\ref{verb-classes} has a
brief discussion and a declarative tree for every tree family, page references
are given only for other sections in which discussion or tree diagrams appear.}
Otherwise, a \xtagcheck \space indicates inclusion in the tree family but no
figure or discussion related specifically to that tree in this report.  Blank
cells indicate that there are no trees for the construction named in the row
header in the tree family named in the column header.  The table below gives
the expansion of abbreviations in the table headers.

\vspace{0.3in}

\small
\begin{tabular}{ll}
Abbreviation&Full Name\\
\hline
Sentential Comp. with NP&Sentential Complement with NP\\
Ditrans. Light Verbs w. PP Shift&Ditransitive Light Verbs with PP Shift\\
Ditrans. Light Verbs w/o PP Shift&Ditransitive Light Verbs without PP Shift\\
Adj. Sm. Cl. w. Sentential Subj.&Adjective Small Clause with Sentential Subject\\
NP Sm. Clause w. Sentential Subj.&NP Small Clause with Sentential Subject\\
PP Sm. Clause w. Sentential Subj.&PP Small Clause with Sentential Subject\\
Y/N question&Yes/No question \\
Wh-mov. NP complement&Wh-moved NP complement \\
Wh-mov. S comp.&Wh-moved S complement \\
Wh-mov. Adj comp.&Wh-moved Adjective complement \\
Wh-mov. object of a P&Wh-moved object of a P \\
Wh-mov. PP&Wh-moved PP \\
Topic. NP complement&Topicalized NP complement \\
Det. gerund&Determiner gerund \\
Rel. cl. on NP comp.&Relative clause on NP complement \\
Rel. cl. on PP comp.& Relative clause on PP complement\\
Rel. cl. on NP object of P& Relative clause on NP object of P\\
Pass. with wh-moved subj.&Passive with wh-moved subject (with and without {\it by} phrase) \\
Pass. w. wh-mov. ind. obj.&Passive with wh-moved indirect object (with and without {\it by} phrase) \\
Pass. w. wh-mov. obj. of the {\it {\it by} phrase}&Passive with wh-moved object of the {\it by} phrase \\
Pass. w. wh-mov. {\it by} phrase&Passive with wh-moved {\it by} phrase \\
Adj. Sm. Cl. w. Sent. Comp.&Adjective Small Clause with Sentential
Complement\\
NP Sm. Cl. w. Sent. Comp.&NP Small Clause with Sentential Complement\\
Sent. Subj. w. {\it to} & Sentential Subject with {\it to} PP complement \\
Pred. Mult-wd. ARB, P & Predicative Multi-word PP with Adv, Prep anchors\\
Pred. Mult-wd. A, P & Predicative Multi-word PP with Adj, Prep anchors\\
Pred. Mult-wd. N, P & Predicative Multi-word PP with Noun, Prep
anchors\\
Pred. Mult-wd. P, P & Predicative Multi-word PP with two Prep
anchors\\
Pred. Mult-wd. no int. mod. & Predicative Multi-word PP with no internal
modification\\
Pred. Sent. Subj., ARB, P & Predicative PP with Sentential Subject, and
Adv, Prep anchors\\
Pred. Sent. Subj., A, P & Predicative PP with Sentential Subject, and
Adj, Prep anchors\\
Pred. Sent. Subj., Conj, P & Predicative PP with Sentential Subject, and
Conj, Prep anchors\\
Pred. Sent. Subj., N, P & Predicative PP with Sentential Subject, and
Noun, Prep anchors\\
Pred. Sent. Subj., P, P & Predicative PP with Sentential Subject, and two
Prep anchors\\
Pred. Sent. Subj., no int-mod & Predicative PP with Sentential Subject,
no internal modification\\
Pred. Locative & Predicative anchored by a Locative Adverb\\
Pred. A Sent. Subj., Comp. & Predicative Adjective with Sentential
Subject and Complement\\
 N predicative tree with sentential subject & Pred. N w/ Sent. Subj.\\
 Predicative Multi-word V, P & Pred. Mult wd. V, P
\end{tabular}
\normalsize
















