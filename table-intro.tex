\chapter{Where to Find What}
\label{table-intro}

The two tables that follow give an overview of what types of trees occur in
various tree families, with pointers to discussion in this report.  The
first table gives the expansion of abbreviations in the headers of the
second table. The abbeviations and their expansions can be matched in the
second table in the order in which they appear - first for the column
headers (left-right) and second for the row header (top-down). Notice that
the non-abbreviated headers will not be seen in this table, so they should
be skipped when following the order of appearance. The second table
contains two kinds of information related to the XTAG grammar. Firstly, the
column headers give the list all the tree families in the grammar (along
with the {\sc Xtag} names for them). These can be also be referenced in the
more detailed decsriptions of each tree family in
Chapter~\ref{verb-classes}. Secondly, each column, corresponding to a tree
family, contains a listing of the constructions available for that tree
family so that the row headers gives the constructions available in the
grammar across the tree families. An entry in a cell of the table indicates
that the tree(s) for the construction named in the row header are included
in the tree family named in the column header. Entries are of two types.
If the particular tree(s) are displayed and/or discussed in this report the
entry gives a page number reference to the relevant discussion or
figure.\footnote{Since Chapter~\ref{verb-classes} has a brief discussion
and a declarative tree for every tree family, page references are given
only for other sections in which discussion or tree diagrams appear.}
Otherwise, a \xtagcheck \space indicates inclusion in the tree family but
no figure or discussion related specifically to that tree in this report.
Blank cells indicate that there are no trees for the construction named in
the row header in the tree family named in the column header.

%The second table gives the name given to each tree family in the actual
%XTAG grammar. This makes it easier to find the description of each tree
%family in Chapter~\ref{verb-classes} and to compare the description with
%the online XTAG grammar.



\vspace{0.3in}

\footnotesize
\begin{tabular}{ll}
Abbreviation&Full Name\\
\hline
{\bf COLUMN HEADERS} &\\
\hline
Intr. Sent. Subj. & Intransitive Sentential Subject
Sent. Subj. w. {\it to} \& Sentential Subject with {\it to} PP complement \\
Pred. Mult-wd. ARB, P & Predicative Multi-word PP with Adv, Prep anchors\\
Pred. Mult-wd. A, P & Predicative Multi-word PP with Adj, Prep anchors\\
Pred. Mult-wd. N, P & Predicative Multi-word PP with Noun, Prep
anchors\\
Pred. Mult-wd. P, P & Predicative Multi-word PP with two Prep
anchors\\
Pred. Mult-wd. no int. mod. & Predicative Multi-word PP with no internal
modification\\
Pred. Sent. Subj., ARB, P & Predicative PP with Sentential Subject, and
Adv, Prep anchors\\
Pred. Sent. Subj., A, P & Predicative PP with Sentential Subject, and
Adj, Prep anchors\\
Pred. Sent. Subj., N, P & Predicative PP with Sentential Subject, and
Noun, Prep anchors\\
Pred. Sent. Subj., P, P & Predicative PP with Sentential Subject, and two
Prep anchors\\
Pred. Sent. Subj., no int-mod & Predicative PP with Sentential Subject,
no internal modification\\
ECM & Exceptional Case Marking\\
Pred. Locative & Predicative anchored by a Locative Adverb\\
Sent. Subj. w. Sm. Cl. Comp. & Sentential Subject with Small Clause
Complement\\
Ditransitive w/ PP w/ P co-anchor & Ditransitive with PP with Verb and Prep
anchors\\
Adj. Sm. Cl. w. Sentential Subj. & Adjective Small Clause with Sentential Subject\\
NP Sm. Clause w. Sentential Subj. & NP Small Clause with Sentential Subject\\
PP Sm. Clause w. Sentential Subj. & PP Small Clause with Sentential Subject\\
Pred. Mult wd. V, P & Predicative Multi-word with Verb, Prep anchors \\
Adj. Sm. Cl. w. Sent. Comp.&Adjective Small Clause with Sentential
Complement\\
NP Sm. Cl. w. Sent. Comp.&NP Small Clause with Sentential Complement\\
Exhaustive PP Sm. Cl.&Exhaustive PP Small Clause\\
Idiom with V, D, N & Idiom with V, D, and N anchors \\
Idiom with V, D, A, N & Idiom with V, D, A, and N anchors \\
Idiom with V, N & Idiom with V, and N anchor \\
Idiom with V, A, N & Idiom with V, A, and N anchors \\
Idiom with V, D, N, P & Idiom with V, D, N, and Prep anchors \\
Idiom with V, D, A, N, P & Idiom with V, D, A, N, and Prep anchors \\
Idiom with V, N, P & Idiom with V, N, and Prep anchors \\
Idiom with V, A, N, P & Idiom with V, A, N, and Prep anchors \\
Tr/In. Resultative with A. & Transitive and Intranstive verbs with Adj
resultative predicate\\
Tr/In. Resultative with P. & Transitive and Intranstive verbs with Prep
resultative predicate\\
Erg. Resultative with A. & Ergative verbs with Adj
resultative predicate\\
Erg. Resultative with P. & Ergative verbs with Prep
resultative predicate\\
& \\
\hline
{\bf ROW HEADERS} & \\
\hline
Y/N question&Yes/No question \\
Wh-mov. NP complement&Wh-moved NP complement \\
Wh-mov. S comp.&Wh-moved S complement \\
Wh-mov. Adj comp.&Wh-moved Adjective complement \\
Wh-mov. object of a mod.&Wh-moved object of a modifier \\
Wh-mov. PP&Wh-moved PP \\
Topic. NP complement&Topicalized NP complement \\
Det. gerund&Determiner gerund \\
Rel. cl. on NP comp.&Relative clause on NP complement \\
Rel. cl. on PP comp.& Relative clause on PP complement\\
Rel. cl. on NP object of P& Relative clause on NP object of P\\
Pass. with wh-moved subj.&Passive with wh-moved subject (with and without {\it by} phrase) \\
Pass. w. wh-mov. ind. obj.&Passive with wh-moved indirect object (with and without {\it by} phrase) \\
Pass. w. wh-mov. obj. of the {\it {\it by} phrase}&Passive with wh-moved object of the {\it by} phrase \\
Pass. w. wh-mov. {\it by} phrase&Passive with wh-moved {\it by} phrase \\
Trans. Idiom with V, D and N & Transitive Idiom with Verb, Det and
Noun anchors\\

\end{tabular}
\normalsize

\footnotesize
\begin{tabular}{ll}
Full Name&XTAG Name\\
\hline
Transitive Ergative  &  TEnx1V\\
Intransitive Sentential Subject &  Ts0V\\
Sentential Subject with `to' complement &  Ts0Vtonx1\\
PP Small Clause, with Adv and Prep anchors & Tnx0ARBPnx1\\
PP Small Clause, with Adj and Prep anchors & Tnx0APnx1\\
PP Small Clause, with Noun and Prep anchors & Tnx0NPnx1\\
PP Small Clause, with Prep anchors & Tnx0PPnx1\\
PP Small Clause, with Prep and Noun anchors & Tnx0PNaPnx1\\
PP Small Clause with Sentential Subject, and Adv and Prep anchors & Ts0ARBPnx1\\
PP Small Clause with Sentential Subject, and Adj and Prep anchors & Ts0APnx1\\
PP Small Clause with Sentential Subject, and Noun and Prep anchors & Ts0NPnx1\\
PP Small Clause with Sentential Subject, and Prep anchors & Ts0PPnx1\\
PP Small Clause with Sentential Subject, and Prep and Noun anchors & Ts0PNaPnx1\\
Exceptional Case Marking & TXnx0Vs1\\
Locative Small Clause with Ad anchor & Tnx0nx1ARB\\
Sentential Subject with Small Clause Complement & Ts0Vs1\\
Transitive & Tnx0Vnx1\\
Ditransitive with PP (where Prep is co-anchor) & Tnx0Vnx1Pnx2\\
Ditransitive & Tnx0Vnx2nx1\\
Ditransitive with PP & Tnx0Vnx1pnx2\\
Sentential Complement with NP & Tnx0Vnx1s2\\
Intransitive Verb Particle & Tnx0Vpl\\
Transitive Verb Particle & Tnx0Vplnx1\\
Ditransitive Verb Particle & Tnx0Vplnx2nx1\\
Intransitive with PP & Tnx0Vpnx1\\
Sentential Complement & Tnx0Vs1\\
Transitive Light Verbs & Tnx0lVN1\\
Ditransitive Light Verbs with PP & Tnx0lVN1Pnx2\\
Adjective Small Clause with Sentential Subject & Ts0Ax1\\
NP Small Clause with Sentential Subject &  Ts0N1\\
PP Small Clause with Sentential Subject & Ts0Pnx1\\
Predicative Multi-word with Verb, Prep anchors & Tnx0VPnx1\\
Adverb It-Cleft & TItVad1s2\\
NP It-Cleft & TItVnx1s2\\
PP It-Cleft & TItVpnx1s2\\
Adjective Small Clause Tree & Tnx0Ax1\\
Adjective Small Clause with Sentential Complement & Tnx0A1s1\\
Equative {\it BE} & Tnx0BEnx1\\
NP Small Clause & Tnx0N1\\
NP with Sentential Complement Small Clause & Tnx0N1s1\\
PP Small Clause & Tnx0Pnx1\\
Exhaustive PP Small Clause & Tnx0Px1\\
Intransitive & Tnx0V\\
Intransitive with Adjective & Tnx0Vax1\\
Transitive Sentential Subject &  Ts0Vnx1\\
Idiom with V, D and N & Tnx0VDN1\\
Idiom with V, D, A, and N anchors & Tnx0VDAN1\\
Idiom with V and N anchors & Tnx0VN1\\
Idiom with V, A, and N anchors & Tnx0VAN1\\
Idiom with V, D, N, and Prep anchors & Tnx0VDN1Pnx2\\
Idiom with V, D, A, N, and Prep anchors & Tnx0VDAN1Pnx2\\
Idiom with V, N, and Prep anchors & Tnx0VN1Pnx2\\
Idiom with V, A, N, and Prep anchors & Tnx0VAN1Pnx2\\
Tr/Intr Resultative with V-Adj anchors & TRnx0Vnx1A2\\
Tr/Intr Resultative with V-Prep anchors & TRnx0Vnx1Pnx2\\
Ergative Resultative with V-Adj anchors & TREnx1VA2\\
Ergative Resultative with V-Prep anchors & TREnx1VPnx2\\
\end{tabular}
\normalsize

\clearpage















