\chapter{Adjunct Clauses}
\label{adjunct-cls}

In each tree family, there is a pair of indicative adjunct clause
trees, exemplified below in Figure \ref{hunt-trans} with a transitive verb.

\begin{figure}[htb]
\centering
\begin{tabular}{ccc}
\psfig{figure=ps/sent-adjs-files/s-init-hunts.ps,height=2.1in}&
\hspace{1.0in}&
\psfig{figure=ps/sent-adjs-files/s-final-hunts.ps,height=2.1in}\\
(a) & &(b)
\end{tabular}
\caption{Auxiliary Trees for Sentence Initial: $\beta$nx0Vnx1s (a) and Sentence
Final: $\beta$vxnx0Vnx1 (b) Adjunct Clauses}
\label{hunt-trans}
\label{2;23,1}
\label{2;24,1}
\end{figure}

Sentence-initial adjuncts adjoin at the root S of the matrix clause
(Figure~\ref{hunt-trans}(a)), while sentence-final adjuncts adjoin at a VP node
(Figure~\ref{hunt-trans}(b)). In this, the XTAG analysis follows the findings
on the attachment sites of adjunct clauses for conditional clauses
(\cite{iatridou91}) and for infinitival clauses (\cite{Browning87}). One
compelling argument is based on Binding Condition C effects.  As can
be seen from examples (\ex{1})-(\ex{3}) below, no Binding Condition
violation occurs when the adjunct is sentence initial, but the
subject of the matrix clause clearly governs the adjunct clause when
it is in sentence final position and co-indexation of the pronoun with
the subject of the adjunct clause is impossible.

\enumsentence{Unless she$_i$ hurries, Mary$_i$ will be late for the meeting.}
\enumsentence{$\ast$She$_i$ will be late for the meeting unless Mary$_i$ hurries.}
\enumsentence{Mary$_i$ will be late for the meeting unless she$_i$ hurries.}


Tree families with direct objects also contain a pair for the passive trees,
and the transitive family (Tnx0Vnx1) contains a pair for the ergative
trees. All of these trees are anchored by the main verb of the adjunct clause,
and adjoin either at S or VP to the matrix clause.  Subordinating conjunctions
adjoin to these sentential adjunct trees, as described in section
\ref{sub-conj} below.  If no conjunction adjoins, only certain modes are
licensed for the adjunct clause.  These are described immediately below.

\section{``Bare'' Adjunct Clauses}

As described in Chapter~\ref{scomps-section} on sentential complements and
complementizers, the features {\bf $<$mode$>$} and {\bf $<$assign-comp$>$} are
used to control the occurrence of complementizers with the various clause
types.  This same mechanism is used here to ensure the correct distribution of
`bare' (i.e. conjunction-less) adjunct clauses.  Three values of {\bf
$<$mode$>$} are licensed: {\bf ind} (indicative), {\bf inf} (infinitive) and
{\bf ger} (gerundive). They interact with complementizers as follows:

\begin{itemize}
\item Participial complements do not license any
complementizers:\footnote{While these sound a bit like extraposed
relative clauses (see \cite{kj87}), those move only to the right and
adjoin to S; as these clauses are equally grammatical both
sentence-initially and sentence-finally, we are analyzing them as
adjunct clauses.}

\begin{itemize}
\enumsentence{ [Destroyed by the fire], the building still stood.}
\enumsentence{The fire raged for days [destroying the building].}
\enumsentence{$\ast$[That destroyed by the fire], the building
still stood.}
%what about: if destroyed by fire, the building would have been rebuilt?
\end{itemize}

\begin{figure}[htb]
\begin{tabular}{cc}
\psfig{figure=/ps/sent-adjs-files/destroyed-by-fire.ps,height=2.7in}&
\psfig{figure=/ps/sent-adjs-files/destroying-the-building.ps,height=2.7in}\\
(a)&(b)
\end{tabular}
\caption{Sample Participial Adjuncts}
\label{destroyed}
\end{figure}

\item Infinitival adjuncts, including purpose clauses, are licensed both with and without the complementizer
{\it for}.
\begin{itemize}
\enumsentence{Harriet bought a Mustang [to impress Eugene].}
\enumsentence{[To impress Harriet], Eugene dyed his hair.}
\enumsentence{Traffic stopped [for Harriet to cross the street].}
\end{itemize}

\item Indicative adjuncts are only licensed with the complementizer {\it
that}; this construction is rather archaic sounding, but we elected to
allow it for the sake of completeness. 

\begin{itemize}
\enumsentence{He died [that others may live].}
\enumsentence{$\ast$He died [others may live].}
\end{itemize}
\end{itemize}

\section{Clauses Conjoined with Subordinating Conjunctions}
\label{sub-conj}
Subordinating conjunctions anchor one of the four auxiliary trees shown in
Figure~\ref{conjs}.\footnote{There is some amount of overlap between
subordinating conjunctions and prepositions. Items which were already in the
grammar as prepositions were not added as subordinating conjunctions where this
would have resulted in duplicate analyses.} The tree in Figure~\ref{conjs}(a)
is selected by a great majority of subordinating conjunctions.
Figure~\ref{conjs}(b) is anchored by multi-word conjunctions.\footnote{UCONJ
means `unanalyzed' conjunction, i.e. both words are not conjunctions
themselves, but together they form a complex subordinating conjunction.}  The
list of multi-word conjunctions was extracted from \cite{quirk85}, and includes
{\it as if}, {\it in order}, and {\it for all (that)}. The remaining two trees,
seen in Figures~\ref{conjs}(c) and \ref{conjs}(d), handle the three word
conjunctions in (\ex{1}) and (\ex{2}) respectively. Thus, the former has two
Conj anchors and an adverb substitution node, while the latter has three
anchors.  This multi-anchor treatment is very similar to that proposed for
idioms in \cite{AS89}, and the analysis of light verbs in the XTAG grammar (see
section~\ref{nx0lVN1-family}).

\enumsentence{{\it as recently/quickly/}etc. {\it as} + indicative complement}
\enumsentence{{\it as soon as} + participial complement}

\begin{figure}[htb]
\centering
\begin{tabular}{ccc}
\psfig{figure=ps/sent-adjs-files/betaCONJs.ps,height=2.7in}&
\hspace*{0.5in} &
\psfig{figure=ps/sent-adjs-files/betaUCONJUCONJs.ps,height=1.3in}\\
(a)&\hspace*{0.5in} &(b)\\

&& \\

\psfig{figure=ps/sent-adjs-files/betaCONJarbCONJs.ps,height=1.3in}&
\hspace*{0.5in} & 
\psfig{figure=ps/sent-adjs-files/betaCONJARBCONJs.ps,height=1.3in}\\
(c)&\hspace*{0.5in} &(d)
\end{tabular}
\caption{Trees Anchored by Subordinating Conjunctions: $\beta$CONJs (a),
$\beta$UCONJUCONJs (b), $\beta$CONJarbCONJs (c) and $\beta$CONJARBCONJs}
\label{conjs}
\end{figure}


Each of these trees adjoins at the interior S$_r$ node of the S and VP
sentential adjunct trees described above and shown in Figure \ref{hunt-trans}.
Subordinating conjunctions are grouped into classes, based on the type of
clause to which they may adjoin and whether they allow a complementizer to also
adjoin to the clause.  Each class instantiates a value for the {\bf
$<$sub-conj$>$} feature at the root S, which prevents subordinating
conjunctions from stacking.  They also instantiate values of the {\bf
$<$mode$>$} and {\bf $<$comp$>$} features of the foot S.  The {\bf $<$mode$>$}
value constrains the types of clauses the subordinating conjunction may adjoin
to and the {\bf $<$comp$>$} value constrains the complementizers which may
adjoin below it.  These classes are:

\begin{itemize}
\item IND1: Indicative clause with optional {\it that} complementizer,
e.g. {\it in order}, {\it so}.
\begin{itemize}
\enumsentence{He died so (that) others could live.} %tested
% weird: given
\end{itemize}

\item IND2: Indicative clause, no complementizer possible, e.g. {\it
in case}, {\it because}.
\begin{itemize}
\enumsentence{Because Bill ate their lettuce the rabbits are sad.} %tested
\enumsentence{$\ast$Because that Bill ate their lettuce the rabbits are sad.} %tested
\end{itemize}

\item IND3: Asymmetric versions of coordinating conjuncts {\it and}
and {\it but}; indicative clause, no complementizer possible, only
allowed in sentence-final clausal adjunct trees.
\begin{itemize}
\enumsentence{Paddington opened the closet and his galoshes were inside.} %tested
\enumsentence{$\ast$And his galoshes were inside Paddington opened the closet.}  %tested
\end{itemize}

\item INF1: Infinitival clause, no complementizer; only {\it so as}, {\it
as if} and {\it as though}.
\begin{itemize}
\enumsentence{As if he had planned it, the door suddenly opened.} %tested
\enumsentence{$\ast$As if for Bill he had planned it, the door suddenly opened.} %tested
\end{itemize}

\item INF2: Infinitival clause optional {\it for} complementizer; only {\it
in order}.
\begin{itemize}
\enumsentence{Max picked the lettuce in order to eat it.}%tested
\enumsentence{Max picked the lettuce in order for us to eat it.}%tested
\end{itemize}

\item GER: Participial ({\bf $<$mode$>$=ger} or {\bf ppart}) complement, no
complementizer possible, e.g. {\it although}, {\it even if}, {\it
when}. 
\begin{itemize}
\enumsentence{Drawn recently, the pictures are valuable.}%tested
\enumsentence{Max ate spinach, impressing Mary.}%tested
\end{itemize}

\end{itemize}

These auxiliary trees are also used to do `discourse' coordination, as in
sentence (\ex{1}).  All subordinating conjunctions which can conjoin indicative
clauses may also adjoin to root matrix sentences, as seen in the derived tree
in Figure~\ref{seuss-sentence}.

\enumsentence{And Truffula trees are what everyone needs! \cite{seuss71}}

\begin{figure}[htb]
\centering
\hspace{0in}
\psfig{figure=ps/sent-adjs-files/disc-conj.ps,height=4.5in}
\caption{Example of discourse conjunction, from Seuss' {\it The
Lorax}\nocite{seuss71}}
\label{seuss-sentence}
\end{figure}

%s{\it and} as an asymmetric as well as coordinating conjunction.





