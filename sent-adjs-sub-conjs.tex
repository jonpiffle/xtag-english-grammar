\chapter{Adjunct Clauses}
\label{adjunct-cls}
\label{sub-conj}

Adjunct clauses include subordinate clauses (i.e. those with overt
subordinating conjunctions), purpose clauses and participial adjuncts.

Subordinating conjunctions each select four trees, allowing them to
appear in four different positions relative to the matrix clause.  The
positions are (1) before the matrix clause, (2) after the matrix
clause, (3) before the VP, surrounded by two punctuation marks, and
(4) after the matrix clause, separated by a punctuation mark. Each of
these trees is shown in Figure \ref{sub-conj-trees}.

\begin{figure}[htb]
\centering
\begin{tabular}{cccc}
\psfig{figure=ps/sent-adjs-files/Pss.ps,height=2.1in}&
\psfig{figure=ps/sent-adjs-files/vxPNs.ps,height=2.1in}&
\psfig{figure=ps/sent-adjs-files/puPPpuvx.ps,height=2.1in}&
\psfig{figure=ps/sent-adjs-files/spuPs.ps,height=2in}\\
(1) $\beta$Pss & (2) $\beta$vxPNs & (3) $\beta$puPPspuvx & (4) $\beta$spuPs \\
\end{tabular}
\caption{Auxiliary Trees for Subordinating Conjunctions}
\label{sub-conj-trees}
\end{figure}

Sentence-initial adjuncts adjoin at the root S of the matrix clause,
while sentence-final adjuncts adjoin at a VP node. In this, the XTAG
analysis follows the findings on the attachment sites of adjunct
clauses for conditional clauses (\cite{iatridou91}) and for
infinitival clauses (\cite{Browning87}). One compelling argument is
based on Binding Condition C effects.  As can be seen from examples
(\ex{1})-(\ex{3}) below, no Binding Condition violation occurs when
the adjunct is sentence initial, but the subject of the matrix clause
clearly governs the adjunct clause when it is in sentence final
position and co-indexation of the pronoun with the subject of the
adjunct clause is impossible.

\enumsentence{Unless she$_i$ hurries, Mary$_i$ will be late for the meeting.}
\enumsentence{$\ast$She$_i$ will be late for the meeting unless Mary$_i$ hurries.}
\enumsentence{Mary$_i$ will be late for the meeting unless she$_i$ hurries.}

%Tree families with direct objects also contain a pair for the passive trees,
%and the transitive family (Tnx0Vnx1) contains a pair for the ergative
%trees. All of these trees are anchored by the main verb of the adjunct clause,
%and adjoin either at S or VP to the matrix clause.  Subordinating conjunctions
%adjoin to these sentential adjunct trees, as described in section
%\ref{sub-conj} below.  If no conjunction adjoins, only certain modes are
%licensed for the adjunct clause.  These are described immediately below.

We had previously treated subordinating conjunctions as a subclass of
{\em conjunction}, but are now assigning them the POS {\em
preposition}, as there is such clear overlap between words that
function as prepositions (taking NP complements) and subordinating
conjunctions (taking clausal complements). While there are some
prepositions which only take NP complements and some which only take
clausal complements, many take both as shown in examples
(\ex{1})-(\ex{4}), and it seems to be artificial to assign them two
different parts-of-speech.

\enumsentence{Helen left before the party.}
\enumsentence{Helen left before the party began.}
\enumsentence{Since the election, Bill has been elated.}
\enumsentence{Since winning the election, Bill has been elated.}

Each subordinating conjunction selects the values of the {\bf
$<$mode$>$} and {\bf $<$comp$>$} features of the subordinated S. The
{\bf $<$mode$>$} value constrains the types of clauses the
subordinating conjunction may appear with and the {\bf $<$comp$>$}
value constrains the complementizers which may adjoin to that
clause. For instance, indicative subordinate clauses may appear with
the complementizer {\it that} as in (\ex{1}), while participial
clauses may not have any complementizers (\ex{2}).

\enumsentence{Midge left that car so that Sam could drive to work.}
\enumsentence{*Since that seeing the new VW, Midge could think of
nothing else.}

\subsection{Multi-word Subordinating Conjunctions}

We extracted a list of multi-word conjunctions, such as {\it as if},
{\it in order}, and {\it for all (that)}, from \cite{quirk85}. For the
most part, the components of the complex are all anchors, as shown in
Figures~\ref{conjs}(a). In one case, {\it as ADV as}, there is a great
deal of latitude in the choice of adverb, so this is a substitution
site (Figures~\ref{conjs}(b)). This multi-anchor treatment is very
similar to that proposed for idioms in \cite{AS89}, and the analysis
of light verbs in the XTAG grammar (see section~\ref{nx0lVN1-family}).

\begin{figure}[htb]
\centering
\begin{tabular}{ccc}
\psfig{figure=ps/sent-adjs-files/vxPARBPs.ps,height=2.7in}&
\hspace*{0.5in} &
\psfig{figure=ps/sent-adjs-files/vxParbPs.ps,height=2.7in}\\
(a)&\hspace*{0.5in} &(b)\\
\end{tabular}
\caption{Trees Anchored by Subordinating Conjunctions:  $\beta$vxPARBPs and $\beta$vxParbPs}
\label{conjs}
\end{figure}

\section{``Bare'' Adjunct Clauses}

``Bare'' adjunct clauses do not have an overt subordinating
conjunction, but are typically parallel in meaning to clauses with
subordinating conjunctions. For this reason, we have elected to handle
them using the same trees shown above, but with null anchors. They are
selected at the same time and in the same way the {\it PRO} tree is,
as they all have {\it PRO} subjects.  Three values of {\bf $<$mode$>$}
are licensed: {\bf inf} (infinitive), {\bf ger} (gerundive) and {\bf
ppart} (past participal).\footnote{We considered allowing bare
indicative clauses, such as {\it He died that others may live}, but
these were considered too archaic to be worth the additional ambiguity
they would add to the grammar.} They interact with complementizers as
follows:

\begin{itemize}
\item Participial complements do not license any
complementizers:\footnote{While these sound a bit like extraposed
relative clauses (see \cite{kj87}), those move only to the right and
adjoin to S; as these clauses are equally grammatical both
sentence-initially and sentence-finally, we are analyzing them as
adjunct clauses.}

\enumsentence{[Destroyed by the fire], the building still stood.}
\enumsentence{The fire raged for days [destroying the building].}
\enumsentence{$\ast$[That destroyed by the fire], the building
still stood.}
%what about: if destroyed by fire, the building would have been rebuilt?

\begin{figure}[htb]
\begin{tabular}{cc}
\psfig{figure=ps/sent-adjs-files/destroyed-by-fire.ps,height=2.7in}&
\psfig{figure=ps/sent-adjs-files/destroying-the-building.ps,height=2.7in}\\
(a)&(b)
\end{tabular}
\caption{Sample Participial Adjuncts}
\label{destroyed}
\end{figure}

\item Infinitival adjuncts, including purpose clauses, are licensed both with and without the complementizer
{\it for}.
\enumsentence{Harriet bought a Mustang [to impress Eugene].}
\enumsentence{[To impress Harriet], Eugene dyed his hair.}
\enumsentence{Traffic stopped [for Harriet to cross the street].}
\end{itemize}

\section{Discourse Conjunction}

The CONJs auxiliary tree is used to handle `discourse' conjunction,
as in sentence (\ex{1}).  Only the coordinating conjunctions ({\it
and, or} and {\it but}) are allowed to adjoin to the roots of
matrix sentences. Discourse conjunction with {\it and} is shown in the
derived tree in Figure~\ref{seuss-sentence}.

\enumsentence{And Truffula trees are what everyone needs! \cite{seuss71}}

\begin{figure}[htbp]
\centering
\hspace{0in}
\psfig{figure=ps/sent-adjs-files/disc-conj.ps,height=4.5in}
\caption{Example of discourse conjunction, from Seuss' {\it The
Lorax}\protect\nocite{seuss71}}
\label{seuss-sentence}
\end{figure}





