
\chapter{Imperatives}
\label{imperatives}

\section{Agreement, mode, and the null subject}

Imperatives in English do not require overt subjects.  These subjects, whether
overt or not, are generally interpreted as second person, as is clear from the
verbal agreement and the interpretation. 
Imperatives with non-overt subjects are handled by the
imperative trees discussed in this section.  Imperatives with overt
subjects are not currently handled.  More discussion on imperatives with
overt subjects is given in Sections \ref{sec:vocative} and
\ref{sec:overt-subject}.

The imperative tree(s) in each tree family in the English XTAG grammar are
identical to the declarative tree(s) of that family except that the NP$_{0}$
subject position is filled by an $\epsilon$, the NP$_{0}$ {\bf
$<$agr~pers$>$} feature is set to the value {\bf 2nd}, and the {\bf
$<$mode$>$} feature on the root node has the value {\bf imp} (see equations
\ex{1} -- \ex{2}). Hardwiring the {\bf $<$agr~pers$>$} feature into the
tree ensures the proper verbal agreement for an imperative.  The {\bf
$<$mode$>$} value of {\bf imp} on the root node is recognized as a valid
mode for a matrix clause.\footnote{%
%
The other valid {\bf $<$mode$>$} for a matrix clause is {\bf ind}.%
%
} The {\bf imp} value for {\bf $<$mode$>$} also prevents imperatives from
appearing as embedded clauses.  Figure \ref{alphaInx0Vnx1} shows the
imperative tree in the transitive tree family.

\enumsentence{{\bf NP$_0$.t:$<$agr~pers$>$ = 2}}
\enumsentence{{\bf S$_r$.b:$<$mode$>$ = imp}}


\begin{figure}[htbp]
\centering{
\begin{tabular}{c}
\psfig{figure=ps/imperatives-files/alphaInx0Vnx1.ps,height=6in}
\end{tabular}
}
\caption{Transitive imperative tree: $\alpha$Inx0Vnx1}
\label{alphaInx0Vnx1}
\label{2;11,1}
\end{figure}


Moreover, the {\bf $<$mode$>$} feature on the anchor is unspecified, and
the {\bf $<$mode$>$} feature on the top feature structure associated with
the VP has the value {\bf base} (see equation in \ex{1}).

\enumsentence{{\bf VP.t:$<$mode$>$ = base}}

This allows the lexical verb of the imperative to be any type of verb, as
long as the left-most verb has {\bf $<$mode$>$ = base}.  For instance, in a
simple transitive imperative as in \ex{1}, the verb {\it eat}, which is
specified with {\bf $<$mode$>$ = base}, anchors the imperative tree,
unifying with {\bf VP.t:$<$mode$>$ = base}.  In an imperative with
auxiliary {\it be} as in \ex{2}, the verb {\it waiting}, which is specified
with {\bf $<$mode$>$ = ger}, anchors the imperative tree, and the auxiliary
{\it be}, which is specified with {\bf $<$mode$>$ = base}, adjoins onto the
VP, unifying with {\bf VP.t:$<$mode$>$ = base}.

\enumsentence{Eat the cake!}
\enumsentence{Be waiting for me!}


\section{Negative Imperatives}
\label{neg-imp}

\subsection{{\it Don't} imperatives}

All Negative imperatives in English require {\it do}-support, even those
that are formed with {\it be} and auxiliary {\it have}.

\enumsentence{Don't leave!}
\enumsentence{*Not leave!}

\enumsentence{Do not eat the cake!}
\enumsentence{*Not eat the cake!}

%\enumsentence{Do not be talking so loud!}
%\enumsentence{*Not be talking so loud!}

\enumsentence{Don't have eaten everything when the guests arrive!}
\enumsentence{*Not have eaten everything when the guests arrive!}
\enumsentence{*Have not eaten everything when the guests arrive!}

In English XTAG grammar, negative imperatives receive a similar structural analysis
to {\it yes-no} questions, as in \cite{potsdamdiss97} and \cite{handiss}.
That is, {\it do} and {\it don't} in negative imperatives are treated as an
instance of {\it do}-support and adjoin to a clause.  The crucial
structural evidence for this analysis is that when there is an overt subject
in negative imperatives formed with {\it don't}, the subject must follow
{\it don't}, just as it does in {\it yes-no} questions.

%\enumsentence{Don't you worry!}
\enumsentence{Don't you move!}

\enumsentence{Don't you like carrots?}
%\enumsentence{Didn't you finish your paper yet?}

{\it Do}-support in negative imperatives is handled by the elementary tree
$\beta$IVs anchored by {\it do} and {\it don't}, as shown in Figure
\ref{fig:doimp}.  This tree adjoins onto the root node of the imperative
tree.  The feature {\bf $<$mode$>$ = imp} on the S foot node restricts this
tree to adjoin only to imperative trees. Furthermore, the S root node of
$\beta$IVs is specified with {\bf $<$mode$>$ = imp}, which prevents
imperatives with {\it do}-support from appearing as embedded clauses.
 
\begin{figure}[htbp]
\centering
\begin{tabular}{ccc}
{\psfig{figure=ps/imperatives-files/betaIVs-do.ps,height=10cm}} &
{\ } & 
{\psfig{figure=ps/imperatives-files/betaIVs-dont.ps,height=10cm}} \\
$\beta$IVs[do] & {\ } & $\beta$IVs[don't] 
\end{tabular}
\caption{Trees anchored by imperative {\it do} and {\it don't}}
\label{fig:doimp}
\end{figure}
 
In negative imperatives formed with {\it don't}, the $\beta$IVs[don't] tree in
Figure \ref{fig:doimp} adjoins to the root node of the imperative tree. The
derived tree for the negative imperative {\it Don't leave!} is given in
Figure \ref{fig:dont-leave}.

\begin{figure}[htbp]
  \begin{center} \leavevmode \psfig{figure=ps/imperatives-files/derived_dont-leave.ps,height=8cm}
  \end{center}
  \caption{Derived tree for {\it Don't leave!}}
\label{fig:dont-leave}
\end{figure} 

\subsection{{\it Do not} imperatives}

In negative imperatives formed with {\it do not}, the $\beta$IVs[do] tree
in Figure \ref{fig:doimp} adjoins to the root node of the imperative tree
and the $\beta$NEGvx tree that anchors {\it not}, shown in Figure
\ref{fig:not}, adjoins to the VP node of the imperative tree.

\begin{figure}[htbp]
  \begin{center} \leavevmode
\psfig{figure=ps/imperatives-files/betaNEGvx_not_.ps,height=10cm} 
  \end{center}
  \caption{Tree anchored by {\it not}}
\label{fig:not}
\end{figure} 

The  derived tree for the negative imperative {\it Do not eat the cake!} is
given in Figure \ref{fig:do-not-leave}.    


\begin{figure}[htbp]
\begin{center} \leavevmode 
{\psfig{figure=eps/imperatives-files/do-not-eat.eps,height=11cm}} 
\end{center}
\caption{Derived trees for {\it Do not eat the cake!}}
\label{fig:do-not-leave}
\end{figure} 

%Note that  trees in Figure \ref{fig:dont-leave} and Figure
%\ref{fig:do-not-leave} have an empty verb.  This is due to the feature
%{\bf $<$displ-const$>$} in $\beta$IVs.\footnote{The other possibility of
%adjoining {\it not} above the VP that projects from the empty verb is ruled
%out because {\it not} is made to select VP with the following equation:
%{\bf $<$displ-const set1=-$>$}.  Since the empty verb tree has {\bf
%$<$displ-const set1=+$>$}, {\it not} cannot adjoin onto the VP that
%projects from it.} This feature ensures that when $\beta$IVs is adjoined to
%an elementary tree, $\beta$Vvx that anchors an empty verb must also adjoin
%onto the VP of that same elementary tree.  This tree is represented in
%Figure \ref{fig:epsilon}.  The empty verb represents the originating
%position of {\it do} and {\it don't}.  This mechanism is also used in
%interrogatives that have subject-verb inversion to simulate auxiliary verb
%movement.  For more on this, see Section~\ref{auxs.dosupport} on {\it
%do}-support and inversion.

%\begin{figure}[htbp]
%  \begin{center} \leavevmode % 
%  \psfig{figure=ps/imperatives-files/betaVvx-epsilon.ps,height=12cm} 
%  \end{center}
%  \caption{$\beta$Vvx[$\epsilon$]}
%\label{fig:epsilon}
%\end{figure} 

If {\it do} in negative imperatives is in the same position as {\it do} in
{\it yes-no} questions, the fact that an overt subject cannot intervene
between {\it do} and {\it not} is puzzling.

\enumsentence{Do not open the window!}
\enumsentence{*Do you not open the window!}

We adopt the account given in \cite{akmajian84} that this fact is not due
to syntax but due to an intonational constraint in imperatives.  He argues
that (i) when an imperative sentence has an overt subject, the subject must be
the only intonation center preceding the verb phrase, and (ii) in
negative imperatives with {\it do} and {\it not}, either {\it do} or {\it
not} must be the intonation center.  These two contraints conspire to rule
out {\it do not} imperatives with an overt subject. 
Since the constraint ruling out \ex{0} is not taken to be a syntactic one, the
sentence is not ruled out by the XTAG grammar.

\subsection{Negative Imperatives with {\it be} and {\it have}}

Another puzzling fact that needs to be explained is that negative imperatives
require {\it do}-support even with {\it be} and auxiliary {\it have}, unlike
negative declaratives \ex{1}-\ex{2} and negative questions \ex{3}-\ex{4} where
{\it do} cannot co-occur with {\it have} or {\it be}.

\enumsentence{He isn't talking loud.}
\enumsentence{*He doesn't be talking loud.}

\enumsentence{Isn't he talking loud?}
\enumsentence{*Doesn't he be talking loud?}

%This fact does not pose a problem for the XTAG analysis of negative
%imperatives if we adopt the line of approach given in \cite{handiss}.  

\cite{handiss} accounts for this fact in terms of V to I movement. She
notes that while declaratives and questions are tensed, imperatives are
not. She argues that this tense difference is responsible for why negative
imperatives require {\it do}-support even for {\it be} and auxiliary {\it
have}.  Assuming a clause structure in which CP dominates IP and IP dominates
VP, she argues that it is the tense features in I$^0$
that attract {\it be} and auxiliary {\it have} in declaratives and questions.
In declaratives, {\it be} or auxiliary {\it have} moves to and stays in I$^0$,
and in questions, once {\it be} or auxiliary {\it have} moves to I$^0$, it can
further move to C$^0$.  Moreover, main verbs cannot move at all to I$^0$ in the
overt syntax.  Instead, they undergo movement at LF.  But negation blocks LF
movement and so as a last resort {\it do} is inserted in I$^0$ to support INFL.
In imperatives, I$^0$ does not have tense features and so it cannot attract
{\it be} and auxiliary {\it have}.  Thus, {\it be} and auxiliary {\it have} as
well as main verbs undergo movement at LF in imperatives.  And so in negative
imperatives, since negation blocks LF verb movement, {\it do} is inserted in
I$^0$ as a last resort device even for {\it be} and auxiliary {\it have} and it
further moves to C$^0$ in the overt syntax.

In the XTAG grammar we do not represent V to I movement or LF movement, and so
our analysis cannot directly follow that of \cite{handiss}. Instead, we derive
the facts by having a separate entry for {\it do} (which anchors a separate
auxiliary tree $\beta$IVs) which is used only with imperatives. Unlike the
other entry for {\it do}, it does not require the verb below it to be {\bf
$<$mainv = $+>$ }. Thus, this {\it do} can co-occur with the verbs {\it have}
and {\it be} which have the feature {\bf $<$mainv = $->$ }

\section{Emphatic Imperatives}

Another case where imperatives have {\it do}-support is emphatic
imperatives.

\enumsentence{Do open the window!}
\enumsentence{Do show up for the lecture!}

In English XTAG grammar, {\it do} in emphatic imperatives is treated just as
{\it do} in negative imperatives.  It is adjoined to the S node of an
imperative clause with an empty subject.  Again, the crucial evidence for this
analysis comes from word order facts.  When emphatic imperatives have an overt
subject, it must follow {\it do}.

\enumsentence{Do somebody bring me some water!}
\enumsentence{Do at least some of you show up for the lecture!}

\section{Cases not handled}

\subsection{Overt subjects before {\it do/don't}}
\label{sec:vocative}

Given our analsis of negative imperatives, if the subject precedes {\it do}
or {\it don't}, we are forced to treat it as a vocative rather than as a true subject.  Vocatives are considered to be outside the clause
structure and do not have any structural relation with any element in the
clause.

\enumsentence{You don't drink the water. = (You! Don't drink the water!)}
\enumsentence{You do not leave the room. = (You! Do not leave the room!)}

Given the fact that the imperatives in \ex{-1} and \ex{0} seem to be degraded
unless there is an intonational break between {\it you} and the rest of the
sentence, treating {\it you} as a vocative seems to be the correct approach.
Currently, however, the XTAG grammar does not handle vocatives.

\subsection{Overt subjects after {\it do/don't}}
\label{sec:overt-subject}

One remaining task for imperatives is to handle those with overt subjects
such as the examples in \ex{-3} and \ex{-2}.
The type of overt subjects allowed in imperatives is restricted to 2nd
person pronouns and some quantified noun phrases. Currently, the English
XTAG grammar only has imperative trees with empty subjects.  

\subsection{Passive Imperatives}

Passive imperatives such as \ex{1} are currently not handled.

\enumsentence{Don't be defeated at the race today!}

One way to account for these would be to make separate passive imperative trees
in each tree family, as is done for the declaratives and otehr clause types in
each family. 

Another way to allow for passive imperatives would be to remove the equation
{\bf V.t:$<$passive$> = -$ from the imperative trees. However, this option may
affect the consistency of the treatment of passives. A decision in this respect
will have to be made before implementing imperative passives.

\subsection{Overgeneration}

As discussed above, imperatives can also be formed with auxiliaries
like {\it have} and {\it be}, as in \ex{1} and \ex{2}.

\enumsentence{Don't have fallen asleep when I come back!}
\enumsentence{Be waiting for me when I return!}
%\enumsentence{Don't be sleeping while reading your book!}


The auxiliary {\it be} can form affirmative as well as negative
imperatives. However, {\it have} can only form a negative imperative, as
can be seen from \ex{0} and the ungrammaticality of \ex{1}:

\enumsentence{* Have eaten your meal by the time I return!}

The current analysis of imperative, however, does not rule out \ex{0}.










