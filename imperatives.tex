\section{Imperatives}
Imperatives in English do not require overt subjects.  The subject in
imperatives is second person, i.e. {\it you}, whether it is overt or
not as is clear from the verbal agreement and the interpretation.
Imperatives with overt subjects can be  parsed using the trees already
needed for declaratives.  The imperative cases in which the subject is
not overt are handled by the imperative trees discussed in this section.

The imperative trees in English LTAG grammar are identical to the
declarative tree except that the NP$_{0}$ subject position is filled
by an $\epsilon$, the NP$_{0}$ agr-pers feature is set in the tree to
the value 2 and the mode feature on the root node has the value imp.
The value for agr-pers that comes from the epsilon node insures the
proper verbal agreement for an imperative.  The mode value of imp on
the root node is recognized as a valid mode for a matrix clause.  The
imp value for mode also allows imperatives to be blocked from
appearing as embedded clauses.  Figure \ref{Inx0Vnx1} is
the imperative tree for the transitive tree family.

\begin{figure}[htbp]
\center{
\psfig{figure=/mnt/linc/extra/xtag/work/doc/tech-rept/ps/alphaInx0Vnx1.ps,height=10.0cm}
\caption{ \label{{alphaInx0Vnx1} Tree:  $\alpha$Inx0Vnx1}
}
\end{figure}
