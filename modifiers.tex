\chapter{Modifiers}
\label{modifiers}

This chapter covers various types of modifiers: adverbs, prepositions,
adjectives, and noun modifiers in noun-noun compounds.\footnote{Relative
clauses are discussed in Chapter~\ref{rel_clauses}.}  These categories
optionally modify other lexical items and phrases by adjoining onto them.  In
their modifier function these items are adjuncts; they are not part of the
subcategorization frame of the items they modify.  Examples of some of these
modifiers are shown in (\ex{1})-(\ex{3}).

\enumsentence{[$_{ADV}$ certainly $_{ADV}$], the October 13 sell-off
didn't settle any stomachs . (WSJ)}

\enumsentence{Mr. Bakes [$_{ADV}$ previously $_{ADV}$] had a turn at running
Continental . (WSJ)}
pp
\enumsentence{most [$_{ADJ}$ foreign $_{ADJ}$] [$_{N}$ government
$_{N}$] [$_{N}$ bond $_{N}$] [prices] rose [$_{PP}$ during the week
$_{PP}$]. }

The trees used for the various modifiers are quite similar in form.  The
modifier anchors the tree and the root and foot nodes of the tree are of the
category that the particular anchor modifies. Some modifiers,
e.g. prepositions, have arguments that are also included in the tree.  The foot
node may be to the right or the left of the anchoring modifier (and its
arguments) depending on whether that modifier occurs before or after the
category it modifies. For example, almost all adjectives appear to the left of
the nouns they modify, while prepositions appear to the right when modifying
nouns.


\section{Adjectives}
\label{adj-modifier}

In addition to being modifiers, adjectives in the XTAG English grammar can be
also anchor clauses (see Adjective Small Clauses in
Chapter~\ref{small-clauses}).  There is also one tree family, Intransitive with
Adjective (Tnx0Va1), that has an adjective as an argument and is used for
sentences such as {\it Seth felt happy}. In that tree family the adjective
substitutes into the tree rather than adjoining as is the case for modifiers.


\begin{figure}[htb]
\centering
\begin{tabular}{cc}
{\psfig{figure=ps/modifiers-files/betaAn-features.ps,height=3.5in}}
\end{tabular}\\
\caption {Standard Tree for Adjective modifying a Noun: $\beta$An}
\label {An-tree}
\end{figure}

As modifiers, adjectives anchor the tree shown in Figure~\ref{An-tree}.  The
features of the N node onto which the $\beta$An tree adjoins are passed through
to the top node of the resulting N.  The null adjunction marker (NA) on the N
foot node imposes right binary branching such that each subsequent adjective
must adjoin on top of the leftmost adjective that has already adjoined.  Due to
the NA constraint, a sequence of adjectives will have only one derivation in
the XTAG grammar. The adjective's morphological features such as superlative or
comparative are not currently used in the tree.  At this point, the treatment
of adjectives in the XTAG English grammar does not include selectional or
ordering restrictions. Consequently, any adjective can adjoin onto any noun and
on top of any other adjective already modifying a noun. All of the modified
noun phrases shown in (\ex{1})-(\ex{4}) currently parse with the same structure
shown for {\it colorless green ideas\/} in Figure
\ref{colorless-green-adj}.

\enumsentence{big green bugs}
\enumsentence{big green ideas}
\enumsentence{colorless green ideas}
\enumsentence{$\ast$green big ideas}


\begin{figure}[htb]
\centering
\begin{tabular}{cc}
{\psfig{figure=ps/modifiers-files/colorless-green-ideas.ps,height=2.3in}}
\end{tabular}\\
\caption {Multiple adjectives modifying a noun}
\label {colorless-green-adj}
\end{figure}


While (\ex{-2})-(\ex{0}) are all semantically anomalous, (\ex{0}) also suffers
from an ordering problem that makes it seem ungrammatical as
well.\footnote{This is, in fact, the point of the famous linguistic example in
(\ex{-1}).} We would argue that the grammar should accept (\ex{-3})-(\ex{-1})
but not (\ex{0}).  One of the future goals for the grammar is to develop a
treatment of adjective ordering similar to that developed by
\cite{HockeyEgedi94} for determiners\footnote{See
Chapter~\ref{det-comparitives} or \cite{HockeyEgedi94} for details of the
determiner analysis.}. An adequate implementation of ordering restrictions for
adjectives would rule out (\ex{0}).

Another area in which we plan to have future grammar development is
comparatives.  Comparatives that involve ellipsis will require a general
solution of the problem of representing ellipsis.  Simpler comparatives without
ellipsis, such as {\it fewer than nine\/} in (\ex{1}), should be amenable to
analysis as complex determiners.


\enumsentence{cats actually have fewer than nine lives .}


\section{Noun-Noun Modifiers}
\label{noun-modifier}

Noun-noun compounding in the English XTAG grammar is very similar to
adjective-noun modification.  The noun modifier tree, shown in
Figure~\ref{noun-compound-tree}, has essentially the same structure as the
adjective modifier tree in Figure~\ref{An-tree}, except for the syntactic
category label of the anchor.  

\begin{figure}[htb]
\centering
\begin{tabular}{c}
{\psfig{figure=ps/modifiers-files/betaNn.ps,height=3.5in}}
\end{tabular}
\caption {Noun-noun compounding tree: $\beta$Nn}
\label {noun-compound-tree}
\end{figure}


Noun compounds have a variety of scope possibilities not available to
adjectives, as illustrated by the single bracketing possibility in (\ex{1}) and
the two possibilities for (\ex{2}).  This ambiguity is manifested in the XTAG
grammar by the two possible adjunction sites in the noun-noun compound tree
itself.  Subsequent modifying nouns can adjoin either onto the N$_r$ node or
onto the N anchor node of that tree, which results in exactly the two
bracketing possibilities shown in (\ex{2}).  This inherent structural ambiguity
results in noun-noun compounds regularly having multiple derivations. However,
the multiple derivations are not a defect in the grammar because they are
necessary to correctly represent the genuine ambiguity of these phrases.

\enumsentence{[$_{N}$ big [$_{N}$ green design $_{N}$]$_{N}$]}

\enumsentence{[$_{N}$ computer [$_{N}$ furniture design $_{N}$]$_{N}$]\\
\/~~[$_{N}$ [$_{N}$ computer furniture $_{N}$] design $_{N}$]}

Noun-noun compounds have no restriction on number.  XTAG allows nouns to be either singular or plural as in (\ex{1})-(\ex{3}).
\enumsentence{Hyun is taking an algorithms course .}
\enumsentence{waffles are in the frozen foods section .}
\enumsentence{I enjoy the dog shows .}

\section{Prepositions}
\label{prep-modifier}

There are three basic types of prepositional phrases, and three places
at which they can adjoin.  The three types of prepositional phrases
are: Preposition with NP Complement, Preposition with Sentential
Complement, and Exhaustive Preposition.  The three places are to the
right of an NP, to the right of a VP, and to the left of an S.  Each
of the three types of PP can adjoin at each of these three places, for
a total of nine PP modifier trees. Table \ref{prep-summary} gives the
tree family names for the various combinations of type and
location. (See Section \ref{post-PP} for discussion of the
$\beta$spuPnx, which handles post-sentential comma-separated PPs.)

\begin{table}[htb]
\centering
\begin{tabular}{|l||c|c|c|}
\hline
\multicolumn{1}{|c||}{}&\multicolumn{3}{c|}{position and category modified}\\
\cline{2-4}
\multicolumn{1}{|c||}{}&pre-sentential&post-NP&post-VP\\
\multicolumn{1}{|c||}{Complement type}&S modifier&NP modifier&VP modifier\\
\hline
\hline
S-complement&$\beta$Pss&$\beta$nxPs&$\beta$vxPs\\
\hline
NP-complement&$\beta$Pnxs&$\beta$nxPnx&$\beta$vxPnx\\
\hline
no complement&$\beta$Ps&$\beta$nxP&$\beta$vxP\\
(exhaustive)&&&\\
\hline
\end{tabular}
\caption{Preposition Anchored Modifiers}
\label{prep-summary}
\end{table}

The subset of preposition anchored modifier trees in Figure~\ref{prep-trees}
illustrates the locations and the four PP types.  Example sentences using the 
trees in Figure \ref{prep-trees} are shown in (\ex{1})-(\ex{4}). There are also
more trees with multi-word prepositions as anchors. Examples of these are: 
{\it ahead of}, {\it contrary to}, {\it at variance with} and {\it as recently
as}.

\begin{figure}[htb]
\centering
\begin{tabular}{ccccccc}
{\psfig{figure=ps/modifiers-files/betaPss.ps,height=1.5in}}
& \hspace{.5in} &
{\psfig{figure=ps/modifiers-files/betanxPnx.ps,height=1.5in}}
&  \hspace{.5in} &
{\psfig{figure=ps/modifiers-files/betavxP.ps,height=1.5in}}
&  \hspace{.5in} &
{\psfig{figure=ps/betavxPPnx.ps,height=1.75in}}
\\
$\beta$Pss&&$\beta$nxPnx&&$\beta$vxP&&$\beta$vxPPnx\\
\end{tabular}\\
\caption {Selected Prepositional Phrase Modifier trees:
$\beta$Pss, $\beta$nxPnx, $\beta$vxP and $\beta$vxPPnx}
\label {prep-trees}
\end{figure}

\enumsentence{[$_{PP}$ with Clove healthy $_{PP}$], the veterinarian's
bill will be more affordable . ($\beta$Pss\footnote{{\it Clove healthy} is an adjective small clause})}
\enumsentence{The frisbee [$_{PP}$ in the brambles $_{PP}$] was hidden .
($\beta$nxPnx)}
\enumsentence{Clove played frisbee [$_{PP}$ outside $_{PP}$] . ($\beta$vxP)}
\enumsentence{Clove played frisbee [$_{PP}$ outside of the house
$_{PP}$] . ($\beta$vxPPnx)}

Prepositions that take NP complements assign accusative case to those
complements (see section~\ref{prep-case} for details).  Most prepositions take
NP complements.  There are just a few prepositions that take sentential
complements (see section~\ref{NPA}).


\section{Adverbs}
\label{adv-modifier}

In the English XTAG grammar, VP and S-modifying adverbs anchor the
auxiliary trees $\beta$ARBs, $\beta$sARB, $\beta$vxARB and
$\beta$ARBvx,\footnote{In the naming conventions for the XTAG trees,
ARB is used for {\underline a}dve{\underline {rb}}s.  Because the
letters in A, Ad, and Adv are all used for other parts of speech
({\underline a}djective, {\underline d}eterminer and {\underline
v}erb), ARB was chosen to eliminate ambiguity.
Appendix~\ref{tree-naming} contains a full explanation of naming
conventions.}  allowing pre and post modification of S's and VP's.
Besides the VP and S-modifying adverbs, the grammar includes adverbs
that modify other categories. Examples of adverbs modifying an
adjective, an adverb, a PP, an NP, and a determiner are shown in
(\ex{1})-(\ex{8}). (See Sections \ref{par-adverb} and
\ref{post-adverb} for discussion of the $\beta$puARBpuvx and
$\beta$spuARB, which handle pre-verbal parenthetical adverbs and
post-sentential comma-separated adverbs.)

\begin{itemize}
\item{Modifying an adjective}
\enumsentence{{\bf extremely} good}
\enumsentence{{\bf rather} tall}
\enumsentence{rich {\bf enough}}

\item{Modifying an adverb}
\enumsentence{oddly {\bf enough}} 
\enumsentence{{\bf very} well}

\item{Modifying a PP}
\enumsentence{{\bf right} through the wall}

\item{Modifying a NP}
\enumsentence{{\bf quite} some time}

\item{Modifying a determiner}
\enumsentence{{\bf exactly} five men}

\end{itemize}

XTAG has separate trees for each of the modified categories and for pre and
post modification where needed.  The kind of treatment given to adverbs here is
very much in line with the base-generation approach proposed by \cite{Ernst84},
which assumes all positions where an adverb can occur to be base-generated, and
that the semantics of the adverb specifies a range of possible positions
occupied by each adverb. While the relevant semantic features of the adverbs
are not currently implemented, implementation of semantic features is scheduled
for future work.  The trees for adverb anchored modifiers are very similar in
form to the adjective anchored modifier trees.  Examples of two of the basic
adverb modifier trees are shown in Figure~\ref{adv-trees}.

\begin{figure}[hb]
\centering
\begin{tabular}{ccc}
{\psfig{figure=ps/modifiers-files/betaARBs.ps,height=4.5in}}&
\hspace*{1.0in}&
{\psfig{figure=ps/modifiers-files/betavxARB.ps,height=4in}}\\
(a)&&(b)\\
\end{tabular}
\caption {Adverb Trees for pre-modification of S: $\beta$ARBs (a) and
post-modification of a VP: $\beta$vxARB (b)}
\label{adv-trees}
\end{figure}

\newpage

Like the adjective anchored trees, these trees also have the NA
constraint on the foot node to restrict the number of derivations
produced for a sequence of adverbs.  Features of the modified category
are passed from the foot node to the root node, reflecting correctly
that these types of properties are unaffected by the adjunction of an
adverb.  A summary of the categories modified and the position of
adverbs is given in Table \ref{adv-summary}.

\begin{table}[h]
\centering
\begin{tabular}{|c||c|c|}
\hline
&\multicolumn{2}{c|}{Position with respect to item modified}\\
\cline{2-3}
Category Modified&Pre&Post\\
\hline
\hline
S&$\beta$ARBs&$\beta$sARB\\
\hline
VP&$\beta$ARBvx,$\beta$puARBpuvx&$\beta$vxARB\\
\hline
A&$\beta$ARBa&$\beta$aARB\\
\hline
PP&$\beta$ARBpx&$\beta$pxARB\\
\hline
ADV&$\beta$ARBarb&$\beta$arbARB\\
\hline
NP&$\beta$ARBnx&\\
\hline
Det&$\beta$ARBd&\\
\hline
\end{tabular}
\caption{Simple Adverb Anchored Modifiers}
\label{adv-summary}
\end{table}


In the English XTAG grammar, no traces are posited for wh-adverbs, in-line with
the base-generation approach (\cite{Ernst84}) for various positions of
adverbs. Since convincing arguments have been made against traces for adjuncts
of other types (e.g. \cite{Baltin}), and since the reasons for wanting traces
do not seem to apply to adjuncts, we make the general assumption in our grammar
that adjuncts do not leave traces.  Sentence initial wh-adverbs select the same
auxiliary tree used for other sentence initial adverbs ($\beta$ARBs) with the
feature {\bf $<$wh$>$=+}.  Under this treatment, the derived tree for the
sentence {\it How did you fall?} is as in Figure (\ref{how-did-you-fall}), with
no trace for the adverb.


\begin{figure}[h]
\centering
\begin{tabular}{c}
{\psfig{figure=ps/modifiers-files/how-did-you-fall.ps,height=3.5in}}
\end{tabular}
\caption {Derived tree for {\it How did you fall?}}
\label {how-did-you-fall}
\end{figure}


\begin{figure}[h]
\centering
\begin{tabular}{c}
{\psfig{figure=ps/modifiers-files/betaARBarbs.ps,height=6.0in}}
\end{tabular}
\caption {Complex adverb phrase modifier: $\beta$ARBarbs}
\label{weird-adv-tree}
\end{figure}

There is one more adverb modifier tree in the grammar which is not included in
Table \ref{adv-summary}.  This tree, shown in Figure~\ref{weird-adv-tree}, has
a complex adverb phrase and is used for wh+ two-adverb phrases that occur
sentence initially, such as in sentence (\ex{1}).  Since {\it how} is the only
wh+ adverb, it is the only adverb that can anchor this tree.

\enumsentence{how quickly did Srini fix the problem ?}

Focus adverbs such as {\it only}, {\it even}, {\it just} and {\it at least} 
are also handled by the system.  Since the syntax allows focus adverbs to 
appear in practically any position, these adverbs select most of the trees 
listed in Table \ref{adv-summary}.  It is left up to the semantics or 
pragmatics to decide the correct scope for the focus adverb for a given 
instance.  In terms of the ability of the focus adverbs to modify at different
levels of a noun phrase, the focus adverbs can modify either cardinal 
determiners or noun-cardinal noun phrases, and cannot modify at the level of 
noun.  The tree for adverbial modification of noun phrases is in shown
Figure~\ref{other-adv-trees}(a). 

In addition to {\it at least}, the system handles the other two-word adverbs, 
{\it at most} and {\it up to}, and the three-word {\it as-as} adverb
constructions, where an adjective substitutes between the two occurences of 
{\it as}.  An example of a three-word as-as adverb is {\it as little as}.  
Except for the ability of {\it at least} to modify many different types of 
constituents as noted above, the multi-word adverbs are restricted to 
modifying cardinal determiners.  Example sentences using the trees in 
Figure~\ref{other-adv-trees} are shown in (\ex{1})-(\ex{5}).

\begin{itemize}
\item{Focus Adverb modifying an NP}
\enumsentence{{\bf only} a member of our crazy family could pull off that kind
of a stunt .}
\enumsentence{{\bf even} a flying saucer sighting would seem interesting in
comparison with your story .}
\enumsentence{The report includes a proposal for {\bf at least} a partial 
impasse in negotiations .}

\item{Multi-word adverbs modifying cardinal determiners}
\enumsentence{{\bf at most} ten people came to the party .} 
\enumsentence{They gave monetary gifts of {\bf as little as} five dollars .}

\end{itemize}

\begin{figure}[htb]
\centering
\begin{tabular}{ccccccc}
{\psfig{figure=ps/modifiers-files/betaARBnx.ps,height=1.5in}}
& \hspace{.5in} &
{\psfig{figure=ps/modifiers-files/betaPARBd.ps,height=1.5in}}
&  \hspace{.5in} &
{\psfig{figure=ps/modifiers-files/betaPaPd.ps,height=1.5in}}
\\
$\beta$ARBnx&&$\beta$PARBd&&$\beta$PaPd&&\\
(a)&&(b)&&(c)\\
\end{tabular}\\
\caption {Selected Focus and Multi-word Adverb Modifier trees:
$\beta$ARBnx, $\beta$PARBd and $\beta$PaPd}
\label {other-adv-trees}
\end{figure}









