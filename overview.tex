\chapter{Overview of the XTAG System}
\label{overview}

This section focuses on the various components that comprise the
parser and English grammar in the XTAG system.  Persons interested
only in the linguistic analyses in the grammar may skip this section
without loss of continuity, although a quick glance at the tagset used
in XTAG and the set of non-terminal labels used will be useful. We may
occasionally refer back to the various components mentioned in this
section.

\section{System Description}

Figure~{\ref{flowchart}} shows the overall flow of the system when
parsing a sentence; a summary of each component is presented in
Table~\ref{sys-table}. At the heart of the system is a parser for
lexicalized TAGs (\cite{schabesjoshi88,schabes90}) which produces all
legitimate parses for the sentence. The parser has two phases: {\bf
Tree Selection} and {\bf Tree Grafting}.

\begin{figure}[t]
\hspace{0.35in}
\centering
{\psfig{figure=ps/flowchart.eps,height=3.0in,angle=270}}
\caption[XTAG system diagram]{Overview of XTAG system }
\label{flowchart}
\end{figure}

\begin{table}[ht]
\small
\centering
\begin{tabular}{|l|l|} \hline
Component & Details \\ \hline
Morphological & Consists of approximately 317,000 inflected items \\ 
Analyzer and & derived from over 90000 stems. \\ 
Morph Database & Entries are indexed on the inflected form and return \\
& the root form, POS, and inflectional information.\\ \hline
POS Tagger & Wall Street Journal-trained
trigram tagger (\cite{kwc88})  \\ 
and  Lex Prob & extended to output N-best POS sequences  \\
Database & (\cite{soong90}). Decreases the time to parse \\
&a sentence by an average of 93\%. \\\hline
Syntactic &  More than 30,000 entries. \\
Database & Each entry consists of: the uninflected form of the word, \\
& its POS, the list of trees or tree-families associated with \\
& the word, and a list of feature equations that capture \\
&lexical idiosyncrasies. \\ \hline
Tree Database &  1094 trees, divided into 52 tree families and 218 individual \\
& trees. Tree families represent subcategorization frames; \\
& the trees in a tree family would be related to each other \\

& transformationally in a movement-based approach. \\ \hline
X-Interface & Menu-based facility for creating and modifying tree files. \\
&  User controlled parser parameters: parser's start category, \\ 
& enable/disable/retry on failure for POS tagger. \\
& Storage/retrieval facilities for elementary and parsed trees.\\
& Graphical displays of tree and feature data structures. \\
& Hand combination of trees by adjunction or substitution \\
& for grammar development. \\ 
& Ability to manually assign POS tag \\
& and/or Supertag before parsing \\ \hline
\end{tabular}
\caption{System Summary}
\label{sys-table}
\end{table}

\subsection{Tree Selection}

Since we are working with lexicalized TAGs, each word in the sentence
selects at least one tree. The advantage of a lexicalized formalism
like LTAGs is that rather than parsing with all the trees in the
grammar, we can parse with only the trees selected by the words in the
input sentence.

In the XTAG system, the selection of trees by the words is done in
several steps. Each step attempts to reduce ambiguity, i.e. reduce the
number of trees selected by the words in the sentence.

\begin{description}
\item[Morphological Analysis and POS Tagging] The input sentence is
  first submitted to the {\bf Morphological Analyzer} and the {\bf
    Tagger}. The morphological analyzer~(\cite{karp92}) consists of a
  disk-based database (a compiled version of the derivational rules)
  which is used to map an inflected word into its stem, part of speech
  and feature equations corresponding to inflectional information.
  These features are inserted at the anchor node of the tree
  eventually selected by the stem. The POS Tagger can be disabled in
  which case only information from the morphological analyzer is used.
  The morphology data was originally extracted from the Collins
  English Dictionary (\cite{ced79}) and Oxford Advanced Learner's
  Dictionary (\cite{oald74}) available through ACL-DCI
  (\cite{liberman89}), and then cleaned up and augmented by hand
  (\cite{karp92}).
    
\item[POS Blender] The output from the morphological analyzer and the
  POS tagger go into the {\bf POS Blender} which uses the output of
  the POS tagger as a filter on the output of the morphological
  analyzer. Any words that are not found in the morphological database
  are assigned the POS given by the tagger.
  
\item[Syntactic Database] The syntactic database contains the mapping
  between particular stem(s) and the tree templates or tree-families
  stored in the {\bf Tree Database} (see Table~\ref{sys-table}). The
  syntactic database also contains a list of feature equations that
  capture lexical idiosyncrasies. The output of the POS Blender is
  used to search the {\bf Syntactic Database} to produce a set of
  lexicalized trees with the feature equations associated with the
  word(s) in the syntactic database unified with the feature equations
  associated with the trees. Note that the features in the syntactic
  database can be assigned to any node in the tree and not just to the
  anchor node. The syntactic database entries were originally
  extracted from the Oxford Advanced Learner's Dictionary
  (\cite{oald74}) and Oxford Dictionary for Contemporary Idiomatic
  English (\cite{cie75}) available through ACL-DCI
  (\cite{liberman89}), and then modified and augmented by hand
  (\cite{EgediMartin94}).  There are more than 31,000 syntactic
  database entries.\footnote{This number does not include trees
    assigned by default based on the part-of-speech of the word.}
  Selected entries from this database are shown in
  Table~\ref{syn-entries}.
    
\item[Default Assignment] For words that are not found in the
  syntactic database, default trees and tree-families are assigned
  based on their POS tag.
  
\item[Filters] Some of the lexicalized trees chosen in previous stages
  can be eliminated in order to reduce ambiguity. Two methods are
  currently used: structural filters which eliminate trees which have
  impossible spans over the input sentence and a statistical filter
  based on unigram probabilities of non-lexicalized trees (from a hand
  corrected set of approximately 6000 parsed sentences). These methods
  speed the runtime by approximately 87\%.
  
\item[Supertagging] Before parsing, one can avail of an optional step
  of {\em supertagging} the sentence. This step uses statistical
  disambiguation to assign a unique elementary tree (or {\em
    supertag}) to each word in the sentence. These assignments can
  then be hand-corrected. These supertags are used as a filter on the
  tree assignments made so far. More information on supertagging can
  be found in (\cite{srini97diss,srini97iwpt}).

\end{description}

\begin{table}[htb]
\begin{verbatim}
<<INDEX>>porousness<<ENTRY>>porousness<<POS>>N
<<TREES>>^BNXN ^BN ^CNn
<<FEATURES>>#N_card- #N_const- #N_decreas- #N_definite- #N_gen- 
#N_quan- #N_refl-

<<INDEX>>coo<<ENTRY>>coo<<POS>>V<<FAMILY>>Tnx0V

<<INDEX>>engross<<ENTRY>>engross<<POS>>V<<FAMILY>>Tnx0Vnx1
<<FEATURES>>#TRANS+

<<INDEX>>forbear<<ENTRY>>forbear<<POS>>V<<FAMILY>>Tnx0Vs1
<<FEATURES>>#S1_WH- #S1_inf_for_nil

<<INDEX>>have<<ENTRY>>have<<POS>>V<<ENTRY>>out<<POS>>PL
<<FAMILY>>Tnx0Vplnx1
\end{verbatim}   
\caption{Example Syntactic Database Entries.}

\label{syn-entries}
\end{table}

\subsection{Tree Database}
\label{tree-db}

The {\bf Tree Database} contains the tree templates that are
lexicalized by following the various steps given above. The lexical
items are inserted into distinguished nodes in the tree template
called the {\em anchor nodes}.  The part of speech of each word in the
sentence corresponds to the label of the anchor node of the trees.
Hence the tagset used by the POS Tagger corresponds exactly to the
labels of the anchor nodes in the trees.  The tagset used in the XTAG
system is given in Table~\ref{tagset}. The tree templates are
subdivided into tree families (for verbs and other predicates), and
tree files which are simply collections of trees for lexical items
like prepositions, determiners, etc%
\footnote{ The nonterminals in the tree database are {\tt A, AP, Ad,
    AdvP, Comp, Conj, D, N, NP, P, PP, Punct, S, V, VP}.}%
.

\subsection{Tree Grafting}

Once a particular set of lexicalized trees for the sentence have been
selected, XTAG uses an Earley-style predictive left-to-right parsing
algorithm for LTAGs (\cite{schabesjoshi88,schabes90}) to find all
derivations for the sentence. The derivation trees and the associated
derived trees can be viewed using the X-interface (see
Table~\ref{sys-table}). The X-interface can also be used to save
particular derivations to disk.

The output of the parser for the sentence {\it I had a map yesterday} is
illustrated in Figure~\ref{sentence}. The parse tree\footnote{The feature
structures associated with each note of the parse tree are not shown here.}
represents the surface constituent structure, while the derivation tree
represents the derivation history of the parse. The nodes of the derivation
tree are the tree names anchored by the lexical items\footnote{Appendix
\ref{tree-naming} explains the conventions used in naming the trees.}.  The
composition operation is indicated by the nature of the arcs: a dashed line is
used for substitution and a bold line for adjunction.  The number beside each
tree name is the address of the node at which the operation took place.  The
derivation tree can also be interpreted as a dependency graph with unlabeled
arcs between words of the sentence.

\begin{figure}[htb]
\centering
\begin{tabular}{cc}
{{\psfig{figure=ps/overview-files/derived.ps,height=3.0in}}}  &
{{\psfig{figure=ps/overview-files/derivation.ps,height=2.0in,width=2.7in}}} \\
Parse Tree  & Derivation Tree \\
\end{tabular}
\caption{Output Structures from the Parser}
\label{sentence}
\end{figure}

\begin{table*}[ht]
\small
\centering
\begin{tabular}{|l|l|} \hline
Part of Speech & Description \\ \hline
A & Adjective \\ \hline
Ad & Adverb \\ \hline
Comp & Complementizer \\ \hline
D & Determiner \\ \hline
G & Genitive Noun \\ \hline
I & Interjection \\ \hline
N & Noun \\ \hline
P & Preposition \\ \hline
PL & Particle \\ \hline
Punct & Punctuation \\ \hline
V & Verb \\ \hline
\end{tabular}
\caption{XTAG tagset}
\label{tagset}
\end{table*}


\subsection{The Grammar Development Environment}

Working with and developing a large grammar is a challenging process,
and the importance of having good visualization tools cannot be
over-emphasized. Currently the XTAG system has X-windows based tools
for viewing and updating the morphological and syntactic databases
(\cite{karp92,EgediMartin94}). These are available in both ASCII and
binary-encoded database format. The ASCII format is well-suited for
various UNIX utilities (awk, sed, grep) while the database format is
used for fast access during program execution.  However even the ASCII
formatted representation is not well-suited for human readability. An
X-windows interface for the databases allows users to easily examine
them.  Searching for specific information on certain fields of the
syntactic database is also available. Also, the interface allows a
user to insert, delete and update any information in the databases.
Figure~\ref{morphsyn-tool}(a) shows the interface for the morphology
database and Figure~\ref{morphsyn-tool}(b) shows the interface for the
syntactic database.

\begin{figure}[htb]
\begin{tabular}{cc}
{\psfig{figure=ps/morph.ps,height=3.0in}}&
{\psfig{figure=ps/syn.ps,height=3.0in,width=2.0in}}\\
(a) Morphology database&(b) Syntactic database
\end{tabular}
\caption[Interfaces database]{Interfaces to the database maintenance tools}
\label{morphsyn-tool}
\end{figure}


%\begin{figure}[h]
%\centering
%\mbox{}
%{\psfig{figure=ps/syn.ps,height=3.0in}}
%\caption[Syntactic database]{Interface to the Syntactic database}
%\label{syn-tool}
%\end{figure}

XTAG also has a parsing and grammar development interface
(\cite{PSJ92}). This interface includes a tree editor, the ability to
vary parameters in the parser, work with multiple grammars and/or
parsers, and use metarules for more efficient tree editing and
construction (\cite{becker94}). The interface is shown in
Figure~\ref{xtag-interface}. It has the following features:

\begin{itemize}

\item Menu-based facility for creating and modifying tree files and 
loading grammar files.

\item User controlled parser parameters, including the root
category (main S, embedded S, NP, etc.), and the use of the tagger
(on/off/retry on failure).

\item Storage/retrieval facilities for elementary and parsed trees.

\item The production of postscript files corresponding to elementary
and parsed trees.

\item Graphical displays of tree and feature data structures,
including a scroll `web' for large tree structures.

\item Mouse-based tree editor for creating and modifying trees and
feature structures.

\item Hand combination of trees by adjunction or substitution for use
in diagnosing grammar problems.

\item Metarule tool for automatic aid to the generation of trees by using 
tree-based transformation rules
 
\end{itemize}

\begin{figure}[htb]
\centering
\mbox{}
{\psfig{figure=ps/xtag-interface.ps,height=3.0in}}
\caption[XTAG Interface]{Interface to the XTAG system}
\label{xtag-interface}
\end{figure}

\section{Computer Platform}


XTAG was developed on the Sun SPARC station series. It has been tested
on various Sun platforms including Ultra-1, Ultra-Enterprise. XTAG is
freely available from the XTAG web page at {\tt
  http://www.cis.upenn.edu/\~{}xtag/}. It requires 75 MB of disk space
(once all binaries and databases are created after the install). XTAG
requires the following software to run:

\begin{itemize}
  
\item A machine running UNIX and X11R4 (or higher). Previous releases
  of X will not work. X11R4 is free software which usually comes
  bundled with your OS. It is also freely available for various
  platforms at {\tt http://www.xfree86.org/}
  
\item A Common Lisp compiler which supports the latest definition of
  Common Lisp (Steele's Common Lisp, second edition). XTAG has been
  tested on Lucid Common Lisp/SPARC Solaris, Version: 4.2.1. Allegro
  CL is no longer directly supported, however there have been third
  party ports to recent versions of Allegro CL.
  
\item CLX version 4 or higher. CLX is the Lisp equivalent to the Xlib
  package written in C.
  
\item Mark Kantrowitz's Lisp Utilities from CMU: logical-pathnames and
  defsystem.

\end{itemize}

A patched version of CLX (Version 5.02) for SunOS 5.5.1 and the CMU
Lisp Utilities are provided in our ftp directory for your convenience.
However, we ask that you refer to the appropriate sources for updates.

The morphology database component (\cite{karp92}), no longer under
licensing restrictions, is available as a separate download from the
XTAG web page (see above for URL).

The syntactic database component is also available as part of the XTAG
system (\cite{EgediMartin94}). 

More information can be obtained on the XTAG web page at \\
{\tt http://www.cis.upenn.edu/\~{}xtag/}.

