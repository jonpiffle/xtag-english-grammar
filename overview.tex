\chapter{Overview of the XTAG System}
\label{overview}

This section is derived in large part from the XTAG project notes
(\cite{Xtag-notes}).  An additional section on Corpus Parsing and Evaluation
has not been replicated here (but see Appendix \ref{evaluation}). This section
focuses on the various components that comprise the parser and English grammar
in the XTAG system.  Persons interested only in the linguistic analyses in the
grammar may skip this section without loss of continuity, although we may
occasionally refer back to the various components mentioned here.

\section{System Description}

Figure~{\ref{flowchart}} shows the overall flow of the system when parsing a
sentence. The input sentence is submitted to the {\bf Morphological Analyzer}
and the {\bf Tagger}. The morphological analyzer retrieves the morphological
information for each individual word in the sentence from the {\bf
Morphological Database}. The result is filtered in the {\bf P.O.S. Blender}
using the output of the {\bf Trigram Tagger} to reduce the part of speech
ambiguity of the words. The augmented sentence, with each word annotated with
part of speech tags and morphological information, is input to the {\bf
Parser}, which then consults the {\bf Syntactic Database} and the {\bf Tree
Database} to retrieve the appropriate tree structures for each word in the
sentence.  Information from the {\bf Statistical Database}, along with a
variety of heuristics, is used to reduce the number of trees selected. The
parser then composes the structures to obtain the parse(s) of the sentence.

\begin{figure}[htb]
\centering
\mbox{}
{\psfig{figure=ps/overview-files/overview.eps,height=3.0in}}
\caption{Overview of the XTAG system}
\label{flowchart}
\end{figure}

\section{Morphological Analyzer}

The morphology data was originally extracted from the Collins English
Dictionary ({\cite{ced79}) and Oxford Advanced Learner's Dictionary
(\cite{oald74}) available through ACL-DCI (\cite{liberman89}), and then cleaned
up and augmented by hand (\cite{karp92}).  The database consists of
approximately 317,000 inflected items, along with their root forms and
inflectional information (such as case, number, tense).  Thirteen different
parts of speech are differentiated: Noun, Proper Noun, Pronoun, Verb, Verb
Particle, Adverb, Adjective, Preposition, Complementizer, Determiner,
Conjunction, Interjection, and Noun/Verb Contraction.  Nouns and Verbs are the
largest categories, with approximately 213,000 and 46,500 inflected forms
respectively.  This information is maintained in database form for quick
access.  Retrieval time for a given inflected entry is approximately 0.6 msec.

\section{Part of Speech Tagger}

A trigram part of speech tagger (\cite{kwc88}), trained on the Wall Street
Journal Corpus, is incorporated in XTAG. The trigram tagger has been extended
to output the N-best parts of speech sequences (\cite{soong90}).  XTAG uses
this information to reduce the number of specious parses by filtering the
possible parts of speech provided by the morphological analyzer for each word.
When the correct part of speech sequence is returned, the time required to
parse a sentence decreases by an average of 93\%.

\section{Parser}

XTAG uses an Earley-style parser which has been extended to handle
feature structures associated with trees (\cite{schabes90}). The parser uses a
general two-pass parsing strategy for lexicalized grammars (\cite{schabes88}).
In the tree-selection pass, the parser uses the syntactic database entry for
each lexical item in the sentence to select a set of elementary structures from
the tree database.  The tree-grafting pass composes the selected trees using
substitution and adjunction operations to obtain the parse of the sentence.
The output of the parser for the sentence {\it I had a map yesterday} is
illustrated in Figure~\ref{sentence}. The parse tree\footnote{The feature
structures associated with each note of the parse tree are not shown here.}
represents the surface constituent structure, while the derivation tree
represents the derivation history of the parse. The nodes of the derivation
tree are the tree names anchored by the lexical items\footnote{Appendix
\ref{tree-naming} explains the conventions used in naming the trees.}.  The
composition operation is indicated by the nature of the arcs: a dashed line is
used for substitution and a bold line for adjunction.  The number beside each
tree name is the address of the node at which the operation took place.  The
derivation tree can also be interpreted as a dependency graph with unlabeled
arcs between words of the sentence.

\begin{figure}[htb]
\centering
\begin{tabular}{cc}
{{\psfig{figure=ps/overview-files/derived.ps,height=3.0in}}}  &
{{\psfig{figure=ps/overview-files/derivation.ps,height=2.0in,width=2.7in}}} \\
Parse Tree  & Derivation Tree \\
\end{tabular}
\caption{Output Structures from the Parser}
\label{sentence}
\end{figure}

Additional methods that take advantage of FB-LTAGs have been implemented to
improve the performance of the parser.  For instance, the span\footnote{The
\xtagdef{span} of a tree is the number of terminals and non-terminals along its
frontier.} of the tree and the position of the anchor in the tree are used to
weed out unsuitable trees in the first pass of the parser.  Statistical
information about the usage frequency of the trees has been acquired by parsing
corpora. This information has been compiled into a statistical database that is
used by the parser. These methods speed the runtime by approximately 87\%.

\section{Syntactic Database}
\label{description-syn-entries}

The syntactic database associates lexical items with the appropriate
trees and tree families based on various selectional information.  The
syntactic database entries were originally extracted from the Oxford
Advanced Learner's Dictionary (\cite{oald74}) and Oxford Dictionary
for Contemporary Idiomatic English (\cite{cie75}) available through
ACL-DCI (\cite{liberman89}), and then modified and augmented by hand
(\cite{EgediMartin94}).  There are more than 31,000 syntactic database
entries.\footnote{This number does not include trees assigned by
default based on the part-of-speech of the word.}  Selected entries
from this database are shown in Table~\ref{syn-entries}.

Each syntactic entry consists of an {\sc index} field, the uninflected form
under which the entry is compiled in the database, an {\sc entry} field, which
contains all of the lexical items that will anchor the associated tree(s), a
{\sc pos} field, which gives the part of speech for the lexical item(s) in the
{\sc entry} field, and then either (but not both) a {\sc trees} or {\sc fam}
field.  The {\sc trees} field indicates a list of individual trees to be
associated with the entry, while the {\sc fam} field indicates a list of tree
families. A tree family may contain a number of trees.  A syntactic entry may
also contain a list of feature templates ({\sc fs}) which expand out to feature
equations to be placed in the specified tree(s). Any number of {\sc ex} fields
may be provided for example sentences. Note that lexical items may have more
than one entry and may select the same tree more than once, using different
features to capture lexical idiosyncrasies.

\begin{table}[htb]
\begin{tabular}{lll}

\hspace{0.5in}&INDEX: & have/26 \\
&ENTRY: & have \\
&POS: & V \\
&TREES:& $\beta$Vvx \\
&FS:&\#VPr\_ind, \#VPr\_past, \#VPr\_perfect+, \#VP\_ppart, \#VP\_pass- \\
&EX:&he had died; we had died \\
\\
&INDEX: & have/50 \\
&ENTRY:& have \\ 
&POS:& V \\ 
&TREES:& $\beta$Vvx \\
&FS:&\#VP\_inf  \\ 
&EX:&John has to go to the store. \\
\\
&INDEX:&have/69\\
&ENTRY:&NP0 have NP1\\
&POS:&NP0 V NP1 \\
&FAM: &Tnx0Vnx1 \\
&FS:&\#TRANS+ \\
&EX:&John has a problem. \\
\\
&INDEX:&map/1 \\
&ENTRY:&NP0 map out NP1 \\
&POS:&NP0 V PL NP1 \\
&FAM: &Tnx0Vplnx1 \\
\\
&INDEX:& map/3 \\
&ENTRY:& map \\
&POS: & N \\
&TREES: & $\alpha$N,  $\alpha$NXdxN, $\beta$Nn \\
&FS: & \#N\_wh-, \#N\_refl- \\
\\
&INDEX: & map/4 \\
&ENTRY: & map \\
&POS: & N \\
&TREES: & $\alpha$NXN \\
&FS: & \#N\_wh-, \#N\_refl-, \#N\_plur \\
\end{tabular}	
\caption{Example Syntactic Database Entries}
\label{syn-entries}
\end{table}

The syntactic database is currently undergoing a series of changes
designed to make it easier to use and update.  In addition, the number
of entries will be augmented to increase the coverage of the database,
and the defaults used by the XTAG system will be accessible from the
database itself.  The format of the entries as seen in Table
\ref{syn-entries} will change slightly in the new version, but the
same basic information will be included.

\section{Tree Database} 
\label{tree-db}

Trees in the English XTAG grammar fall into two conceptual classes.
The smaller class consists of individual trees such as NP and adverb
trees.  The trees in this class are generally anchored by non-verbal
lexical items. The larger class consists of trees that are grouped
into tree families.  These tree families represent subcategorization
frames (see section~\ref{subcat-frames}).  Currently, there are 739
trees that compose 49 tree families, along with 218 individually
selected trees in the tree database.


\section{Statistics Database}
\label{stat-db}
The statistics database contains tree unigram frequencies which have been
collected by parsing the Wall Street Journal, IBM manual, and ATIS corpora
using the XTAG English grammar. The parser, augmented with the statistics
database, assigns each word of the input sentence the top three most frequently
used trees given the part of speech of the word. On failure the parser retries
using all the trees suggested by the syntactic database for each word.  The
augmented parser has been observed to have a success rate of 50\% without
retries.

\section{X-Interface}

In addition to the parser and English grammar, XTAG provides a graphical
interface for manipulating TAGs.  The interface offers the following:

\begin{itemize}

\item Menu-based facility for creating and modifying tree files and 
loading grammar files.

\item User controlled parser parameters, including the root
category (main S, embedded S, NP, etc.), and the use of the tagger
(on/off/retry on failure).

\item Storage/retrieval facilities for elementary and parsed trees as
text files.

\item The production of postscript files corresponding to elementary
and parsed trees.

\item Graphical displays of tree and feature data structures,
including a scroll `web' for large tree structures.

\item Mouse-based tree editor for creating and modifying trees and
feature structures.

\item Hand combination of trees by adjunction or substitution for use
in diagnosing grammar problems.

\end{itemize}


\section{Computer Platform}

XTAG was developed on the Sun SPARC station series, and has been
tested on the Sun 4 and HP BOBCATs series 9000.  It is available
through anonymous ftp, and requires 20MB of disk space.  Please send
mail to xtag-request@linc.cis.upenn.edu for ftp instructions or more
information.  XTAG requires the following software to run:

\begin{itemize}

\item A machine running UNIX and X11R4 (or higher). Previous releases
of X will not work.  X11R4 is free software available from MIT.

\item A Common Lisp compiler which supports the latest definition of Common 
Lisp (Steele's Common Lisp, second edition).  XTAG has been tested
with Lucid Common Lisp 4.0 and Allegro 4.0.1.

\item CLX version 4 or higher. CLX is the lisp equivalent to the Xlib package 
written in C.

\item Mark Kantrowitz's Lisp Utilities from CMU: logical-pathnames and
defsystem.

\end{itemize}

The latest version of CLX (R5.0) and the CMU Lisp Utilities are
provided in our ftp directory for your convenience.  However, we ask
that you refer to the appropriate source for updates.

The morphology database component (\cite{karp92}), no longer under
licensing restrictions, is available as a separate system from the
XTAG system.  FTP instructions and more information can be obtained by
mailing requests to lex-request@linc.cis.upenn.edu.

The syntactic database component is also available as a separate
system (\cite{EgediMartin94}).  The new format of the database is
expected to be available in 1995.  FTP instructions and more
information can be obtained by mailing requests to
lex-request@linc.cis.upenn.edu.
