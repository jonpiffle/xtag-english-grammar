\section{Auxiliaries}
\label{auxiliaries}

Although there has been some debate about the lexical category of auxiliaries,
the English XTAG grammar follows \cite{mccawley88}, \cite{haegeman91}, and
others in classifying auxiliaries as verbs. The category of verbs can therefore
be divided into two sets, main or lexical verbs, and auxiliary verbs, which can
co-occur in a verbal sequence.  Only the highest verb in a verbal sequence is
marked for tense and agreement regardless of whether it is a main or auxiliary
verb.  Some auxiliaries ({\it be}, {\it do}, and {\it have}) share with main
verbs the property of having overt morphological marking for tense and
agreement, while the modal auxiliaries do not.  However, all auxiliary verbs
differ from main verbs in several crucial ways.

\begin{itemize}

\item Multiple auxiliaries can occur in a single sentence, while a matrix
sentence may have at most one main verb. 

\item Auxiliary verbs cannot occur as the sole verb in the sentence, but must
be followed by a main verb.

\item All auxiliaries precede the main verb in verbal sequences.

\item Auxiliaries do not subcategorize for any arguments.

\item Auxiliaries impose requirements on the morphological form of the verbs
that immediately follow them.

\item Only auxiliary verbs invert in questions (with the sole exception in 
American English of main verb {\it be}\footnote{Some dialects, particularly
British English, can also invert main verb {\it have} in yes/no questions: {\it
Have you any Grey Poupon?}  This is usually attributed to the influence of
auxiliary {\it have}, coupled with the historic fact that English once allowed
this movement for all verbs.\label{have-footnote}}).

\item An auxiliary verb must precede sentential negation (e.g. *{\it John not goes}).

\item Auxiliaries can form contractions with subjects and negation (e.g. {\it
he'll}, {\it won't}).

\end{itemize}


\noindent The restrictions that an auxiliary verb imposes on the succeeding verb limits
the sequence of verbs that can occur.  In English, sequences of up to five
verbs are allowed, as in sentence (\ex{1}).

\enumsentence{The music should have been being played [when the president arrived].}

\noindent 
The required ordering of verb forms when all five verbs are present is:

\begin{quote}
\begin{tabular}{ccl}
& & {\bf modal base perfective progressive passive}
\end{tabular}
\end{quote}

\noindent
The rightmost verb is the main verb of the sentence.  While a main verb
subcategorizes for the arguments that appear in the sentence, the auxiliary
verbs select the particular morphological forms of the verb to follow each of
them.  The auxiliaries included in the English XTAG grammar are listed in Table
\ref{aux-table} by type.  The third column of Table \ref{aux-table} lists the
verb forms that are required to follow each type of auxiliary verb.

\begin{table}[ht]
\centering
\begin{tabular}{|l|c|c|}  
\hline
TYPE&LEX ITEMS&SELECTS FOR\\     
\hline
modals & can, could, may, might, will, & base form\footnotemark
\\ & would, ought, shall, should & e.g. will
go, might come\\
\hline
perfective & have & past participle\\
& & (e.g. has gone)\\  
\hline
progressive & be & gerund\\
& & (e.g. is going, was coming)\\  
\hline
passive & be & past participle\\
& & (e.g. was helped by Jane)\\  
\hline
do support & do &base form\\
& & (e.g. did go, does come)\\  
\hline
infinitive to & to & base form\\
& & (e.g. to go, to come)\\  
\hline
\end{tabular}
\caption{Auxiliary Verb Properties}
\label{aux-table}
\end{table}

%  This text belong with the table above.  It is put here so that it gets on
%  the right page.
\footnotetext{There are American dialects, particularly in the South, which
allow double modals such as {\it might could} and {\it might should}. These
constructions are not allowed in the XTAG English grammar.}

\subsection{Non-inverted sentences}
\label{aux-non-inverted}

This section and the sections that follow describe how the English XTAG
implementation accounts for properties of the auxiliary system described above.

In our grammar, auxiliary trees are added to the main verb tree by adjunction.
Figure \ref{Vvx} shows the adjunction tree for non-inverted
sentences\footnote{We saw this tree briefly in section \ref{case-for-verbs},
but with most of its features missing.  The full tree is presented here.}.

\begin{figure}[htb]
\centering
\rule[.1in]{4.0in}{0.01in}\\ 
\begin{tabular}{c}
\psfig{figure=ps/auxs-files/betaVvx-with-features.ps,height=3.0in}
\end{tabular}
\caption{Auxiliary verb tree for non-inverted sentences}
\rule[.1in]{4.0in}{0.01in}\\
\label{Vvx} 
\end{figure}

\begin{figure}[htb]
\centering
\rule[.1in]{6.0in}{0.01in} \\
\begin{tabular}{cc}
{\psfig{figure=ps/auxs-files/betaVvx_should-with-features.ps,height=3.45in}} &
{\psfig{figure=ps/auxs-files/betaVvx_have-with-features.ps,height=3.45in}} \\
\\
{\psfig{figure=ps/auxs-files/betaVvx_been-with-features.ps,height=3.45in}} &
{\psfig{figure=ps/auxs-files/betaVvx_being-with-features.ps,height=3.45in}} \\
\end{tabular}
\caption{Auxiliary trees for {\it The music should have been being played.}}
\rule[.1in]{6.0in}{0.01in}
\label{anchored-aux-trees}
\end{figure}

The restrictions outlined in Column 3 of Table \ref{aux-table} are implemented
through the features {\it $<$mode$>$}, {\it $<$perfective$>$}, {\it
$<$progressive$>$}, and {\it $<$passive$>$}.  The syntactic lexicon entries for
the auxiliaries gives values for these features on the foot node~(VP$_{*}$) in
Figure \ref{Vvx}.  Since the top features of the foot node must unify with the
bottom features of the node it adjoins onto for the sentence to be valid, this
enforces the restrictions made by the auxiliary node.  In addition to these
feature values, each auxiliary also gives values to the anchoring
node~(V$\diamond$), to be passed up the tree to the root VP~(VP$_{r}$) node;
there they will become the new features for the top VP node of the sentential
tree.  Another auxiliary may now adjoin on top of it, and so forth.  These
feature values thereby ensure the proper auxiliary sequencing.  Figure
\ref{anchored-aux-trees} shows the auxiliary trees anchored by the 4 auxiliary
verbs in sentence (\ex{0}).  Figure \ref{non-inverted-sentence} shows the
final tree created for this sentence.

\begin{figure}[htb]
\centering
\rule[.1in]{4.0in}{0.01in} \\
\begin{tabular}{c}
{\psfig{figure=ps/auxs-files/non-inverted-sentence.ps,height=3.0in}} \\
\end{tabular}
\caption{{\it The music should have been being played.}}
\rule[.1in]{4.0in}{0.01in}
\label{non-inverted-sentence}
\end{figure}

The general English restriction that matrix clauses must have tense (or be
imperatives) is enforced by requiring the top S-node of a sentence to have {\it
$<$mode$>$= ind/imp} (indicative or imperative).  Since only the indicative
sentences have tense, non-tensed clauses are restricted to occurring in embedded
environments. 

\subsection{Inverted Sentences}

In inverted sentences, the two trees shown in Figure \ref{inverted-trees}
adjoin to an S tree anchored by a main verb.  The tree in Figure 
\ref{inverted-trees}a is anchored by the auxiliary verb and adjoins to the S
node, while Tree \ref{inverted-trees}b is anchored by an empty element and
adjoins at the VP node.  Figure \ref{yes/no-question} shows these trees
(anchored by {\it will}) adjoined to the declarative transitive
tree\footnote{The declarative transitive tree was seen in section
\ref{subcat-frames}.} (anchored by main verb {\it buy}).


\begin{figure}[htb]
\centering
\rule[.1in]{4.0in}{0.01in} \\
\begin{tabular}{cc}
{\psfig{figure=ps/auxs-files/betaVs-with-features.ps,height=3.0in}} &
{\psfig{figure=ps/auxs-files/betaVvx_epsilon-with-features.ps,height=3.2in}} \\
(a) & (b) \\ 
\end{tabular}
\caption{Trees for auxiliary verb inversion}
\rule[.1in]{4.0in}{0.01in}
\label{inverted-trees}
\end{figure}

\begin{figure}[htb]
\centering
\rule[.1in]{4.0in}{0.01in} \\
\begin{tabular}{c}
{\psfig{figure=ps/auxs-files/yes-no-question.ps,height=3.0in}} \\
\end{tabular}
\caption{{\it Will John buy a backpack?}}
\rule[.1in]{4.0in}{0.01in}
\label{yes/no-question}
\end{figure}

The feature {\it $<$displ-const$>$} ensures that both of the trees in Figure
\ref{inverted-trees} must adjoin to an elementary tree whenever one of them does. For
more discussion on this mechanism, which simulates tree local multi-component
adjunction, see \cite{hockeysrini93} [NOTE: refer to section in this report, if
we include it].  Tree \ref{inverted-trees}b, anchored by $\epsilon$, represents
the originating position of the inverted auxiliary. Its adjunction blocks the
{\it $<$assign-case$>$} values of the VP it dominates from being coindexed with
the {\it $<$case$>$} value of the subject. Since {\it $<$assign-case$>$} values
from the VP are blocked, the {\it $<$case$>$} value of the subject can only be
coindexed with the {\it $<$assign-case$>$} value of the inverted auxiliary
(Tree \ref{inverted-trees}a).  Consequently, the inverted auxiliary functions
as the case-assigner for the subject in these inverted structures.  This is in
contrast to the situation in uninverted structures where the anchor of the
highest (leftmost) VP assigns case to the subject (see Section
\ref{case-for-verbs} for more on case assignment).  The XTAG analysis is similar to
GB accounts where the inverted auxiliary plus the $\epsilon$-anchored tree are
taken as representing I to C movement.

\subsection{Do-Support}

It is well-known that English requires a mechanism called 'do-support' for
negated sentences and for inverted yes-no questions without auxiliaries.

\enumsentence {John does not want a car.}
\enumsentence {*John not wants a car.}
\enumsentence {John will not want a car.}
\enumsentence {Do you want to leave home?}
\enumsentence {*Want you to leave home?}
\enumsentence {Will you want to leave home?}

\subsubsection{In negated sentences}
\label{do-support-negatives}

The GB analysis of do-support in negated sentences hinges on the separation of
the INFL and VP nodes (see \cite{chomsky65}, \cite{jackendoff72}, and
\cite{chomsky86}).  The claim is that the presence of the negative morpheme
blocks the main verb from getting tense from the INFL node, thereby forcing the
addition of a verbal lexeme to carry the inflectional elements.  If an
auxiliary verb is present, then it carries tense, but if not, periphrastic or
`dummy', {\it do} is required.  This seems to indicate that {\it do} and other
auxiliary verbs would not co-occur, and indeed this is the case (see sentences
(\ex{1}) - (\ex{2})).  Auxiliary {\it do} is allowed in English when no
negative morpheme is present, but this usage is marked as emphatic.  Emphatic
{\it do} is also not allowed to co-occur with auxiliary verbs (sentences
(\ex{3}) - (\ex{6})).

\enumsentence {*We will have do bought a sleeping bag.}
\enumsentence {*We do will have bought a sleeping bag.}
\enumsentence {You {\bf do} have a backpack, don't you?}
\enumsentence {I {\bf do} want to go!}
\enumsentence {*You {\bf do} can have a backpack, don't you?}
\enumsentence {*I {\bf did} have had a backpack!}

In the XTAG grammar, {\it do} is prevented from co-occurring with other
auxiliary verbs by a requirement that it adjoin only onto main verbs.  It has
indicative mode, so no other auxiliaries can adjoin above it\footnote{Earlier,
we said that indicative mode carries tense with it.  Since only the topmost
auxiliary carries the tense, any subsequent verbs must {\bf not} have
indicative mode.}.  {\it Do} or another auxiliary is required to adjoin on top
of a negative morpheme by allowing {\it not} to only adjoin onto an
non-indicative (and therefore non-tensed) verb.  Since all matrix clauses must
be indicative (or imperative), a negative sentence will fail without an
auxiliary verb.  This analysis allows {\it not} the freedom to move around in
the auxiliaries, as seen in the sentences (\ex{1}) - (\ex{4}).

\enumsentence {John will have had a backpack.}
\enumsentence {*John not will have had a backpack.}
\enumsentence {John will not have had a backpack.}
\enumsentence {John will have not had a backpack.}

\subsubsection{In inverted yes/no questions}

In inverted yes/no questions, {\it do} is required if there is no auxiliary
verb to invert, as seen in sentences (\ex{-12}) - (\ex{-10}), replicated here
as (\ex{1}) - (\ex{3}).

\enumsentence {Do you want to leave home?}
\enumsentence {*Want you to leave home?}
\enumsentence {Will you want to leave home?}
\enumsentence {*Do you will want to leave home?}

In English, unlike other Germanic languages, the main verb cannot move to the
beginning of a clause, with the exception of main verb {\it be}\footnote{The
inversion of main verb {\it have} in British English was previously noted.}.
In a GB account of inverted yes/no questions, the tense feature is said to be
in C$^{0}$ at the front of the sentence.  Since main verbs cannot move, they
cannot pick up the tense feature, and do-support is again required if there is
no auxiliary verb to perform the role.  Sentence (\ex{0}) shows that {\it do}
does not interact with other auxiliary verbs, even when in the inverted
position.

In XTAG, trees anchored by a main verb that lacks tense are required to have an
auxiliary verb adjoin onto them, whether at the VP node to form a declarative
sentence, or at the S node to form an inverted question.  {\it Do} selects the
inverted auxiliary trees given in Figure \ref{inverted-trees}, just as other
auxiliaries do, so it is available to adjoin onto a tree at the S node to form
a yes/no question.  The mechanism described in Section \ref{do-support-negatives} 
prohibits {\it do} from co-occuring with other auxiliary verbs even in the
inverted position.


\subsection{Infinitives}

The infinitive {\it to} is considered an auxiliary verb in the XTAG system, and
selects the auxiliary tree in Figure \ref{Vvx}.  {\it To}, like {\it do}, does
not interact with the other auxiliary verbs, adjoining only to main verb base
forms, and carrying infinitive mode.  It is used in embedded clauses, both with
and without a complementizer, as in sentences (\ex{1}) - (\ex{3}).  Since it
cannot be inverted, it simply does not select the trees in Figure
\ref{inverted-trees}.

\enumsentence {John wants to have a backpack.}
\enumsentence {John wants Mary to have a backpack.}
\enumsentence {John wants for Mary to have a backpack.}

The usage of inifinitival {\em to} interacts closely with the
distribution of null subjects (PRO), and is described in more detail
in Section \ref{for-complementizer}.






