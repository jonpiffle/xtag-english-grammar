\chapter{Auxiliaries}
\label{auxiliaries}

Although there has been some debate about the lexical category of auxiliaries,
the English XTAG grammar follows \cite{mccawley88}, \cite{haegeman91}, and
others in classifying auxiliaries as verbs. The category of verbs can therefore
be divided into two sets, main or lexical verbs, and auxiliary verbs, which can
co-occur in a verbal sequence.  Only the highest verb in a verbal sequence is
marked for tense and agreement regardless of whether it is a main or auxiliary
verb.  Some auxiliaries ({\it be}, {\it do}, and {\it have}) share with main
verbs the property of having overt morphological marking for tense and
agreement, while the modal auxiliaries do not.  However, all auxiliary verbs
differ from main verbs in several crucial ways.

\begin{itemize}

\item Multiple auxiliaries can occur in a single sentence, while a matrix
sentence may have at most one main verb. 

\item Auxiliary verbs cannot occur as the sole verb in the sentence, but must
be followed by a main verb.

\item All auxiliaries precede the main verb in verbal sequences.

\item Auxiliaries do not subcategorize for any arguments.

\item Auxiliaries impose requirements on the morphological form of the verbs
that immediately follow them.

\item Only auxiliary verbs invert in questions (with the sole exception in 
American English of main verb {\it be}\footnote{Some dialects, particularly
British English, can also invert main verb {\it have} in yes/no questions
(e.g. {\it have you any Grey Poupon ?}).  This is usually attributed to the
influence of auxiliary {\it have}, coupled with the historic fact that English
once allowed this movement for all verbs.\label{have-footnote}}).

\item An auxiliary verb must precede sentential negation (e.g. $\ast${\it John not goes}).

\item Auxiliaries can form contractions with subjects and negation (e.g. {\it
he'll}, {\it won't}).

\end{itemize}

\noindent The restrictions that an auxiliary verb imposes on the succeeding verb limits
the sequence of verbs that can occur.  In English, sequences of up to five
verbs are allowed, as in sentence (\ex{1}).

\enumsentence{The music should have been being played [for the president] .}

\noindent 
The required ordering of verb forms when all five verbs are present is:

\begin{quote}
\begin{tabular}{ccl}
& & {\bf modal base perfective progressive passive}
\end{tabular}
\end{quote}

\noindent
The rightmost verb is the main verb of the sentence.  While a main verb
subcategorizes for the arguments that appear in the sentence, the auxiliary
verbs select the particular morphological forms of the verb to follow each of
them.  The auxiliaries included in the English XTAG grammar are listed in Table
\ref{aux-table} by type.  The third column of Table \ref{aux-table} lists the
verb forms that are required to follow each type of auxiliary verb.

\vspace*{0.2in}

\begin{table}[ht]
\centering
\begin{tabular}{|l|c|c|}  
\hline
TYPE&LEX ITEMS&SELECTS FOR\\     
\hline
modals & {\it can}, {\it could}, {\it may}, {\it might}, {\it will}, & base form\footnotemark
\\ & {\it would}, {\it ought}, {\it shall}, {\it should} & (e.g. {\it will
go}, {\it might come})\\ & {\it need} &\\
\hline
perfective & {\it have} & past participle\\
& & (e.g. {\it has gone})\\  
\hline
progressive & {\it be} & gerund\\
& & (e.g. {\it is going}, {\it was coming})\\  
\hline
passive & {\it be} & past participle\\
& & (e.g. {\it was helped by Jane})\\  
\hline
do support & {\it do} &base form\\
& & (e.g. {\it did go}, {\it does come})\\  
\hline
infinitive to & {\it to} & base form\\
& & (e.g. {\it to go}, {\it to come})\\  
\hline
\end{tabular}
\caption{Auxiliary Verb Properties}
\label{aux-table}
\end{table}

\vspace*{0.2in}

%  This text belong with the table above.  It is put here so that it gets on
%  the right page.
\footnotetext{There are American dialects, particularly in the South, which
allow double modals such as {\it might could} and {\it might should}. These
constructions are not allowed in the XTAG English grammar.}

\section{Non-inverted sentences}
\label{aux-non-inverted}

This section and the sections that follow describe how the English XTAG grammar
accounts for properties of the auxiliary system described above.

In our grammar, auxiliary trees are added to the main verb tree by adjunction.
Figure~\ref{Vvx} shows the adjunction tree for non-inverted
sentences.\footnote{We saw this tree briefly in section~\ref{case-for-verbs},
but with most of its features missing.  The full tree is presented here.}

\begin{figure}[htb]
\centering
\begin{tabular}{c}
\psfig{figure=ps/auxs-files/betaVvx-with-features.ps,height=5.2in}
\end{tabular}
\caption{Auxiliary verb tree for non-inverted sentences: $\beta$Vvx }
\label{Vvx} 
\end{figure}

\begin{figure}[htbp]
\centering
\begin{tabular}{ccc}
{\psfig{figure=ps/auxs-files/betaVvx_should-with-features.ps,height=3.9in}} &
\hspace*{1in}&
{\psfig{figure=ps/auxs-files/betaVvx_have-with-features.ps,height=3.9in}} \\
\\
{\psfig{figure=ps/auxs-files/betaVvx_been-with-features.ps,height=3.9in}} &
\hspace*{1in}&
{\psfig{figure=ps/auxs-files/betaVvx_being-with-features.ps,height=3.9in}} \\
\end{tabular}
\caption{Auxiliary trees for {\it The music should have been being played .}}
\label{anchored-aux-trees}
\end{figure}

The restrictions outlined in column 3 of Table \ref{aux-table} are
implemented through the features {\bf $<$mode$>$}, {\bf
$<$perfect$>$}, {\bf $<$progressive$>$} and {\bf $<$passive$>$}.
The syntactic lexicon entries for the auxiliaries give values for
these features on the foot node~(VP$^{*}$) in Figure~\ref{Vvx}.  Since
the top features of the foot node must eventually unify with the
bottom features of the node it adjoins onto for the sentence to be
valid, this enforces the restrictions made by the auxiliary node.  In
addition to these feature values, each auxiliary also gives values to
the anchoring node~(V$\diamond$), to be passed up the tree to the root
VP~(VP$_{r}$) node; there they will become the new features for the
top VP node of the sentential tree.  Another auxiliary may now adjoin
on top of it, and so forth.  These feature values thereby ensure the
proper auxiliary sequencing.  Figure~\ref{anchored-aux-trees} shows the auxiliary trees anchored by the four 
auxiliary verbs in sentence (\ex{0}).  Figure~\ref{non-inverted-sentence} shows
the final tree created for this sentence.

\begin{figure}[htb]
\centering
\begin{tabular}{c}
{\psfig{figure=ps/auxs-files/non-inverted-sentence.ps,height=5.1in}}
\end{tabular}
\caption{{\it The music should have been being played .}}
\label{non-inverted-sentence}
\end{figure}

The general English restriction that matrix clauses must have tense
(or be imperatives) is enforced by requiring the top S-node of a
sentence to have {\bf $<$mode$>$=ind/imp} (indicative or imperative).
Since only the indicative and imperative sentences have tense,
non-tensed clauses are restricted to occurring in embedded
environments.

Noun-verb contractions are labeled NVC in their part-of-speech field
in the morphological database and then undergo special processing to
split them apart into the noun and the reduced verb before
parsing. The noun then selects its trees in the normal fashion. The
contraction, say {\it 'll} or {\it 'd}, likewise selects the normal
auxiliary verb tree, $\beta$Vvx. However, since the contracted form,
rather than the verb stem, is given in the morphology, the contracted
form must also be listed as a separate syntactic entry. These entries
have all the same features of the full form of the auxiliary verbs,
with tense constraints coming from the morphological entry (e.g. {\it
it's} is listed as {\sc it 's NVC 3sg PRES}). The ambiguous
contractions {\it 'd} ({\it had/would}) and {\it 's} ({\it has/is})
behave like other ambiguous lexical items; there are simply multiple
entries for those lexical items in the lexicon, each with different
features. In the resulting parse, the contracted form is shown with
features appropriate to the full auxiliary it represents.

\section{Inverted Sentences}

In inverted sentences, the two trees shown in Figure~\ref{inverted-trees}
adjoin to an S tree anchored by a main verb.  The tree in
Figure~\ref{inverted-trees}(a) is anchored by the auxiliary verb and adjoins to
the S node, while the tree in Figure~\ref{inverted-trees}(b) is anchored by an
empty element and adjoins at the VP node.  Figure~\ref{yes/no-question} shows
these trees (anchored by {\it will}) adjoined to the declarative transitive
tree\footnote{The declarative transitive tree was seen in
section~\ref{nx0Vnx1-family}.} (anchored by main verb {\it buy}).


\begin{figure}[htbp]
\centering
\begin{tabular}{ccc}
{\psfig{figure=ps/auxs-files/betaVs-with-features.ps,height=4.5in}} &
\hspace*{0.5in} &
{\psfig{figure=ps/auxs-files/betaVvx_epsilon-with-features.ps,height=5in}} \\
(a) &&(b) \\ 
\end{tabular}
\caption{Trees for auxiliary verb inversion: $\beta$Vs (a) and $\beta$Vvx (b)}
\label{inverted-trees}
\end{figure}

\begin{figure}[htb]
\centering
\begin{tabular}{c}
{\psfig{figure=ps/auxs-files/yes-no-question.ps,height=4.0in}} \\
\end{tabular}
\caption{{\it will John buy a backpack ?}}
\label{yes/no-question}
\end{figure}

The feature {\bf $<$displ-const$>$} ensures that both of the trees in
Figure~\ref{inverted-trees} must adjoin to an elementary tree whenever one of
them does. For more discussion on this mechanism, which simulates tree local
multi-component adjunction, see \cite{hockeysrini93}.  The tree in
Figure~\ref{inverted-trees}(b), anchored by $\epsilon$, represents the
originating position of the inverted auxiliary. Its adjunction blocks the {\bf
$<$assign-case$>$} values of the VP it dominates from being co-indexed with the
{\bf $<$case$>$} value of the subject. Since {\bf $<$assign-case$>$} values
from the VP are blocked, the {\bf $<$case$>$} value of the subject can only be
co-indexed with the {\bf $<$assign-case$>$} value of the inverted auxiliary
(Figure~\ref{inverted-trees}(a)).  Consequently, the inverted auxiliary
functions as the case-assigner for the subject in these inverted structures.
This is in contrast to the situation in uninverted structures where the anchor
of the highest (leftmost) VP assigns case to the subject (see
section~\ref{case-for-verbs} for more on case assignment).  The XTAG analysis
is similar to GB accounts where the inverted auxiliary plus the
$\epsilon$-anchored tree are taken as representing I to C movement.

\section{Do-Support}

It is well-known that English requires a mechanism called `do-support' for
negated sentences and for inverted yes-no questions without auxiliaries.

\enumsentence {John does not want a car .}
\enumsentence {$\ast$John not wants a car .}
\enumsentence {John will not want a car .}
\enumsentence {Do you want to leave home ?}
\enumsentence {$\ast$want you to leave home ?}
\enumsentence {will you want to leave home ?}

\subsection{In negated sentences}
\label{do-support-negatives}

The GB analysis of do-support in negated sentences hinges on the separation of
the INFL and VP nodes (see \cite{chomsky65}, \cite{jackendoff72} and
\cite{chomsky86}).  The claim is that the presence of the negative morpheme
blocks the main verb from getting tense from the INFL node, thereby forcing the
addition of a verbal lexeme to carry the inflectional elements.  If an
auxiliary verb is present, then it carries tense, but if not, periphrastic or
`dummy', {\it do} is required.  This seems to indicate that {\it do} and other
auxiliary verbs would not co-occur, and indeed this is the case (see sentences
(\ex{1})-(\ex{2})).  Auxiliary {\it do} is allowed in English when no
negative morpheme is present, but this usage is marked as emphatic.  Emphatic
{\it do} is also not allowed to co-occur with auxiliary verbs (sentences
(\ex{3})-(\ex{6})).

\enumsentence {$\ast$We will have do bought a sleeping bag .}
\enumsentence {$\ast$We do will have bought a sleeping bag .}
\enumsentence {You {\bf do} have a backpack, don't you ?}
\enumsentence {I {\bf do} want to go !}
\enumsentence {$\ast$You {\bf do} can have a backpack, don't you ?}
\enumsentence {$\ast$I {\bf did} have had a backpack !}

At present, the XTAG grammar does not have analyses for emphatic {\it do}.

In the XTAG grammar, {\it do} is prevented from co-occurring with other
auxiliary verbs by a requirement that it adjoin only onto main verbs
({\bf $<$mainv$>$ = $+$}).  It has
indicative mode, so no other auxiliaries can adjoin above it.\footnote{Earlier,
we said that indicative mode carries tense with it.  Since only the topmost
auxiliary carries the tense, any subsequent verbs must {\bf not} have
indicative mode.}  The lexical item {\it not} is only allowed to adjoin onto a
non-indicative (and therefore non-tensed) verb.  Since all matrix clauses must
be indicative (or imperative), a negated sentence will fail unless an auxiliary
verb, either {\it do} or another auxiliary, adjoins somewhere above the
negative morpheme, {\it not}. In addition to forcing adjunction of an
auxiliary, this analysis of {\it not} allows it freedom to move around in the
auxiliaries, as seen in the sentences (\ex{1})-(\ex{4}).

\enumsentence {John will have had a backpack .}
\enumsentence {$\ast$John not will have had a backpack .}
\enumsentence {John will not have had a backpack .}
\enumsentence {John will have not had a backpack .}

\subsection{In inverted yes/no questions}

In inverted yes/no questions, {\it do} is required if there is no auxiliary
verb to invert, as seen in sentences (\ex{-12})-(\ex{-10}), replicated here
as (\ex{1})-(\ex{3}).

\enumsentence {do you want to leave home ?}
\enumsentence {$\ast$want you to leave home ?}
\enumsentence {will you want to leave home ?}
\enumsentence {$\ast$do you will want to leave home ?}

In English, unlike other Germanic languages, the main verb cannot move to the
beginning of a clause, with the exception of main verb {\it be}.\footnote{The
inversion of main verb {\it have} in British English was previously noted.}  In
a GB account of inverted yes/no questions, the tense feature is said to be in
C$^{0}$ at the front of the sentence.  Since main verbs cannot move, they
cannot pick up the tense feature, and do-support is again required if there is
no auxiliary verb to perform the role.  Sentence (\ex{0}) shows that {\it do}
does not interact with other auxiliary verbs, even when in the inverted
position.

In XTAG, trees anchored by a main verb that lacks tense are required to have an
auxiliary verb adjoin onto them, whether at the VP node to form a declarative
sentence, or at the S node to form an inverted question.  {\it Do} selects the
inverted auxiliary trees given in Figure~\ref{inverted-trees}, just as other
auxiliaries do, so it is available to adjoin onto a tree at the S node to form
a yes/no question.  The mechanism described in
section~\ref{do-support-negatives} prohibits {\it do} from co-occurring with
other auxiliary verbs, even in the inverted position.


\section{Infinitives}

The infinitive {\it to} is considered an auxiliary verb in the XTAG system, and
selects the auxiliary tree in Figure~\ref{Vvx}.  {\it To}, like {\it do}, does
not interact with the other auxiliary verbs, adjoining only to main verb base
forms, and carrying infinitive mode.  It is used in embedded clauses, both with
and without a complementizer, as in sentences (\ex{1})-(\ex{3}).  Since it
cannot be inverted, it simply does not select the trees in
Figure~\ref{inverted-trees}.

\enumsentence {John wants to have a backpack .}
\enumsentence {John wants Mary to have a backpack .}
\enumsentence {John wants for Mary to have a backpack .}

The usage of infinitival {\em to} interacts closely with the distribution of
null subjects (PRO), and is described in more detail in
section~\ref{for-complementizer}.

\section{Semi-Auxiliaries}

Under the category of semi-auxiliaries, we have placed several verbs that do
not seem to closely follow the behavior of auxiliaries.  One of these 
auxiliaries, {\it dare}, mainly behaves as a modal and selects for the base 
form of the verb.  The other semi-auxiliaries all select for the infinitival 
form of the verb.  Examples of this second type of semi-auxiliary are {\it 
used to}, {\it ought to}, {\it get to}, {\it have to}, and {\it BE to}.  

\subsection{Marginal Modal {\it dare}}

The auxiliary {\it dare} is unique among modals in that it both allows
DO-support and exhibits a past tense form.  It clearly falls in modal position
since no other auxiliary (except {\it do}) may precede it in linear
order\footnote{Some speakers accept {\it dare} preceded by a modal, as in {\it
I might dare finish this report today}.  In the XTAG analysis, this particular
double modal usage is accounted for.  Other cases of double modal occurrence
exist in some dialects of American English,  although these are not accounted
for in the system, as was mentioned earlier.\label{dare-footnote}}.  Examples appear below.

\enumsentence{she {\bf dare} not have been seen .}
\enumsentence{she does not {\bf dare} succeed .}
\enumsentence{Jerry {\bf dared} not look left or right .}
\enumsentence{only models {\bf dare} wear such extraordinary outfits .}
\enumsentence{{\bf dare} Dale tell her the secret ?}
\enumsentence{$\ast$Louise had dared not tell a soul .}

As mentioned above, auxiliaries are prevented from having DO-support within the
XTAG system.  To allow for DO-support in this case, we had to create a lexical
entry for {\it dare} that allowed it to have the feature {\bf
mainv+} and to have {\bf base} mode (this measure is
what also allows {\it dare} to occur in double-modal sequences).  A second
lexical entry was added to handle the regular modal occurrence of {\it dare}.
Additionally, all other modals are classified as being present tense, while
{\it dare} has both present and past forms.  To handle this behavior, {\it
dare} was given similar features to the other modals in the morphology minus
the specification for tense. 

\subsection{Other semi-auxiliaries}

The other semi-auxiliaries all select for the infinitival form of the verb.
Many of these auxiliaries allow for DO-support and can appear in both base and
past participle forms, in addition to being able to stand alone (indicative 
mode).  Examples of this type appear below.

\enumsentence{Alex {\bf used} to attend karate workshops .}
\enumsentence{Angelina might have {\bf used} to believe in fate .}
\enumsentence{Rich did not {\bf used} to want to be a physical therapist .}
\enumsentence{Mick might not {\bf have} to play the game tonight .}
\enumsentence{Singer {\bf had} to have been there .}
\enumsentence{Heather has {\bf got} to finish that project before she goes
insane .}

The auxiliaries {\it ought to} and {\it BE to} may not be preceded by any other
auxiliary.  

\enumsentence{Biff {\bf ought} to have been working harder .}
\enumsentence{$\ast$Carson does {\bf ought} to have been working harder .}
\enumsentence{the party {\bf is} to take place this evening .}
\enumsentence{$\ast$the party had {\bf been} to take place this evening .}

The trickiest element in this group of auxiliaries is {\it used to}.  While the
other verbs behave according to standard inflection for auxiliaries, {\it used
to} has the same form whether it is in mode base, past participle, or
indicative forms.  The only connection {\it used to} maintains with the
infinitival form {\it use} is that occasionally, the bare form {\it use} will
appear with DO-support.  Since the three modes mentioned above are mutually
exclusive in terms of both the morphology and the lexicon, {\it used} has three
entries in each.  

\subsection{Other Issues}

There is a lingering problem with the auxiliaries that stems from the fact that
there currently is no way to distinguish between the main verb and auxiliary verb
behaviors for a given letter string within the morphology.  This situation
results in many unacceptable sentences being successfully parsed by the system.
Examples of the unacceptable sentences are given below.

\enumsentence{the miller {\bf cans} tell a good story .  (vs  the farmer {\bf
cans} peaches in July .)}
\enumsentence{David {\bf wills} have finished by noon .  (vs  the old man {\bf
wills} his fortune to me .)}
\enumsentence{Sarah {\bf needs} not leave .  (vs  Sarah {\bf needs} to leave .)}
\enumsentence{Jennifer {\bf dares} not be seen .  (vs  the young woman {\bf
dares} him to do the stunt .)}
\enumsentence{Lila {\bf does use} to like beans .  (vs  Lila {\bf does use} her
new cookware .)}




        




