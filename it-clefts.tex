\chapter{It-clefts}
\label{it-clefts}

There are several varieties of it-clefts in English.  All the
it-clefts have four major components:

\begin{itemize}
\item {\bf the dummy subject:}  {\it it},
\item {\bf the main verb:}  {\it be},
\item {\bf the clefted element:}  A constituent (XP) compatible with
any gap in the clause,
\item {\bf the clause:}  A clause (e.g. S) with or without a gap.
\end{itemize}

\noindent
Examples of it-clefts are shown in (\ex{1})-(\ex{4}).

\enumsentence{it was [$_{XP}$ here $_{XP}$] [$_{S}$ that the ENIAC was
created . $_{S}$]}
\enumsentence{it was [$_{XP}$ at MIT $_{XP}$]  [$_{S}$ that colorless green
ideas slept furiously . $_{S}$]}
\enumsentence{it is [$_{XP}$ happily $_{XP}$]  [$_{S}$ that Seth quit Reality . $_{S}$]}
\enumsentence{it was  [$_{XP}$ there $_{XP}$]  [$_{S}$ that she would
have to enact her renunciation . $_{S}$]}

The clefted element can be of a number of categories, for example NP, PP or
adverb. The clause can also be of several types. The English XTAG grammar
currently has a separate analysis for only a subset of the `specificational'
it-clefts\footnote{See e.g. \cite{Ball91},
\cite{Delin89} and \cite{Delahunty84} for more detailed discussion of
types of it-clefts.}, in particular the ones without gaps in the clause
(e.g. (\ex{-1}) and (\ex{-0})).  It-clefts that have gaps in the clause, such
as (\ex{-3}) and (\ex{-2}) are currently handled as relative clauses. Although
arguments have been made against treating the clefted element and the clause as
a constituent (\cite{Delahunty84}), the relative clause approach does capture
the restriction that the clefted element must fill the gap in the clause, and
does not require any additional trees.

In the `specificational' it-cleft without gaps in the clause, the
clefted element has the role of an adjunct with respect to the clause.
For these cases the English XTAG grammar requires additional trees.
These it-cleft trees are in separate tree families because, although
some researchers (e.g. \cite{Akmajian70}) derived it-clefts through
movement from other sentence types, most current researchers
(e.g. \cite{Delahunty84}, \cite{Knowles86}, \cite{gazdar85},
\cite{Delin89} and \cite{Sornicola88}) favor base-generation of the
various cleft sentences.  Placing the it-cleft trees in their own tree
families is consistent with the current preference for base
generation, since in the XTAG English grammar, structures that would
be related by transformation in a movement-based account will appear
in the same tree family. Like the base-generated approaches, the
placement of it-clefts in separate tree families makes the claim that
there is no derivational relation between it-clefts and other sentence
types.

The three it-cleft tree families are virtually identical except for the
category label of the clefted element.  Figure~\ref{pp-it-clefts} shows the
declarative tree and an inverted tree for the PP It-cleft tree family.

\begin{figure}[htb]
\centering
\begin{tabular}{ccc}
{\psfig{figure=ps/it-cleft-files/alphaItVpnx1s2.ps,height=2.0in}} &
\hspace*{0.5in} &
{\psfig{figure=ps/it-cleft-files/alphaInvItVpnx1s2.ps,height=2.5in}} \\
(a)&\hspace*{0.5in}&(b)\\
\end{tabular}
\caption{It-cleft with PP clefted element: $\alpha$ItVpnx1s2 (a) and
$\alpha$InvItVpnx1s2 (b)}
\label{pp-it-clefts}
\label{1;1,3}
\label{1;3,3}
\end{figure}


The extra layer of tree structure in the VP represents that, while {\it be} is
a main verb rather than an auxiliary in these cases, it retains some auxiliary
properties. The VP structure for the equative/it-cleft-{\it be} is identical to
that obtained after adjunction of predicative-{\it be} into
small-clauses.\footnote{For additional discussion of equative or
predicative-{\it be} see Chapter~\ref{small-clauses}.}  The inverted tree in
Figure~\ref{pp-it-clefts}(b) is necessary because of {\it be}'s auxiliary-like
behavior.  Although {\it be} is the main verb in it-clefts, it inverts like an
auxiliary.  Main verb inversion cannot be accomplished by adjunction as is done
with auxiliaries and therefore must be built into the tree family. The tree in
Figure~\ref{pp-it-clefts}(b) is used for yes/no questions such as (\ex{1}).

\enumsentence{was it in the forest that the wolf talked to the little girl ?}





















































