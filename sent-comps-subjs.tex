%talk about ECM and object control verbs
%being treated the same (at least, sharing a tree family).
%format for features? angle brackets or no

\chapter{Sentential Subjects and Sentential Complements}
\label{scomps-section}

In the XTAG grammar, arguments of a lexical item, including
subjects, appear in the elementary tree anchored by that lexical item.  A
sentential argument appears as an S node in the appropriate position
within an elementary tree anchored by the lexical item that selects
it. This is the case for sentential complements of verbs, prepositions
and nouns and for sentential subjects. The distribution of
complementizers in English is intertwined with the distribution of
embedded sentences.  A successful analysis of complementizers in
English must handle both the cooccurrence restrictions between
complementizers and various types of clauses, and the distribution of
the clauses themselves, in both subject and complement positions.

\section{S or VP complements?}
 
Two comparable grammatical formalisms, Generalized Phrase Structure
Grammar (GPSG) \cite{gazdar85} and Head-driven Phrase Structure
Grammar (HPSG) \cite{PollardSag94:BK}, have rather different
treatments of sentential complements (S-comps). Both treat
embedded sentences as VP's with subjects, generating the correct
structures but missing the generalization that S's behave similarly in
both matrix and embedded environments, and that VP's behave quite
differently.  Neither account has PRO\label{PRO} subjects of
infinitival clauses-- they have subjectless VP's instead.  GPSG has a
complete complementizer system, which appears to cover the same range
of data as our analysis.  It is not clear what sort of complementizer
analysis could be implemented in HPSG.

Following the standard GB approach, the English XTAG grammar does not
allow VP complements but treats verb-anchored structures without overt
subjects as having PRO subjects. Thus, indicative clauses, infinitives
and gerunds all have a uniform treatment as embedded clauses using the
same trees under this approach. Furthermore, our analysis is able to
preserve the selectional and distributional distinction between S's and
VP's, in the spirit of GB theories, but without having to posit `extra'
empty categories, such as empty complementizers.\footnote{We do have PRO
and NP traces in the grammar.} Consider the alternation between {\it
that} and the null COMP,\footnote{Although we will continue
to refer to `null' complementizers, in our analysis this is actually
the absence of a complementizer.} shown in sentences \ex{1} and \ex{2}.

\enumsentence{He hopes $\emptyset$ Muriel wins .}
\enumsentence{He hopes that Muriel wins .}

 In GB both {\it Muriel wins} in \ex{-1} and {\it that Muriel wins} in
\ex{0} are CPs even though there is no overt complementizer to head the
phrase in \ex{-1}.  Our grammar does not distinguish by category label
between the phrases that would be labeled in GB as IP and CP.  We label
both of these phrases S.  The difference between these two levels is the
presence or absence of the complementizer (or extracted WH constituent), and is
represented in our system as a difference in feature values (here, of the {\bf
$<$comp$>$} feature), and the presence of the additional structure contributed
by the complementizer or extracted constituent.  This illustrates an important
distinction in XTAG, that between features and node labels.  Because we have a
sophisticated feature system, we are able to make fine-grained distinctions
between nodes with the same label which in another system might have to be
realized by using distinguishing node labels.
 
\section{Complementizers and Embedded Clauses in English:  The
Data}
\label{data}

Verbs selecting sentential complements place restrictions on
their complements, in particular, on the form of the embedded verb
phrase.\footnote{Other considerations, such as the relationship between the
tense/aspect of the matrix clause and the tense/aspect of a complement clause
are also important but are not currently addressed in the current English XTAG
grammar.}  Furthermore, complementizers are constrained to appear with certain
types of clauses, again, based primarily on the form of the embedded VP.  For
example, {\it hope\/} selects both indicative and infinitival complements. With
an indicative complement, it may only have {\it that\/} or null as possible
complementizers; with an infinitival complement, it may only have a null
complementizer.  Verbs that allow wh+ complementizers, such as {\it ask}, can
take {\it whether} and {\it if} as complementizers.  The possible combinations
of complementizers and clause types is summarized in Table \ref{facts}.

As can be seen in Table \ref{facts}, sentential subjects differ from
sentential complements in requiring the complementizer {\it that\/}
for all indicative and subjunctive clauses.  In sentential complements,
{\it that\/} often varies freely with a null complementizer, as
illustrated in \ex{1}-\ex{6}.

\enumsentence{Christy hopes that Mike wins .}
\enumsentence{Christy hopes Mike wins .}
\enumsentence{Dania thinks that Newt is a liar .}
\enumsentence{Dania thinks Newt is a liar .}
\enumsentence{That Helms won so easily annoyed me .}
\enumsentence{$\ast$Helms won so easily annoyed me .}


\begin{table}[ht]
\centering
\begin{tabular}{|l|llllll|} \hline
Complementizer:&&that&whether&if&for&null\\
\hline
Clause type&&&&&&\\
\hline
indicative&subject&Yes&Yes&No&No&No\\
&complement&Yes&Yes&Yes&No&Yes\\
\hline
infinitive&subject&No&Yes&No&Yes&Yes\\
&complement&No&Yes&No&Yes&Yes\\
\hline
subjunctive&subject&Yes&No&No&No&No\\
&complement&Yes&No&No&No&Yes\\
\hline
gerundive\footnotemark\ &complement&No&No&No&No&Yes\\
\hline
base & complement & No & No & No & No & Yes \\
\hline
small clause & complement & No & No & No & No & Yes \\
\hline
\end{tabular}
\vspace{.2in}
\caption{Summary of Complementizer and Clause Combinations}
\label{facts}
\end{table}
\footnotetext{Most gerundive phrases are treated as NP's.  In
fact, all gerundive subjects are treated as NP's, and the only gerundive
complements which receive a sentential parse are those for which there is no
corresponding NP parse.  This was done to reduce duplication of parses. See
Chapter~\ref{gerunds-chapter} for further discussion of
gerunds.\label{gerund-footnote}}


Another fact which must be accounted for in the analysis is that in infinitival
clauses, the complementizer {\it for} must appear with an overt subject NP,
whereas a complementizer-less infinitival clause never has an overt subject, as
shown in \ex{1}-\ex{4}. (See section~\ref{for-complementizer} for more
discussion of the case assignment issues relating to this construction.)

\enumsentence{To lose would be awful .}
\enumsentence{For Penn to lose would be awful .}
\enumsentence{$\ast$For to lose would be awful .}
\enumsentence{$\ast$Penn to lose would be awful .}

In addition, some verbs select {\bf $<$wh$>$=+} complements (either questions
or clauses with {\it whether} or {\it if}) \cite{grimshaw90}:

\enumsentence{Jesse wondered who left .}
\enumsentence{Jesse wondered if Barry left .}
\enumsentence{Jesse wondered whether to leave .}
\enumsentence{Jesse wondered whether Barry left .}
\enumsentence{$\ast$Jesse thought who left .}
\enumsentence{$\ast$Jesse thought if Barry left .}
\enumsentence{$\ast$Jesse thought whether to leave .}
\enumsentence{$\ast$Jesse thought whether Barry left .}

\section{Features Required}
\label{s-features}

As we have seen above, clauses may be {\bf $<$wh$>$=+} or {\bf $<$wh$>$=--},
may have one of several complementizers or no complementizer, and can be of
various clause types.  The XTAG analysis uses three features to capture these
possibilities: {\bf $<$comp$>$} for the variation in complementizers,
{\bf$<$wh$>$} for the question vs.  non-question alternation and {\bf
$<$mode$>$}\footnote{{\bf $<$mode$>$} actually conflates several types of
information, in particular verb form and mood.} for clause types.  In addition
to these three features, the {\bf $<$assign-comp$>$} feature represents
complementizer requirements of the embedded verb.  More detailed discussion of
the {\bf $<$assign-comp$>$} feature appears below in the discussions of
sentential subjects and of infinitives.  The four features and their possible
values are shown in Table \ref{feat}.


\begin{table}[th]
\centering
\begin{tabular}{|l|c|} \hline
Feature&Values\\
\hline
{\bf $<$comp$>$}&that, if, whether, for, rel, nil\\
\hline
{\bf$<$mode$>$}&ind, inf, subjnt, ger, base, ppart, nom/prep\\
\hline
{\bf$<$assign-comp$>$}&that, if, whether, for, rel, ind\underline{~}nil, inf\underline{~}nil\\
\hline
{\bf$<$wh$>$}&+,--\\
\hline
\end{tabular}
\caption{Summary of Relevant Features}
\label{feat}
\end{table}


\section{Distribution of Complementizers}
\label{comp-distr}

Like other non-arguments, complementizers anchor an auxiliary tree (shown in
Figure \ref{comp-tree}) and adjoin to elementary clausal trees.  The auxiliary
tree for complementizers is the only alternative to having a complementizer
position `built into' every sentential tree.  The latter choice would mean
having an empty complementizer substitute into every matrix sentence and a
complementizerless embedded sentence to fill the substitution node.  Our choice
follows the XTAG principle that initial trees consist only of the arguments of
the anchor\footnote{See section~\ref{compl-adj} for a discussion of the
difference between complements and adjuncts in the XTAG grammar.} -- the S tree
does not contain a slot for a complementizer, and the $\beta$COMP tree has only
one argument, an S with particular features determined by the complementizer.
Complementizers select the type of clause to which they adjoin through
constraints on the {\bf $<$mode$>$} feature of the S foot node in the tree
shown in Figure~\ref{comp-tree}.  These features also pass up to the root node,
so that they are `visible' to the tree where the embedded sentence
adjoins/substitutes.

\begin{figure}[hbt]
\centering
\hspace{0.0in}
\psfig{figure=ps/sent-comps-subjs-files/betaCOMPs_that_.ps,height=8.2cm}
\caption{Tree $\beta$COMPs, anchored by {\it that}}
\label{comp-tree}
\end{figure}

The grammar handles the following complementizers: {\it that\/}, {\it
whether\/}, {\it if\/}, {\it for\/}, and no complementizer, and the
clause types: indicative, infinitival, gerundive, past participial,
subjunctive and small clause ({\bf nom/prep}).  The {\bf
$<$comp$>$} feature in a clausal tree reflects the value of the
complementizer if one has adjoined to the clause. 

The {\bf $<$comp$>$} and {\bf $<$wh$>$} features receive their root
node values from the particular complementizer which anchors the tree.
The $\beta$COMPs tree adjoins to an S node with the feature {\bf
$<$comp$>$=nil}; this feature indicates that the tree does not already
{\bf have} a complementizer adjoined to it.\footnote{ Because root S's
cannot have complementizers, the parser checks that the root S has {\bf
$<$comp$>$=nil} at the end of the derivation, when the S is also checked for
a tensed verb.} We ensure that there are no stacked complementizers by
requiring the foot node of $\beta$COMPs to have {\bf $<$comp$>$=nil}.

% as well
%as using the {\bf $<$conj$>$=nil} feature to prevent complementizers from
%adjoining above subordinating conjunctions.

\section{Complementizer {\it for\/} and Case Assignment of the Subject}
\label{for-complementizer}

The {\bf $<$assign-comp$>$} feature is used to represent the
requirements of particular types of clauses for particular
complementizers.  So while the {\bf $<$comp$>$} feature represents
constraints originating from the VP dominating the clause, the {\bf
$<$assign-comp$>$} feature represents constraints originating from the
highest VP in the clause. {\bf $<$assign-comp$>$} is used to control
%appearance of subjects in infinitival clauses,  to
%ensure the correct distribution of complementizers in sentential
%subjects, and to block `that-trace' violations.
the appearance of subjects in infinitival clauses (see discussion of
ECM constructions in \ref{ecm-verbs}), to block bare indicative
sentential subjects (bare infinitival subjects are allowed), and to
block `that-trace' violations.

Examples \ex{2}, \ex{3} and \ex{4} show that an accusative
case subject is obligatory in an infinitive clause if the
complementizer {\it for\/} is present. The infinitive clause in
\ex{1} is analyzed in the English XTAG grammar as
having a PRO subject.  

%The apparent subject of {\it to win\/} in
%(\ex{1}) is taken to be an object of the verb rather than the subject
%of the infinitive clause. 
%\enumsentence{Mike wants her to pass the exam.}
%Note: I (Seth) took out this sentence, since the tech report 
%claims it gets an object-control analysis, while it in fact gets an
% ECM analysis.  It may be the case that it *should* get an ECM analysis,
% but for now I took it out, because it doesn't seem to have anything to
% do anyway with the point of this section.

\enumsentence{Christy wants to pass the exam .}
\enumsentence{Mike wants for her to pass the exam .}
\enumsentence{$\ast$Mike wants for she to pass the exam .}
\enumsentence{$\ast$Christy wants for to pass the exam .}
 
%The {\it for-to\/} construction is particularly illustrative of the
%difficulties and benefits faced in using a lexicalized grammar.  

It is commonly accepted that {\it for\/} behaves as a case-assigning
complementizer in this construction. It can assign accusative case to the
embedded subject since the infinitival verb can not assign
(nominative) case to this position.  
%However, in our featurized grammar, the
%absence of a feature licenses anything, so we must have overt null
%case assigned by infinitives to ensure the correct distribution of PRO
%subjects. (See section~\ref{case-assignment} for more discussion of
%case assignment.)  This null case assignment clashes with accusative
%case assignment if we simply add {\it for\/} as a standard
%complementizer, since NP's (including PRO) are drawn from the lexicon
%already marked for case.  
In \ex{-1} there is a feature clash between the nominative case subject {\it
she} and the accusative case assigning complementizer, thus accounting for its
ungrammaticality. Similarly, the sentence in \ex{0} is ruled out because PRO
has a feature {\bf $<$case$>$=none} which is coindexed with with the {\bf
$<$assign-case$>$} feature on S. This feature clashes with the {\bf
$<$assign-case$>$=acc} feature in the {\it for} auxiliary tree.


%Thus, we must use the {\bf
%$<$assign-comp$>$} feature to pass information about the verb up to
%the root of the embedded sentence.  To capture these facts, two
%infinitive {\it to}'s are posited. One infinitive {\it to\/} has {\bf
%$<$assign-case$>$=none} which forces a PRO subject, and {\bf
%$<$assign-comp$>$=inf\_nil} which prevents {\it for\/} from
%adjoining. The other infinitive {\it to\/} has no value at all for
%{\bf $<$assign-case$>$} and has {\bf $<$assign-comp$>$=for/ecm}, so that
%it can only occur either with the complementizer {\it for\/} or with
%ECM constructions. In those
%instances either {\it for} or the ECM verb
%supplies the {\bf $<$assign-case$>$} value, assigning
% accusative case to the overt subject.
%%
%%{This whole thing about the two to's may still be needed, and in fact I think
%%they are still in the grammar. However, they are not required to account for
%%the examples that are given in this section, so I took this discussion
%%out. The problem is that we can't rule out sentences where no case at all is
%%assigned to an overt subject, as in ``Mike hopes Christy to pass the exam.''
%%This is what the two ``to''s may still be needed for -- but it is still far
%%from clear how this will work. -- Tonia


\section{Sentential Complements of Verbs}
\label{sent-complements}
{\bf Tree families}: Tnx0Vs1, Tnx0Vnx1s2, TItVnx1s2, TItVpnx1s2, TItVad1s2.

\subsection{Long Distance Extraction}
Verbs that select sentential complements restrict the {\bf $<$mode$>$} and {\bf
$<$comp$>$} values for those complements. Since with very few exceptions long
distance extraction is possible from sentential complements,\footnote{Although
see the discussion of non-bridge verbs in the next section.} the S complement
nodes are adjunction nodes. 
Figure \ref{think} shows the declarative
tree for a sentential complement-taking bridge verb, anchored by {\it think}.

\begin{figure}[hbt]
\centering
\hspace{0.0in}
\psfig{figure=ps/sent-comps-subjs-files/think.ps,height=1.7in}
\caption{Sentential complement tree: $\beta$nx0Vs1}
\label{think}
\label{2;1,10}
\end{figure}

The need for an adjunction node rather than a substitution node at S$_{1}$
may not be obvious until one considers the derivation of sentences with
long distance extractions.  For example, the declarative in \ex{1} is
derived by adjoining the auxiliary tree in Figure~\ref{aard-emu}(b) to the
S$_{r}$ node of the tree in Figure~\ref{aard-emu}(a).  Since there are no
bottom features on S$_{1}$ in the auxiliary tree, the same final result
could have been achieved with a substitution node at S$_{1}$.

\enumsentence{The emu thinks that the aardvark smells terrible .}

\begin{figure}[htb]
\centering
\begin{tabular}{ccc}
\psfig{figure=ps/sent-comps-subjs-files/aard-smells.ps,height=2.1in}&
\hspace{0.3in}&
\psfig{figure=ps/sent-comps-subjs-files/emu-thinks.ps,height=2.1in}\\
(a)&&(b)\\
\end{tabular}
\caption{Trees for {\it The emu thinks that the aardvark smells terrible .}}  
\label{aard-emu}
\label{1;4,4}
\end{figure}

However, adjunction is crucial in deriving sentences with
long distance extraction, as in sentences \ex{1} and \ex{2}.  

\enumsentence{Who does the emu think smells terrible ?}
\enumsentence{Who did the elephant think the panda heard the emu say
smells terrible ?} 

The example in (\ex{-1}) is derived from the trees for {\it who smells
terrible?}  shown in Figure ~\ref{who-smells} and {\it the emu thinks} S shown
in Figure~\ref{aard-emu}(b), by adjoining the latter at the S$_r$ node of the
former.\footnote{See Chapter~\ref{auxiliaries} for a discussion of do-support.}
This process is recursive, allowing sentences like (\ex{0}). Such a
representation has been shown by \cite{kj85} to be well-suited for describing
unbounded dependencies.

\begin{figure}[thb]
\centering
\hspace{0.0in}
\psfig{figure=ps/sent-comps-subjs-files/who-smells.ps,height=2.3in}
\caption{Tree for {\it Who smells terrible?}}
\label{who-smells}
\label{1;4,14}
\end{figure}

In English, a complementizer may not appear on a complement with an extracted
subject (the `{\it that}-trace' configuration). This phenomenon
is illustrated in \ex{1}-\ex{3}:

\enumsentence{Which animal did the giraffe say that he likes ?}
\enumsentence{$\ast$Which animal did the giraffe say that likes him ?}
\enumsentence{Which animal did the giraffe say likes him ?}

These sentences are derived in XTAG by adjoining the tree for {\it did the
giraffe say} S at the S$_r$ node of the tree for either {\it which animal likes
him} (to yield sentence~\ex{0}) or {\it which animal he likes} (to yield
sentence~\ex{-2}).  That-trace violations are blocked by the presence of the
feature {\bf $<$assign-comp$>$=inf\underline{~}nil/ind\underline{~}nil/ecm}
feature on the bottom of the S$_r$ node of trees with extracted subjects (W0),
i.e. those used in sentences such as \ex{-1} and \ex{0}.  
If a complementizer tree, $\beta$COMPs, adjoins to a subject
extraction tree at $S_r$, its {\bf $<$assign-comp$>$ =
that/whether/for/if} feature will clash and the derivation will
fail. If there is no complementizer, there is no feature clash, and this will
permit the derivation of sentences like \ex{0}, or of ECM constructions, in
which case the ECM verb will have {\bf $<$assign-comp$>$=ecm} (see
section~\ref{ecm-verbs} for more discussion of the ECM case).
Complementizers may adjoin normally to object extraction trees such as those
used in sentence~\ex{-2}, and so object extraction trees have no value 
for the {\bf $<$assign-comp$>$} feature.
%This blocks (or
%`filters') any other values of {\bf $<$assign-comp$>$} projected by the verb,
%and ensures that no complementizer is able to adjoin at this node.


% Tonia: 
\subsection{Bridge vs. Non-bridge Verbs}

There is a class of {\it non-bridge verbs} (such as the manner-of-speaking
verbs, factives, and negative verbs) which do not allow extraction from their
sentential complement.  In contrast to a bridge verb {\it say} which allows
extraction from its complement, as in~\ex{1}, the non-bridge verb {\it whisper}
does not allow this extraction, as shown in~\ex{2}.

\enumsentence{Who did the elephant say that the emu saw ?}
\enumsentence{$\ast$ Who did the elephant whisper that the emu saw ?}

Similarly, adjunct extraction with a matrix bridge verb yields an ambiguity. In
\ex{1}, the adjunct wh-expression {\it when} can be interpreted as modifying
either the matrix predicate or the embedded predicate. In the non-bridge
example \ex{2}, there is no such ambiguity. The wh-expression can only be
construed as modifying the matrix predicate.

\enumsentence{When did Laura say she would be back ?}
\enumsentence{When did Laura whisper she would be back ?}

At this time, however,  we do not account for the bridge/non-bridge
distinction. 

%One way to account for this distinction would be to analyze non-bridge verbs as
%anchoring initial trees whose S-node complement was a substitution site, as
%opposed to bridge verbs which anchor auxiliary trees recursive on S. If this
%were so, non-bridge verbs would not allow any long-distance extraction from
%their complements. However, not all examples are as bad as one would expect
%them to be under this analysis. For example, some verbs do not allow
%long-distance adjunct extraction, but seem to marginally allow an object to be
%extracted from its complement S.
%
%\enumsentence{$\ast$ Why$_{i}$ do you doubt that Dr. Joshi left e$_{i}$ ?}
%\enumsentence{? What$_{i}$ do you doubt that Dr. Joshi saw e$_{i}$ ?}
%
%In order to account for \ex{-1}, the verb {\it doubt} can be analyzed as
%anchoring an initial tree, rather than an auxiliary tree. However, then we
%categorically rule out \ex{0}, as well. 
%
%Frank (1992) offers an analysis of the bridge/non-bridge distinction in which
%bridge verbs adjoin to their complement clause at C', whereas non-bridge verbs
%adjoin at CP. {etc, etc. I'm not sure how much of this to include. -- Tonia}

\subsection{Subjacency Violations: Multiple Wh-extraction}

In the case of indirect questions, subjacency follows from the principle that a
given tree cannot contain more than one wh-element. Extraction out of an
indirect question is ruled out because a sentence like~\ex{1} would have to be
derived from the adjunction of {\it do you wonder} into {\it who$_{i}$
who$_{j}$ e$_{j}$ loves e$_{i}$}, which is an ill-formed elementary
tree.\footnote{This does not mean that elementary trees with more than one gap
should be ruled out across the grammar. Such trees might be required for
dealing with parasitic gaps or gaps in coordinated structures.}


\enumsentence{$\ast$ Who$_{i}$ do you wonder who$_{j}$ e$_{j}$ loves e$_{i}$ ?}


\subsection{Exceptional Case Marking Verbs and Bare Infinitives}
\label{ecm-verbs}

{\bf Tree family}: TXnx0Vs1, Ts0Vs1

Exceptional Case Marking verbs are those which assign accusative case to the
subject of the sentential complement. This is in contrast to verbs
in the Tnx0Vnx1s2 family (section~\ref{nx0Vnx1s2-family}), which assign 
accusative case to an NP which is not part of the sentential complement.  

The subject of an ECM infinitive complement is assigned accusative case in a
manner analogous to that of a subject in a {\it for-to\/} construction, as
described in section~\ref{for-complementizer}.  As in the {\it for-to\/} case,
the ECM verb assigns accusative case into the subject of the lower infinitive,
and so the infinitive uses the {\it to} which has no value for {\bf
$<$assign-case$>$} and has {\bf $<$assign-comp$>$=for/ecm}.  The ECM verb has
{\bf $<$assign-comp$>$=ecm} and {\bf $<$assign-case$>$=acc} on its foot.  The
former allows the {\bf $<$assign-comp$>$} features of the ECM verb and the {\it
to} tree to unify, and so be used together, and the latter assigns the
accusative case to the lower subject.

Figure~\ref{expects-decl} shows the declarative tree for the TXnx0Vs1 family,
in this case anchored by {\it expects}.  Figure~\ref{van-expects} shows a parse
for {\it Van expects Bob to talk}

\begin{figure}[hbt]
\centering
\hspace{0.0in}
\psfig{figure=ps/sent-comps-subjs-files/expects.ps,height=3.3in}
\caption{ECM tree: $\beta$Xnx0Vs1}
\label{expects-decl}
\label{3;1,15}
\end{figure}

\begin{figure}[hbt]
\centering
\hspace{0.0in}
\psfig{figure=ps/sent-comps-subjs-files/van-expects.ps,height=3.3in}
\caption{Sample ECM parse}
\label{van-expects}
\end{figure}

The ECM and {\it for-to\/} cases are analogous in how they are used together
with the correct infinitival {\it to} to assign accusative case to the 
subject of the lower infinitive.  However, they are different in that
{\it for} is blocked along with other complementizers in subject extraction
contexts, as discussed in section~\ref{sent-complements}, as in
\ex{1}, while subject extraction is compatible with ECM cases, 
as in \ex{2}.

\enumsentence{$\ast$What child did the giraffe ask for to leave ?}
\enumsentence{Who did Bill expect to eat beans ?}

Sentence \ex{-1} is ruled out by the {\bf $<$assign-comp$>$=
inf\underline{~}nil/ind\underline{~}nil/ecm} feature
on the subject extraction tree for {\it ask}, since the
{\bf $<$assign-comp$>$=for} feature from the {\it for} tree will fail to 
unify.  However, 
\ex{0} will be allowed since {\bf $<$assign-comp$>$=ecm} feature on the
{\it expect} tree will unify with the foot of the ECM verb tree.  
The use of features allows the ECM and
{\it for-to\/} constructions to act the same for exceptional case assignment,
while also being distinguished for {\it that}-trace violations.


\subsubsection{ECM Passives}

Passivized ECM verbs are treated as raising verbs, meaning that they are
auxiliary trees recursive on VP.
Since the
subject of the infinitive is not thematically selected by the ECM verb,
it is not part of the ECM verb's tree, and so it cannot be part of the
passive tree. Therefore, the passive acts as a raising verb, and so for
example, 
the sentence {\it John is believed to be happy} would be derived by
adjoining {\it believed} in as a raising verb.  For
further discussion, see section~\ref{sm-clause-xtag-ECM}.

\subsubsection{Bare Infinitives}

Verbs that take bare infinitives, as in \ex{1}, are also treated as ECM
verbs, the only difference being that their foot feature has {\bf
$<$mode$>$=base} instead of {\bf $<$mode$>$=inf}.  Since the complement does
not have {\it to}, there is no question of using the {\it to} tree for allowing
accusative case to be assigned.  Instead, verbs with {\bf $<$mode$>$=base}
allow either accusative or nominative case to be assigned to the subject. The
foot of the ECM bare infinitive tree forces the subject to be accusative by its
{\bf $<$assign-case$>$=acc} value at its foot node which unifies with the {\bf
$<$assign-case$>$=nom/acc} value of the bare infinitive clause. 


\enumsentence{Bob sees the harmonica fall .}

Verbs
taking a bare infinitive complement and an NP subject select the
TXnx0Vs1 family.  Verbs taking a bare infinitive complement and a sentential
subject ({\it make} and {\it let}) select the Ts0Vs1 family (see 
section~\ref{s0Vs1-family}).  These verbs ({\it make} and {\it let}) also
take a predicative small clause (see section~\ref{sm-clause-xtag-analysis}),
and so the foot node has value {\bf $<$mode$>$=nom/prep/base}, thus allowing
either a bare infinitive or a {\bf nom/prep} complement.


\section{Sentential Subjects}
\label{sent-subjs}

{\bf Tree families}: Ts0Vnx1, Ts0Ax1, Ts0N1, Ts0Pnx1, Ts0ARBPnx1, 
Ts0PPnx1, Ts0PNaPnx1, Ts0V, Ts0Vtonx1, Ts0NPnx1, Ts0APnx1, Ts0Vs1.

Verbs that select sentential subjects anchor trees that have an S node
in the subject position rather than an NP node.  Since extraction is
not possible from sentential subjects, they are implemented as
substitution nodes in the English XTAG grammar.  Restrictions on
sentential subjects, such as the required {\it that} complementizer for
indicatives, are enforced by feature values specified on the S
substitution node in the elementary tree.  

Sentential subjects behave essentially like sentential complements, with a few
exceptions.  In general, all verbs which license sentential subjects license
the same set of clause types. Thus, unlike sentential complement verbs which
select particular complementizers and clause types, the matrix verbs licensing
sentential subjects merely license the S argument. Information about the
complementizer or embedded verb is located in the tree features, rather than in
the features of each verb selecting that tree.  Thus, all sentential subject
trees have the same {\bf $<$mode$>$}, {\bf $<$comp$>$} and {\bf
$<$assign-comp$>$} values shown in Figure~\ref{comparison}(a).

\begin{figure}[htb]
\centering
\begin{tabular}{ccc}
\psfig{figure=ps/sent-comps-subjs-files/perplexes-feats.ps,height=2.2in}&
\hspace{0.5in}&
\psfig{figure=ps/sent-comps-subjs-files/think-feats.ps,height=2.6in}\\
(a)&&(b)\\
\end{tabular}
\caption{Comparison of {\bf $<$assign-comp$>$} values for sentential
subjects: $\alpha$s0Vnx1 (a) and sentential complements: $\beta$nx0Vs1 (b)}
\label{comparison}
\label{1;1,16}
\end{figure}

The major difference in clause types licensed by S-subjs and S-comps is that
indicative S-subjs obligatorily have a complementizer (see examples in
section~\ref{data}). The {\bf $<$assign-comp$>$} feature is used here to
license a null complementizer for infinitival but not indicative clauses. {\bf
$<$assign-comp$>$} has the same possible values as {\bf $<$comp$>$}, with the
exception that the {\bf nil} value is `split' into {\bf ind\_nil} and {\bf
inf\_nil}.  This difference in feature values is illustrated in
Figure~\ref{comparison}.
%This allows us to specify precisely which environments license null
%complementizers. 
%Intuitively, {\bf $<$assign-comp$>$} passes information about what
%complementizers are licensed from the verb \underline{up} to its root,
%where it is `visible' to the extra-clausal environment.  

Another minor difference is that {\it whether\/} but not {\it if\/} is
grammatical with S-subjs.\footnote{Some speakers also find {\it if\/}
  as a complementizer only marginally grammatical in S-comps.} Thus,
{\it if} is not among the {\bf $<$comp$>$} values allowed in S-subjs.
The final difference from S-comps is that there are no S-subjs with
{\bf $<$mode$>$=ger}. As noted in footnote~\ref{gerund-footnote} of
this chapter, gerundive complements are only allowed when there is no
corresponding NP parse. In the case of gerundive S-subjs, there is
always an NP parse available.

\section{Nouns and Prepositions taking Sentential Complements}
\label{NPA}

{\bf Trees}: $\alpha$NXNs, $\beta$vxPs, $\beta$Pss, $\beta$nxPs,
Tnx0N1s1, Tnx0A1s1.

\begin{figure}[thb]
\centering
\begin{tabular}{ccc}
\psfig{figure=ps/sent-comps-subjs-files/betaPss.ps,height=5.6cm}&
\hspace{0.3in}&
\psfig{figure=ps/sent-comps-subjs-files/alphaNXNs.ps,height=4cm}
\\
(a) && (b)\\
\end{tabular}
\caption{Sample trees for preposition: $\beta$Pss (a) and noun: $\alpha$NXNs (b) taking
sentential complements}
\label{nounprep}
\end{figure}

Prepositions and nouns can also select sentential complements, using
the trees listed above.  These trees use the {\bf $<$mode$>$} and {\bf
$<$comp$>$} features as shown in Figure~\ref{nounprep}.  For example,
the noun {\it claim} takes only indicative complements with {\it
that}, while the preposition {\it with} takes small clause
complements, as seen in sentences \ex{1}-\ex{4}.

\enumsentence{Beth's claim that Clove was a smart dog ....}
\enumsentence{$\ast$Beth's claim that Clove a smart dog ....}
\enumsentence{Dania wasn't getting any sleep with Doug sick .}
\enumsentence{$\ast$Dania wasn't getting any sleep with Doug was sick .}

%%Comparative adjs also take s-comps, e.g. the boys easiest to teach.
%%See Quirk, section 7.20 and others.


\section{PRO control}
\label{PRO-control}

\subsection{Types of control}

In the literature on control, two types are often distinguished: obligatory
control, as in sentences~\ex{1}, \ex{2}, \ex{3}, and \ex{4} and optional 
control, as in sentence~\ex{5}.

\enumsentence{Srini$_i$ promised Mickey [PRO$_i$ to leave] .}
\enumsentence{Srini persuaded Mickey$_{i}$ [PRO$_i$ to leave] .}
\enumsentence{Srini$_{i}$ wanted [PRO$_i$ to leave] .}
\enumsentence{Christy$_{i}$ left the party early [PRO$_i$ to go to the airport] .}
\enumsentence{[PRO$_{arb/i}$ to dance] is important for Bill$_{i}$ .}

At present, an analysis for obligatory control into complement clauses (as
in sentences~\ex{-4}, \ex{-3}, and \ex{-2}) has been implemented. An
analysis for cases of obligatory control into adjunct clauses and optional
control exists and can be found in \cite{bhatt94}.

\subsection{A feature-based analysis of PRO control}
The analysis for obligatory control involves co-indexation of the control
feature of the NP anchored by PRO to the control feature of the controller.
A feature equation in the tree anchored by the control verb co-indexes the
control feature of the controlling NP with the foot node of the tree.  All
sentential trees have a co-indexed control feature from the root S to the
subject NP.

When the tree containing the controller adjoins onto the complement clause
tree containing the PRO, the features of the foot node of the auxiliary
tree are unified with the bottom features of the root node of the
complement clause tree containing the PRO. This leads to the control
feature of the controller being co-indexed with the control feature of the
PRO.

Depending on the choice of the controlling verb, the control propagation
paths in the auxiliary trees are different.  In the case of subject control
(as in sentence~\ex{-4}), the subject NP and the foot node have
co-indexed control features, while for object control
(e.g. sentence~\ex{-3}, the object NP and the foot node are co-indexed
for control. Among verbs that belong to the Tnx0Vnx1s2 family, i.e. verbs
that take an NP object and a clausal complement, subject-control verbs form
a distinct minority, {\em promise} being the only commonly used verb in
this class.


Consider the derivation of sentence~\ex{-3}. The auxiliary tree for
{\em persuade}, shown in Figure \ref{persuade-tree}, has the following
feature equation~\ex{1}.

\enumsentence{  NP$_{1}$:{\bf $<$control$>$} = S$_{2}$.t:{\bf $<$control$>$} }

The auxiliary tree adjoins into the tree for {\em leave}, shown in Figure
\ref{leave-tree}, which has the following feature equation~\ex{1}.

\enumsentence{S$_{r}$.b:{\bf $<$control$>$} = NP$_{0}$.t:{\bf $<$control$>$}}

\begin{figure}[hbt]
\centering
\hspace{0.0in}
\psfig{figure=ps/sent-comps-subjs-files/betanx0Vnx1s2_persuaded_.ps,height=5.2cm}
\caption{Tree for {\it persuaded}}
\label{persuade-tree}
\end{figure}

\begin{figure}[hbt]
\centering
\hspace{0.0in}
\psfig{figure=ps/sent-comps-subjs-files/alphanx0V_leave_.ps,height=5.2cm}
\caption{Tree for {\it leave}}
\label{leave-tree}
\end{figure}

Since the adjunction takes place at the root node (S$_{r}$) of the {\em
leave} tree, after unification, NP$_{1}$ of the {\em persuade} tree and
NP$_{0}$ of the {\em leave} tree share a control feature. The resulting
derived and derivation trees are shown in Figures
\ref{derived-tree-persuaded} and \ref{derivation-tree-persuaded}.

\begin{figure}[hbt]
\centering
\hspace{0.0in}
\psfig{figure=ps/sent-comps-subjs-files/persuaded-derv.ps,height=8.2cm}
\caption{Derived tree for {\it Srini persuaded Mickey to leave}}
\label{derived-tree-persuaded}
\end{figure}

\begin{figure}[hbt]
\centering
\hspace{0.0in}
\psfig{figure=ps/sent-comps-subjs-files/persuaded-derivation.ps,height=4.2cm}
\caption{Derivation tree for {\it Srini persuaded Mickey to leave}}
\label{derivation-tree-persuaded}
\end{figure}


\subsection{The nature of the control feature}
The control feature does not have any value and is used only for
co-indexing purposes. If two NPs have their control features
co-indexed, it means that they are participating in a relationship
of control; the c-commanding NP controls the c-commanded NP. 

\subsection{Long-distance transmission of control features}
Cases involving embedded infinitival complements with PRO subjects such as
\ex{1} can also be handled.

\enumsentence{ John$_i$ wants [PRO$_i$ to want [PRO$_i$ to dance]] .}

The control feature of `John' and the two PRO's all get co-indexed.
This treatment might appear to lead to a problem. Consider \ex{1}:

\enumsentence{ John$_{*i}$ wants [Mary$_i$ to want [PRO$_i$ to dance]] .}

If both the {\it want} trees have the control feature of their subject
co-indexed to their foot nodes, we would have a situtation where
the PRO is co-indexed for control feature with {\it John}, as well as with {\it Mary}. 
Note that the higher {\it want} in \ex{-1} is {\em want$_{ECM}$} 
- it assigns case
to the subject of the lower clause while the lower {\it want} in \ex{-1} is 
not. Subject control is restricted
to non-ECM (Exceptional Case Marking) verbs that take infinitival 
complements. Since the two {\it want}s in \ex{-1} are different with
respect to their control (and other) properties, the control feature
of PRO stops at {\it Mary} and is not transmitted to the higher clause.


\subsection{Locality constraints on control}
PRO control obeys locality constraints. The controller for PRO has to be
in the immediately higher clause. Consider the ungrammatical sentence~\ex{1}
(\ex{1} is ungrammatical only with the co-indexing indicated below).
\enumsentence{* John$_i$ wants [PRO$_i$ to persuade Mary$_i$ [PRO$_i$ to dance]]}
However, such a derivation is ruled out automatically by the 
mechanisms of a TAG derivation and feature unification. 
Suppose it was possible to first compose the {\em want} tree with the
{\em dance} tree and then insert the {\em persuade} tree. (This is not
possible in the XTAG grammar because of the convention that
auxiliary trees have NA (Null Adjunction) constraints on their foot nodes.)
Even then, at the end of the derivation the control feature of the 
subject of {\em want} would end up co-indexed with the PRO subject of
{\em persuade} and the control feature of {\em Mary} would be co-indexed with the
PRO subject of {\em dance} as desired. There is no way to generate the illegal
co-indexing in \ex{-1}. Thus the locality constraints on PRO control 
fall out from the mechanics of TAG derivation and feature unification. 



\section{Reported speech}

Reported speech is handled in the XTAG grammar by having the reporting
clause adjoin into the quote. Thus, the reporting clause is an
auxiliary tree, anchored by the reporting verb. See \cite{doran-diss}
for details of the analysis. There are trees in both the Tnx0Vs1 and
Tnx0nx1s2 families to handle reporting clauses which precede, follow
and come in the middle of the quote.

