\section{Conjunction}

\subsection{Introduction}
The XTAG system can handle sentences with conjunction of two constituents
of the same syntactical category.  
There are eight syntactical categories that can conjoin,  and
in each case an auxiliary tree is used to implement the conjunction.
These eight categories can be considered as four different
cases, as described in the following subsections.  In all cases the
two constituents are required to be of the same syntactical category, but
there may also be some additional constraints, as described below.

\subsection{Adjective, Adverb, Preposition, Preposition Phrase Conjunction}

%\begin{description}
%
%\item[Adjective] ``the dark and dreary day''
%\item[PP] ``down the street and around the corner''
%\item[Adverb] ``slowly and carefully''
%\item[preposition] ``the student goes to and from the office''
%\item[NP] ``the boy and the girl''
%\item[noun] `` the boy and girl have left''
%\item[sentential] `` the day was dark and the phone never rang''
%\item[determiner] `` all but one have left''
%\end{description}

\begin{figure}[ht]
\centering
\rule[.1in]{5.0in}{0.01in}
\begin{tabular}{cc}
{\psfig{figure=ps/conj-files/betaA1conjA2.ps,height=1in}}&
{\psfig{figure=ps/conj-files/derived-tree-140291.ps,height=1.8in,width=2in}}\\
(a) & (b)\\
\end{tabular}
\caption{Tree for adjective conjunction and a resulting parse tree}
\rule[.1in]{5.0in}{0.01in}
\label{A1conjA2}
\end{figure}

Each of these four categories has an auxiliary tree that is used for
conjunction of two constituents of that category.  The auxiliary tree
gets adjoined into the left-hand-side component, and the right-hand-side
component gets substituted into the auxiliary tree.  

Figure~\ref{A1conjA2}(a)  shows the auxiliary tree for adjective conjunction,
and is used, for example, in the derivation of the parse tree for the 
noun phrase {\it the dark and dreary day}, as shown in
 figure~\ref{A1conjA2}(b).  The auxiliary tree gets adjoined onto the node for
the left adjective, and the right adjective gets substituted into the right
hand side node of the auxiliary
tree.\footnote{see section~\ref{mc-adjunction} for an explanation of the
{\bf displ-const} feature.}
The analysis for adverb, preposition, and PP conjunction is exactly the
same and there is a corresponding auxiliary tree for each of these that
is identical to that of Figure~\ref{A1conjA2}(a) except, of course,
for the node labels.
\subsection{Noun Phrase and Noun Conjunction}

\begin{figure}[ht]
\centering
\rule[.1in]{5.0in}{0.01in}
\psfig{figure=ps/conj-files/betaNP1conjNP2.ps,height=2in}
\caption{Tree for NP conjunction}
\rule[.1in]{5.0in}{0.01in}
\label{NP1conjNP2}
\end{figure}

The tree for NP conjunction, shown in figure~\ref{NP1conjNP2}, has the
same basic analysis as in the previous section except that the {\bf wh}
and {\bf case} features are
used to constrain the two noun phrases to have the same {\bf wh} and 
{\bf case} values.
This allows, for example, {\it he and she wrote the book together} while
disallowing {\it he and her wrote the book together.} 
The {\bf agr} feature of the top node
sets the resulting NP to get plural number.  The  tree for N conjunction
is identical to that for the NP tree except for the node labels. 

\subsection{Sentential Conjunction}
\begin{figure}[ht]
\centering
\rule[.1in]{5.0in}{0.01in}
\psfig{figure=ps/conj-files/betaS1conjS2.ps,height=1.6in}
\caption{Tree for sentential conjunction}
\rule[.1in]{5.0in}{0.01in}
\label{S1conjS2}
\end{figure}

The tree for sentential conjunction, shown in figure~\ref{S1conjS2}, is
based on the same analysis as in the previous two sections except with
some different features.  The {\bf mode}
feature \footnote{see section~\ref{sentence-mode} 
for an explanation of the {\bf mode} feature.} 
is used to 
constrain the two sentences being conjoined to have the same mode so that
{\it the day is dark and the phone never rang} is acceptable, but 
{\it the day is dark and that the phone never rang} is not.  The 
{\bf assign-comp} feature \footnote{see section~\ref{assign-comp} 
for an explanation of the {\bf assign-comp} feature.} 
feature is used to allow conjunction of infinitival sentences, such as
{\it to read and to sleep is a good life}.

\subsection{Determiner Conjunction}
\begin{figure}[ht]
\centering
\rule[.1in]{5.0in}{0.01in}
\psfig{figure=ps/conj-files/betaDX1conjDX2.ps,height=3.5in}
\caption{Tree for determiner conjunction}
\rule[.1in]{5.0in}{0.01in}
\label{DX1conjDX2}
\end{figure}

The tree for determiner conjunction, shown in figure~\ref{DX1conjDX2},
is unlike the other conjunction trees in that the foot node is on the right.
This is because 
determiner phrases generally build to the left. For the same reason,
all the various feature values are taken from the left determiner, and the
only requirement is that the {\bf wh} feature is the same, while the
other features, such as {\bf card}, are unconstrained.  For example,
{\it who and which} and {\it all but one} are both acceptable determiners,
but {\it who and all} is not. 

\subsection{Other Conjunctions}
The conjunction analysis described in the previous sections is designed
to handle only the most straightforward cases of conjunction.  Three
types of conjunction that are not handled are:
\begin{description}
\item [Incomplete Constituents] Whereas the sentence 
{\it John likes bananas and Bill hates bananas} is a simple
case of sentential conjunction, {\it John likes and Bill hates bananas}
cannot be handled by the previous analysis.  Since {\it likes} 
anchors a tree that needs both a subject noun phrase and an object noun
phrase to be substituted in, the latter sentence has an empty
substitution node after {\it John likes}.  

\item [Verb Phrase Conjunction] Since verbs anchor a tree with a root node 
of type {\bf S} and not {\bf VP},
there is no straightforward way to implement verb phrase conjunction.
For example, in the sentence {\it John eats cookies and drinks beer},
there is no point in the derivation at which {\it eats cookies} and
{\it drinks beer} are available as separate trees ready to be conjoined.
They are both only subtrees in their respective {\bf S} trees.  This could
also be considered as a case of incomplete constituents, since
they are both missing a noun phrase.

\item [Gapping]
Sentences such as {\it John likes apples and Bill pears} are also not
handled by the previous analysis.  These could also be considered as a case
of incomplete constituents.
\end{description}

One grammar formalism that is capable of handling these types of 
conjunction is Combinatory Categorial Grammar (CCG) (see \cite{steedman90})
which relies on a nonstandard notion of a constituent in order to accomplish
this.  Proposals have been made (for example, \cite{joshischabes91}),
inspired by the CCG approach, to handle these problematic cases in the
LTAG formalism.  Unlike the CCG analysis, however, the traditional notion
of constituents and phrase structure is maintained.  Such proposals are
as of yet unimplemented.

%test sentences 
%I ran and found a Brickel bush
% you and me and the whole world
% hook and line and bait



