\chapter{Conjunction}
\label{conjunction}

\section{Introduction}

The XTAG system can handle sentences with conjunction of two
constituents of the same syntactic category. The coordinating
conjunctions which select the conjunction trees are {\it and}, {\it
or} and {\it but}.\footnote{We believe that the restriction of {\it
but} to conjoining only two items is a pragmatic one, and our grammars
accepts sequences of any number of elements conjoined by {\it but}.}
There are also multi-word conjunction trees, anchored by {\it
either-or},{\it neither-nor} and {\it both-and}.  There are eight
syntactic categories that can be coordinated, and in each case an
auxiliary tree is used to implement the conjunction.  These eight
categories can be considered as four different cases, as described in
the following sections.  In all cases the two constituents are
required to be of the same syntactic category, but there may also be
some additional constraints, as described below.


\section{Adjective, Adverb, Preposition and PP Conjunction}

Each of these four categories has an auxiliary tree that is used for
conjunction of two constituents of that category.  The auxiliary tree
adjoins into the left-hand-side component, and the right-hand-side
component substitutes into the auxiliary tree.  

\begin{figure}[htb]
\centering
\begin{tabular}{ccc}
{\psfig{figure=ps/conj-files/betaA1conjA2.ps,height=0.8in}}&
\hspace*{0.5in}&
{\psfig{figure=ps/conj-files/derived-tree-140291.ps,height=1.8in}}\\
(a) & \hspace*{0.5in}& (b)\\
\end{tabular}
\caption{Tree for adjective conjunction: $\beta$a1CONJa2 and a resulting parse tree}
\label{A1conjA2}
\end{figure}

Figure~\ref{A1conjA2}(a) shows the auxiliary tree for adjective conjunction,
and is used, for example, in the derivation of the parse tree for the noun
phrase {\it the dark and dreary day}, as shown in Figure~\ref{A1conjA2}(b).
The auxiliary tree adjoins onto the node for the left adjective, and the
right adjective substitutes into the right hand side node of the auxiliary
tree. The analysis for adverb, preposition and PP conjunction is exactly the
same and there is a corresponding auxiliary tree for each of these that is
identical to that of Figure~\ref{A1conjA2}(a) except, of course, for the node
labels.


\section{Noun Phrase and Noun Conjunction}

The tree for NP conjunction, shown in Figure~\ref{NP1conjNP2}(a), has
the same basic analysis as in the previous section except that the
{\bf $<$wh$>$} and {\bf $<$case$>$} features are used to force the two
noun phrases to have the same {\bf $<$wh$>$} and {\bf $<$case$>$}
values.  This allows, for example, {\it he and she wrote the book
together} while disallowing {\it $\ast$he and her wrote the book
together.}  Agreement is lexicalized, since the various conjunctions
behave differently. With {\it and}, the root {\bf $<$agr num$>$} value
is {\bf $<$plural$>$}, no matter what the number of the two
conjuncts. With {\it or}, however, the root {\bf $<$agr num$>$} is
co-indexed with the {\bf $<$agr num$>$} feature of the right
conjunct. This ensures that the entire conjunct will bear the number
of both conjuncts if they agree (Figure~\ref{NP1conjNP2}(b)), or of
the most ``recent'' one if they differ ({\it Either the boys or John
is going to help you.}). There is no rule per se on what the
agreement should be here, but people tend to make the verb agree with
the last conjunct (cf. Quirk et. al \shortcite{quirk85}, section 10.41
for discussion). The tree for N conjunction is identical to that for
the NP tree except for the node labels. (The multi-word conjunctions
do not select the N conjunction tree - {\it $^*$the both dogs and
cats}).

\begin{figure}[htb]
\centering
\begin{tabular}{cc}
{\psfig{figure=ps/conj-files/betaconjNP1conjNP2.ps,height=2.3in}}
\hspace{0.5cm} &
{\psfig{figure=ps/conj-files/derived-tree-129800.ps,height=2.3in}}\\
(a) &  (b)\\
\end{tabular}
\caption{Tree for NP conjunction: $\beta$nx1CONJnx2 and a resulting
parse tree}
\label{NP1conjNP2}
\end{figure}


\section{Determiner Conjunction}

%CDD,10/31/95:We cannot find any reason for this tree to be different,
%so it is now like the other conj trees, with the foot on the left.
%
%The tree for determiner conjunction, shown in Figure~\ref{DX1conjDX2}, is
%unlike the other conjunction trees in that the foot node is on the right.  This
%is because determiner phrases generally build to the left. For the
%same reason, 

In determiner coordination, all of the determiner feature values are
taken from the left determiner, and the only requirement is that the
{\bf $<$wh$>$} feature is the same, while the other features, such as
{\bf $<$card$>$}, are unconstrained.  For example, {\it which and
what} and {\it all but one} are both acceptable determiner
conjunctions, but {\it $\ast$which and all} is not.

\enumsentence{How many and which people camp frequently?}
\enumsentence{$^*$Some or which people enjoy nature.}

\begin{figure}[h*]
\centering
\begin{tabular}{c}
\psfig{figure=ps/conj-files/betaDX1conjDX2.ps,height=4.3in}
\end{tabular}
\vspace{-0.25in}
\caption{Tree for determiner conjunction: $\beta$dx1CONJdx2}
\label{DX1conjDX2}
\end{figure}

\section{Sentential Conjunction}

The tree for sentential conjunction, shown in Figure~\ref{S1conjS2},
is based on the same analysis as the conjunctions in the previous two
sections, with a slight difference in features.  The {\bf $<$mode$>$}
feature\footnote{See section~\ref{s-features} for an explanation of
the {\bf $<$mode$>$} feature.}  is used to constrain the two sentences
being conjoined to have the same mode so that {\it the day is dark and
the phone never rang} is acceptable, but {\it $\ast$the day dark and
the phone never rang} is not. Similarly, the two sentences must agree
in their {\bf $<$wh$>$}, {\bf $<$comp$>$}, {\bf $<$sub-conj$>$} and
{\bf $<$extracted$>$} features.  Co-indexation of the {\bf $<$comp$>$}
and {\bf $<$sub-conj$>$} features ensures that either both conjuncts
have the same complementizer or subordinating conjunction, or there is
a single complementizer/subordinating conjunction adjoined to the
complete conjoined S.  The {\bf $<$assign-comp$>$}
feature\footnote{See section~\ref{for-complementizer} for an
explanation of the {\bf $<$assign-comp$>$} feature.} feature is used
to allow conjunction of infinitival sentences, such as {\it to read
and to sleep is a good life}.

\begin{figure}[htb]
\centering
\begin{tabular}{c}
\psfig{figure=ps/conj-files/betaS1conjS2.ps,height=2.5in}
\end{tabular}
\caption{Tree for sentential conjunction: $\beta$s1CONJs2}
\label{S1conjS2}
\end{figure}

\section{Comma as a conjunction}

We treat comma as a conjunction in conjoined lists. It anchors the
same trees as the lexical conjunctions, but is considerably more
restricted in how it combines with them. The trees anchored by commas
are prohibited from adjoining to anything but another comma conjoined
element or a non-coordinate element. (All scope possibilities are
allowed for elements coordinated with lexical conjunctions.) Thus,
structures such as Tree
\ref{Comma-conj}(a) are permitted, with each element stacking
sequentially on top of the first element of the conjunct, while
structures such as Tree \ref{Comma-conj}(b) are blocked. 

\begin{figure}[htb]
\centering
\begin{tabular}{ccc}
{\psfig{figure=ps/conj-files/good-adj-conj.ps,height=2.5in}}&
\hspace*{0.5in}&
{\psfig{figure=ps/conj-files/bad-adj-conj.ps,height=2.5in}}\\
(a) Valid tree with comma conjunction & \hspace*{0.5in}& (b) Invalid tree\\
\end{tabular}
\caption{}
\label{Comma-conj}
\end{figure}

This is accomplished by using the {\bf $<$conj$>$} feature, which has the
values {\bf and/or/but} and {\bf comma} to differentiate the lexical
conjunctions from commas. The {\bf $<$conj$>$} values for a comma-anchored
tree and {\it and}-anchored tree are shown in Figure
\ref{conj-contrast}. The feature {\bf $<$conj$>$ = comma/none} on
A$_1$ in (a) only allows comma conjoined or non-conjoined elements as
the left-adjunct, and {\bf $<$conj$>$ = none} on A in (a) allows
only a non-conjoined element as the right conjunct. We also need the
feature {\bf $<$conj$>$ = and/or/but/none} on the right conjunct of
the trees anchored by lexical conjunctions like (b), to block
comma-conjoined elements from substituting there. Without this
restriction, we would get multiple parses of the NP in Tree
\ref{Comma-conj}; with the restrictions we only get the derivation
with the correct scoping, shown as (a).

Since comma-conjoined lists can appear without a lexical conjunction
between the final two elements, as shown in example (\ex{1}), we cannot
force all comma-conjoined sequences to end with a lexical conjunction.

\enumsentence{So it is too with many other spirits which we all know: the
spirit of Nazism or Communism, school spirit , the spirit of a street
corner gang or a football team, the spirit of Rotary or the Ku Klux
Klan. \hfill [Brown cd01]}


\begin{figure}[htb]
\centering
\begin{tabular}{cc}
{\psfig{figure=ps/conj-files/adj-comma-conj.ps,height=2.5in}}&
{\psfig{figure=ps/conj-files/adj-and-conj.ps,height=2.5in}}\\
\end{tabular}
\caption{$\beta$a1CONJa2 (a) anchored by comma and (b) anchored by {\it and}}
\label{conj-contrast}
\end{figure}


\section{{\it But-not}, {\it not-but}, {\it and-not} and  {\it
$\epsilon$-not}}

We are analyzing conjoined structures such as {\it The women but not
the men} with a multi-anchor conjunction tree anchored by the
conjunction plus the adverb {\it not}. The alternative is to allow
{\it not} to adjoin to any constituent. However, this is the only
construction where {\it not} can freely occur onto a constituent other
than a VP or adjective (cf. $\beta$NEGvx and $\beta$NEGa trees). It
can also adjoin to some determiners, as discussed in Section
\ref{det-comparitives}. We want to allow sentences like (\ex{1}) and
rule out those like (\ex{2}). The tree for the good example is shown
in Figure \ref{but-not}. There are similar trees for {\it and-not} and
{\it $\epsilon$-not}, where $\epsilon$ is interpretable as either {\it
and} or {\it but}, and a tree with {\it not} on the first conjunct for
{\it not-but}.

\enumsentence{Beth grows basil in the house (but) not in the garden.}
\enumsentence{$^*$Beth grows basil (but) not in the garden.}

\begin{figure}[htb]
\centering
\begin{tabular}{c}
\psfig{figure=ps/conj-files/but-not.ps,height=2.5in}
\end{tabular}
\caption{Tree for conjunction with but-not: $\beta$px1CONJARBpx2}
\label{but-not}
\end{figure}

Although these constructions sound a bit odd when the two conjuncts do
not have the same number, they are sometimes possible. The agreement
information for such NPs is always that of the non-negated conjunct:
{\it His sons, and not Bill, are in charge of doing the laundry} or
{\it Not Bill, but his sons, are in charge of doing the laundry}
(Some people insist on having the commas here, but they are frequently
absent in corpus data.) The agreement feature from the non-negated
conjunct in passed to the root NP, as shown in Figure
\ref{not-but}. Aside from agreement, these constructions behave just
like their non-negated counterparts.

%"Everyone is going on line and not just the younger generation," said Mal

\begin{figure}[htb]
\centering
\begin{tabular}{c}
\psfig{figure=ps/conj-files/not-but.ps,height=3.5in}
\end{tabular}
\caption{Tree for conjunction with not-but: $\beta$ARBnx1CONJnx2} 
\label{not-but}
\end{figure}

\section{{\it To} as a Conjunction}

{\it To} can be used as a conjunction for adjectives
(Fig. \ref{to-conj}) and determiners, when they denote points on a
scale:

\enumsentence{two to three degrees}
\enumsentence{high to very high temperatures}

As far as we can tell, when the conjuncts are determiners they must be
cardinal.

\begin{figure}[htb]
\centering
\begin{tabular}{c}
\psfig{figure=ps/conj-files/to.ps,height=3.5in}
\end{tabular}
\caption{Example of conjunction with {\it to}} 
\label{to-conj}
\end{figure}

\section{Other Conjunctions}

The conjunction analysis described in the previous sections is
designed to handle only the most straightforward cases of conjunction.
Three types of conjunction that are not handled are:

\begin{itemize}
\item {\bf Incomplete Constituents:} The sentence 
{\it John likes and Bill hates bananas} cannot be handled by the
current XTAG grammar.  Since {\it likes} anchors a tree that needs
both a subject noun phrase and an object noun phrase to be substituted
in, the first conjunct would need have an unfilled substitution node
after {\it John likes} for the sentence to parse.

\item {\bf Verb Phrase Conjunction:} Since verbs anchor a tree with a root 
node of type S and not VP, there is no straightforward way to implement verb
phrase conjunction.  For example, in the sentence {\it John eats cookies and
drinks beer}, there is no point in the derivation at which {\it eats cookies}
and {\it drinks beer} are available as separate trees ready to be conjoined.
They are both only subtrees in their respective S trees.  This could also be
considered as a case of incomplete constituents, since {\it drinks beer} is
missing a noun phrase.

\item {\bf Gapping:}
Sentences such as {\it John likes apples and Bill pears} are also not
handled by the previous analysis.  These could also be considered as a case
of incomplete constituents.
\end{itemize}

One grammar formalism that is capable of handling these types of 
conjunction is Combinatory Categorial Grammar (CCG) (\cite{steedman90})
which relies on a nonstandard notion of a constituent in order to accomplish
this.  Proposals have been made (e.g. \cite{joshischabes91}),
inspired by the CCG approach, to handle these problematic cases in the
FB-LTAG formalism.  Unlike the CCG analysis, however, the traditional notion
of constituents and phrase structure is maintained.  Such proposals are
as of yet unimplemented.

%test sentences 
%I ran and found a Brickel bush
% you and me and the whole world
% hook and line and bait



