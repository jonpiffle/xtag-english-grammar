\chapter{Conjunction}
\label{conjunction}

\section{Introduction}

The XTAG system can handle sentences with conjunction of two constituents of
the same syntactic category.  There are eight syntactic categories that can
conjoin, and in each case an auxiliary tree is used to implement the
conjunction.  These eight categories can be considered as four different cases,
as described in the following sections.  In all cases the two constituents are
required to be of the same syntactic category, but there may also be some
additional constraints, as described below.

\section{Adjective, Adverb, Preposition and PP Conjunction}

Each of these four categories has an auxiliary tree that is used for
conjunction of two constituents of that category.  The auxiliary tree
adjoins into the left-hand-side component, and the right-hand-side
component substitutes into the auxiliary tree.  

\begin{figure}[htb]
\centering
\begin{tabular}{ccc}
{\psfig{figure=ps/conj-files/betaA1conjA2.ps,height=0.8in}}&
\hspace*{0.5in}&
{\psfig{figure=ps/conj-files/derived-tree-140291.ps,height=1.8in}}\\
(a) & \hspace*{0.5in}& (b)\\
\end{tabular}
\caption{Tree for adjective conjunction: $\beta$A1conjA2 and a resulting parse tree}
\label{A1conjA2}
\end{figure}

Figure~\ref{A1conjA2}(a) shows the auxiliary tree for adjective conjunction,
and is used, for example, in the derivation of the parse tree for the noun
phrase {\it the dark and dreary day}, as shown in Figure~\ref{A1conjA2}(b).
The auxiliary tree adjoins onto the node for the left adjective, and the
right adjective substitutes into the right hand side node of the auxiliary
tree. The analysis for adverb, preposition and PP conjunction is exactly the
same and there is a corresponding auxiliary tree for each of these that is
identical to that of Figure~\ref{A1conjA2}(a) except, of course, for the node
labels.


\section{Noun Phrase and Noun Conjunction}

The tree for NP conjunction, shown in Figure~\ref{NP1conjNP2}, has the same
basic analysis as in the previous section except that the {\bf $<$wh$>$} and
{\bf $<$case$>$} features are used to force the two noun phrases to have the
same {\bf $<$wh$>$} and {\bf $<$case$>$} values.  This allows, for example,
{\it he and she wrote the book together} while disallowing {\it $\ast$he and
her wrote the book together.}  The {\bf $<$agr$>$} feature of the top node sets
the resulting NP to have plural number.  The tree for N conjunction is
identical to that for the NP tree except for the node labels.

\begin{figure}[htb]
\centering
\mbox{}
\psfig{figure=ps/conj-files/betaNP1conjNP2.ps,height=2.2in}
\caption{Tree for NP conjunction: $\beta$NP1conjNP2}
\label{NP1conjNP2}
\end{figure}


\section{Sentential Conjunction}

The tree for sentential conjunction, shown in Figure~\ref{S1conjS2}, is based
on the same analysis as the conjunctions in the previous two sections, with a
slight difference in features.  The {\bf $<$mode$>$} feature\footnote{See
section~\ref{s-features} for an explanation of the {\bf $<$mode$>$} feature.}
is used to constrain the two sentences being conjoined to have the same mode so
that {\it the day is dark and the phone never rang} is acceptable, but {\it
$\ast$the day dark and the phone never rang} is not.  The {\bf
$<$assign-comp$>$} feature\footnote{See section~\ref{for-complementizer} for an
explanation of the {\bf $<$assign-comp$>$} feature.} feature is used to allow
conjunction of infinitival sentences, such as {\it to read and to sleep is a
good life}.

\begin{figure}[htb]
\centering
\begin{tabular}{c}
\psfig{figure=ps/conj-files/betaS1conjS2.ps,height=1.8in}
\end{tabular}
\caption{Tree for sentential conjunction: $\beta$S1conjS2}
\label{S1conjS2}
\end{figure}

\section{Determiner Conjunction}

The tree for determiner conjunction, shown in Figure~\ref{DX1conjDX2}, is
unlike the other conjunction trees in that the foot node is on the right.  This
is because determiner phrases generally build to the left. For the same reason,
all the various feature values are taken from the left determiner, and the only
requirement is that the {\bf $<$wh$>$} feature is the same, while the other
features, such as {\bf $<$card$>$}, are unconstrained.  For example, {\it who
and which} and {\it all but one} are both acceptable determiner conjunctions,
but {\it $\ast$who and all} is not.

\begin{figure}[htb]
\centering
\begin{tabular}{c}
\psfig{figure=ps/conj-files/betaDX1conjDX2.ps,height=3.5in}
\end{tabular}
\caption{Tree for determiner conjunction: $\beta$DX1conjDX2}
\label{DX1conjDX2}
\end{figure}

\section{Other Conjunctions}

The conjunction analysis described in the previous sections is designed
to handle only the most straightforward cases of conjunction.  Three
types of conjunction that are not handled are:

\begin{itemize}
\item {\bf Incomplete Constituents} Although the sentence 
{\it John likes and Bill hates bananas} is a simple case of sentential
conjunction, it cannot be handled by the current XTAG grammar.  Since {\it
likes} anchors a tree that needs both a subject noun phrase and an object noun
phrase to be substituted in, the latter sentence would need have an unfilled
substitution node after {\it John likes} for the sentence to parse.

\item {\bf Verb Phrase Conjunction} Since verbs anchor a tree with a root node 
of type S and not VP, there is no straightforward way to implement verb phrase
conjunction.  For example, in the sentence {\it John eats cookies and drinks
beer}, there is no point in the derivation at which {\it eats cookies} and {\it
drinks beer} are available as separate trees ready to be conjoined.  They are
both only subtrees in their respective S trees.  This could also be considered
as a case of incomplete constituents, since {\it drinks beer} is missing a noun
phrase.

\item {\bf Gapping}
Sentences such as {\it John likes apples and Bill pears} are also not
handled by the previous analysis.  These could also be considered as a case
of incomplete constituents.
\end{itemize}

One grammar formalism that is capable of handling these types of 
conjunction is Combinatory Categorial Grammar (CCG) (\cite{steedman90})
which relies on a nonstandard notion of a constituent in order to accomplish
this.  Proposals have been made (e.g. \cite{joshischabes91}),
inspired by the CCG approach, to handle these problematic cases in the
FB-LTAG formalism.  Unlike the CCG analysis, however, the traditional notion
of constituents and phrase structure is maintained.  Such proposals are
as of yet unimplemented.

%test sentences 
%I ran and found a Brickel bush
% you and me and the whole world
% hook and line and bait



