\chapter{Ditransitive constructions and dative shift}
\label{double-objs}

Verbs such as {\it give\/} and {\it put\/} that require two objects, as
shown in examples (\ex{1})-(\ex{4}), are termed ditransitive.

\enumsentence{Christy gave a cannoli to Beth Ann .}
\enumsentence{$\ast$Christy gave Beth Ann .}
\enumsentence{Christy put a cannoli in the refrigerator .} 
\enumsentence{$\ast$Christy put a cannoli .}


The indirect objects {\it Beth Ann\/} and {\it refrigerator\/} appear in
these examples in the form of PP's.  Within the set of ditransitive
verbs there is a subset that also allow two NP's as in (\ex{1}). As can
be seen from (\ex{1}) and (\ex{2}) this two NP, or double-object,
construction is grammatical for {\it give\/} but not for {\it put}.  

\enumsentence{Christy gave Beth Ann a cannoli .}
\enumsentence{$\ast$Christy put the refrigerator the cannoli .}

The alternation between (\ex{-5}) and (\ex{-1}) is known as dative
shift.\footnote{In languages similar to English that have overt case marking
indirect objects would be marked with dative case. It has also been suggested
that for English the preposition {\it to} serves as a dative case marker.} In
order to account for verbs with dative shift the English XTAG grammar includes
structures for both variants in the tree family Tnx0Vnx1Pnx2.  The declarative
trees for the shifted and non-shifted alternations are shown in
Figure~\ref{dative-alt}.


\begin{figure}[htb]
\centering
\begin{tabular}{ccc}
{\psfig{figure=ps/double-obj-files/alphanx0Vnx1Pnx2.ps,height=2.0in}}&
\hspace*{0.5in} &
{\psfig{figure=ps/double-obj-files/alphanx0Vnx2nx1.ps,height=1.1in}}
\\
(a)&\hspace*{0.5in}&(b)\\
\end{tabular}
\caption{Dative shift trees: $\alpha$nx0Vnx1Pnx2 (a) and $\alpha$nx0Vnx2nx1 (b)}
\label{dative-alt}
\label{2;1,2}
\end{figure}

The indexing of nodes in these two trees represents the fact that the semantic
role of the indirect object (NP$_2$) in Figure~\ref{dative-alt}(a) is the same
as that of the direct object (NP$_2$) in Figure~\ref{dative-alt}(b) (and vice
versa).  This use of indexing is consistent with our treatment of other
constructions such as passive and ergative.

%Tonia: Tnx0Vnx1Pnx2 does not contain trees for NP NP structure either

Verbs that do not show this alternation and have only the NP PP structure
(e.g. {\it put\/}) select the tree family Tnx0Vnx1pnx2.  Unlike the
Tnx0Vnx1Pnx2 family, the Tnx0Vnx1pnx2 tree family does not contain trees for
the NP NP structure. Other verbs such as {\it ask} allow only the NP NP
structure as shown in (\ex{1}) and (\ex{2}).

\enumsentence{Beth Ann asked Srini a question .}
\enumsentence{$\ast$Beth Ann asked a question to Srini .}

Verbs that only allow the NP NP structure select the tree family
Tnx0Vnx2nx1. This tree family does not have the trees for the NP PP
structure. 

There are other apparent cases of dative shift with {\it for}, such as in
(\ex{1}) and (\ex{2}), that we have taken to be structurally distinct from the
cases with {\it to}.

\enumsentence{Beth Ann baked Dusty a biscuit .}
\enumsentence{Beth Ann baked a biscuit for Dusty .}

\cite{mccawley88} notes:

\begin{quote}
A ``{\it for-dative}'' expression in underlying structure is external
to the V with which it is combined, in view of the fact that the
latter behaves as a unit with regard to all relevant syntactic
phenomena.
\end{quote}


In other words, the {\it for} PP's that appear to undergo dative shift are
actually adjuncts, not complements. Examples (\ex{1}) and (\ex{2}) demonstrate
that PP's with {\it for} are optional while ditransitive {\it to} PP's are not.

\enumsentence{Beth Ann made dinner .}
\enumsentence{$\ast$Beth Ann gave dinner .}

% Fei: 1/13/98 It needs to be changed. Tonia: ``to'' not built in.
Consequently, in the XTAG grammar the apparent dative shift with {\it
  for} PP's is treated as Tnx0Vnx2nx1 for the NP NP structure, and as
a transitive plus an adjoined adjunct PP for the NP PP structure.  To
account for the ditransitive {\it to} PP's, the preposition {\it to}
is built into the tree family Tnx0Vnx1tonx2. This accounts for the
fact that {\it to} is the only preposition allowed in dative shift
constructions.

\cite{mccawley88} also notes that the {\it to} and {\it for} cases
differ with respect to passivization; the indirect objects with {\it to} may be
the subjects of corresponding passives, as seen in ~(\ex{1})-(\ex{2}), while
the alleged indirect objects with {\it for} cannot, as in
sentences~(\ex{3})-(\ex{4}).  Note that the passivisation examples are for NP
NP structures of verbs that take {\it to} or {\it for} PP's.

\enumsentence{Beth Ann gave Clove dinner .}
\enumsentence{Clove was given dinner (by Beth Ann) .}
\enumsentence{Beth Ann made Clove dinner .}
\enumsentence{?Clove was made dinner (by Beth Ann) .} 

However, we believe that this picture is inaccurate, and that the indirect
objects in the {\it for} case are, in fact, allowed to be the subjects of
passives, as in sentences~(\ex{1})-(\ex{2}).  The apparent strangeness of
sentence~(\ex{0}) is caused by interference from other interpretations of {\it
Clove was made dinner .}

\enumsentence{Dania baked Doug a cake .}
\enumsentence{Doug was baked a cake by Dania .}






