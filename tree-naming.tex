\chapter{Tree Naming conventions}
\label{tree-naming}

The various trees within the XTAG grammar are named more or less according to
the following tree naming conventions.  Although these naming conventions are
generally followed, there are occasional trees that do not strictly follow
these conventions.

\section{Tree Families}
Tree families are named according to the basic declarative tree structure in
the tree family (see section~\ref{family-trees}), but with a T as the first
character instead of an $\alpha$ or $\beta$.

\section{Trees within tree families}
\label{family-trees}

Each tree with begin with either an $\alpha$ (alpha) or a $\beta$ (beta)
symbol, indicating whether it is an initial or auxiliary tree, respectively.
Following an $\alpha$ or a $\beta$ the name may additionally contain one of:

\begin{description}
\item\begin{tabular}{ll}
I&imperative\\
E&ergative\\
N{0,1,2}&relative clause\{position\}\\
G&NP gerund\\
D&Determiner gerund\\
pW{0,1,2}&wh-PP extraction\{position\}\\
W{0,1,2}&wh-NP extraction\{position\}\\
\end{tabular}
\end{description}

\noindent Numbers are assigned according to the position of the argument in the
declarative tree, as follows:

\begin{description}
\item\begin{tabular}{ll}
0&subject position\\
1&first argument (e.g. direct object)\\
2&second argument (e.g. indirect object)\\
\end{tabular}
\end{description}

\noindent The body of the name consists of a string of the following 
components, which corresponds to the leaves of the tree.  The anchor(s) of the
trees is(are) indicated by capitalizing the part of speech corresponding to the
anchor.

\begin{description}
\item\begin{tabular}{ll}
s&sentence\\
a&adjective\\
arb&adverb\\
be&{\it be}\\
x&phrasal category\\
d&determiner\\
v&verb\\
lv&light verb\\
conj&conjunction\\
comp&complementizer\\
it&{\it it}\\
n&noun\\
p&preposition\\
pl&particle\\
by&{\it by}\\
neg&negation\\
\end{tabular}
\end{description}

\noindent As an example, the transitive declarative tree consists of a subject
NP, followed by a verb (which is the anchor), followed by the object NP.  This
translates into $\alpha$nx0Vnx1.  If the subject NP had been extracted, then
the tree would be $\alpha$W0nx0Vnx1.  A passive tree with the {\it by} phrase
in the same tree family would be $\alpha$nx1Vbynx0.  Note that even though the
object NP has moved to the subject position, it retains the object encoding
(nx1).

\section{Assorted Initial Trees}

Trees that are not part of the tree families are generally gathered into
several files for convenience.  The various initial trees are located in
lex.trees.  All the trees in this file should begin with an $\alpha$,
indicating that they are initial trees.  This is followed by the root category
which follows the naming conventions in the previous section (e.g. n for noun,
x for phrasal category).  The root category is in all capital letters.  After
the root category, the node leaves are named, beginning from the left, with the
anchor of the tree also being capitalized.  As an example, the $\alpha$NXdxN
tree is rooted by an NP node (NX), with a determiner phrase subnode (dx), and
anchored by a noun (N).  This tree is shown in Figure~\ref{NXdxN}.

\begin{figure}[hbt]
\centering
\begin{tabular}{c}
{\psfig{figure=ps/tree-naming-files/alphaNXdxN.ps,height=2.5cm}}
\end{tabular}
\caption{NP with determiner tree: $\alpha$NXdxN}
\label{NXdxN}
\end{figure}


\section{Assorted Auxiliary Trees}

Most auxiliary trees are contained in modifiers.trees, although a couple of
other files also contain auxiliary trees.  The auxiliary trees follow a
slightly different naming convention from the initial trees.  Since the root
and foot nodes must be the same for the auxiliary trees, the root nodes are not
explicitly mentioned in the names of auxiliary trees.  The trees are named
according to the leaf nodes, starting from the left, and capitalizing the
anchor node.  All auxiliary trees begin with a $\beta$, of course.  For
example, $\beta$ARBs, indicates a tree anchored by an adverb (ARB), that
adjoins onto the left of an S node (Note that S must be the foot node, and
therefore also the root node).

