The grammar also includes auxiliary trees anchored by multi-word adverbs
like {\it a little}, {\it a bit}, {\it a mite}, {\it sort of}, {\it kind
of}, etc.. 

Multi-word adverbs like {\it sort of} and {\it kind of} can pre- modify
almost any non-clausal category. The only strict constraint on their
occurrence is that they can't modify nouns (in which case an adjectival
interpretation would obtain)\footnote{Note that there are semantic/lexical
constraints even for the categories that these adverbs {\it can} modify,
and no doubt invite a more in-depth analysis.}. The category which they
scope over can be directly determined from their position, except for when
they occur sentence finally in which case they are assumed to modify
VP's. The complete list of auxiliary trees anchored by these adverbs are as
follows: $\beta$NPax, $\beta$NPpx, $\beta$NPnx, $\beta$NPvx, $\beta$vxNP,
$\beta$NParb. Selected trees are shown in Figure~\ref{sortof-adv-tree}, and
some examples are given in (\ex{1})-(\ex{4}).

\begin{figure}[htb]
\centering
\begin{tabular}{ccccccc}
{\psfig{figure=ps/modifiers-files/betaNPax.ps,height=1.5in}}
& \hspace{.5in} & 
{\psfig{figure=ps/modifiers-files/betaNPvx.ps,height=1.5in}}
& \hspace{.5in} &
{\psfig{figure=ps/modifiers-files/betavxNP.ps,height=1.5in}}
\\
$\beta$NPax&&$\beta$NPvx&&$\beta$vxNP&&\\
(a)&&(b)&&(c)&&\\
\end{tabular}\\
\caption{Selected Multi-word Adverb Modifier trees (for adverbs like {\it
sort of}, {\it kind of}): $\beta$NPax, $\beta$NPvx, $\beta$vxNP.}
\label{sortof-adv-tree}
\end{figure}  



\enumsentence{John is {\bf sort of} [$_{AP}$ tired].}

\enumsentence{John is {\bf sort of} [$_{PP}$ to the right].}
 
\enumsentence{John could have been {\bf sort of} [$_{VP}$ eating the cake].}
 
\enumsentence{John has been eating his cake {\bf sort of} [$_{ADV}$ slowly].}


There are some multi-word adverbs that are, however, not so free in their
distribution. Adverbs like {\it a little}, {\it a bit}, {\it a mite} modify
AP's in predicative constructions (sentences with the copula and small
clauses, AP complements in sentences with raising verbs, and AP's when they
are subcategorized for by certain verbs (e.g., {\it John felt angry}). They
can also post-modify VP's and PP's, though not as freely as
AP's\footnote{They can also appear before NP's, as in, ``John wants {\it a
little} sugar''. However, here they function as multi-word determiners and
should not be analyzed as adverbs.}. Finally, they also function as
downtoners for almost all adverbials\footnote{It is to be noted that this
analysis, which allows these multiword adverbs to modify adjectival phrases
as well as adverbials, will yield (not necessarily desirable) multiple
derivations for a sentence like {\it John is a little unecessarily
stupid}. In one derivation, {\it a little} modifies the AP and in the other
case, it modifies the adverb.}. Some examples are provided in
(\ex{5})-(\ex{8}).


\enumsentence{Mickey is {\bf a little} [$_{AP}$ tired].}

\enumsentence{The medicine [$_{VP}$ has eased John's pain] {\bf a little}.}

\enumsentence{John is {\bf a little} [$_{PP}$ to the right].}

\enumsentence{John has been reading his book {\bf a little} [$_{ADV}$ loudly].}

Following their behavior as described above, the auxiliary trees they
anchor are $\beta$DAax, $\beta$DApx, $\beta$vxDA, $\beta$DAarb,
$\beta$DNax, $\beta$DNpx, $\beta$vxDN, $\beta$DNarb. Selected trees are
shown in Figure~\ref{alittle-adv-tree}).


\begin{figure}[htb]
\centering
\begin{tabular}{ccccccc}
{\psfig{figure=ps/modifiers-files/betavxDA.ps,height=1.5in}}
& \hspace{.5in} & 
{\psfig{figure=ps/modifiers-files/betaDAax.ps,height=1.5in}}
& \hspace{.5in} &
{\psfig{figure=ps/modifiers-files/betaDNpx.ps,height=1.5in}}
\\
$\beta$vxDA&&$\beta$DAax&&$\beta$DNpx&&\\
(a)&&(b)&&(c)&&\\
\end{tabular}\\
\caption{Selected Multi-word Adverb Modifier trees (for adverbs like {\it
a little}, {\it a bit}): $\beta$vxDA, $\beta$DAax, $\beta$DNpx.}
\label{alittle-adv-tree}
\end{figure}


